\documentclass[answers,12pt,addpoints]{exam} 
\usepackage{import}

\import{C:/Users/prana/OneDrive/Desktop/MathNotes}{style.tex}

% Header
\newcommand{\name}{Pranav Tikkawar}
\newcommand{\course}{01:XXX:XXX}
\newcommand{\assignment}{Homework n}
\author{\name}
\title{\course \ - \assignment}

\begin{document}
\maketitle
\tableofcontents
\newpage
\section{Problems}
\begin{example}[3 Coloring of NxN]
    what is the number of all 3 colorings of an NxN grid such that no two adjacent cells have the same color?
    LB = $\sqrt{2}^{n^2}$\\
    UB = $3*2^{n^2}$\\
    $$N(n) \sim \frac{4}{3}^{\frac{3}{2}^{n^2}}$$
    this number is the squareice constant\\
    Ice Problem Elliot Lieb
\end{example}
\begin{example}[Domino Tilings of NxN]
    what is the number of ways to tile an NxN grid with 2x1 dominoes?
    Not possible if N is odd\\
    Im thinking LB is $(\frac{N}{2})^2$\\
    UB is $N!$\\
    $$N(n) \sim $$
    For $N(8) = 12988816$\\
    We know for $N$ even it exists\\
\end{example}
\begin{example}[Magic Square]
    MS for n=2 does not exist\\
    \begin{proof}
        Let the square be 
        \[
        \begin{bmatrix}
            a & b\\
            c & d
        \end{bmatrix}
        \]
        then we have $a+b=c+d$ and $a+c=b+d$ and $a+d=b+c$\\
        then $a=d$ and $b=c$ and $a=b=c=d$\\
        so all numbers are equal which is not possible since we want distinct positive integers.
    \end{proof}
\end{example}









\end{document}
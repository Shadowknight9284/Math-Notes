\documentclass{article}
\usepackage{amsmath}
\usepackage{amssymb}

\begin{document}

\title{Homework 1 Solutions}
\author{Student Name}
\date{October 2025}
\maketitle

\section*{Problem 1: Relational Algebra Expressions}

1. Find the name of each employee who lives in city Miami.

\textbf{Solution:}
\[
\pi_{\text{person\_name}}(\sigma_{\text{city} = \text{'Miami'}}(\text{employee}))
\]

2. Find the name of each employee whose salary is greater than 100,000.

\textbf{Solution:}
\[
\pi_{\text{person\_name}}(\sigma_{\text{salary} > 100000}(\text{employee} \bowtie \text{works}))
\]

3. Find the name of each employee who lives in Miami and whose salary is greater than 100,000.

\textbf{Solution:}
\[
\pi_{\text{person\_name}}(\sigma_{\text{city} = 'Miami' \land \text{salary} > 100000}(\text{employee} \bowtie \text{works}))
\]

4. Find the ID and name of each employee who does not work for BigBank.

\textbf{Solution:}
\[
\pi_{\text{ID}, \text{person\_name}}(\sigma_{\text{companyname} \neq 'BigBank'}(\text{employee} \bowtie \text{works}))
\]

5. Find the ID and name of each employee who earns at least as much as every employee in the database.

\textbf{Solution:}
\[
\pi_{\text{ID}, \text{person\_name}} (\text{employee} \bowtie \text{works} - \]
\[\pi_{\text{employee1ID}, \text{personname1}} (\sigma_{\text{salary1} < \text{salary2}}(\rho_{\text{salary1}, \text{employee1ID}, \text{personname1}}(\text{works}) \times \rho_{\text{salary2}}(\text{works}))))
\]


\section*{Problem 2: Schedule}

1. Give the dependency graph for the schedule in the figure.

\textbf{Solution:}
Draw nodes for each transaction ($T_A, T_B, T_C$). For each pair of conflicting actions (read/write), draw edges from one transaction to another if the first's action comes before the other's and both access the same object.

2. Is the schedule conflict-serializable?

\textbf{Solution:}
Yes/No. (Requires analyzing the graph. Yes if acyclic, No if cycles)

3. If yes, serial equivalent; if not, explanation.

\textbf{Solution:}
If the dependency graph is acyclic, list transactions in any topological order. If cyclic, explain which actions cause cycles.


\section*{Problem 3: Serializability True/False}

1. A schedule with exactly one transaction is always conflict-equivalent to a serial schedule.

\textbf{Solution:}
True

2. If each transaction preserves database consistency, every serializable schedule also preserves consistency.

\textbf{Solution:}
True

3. A schedule is conflict serializable if and only if its dependency graph is acyclic.

\textbf{Solution:}
True

4. All schedules that avoid cascading aborts are recoverable.

\textbf{Solution:}
True

5. All schedules that avoid cascading aborts are conflict-serializable.

\textbf{Solution:}
False


\section*{Problem 4: Two-Phase Locking (2PL)}

1. Add lock/unlock instructions so transactions obey 2PL.

\textbf{Solution:}
All locks must be acquired before any unlocks for each transaction. Example order: Lock-SA, Lock-XB, then UnlockA, UnlockB.

2. Can the execution result in deadlock?

\textbf{Solution:}
Yes

\end{document}

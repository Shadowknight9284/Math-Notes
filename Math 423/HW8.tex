\documentclass[answers,12pt,addpoints]{exam}
\usepackage{import}
\usepackage{cancel} % Add this line to include the cancel package

\import{C:/Users/prana/OneDrive/Desktop/MathNotes}{style.tex}

% Header
\newcommand{\name}{Pranav Tikkawar}
\newcommand{\course}{01:XXX:XXX}
\newcommand{\assignment}{Homework n}
\author{\name}
\title{\course \ - \assignment}

\begin{document}
\maketitle


\newpage
\begin{questions}
    \question Section 7.1 Problem 2\\
    Prove the uniqueness up to constants of the Neumann problem using the energy method.
    \begin{solution}
        The Neumann problem is given by
        \begin{align*}
            \Delta u = f \quad \text{in } D\\
            \frac{\partial u}{\partial n} = g \quad \text{on } \partial D
        \end{align*}
        Suppose there is another solution $v$ to the Neumann problem. Then, we have
        \begin{align*}
            \Delta v = f \quad \text{in } D\\
            \frac{\partial v}{\partial n} = g \quad \text{on } \partial D
        \end{align*}
        Subtracting the two equations, we get
        \begin{align*}
            \Delta (u - v) = 0 \quad \text{in } D\\
            \frac{\partial (u - v)}{\partial n} = 0 \quad \text{on } \partial D
        \end{align*}
        If we let $w = u - v$, then we have
        \begin{align*}
            \Delta w = 0 \quad \text{in } D\\
            \frac{\partial w}{\partial n} = 0 \quad \text{on } \partial D
        \end{align*}
        We can use Green's first identity to get
        \begin{align*}
            \int_{\partial D} w \frac{\partial w}{\partial n} dS = \int_D |\nabla w|^2 dV + \int_D w \Delta w dV
        \end{align*}
        Since $\Delta w = 0$, the second term on the right-hand side is zero:
        \begin{align*}
            \int_{\partial D} w \cancel{\frac{\partial w}{\partial n}}^{\text{ goes to }0} dS &= \int_D |\nabla w|^2 dV + \int_D w \cancel{\Delta w}^{\text{ goes to }0} dV\\
            0 &= \int_D |\nabla w|^2 dV\\
            |\nabla w|^2 &= 0\\
            \nabla w &= 0\\
            w &= \text{constant}
        \end{align*}
        Therefore $u - v = w = \text{constant} \implies u = v + \text{constant}$ which proves the uniqueness of the solution up to constants.
    \end{solution}
    \question Section 7.1 Problem 5
    Probe Dirichlet's prinicple for the Neueman boundary condition. It asserts that among all real valued functions $w(x)$ on $D$ the quantity
    $$ E[w] = \frac{1}{2} \iiint_D |\nabla w|^2 \, dV - \iiint_{\partial d} h w \, dS $$
    is the smallest for $w=u$ where $u$ is the solution of the Neumann problem
    \begin{align*}
        \Delta u = 0 \quad \text{in } D\\
        \frac{\partial u}{\partial n} = h(x) \quad \text{on } \partial D
    \end{align*}
    It is required to assume that the average of the given function $h(x)$ is zero.\\
    Notice 3 features of this prinicple:
    \begin{parts}
        \part There is no contraint at all on the trial functions $w(x)$.
        \part the function $h(x)$ appears in the energy
        \part the functional $E[w]$ does not change if a constant is added to $w(x)$.
    \end{parts}

    \begin{solution}
        Let $u$ be the solution to the Neumann problem and $w$ be any other function. Then, we can say that $w = u -v$ for some function $v$. Now the energy of $w$ is given by
        \begin{align*}
            E[w] &= \frac{1}{2} \iiint_D |\nabla w|^2 \, dV - \iiint_{\partial d} h w \, dS\\
            &= \frac{1}{2} \iiint_D |\nabla (u - v)|^2 \, dV - \iiint_{\partial d} h (u - v) \, dS\\
            &= \frac{1}{2} \iiint_D |\nabla u|^2 - 2 \nabla u \cdot \nabla v + |\nabla v|^2 \, dV - \iiint_{\partial d} h u - \iiint_{\partial d} h v \, dS\\
            &= \frac{1}{2} \iiint_D |\nabla u|^2 \, dV - \iiint_{\partial d} h u \, dS - \iiint_{D} \nabla u \nabla v \, dV + \frac{1}{2} \iiint_D |\nabla v|^2 \, dV - \iiint_{\partial d} h v \, dS\\
            &= E[u] - \iiint_{D} \nabla u \nabla v \, dV + \frac{1}{2} \iiint_D |\nabla v|^2 \, dV + \iiint_{\partial d} h v \, dS
        \end{align*}
        Using green's first identity, we can write
        \begin{align*}
            \iiint_{D} \nabla u \nabla v \, dV = \iiint_{\partial D} v \frac{\partial u}{\partial n} \, dS - \iiint_{D} v \Delta u \, dV
        \end{align*}
        \begin{align*}
            E[w] &= E[u] - \cancel{\iiint_{\partial D} v \frac{\partial u}{\partial n} \, dS} + \iiint_{D} v \cancel{\Delta u}^{\text{ goes to }0} \, dV + \frac{1}{2} \iiint_D |\nabla v|^2 \, dV + \cancel{\iiint_{\partial d} h v \, dS}\\
            &= E[u] + \frac{1}{2} \iiint_D |\nabla v|^2 \, dV
        \end{align*}
        Clearly the second term of the RHS is non-negative. Therefore, $E[w] \geq E[u]$ and the energy is minimized if $w = u$.
    \end{solution}
    \question Section 7.2 Problem 2\\
    Let $\phi(x)$ be any $C^2$ function defined on all of three-dimensional space that vanishes outside some sphere. Show that 
    $$
    \phi(0) = -\frac{1}{4\pi} \int \frac{\Delta \phi(x)}{|x|} \, dx.
    $$
    The integration is taken over the region where $\phi(x)$ is not zero.
    \begin{solution}
        We can let $u = \phi$ and $v = \frac{1}{4\pi|x|}$. Note that $\Delta v = 0$ since it is the fundamental solution of the laplacian in $R^3$ Then we can use Green's second identity to get
        \begin{align*}
            \iiint_D u \Delta v - v \Delta u \, dV &= \int_{\partial D} u \frac{\partial v}{\partial n} - v \frac{\partial u}{\partial n} \, dS\\
            \iiint_D \left( \phi \Delta \left(\frac{1}{4\pi|x|}\right) - \frac{1}{4\pi|x|} \Delta \phi \right) \, dV &= \int_{\partial D} \phi \frac{\partial}{\partial n} \left(\frac{1}{4\pi|x|}\right) - \frac{1}{4\pi|x|} \frac{\partial \phi}{\partial n} \, dS\\
            \iiint_D \frac{\Delta \phi}{4 \pi r} dV &= \frac{1}{4\pi} \int_{\partial D} -\phi \frac{\partial}{\partial n} \left(\frac{1}{r}\right) + \left( \frac{1}{r}\right) \frac{\partial \phi}{\partial n} \, dS\\
            \iiint_D \frac{\Delta \phi}{4 \pi r} dV &= \frac{1}{4\pi} \int_{\partial D} \phi  \left(\frac{1}{r^2}\right) \, dS\\
            \iiint_D \frac{\Delta \phi}{4 \pi r} dV &= -\frac{1}{4\pi R^2} \int_{\partial D} \phi \, dS\\
            \iiint_D \frac{\Delta \phi}{4 \pi r} dV &= -\overline{\phi} \quad \text{where } \overline{\phi} \text{ is the average of } \phi \text{ over the sphere}
        \end{align*} 
        Now if we take the lim of $R \to 0$ we see that the RHS becomes $\phi(0)$ and the LHS becomes the integral we want to evaluate. Therefore, we have
        $$
        \phi(0) = -\frac{1}{4\pi} \int \frac{\Delta \phi(x)}{|x|} \, dx.
        $$
    \end{solution}
    \question Section 7.3 Problem 2
    Prove Theorm 2, which gives the solution of Poisson's equation in terms of the Green's Function.
    \begin{solution}
        Theorem 2 states:\\
        The solution of 
        $$\begin{cases}
            \Delta u = f \quad \text{in } D\\
            u = h \quad \text{on } \partial D
        \end{cases}$$
        is given by
        $$u(x_0) = \int_{\partial D} h(x) \frac{\partial G(x_0, x)}{\partial n} \, dS + \int_D f(x) G(x,x_0) \, dV$$
        Let us first consider the $\nabla \cdot (v \nabla u) = \nabla v \cdot \nabla u + v \Delta u$ and integrate over $D$ to get
        \begin{align*}
            \int_D \nabla \cdot (v \nabla u) \, dV &= \int_D \nabla v \cdot \nabla u + v \Delta u \, dV\\
            \int_{\partial D} v \frac{\partial u}{\partial n} \, dS &= \int_D \nabla v \cdot \nabla u + v \Delta u \, dV \quad \text{by divergence theorem}\\
            \int_{\partial D} v \frac{\partial u}{\partial n} \, dS &= \int_D \nabla v \cdot \nabla u \, dV + \int_D v \Delta u \, dV\\
            \int_{\partial D} u \frac{\partial v}{\partial n} \, dS &= \int_D \nabla u \cdot \nabla v \, dV + \int_D u \Delta v \, dV \quad \text{since u, v are arbitrary we can switch them}\\
            \int_{\partial D} u \frac{\partial v}{\partial n} - v \frac{\partial u}{\partial n} \, dS &= \int_D  u \cdot \Delta v -  v \cdot \Delta u \, dV \quad \text{subtracting the two equations}
        \end{align*}
        Replace $v = G(x, x_0)$ and $u = u(x_0)$ where 
        $$\begin{cases}
            \Delta G(x, x_0) = \delta(x - x_0) \quad \text{in } D\\
            G(x, x_0) = 0 \quad \text{on } \partial D
        \end{cases}$$
        then we have 
        \begin{align*}
            \int_{\partial D} u \frac{\partial G(x, x_0)}{\partial n} - G(x, x_0) \frac{\partial u}{\partial n} \, dS &= \int_D u \Delta G(x, x_0) - G(x, x_0) \Delta u \, dV\\
            \int_{\partial D} h \frac{\partial G(x, x_0)}{\partial n} \, dS &= \int_D u \delta(x - x_0) - G(x, x_0) f \, dV\\
            \int_{\partial D} h \frac{\partial G(x, x_0)}{\partial n} \, dS &= u(x_0) - \int_D f(x) G(x, x_0) \, dV\\
            u(x_0) &= \int_{\partial D} h(x) \frac{\partial G(x_0, x)}{\partial n} \, dS + \int_D f(x) G(x,x_0) \, dV
        \end{align*}
        As desired.
    \end{solution}
    \question Section 7.4 Problem 2
    Verify directly from (3) or (4) that the solution of the half-space problem satisfies the condition at infinity: \( u(x) \to 0 \) as \( |x| \to \infty \). Assume that \( h(x,y) \) is a continuous function that vanishes outside some circle.
    \begin{solution}
        The solution of the half space probelm 
        $$\begin{cases}
            \Delta u = 0 \quad z >0 \\
            u(x,y,0) = h(x,y)
        \end{cases}$$
        Is given by 
        $$ u(x_0, y_0, z_0) = \frac{z_0}{2\pi} \int_{R^2} \frac{h(x,y)}{[(x-x_0)^2 + (y-y_0)^2 + z_0^2]^{3/2}} \, dx \, dy$$
        We can then switch $x_0, y_0, z_0$ and $x, y, z$ 
        $$ u(x, y, z) = \frac{z}{2\pi} \int_{R^2} \frac{h(x_0,y_0)}{[(x-x_0)^2 + (y-y_0)^2 + z^2]^{3/2}} \, dx_0 \, dy_0$$
        To verify we can take an easy function for h in the circle and see if the solution goes to zero at infinity. Let $h(x,y) = 1$ for $x\leq 1, y \leq 1$ and $h(x,y) = 0$ otherwise. Then we have
        \begin{align*}
            u(x, y, z) &= \frac{z}{2\pi} \int_0^1 \int_0^1 \frac{1}{[(x-x_0)^2 + (y-y_0)^2 + z^2]^{3/2}} \, dx_0 \, dy_0
        \end{align*}
        Through some calculations that I did on on wolfram alpha, we can se that
        $$ u = \frac{1}{2\pi}[\tan^{-1}\frac{x(1-y)}{z\sqrt{x^2+(1-y)^2+z^2}} - \tan^{-1}\frac{x(-y)}{z\sqrt{x^2+y^2+z^2}}$$
        $$-\tan^{-1}\frac{(x-1)(y-1)}{z\sqrt{(1-x)^2+(1-y)^2+z^2}} - \tan^{-1}\frac{x(y)}{z\sqrt{(x-1)^2+(y)^2+z^2}}] $$
        Now taking the limit as $|x| \to \infty$ we see that the function goes to zero. Therefore, the solution satisfies the condition at infinity.
    \end{solution}
    \question Section 7.4 Problem 5
    Notice that the function \(xy\) is harmonic in the half-plane \(\{y > 0\}\) and vanishes on the boundary line \(\{y = 0\}\). The function \(0\) has the same properties. Does this mean that the solution is not unique? Explain.
    \begin{solution}
        While there may be multiple solutions for the Laplace equation in half space with the boundary line $y=0$, We need to ensure it also satisfies the condition at infinity. In this case, we want to show that $\lim_{x \to \pm \infty} u = 0$ and $\lim_{y \to \infty} u = 0$. We can see that by using green's second identity, we see that 
        $$ 0 = \int_{D} \nabla u \nabla u \, dV$$
        since $\Delta u = 0$ and $u = 0$ on the boundary.
        This implies that 
        \begin{align*}
            \int_D <u_x, u_y>  \cdot <u_x, u_y> \, dV &= 0\\
            \int_D |\nabla u|^2 \, dV &= 0
            u_x^2 + u_y^2 &= 0
            u_x = u_y = 0
        \end{align*}
        And to solve the BC at infinity, $u = 0$\\
        Therefore, the solution is unique.\\
        This emphasizes the importance of end behavior conditions for an unbounded domain. The uniqueness of the solution is guaranteed when the boundary conditions are given at infinity.
    \end{solution}

    \question Section 7.4 Problem 15
    \begin{parts}
        \part Show that if \( v(x,y) \) is harmonic, so is \( u(x,y) = v(x^2 - y^2, 2xy) \).
        \part Show that the transformation \( (x,y) \rightarrow (x^2 - y^2, 2xy) \) maps the first quadrant onto the half-plane \( \{y > 0\} \). (Hint: Use either polar coordinates or complex variables.)
    \end{parts}
    \begin{solution}
        Notice that $z = x + iy =  r e^{i\theta}$. Then $z^2 = x^2 - y^2 + i(2xy)$\\
        Suppose $v(x,y)$ then it satisfies $\frac{\partial v^2}{\partial x^2} + \frac{\partial v^2}{\partial y^2} = 0$.\\
        Make the change of variavles of $j(x,y) = x^2 - y^2$ and $k(x,y) = 2xy$. Then we have
        \begin{align*}
            \frac{\partial v}{\partial x} &= 2x \frac{\partial v}{\partial j} + 2y \frac{\partial v}{\partial k}\\
            \frac{\partial v^2}{\partial x^2} &= 2 \frac{\partial v}{\partial j} + 4x^2 \frac{\partial^2 v}{\partial j^2} + 4xy \frac{\partial^2 v}{\partial j \partial k} + 4yx \frac{\partial^2 v}{\partial k \partial j} + 4y^2 \frac{\partial^2 v}{\partial k^2}\\  
            \frac{\partial v}{\partial y} &= -2y \frac{\partial v}{\partial j} + 2x \frac{\partial v}{\partial k}\\
            \frac{\partial v^2}{\partial y^2} &= 4y^2 \frac{\partial^2 v}{\partial j^2} - 4yx \frac{\partial^2 v}{\partial k \partial j} - 4xy \frac{\partial^2 v}{\partial j \partial k} + 4x^2 \frac{\partial^2 v}{\partial k^2}\\
        \end{align*}
        Adding the two equations we get
        \begin{align*}
            \frac{\partial v^2}{\partial x^2} + \frac{\partial v^2}{\partial y^2} &= 4x^2 \frac{\partial^2 v}{\partial j^2} + 4y^2 \frac{\partial^2 v}{\partial k^2} = 0\\
            \frac{\partial^2 v}{\partial j^2} + \frac{\partial^2 v}{\partial k^2} &= 0
        \end{align*}
        Therefore, $u(x,y) = v(x^2 - y^2, 2xy)$ is harmonic.\\

        We can also see that for the quarter plane $x > 0, y > 0$, with the change of variables we map to $-\infty < x < \infty, y > 0$.
        


    \end{solution}
\end{questions}

\end{document}
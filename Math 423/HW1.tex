\documentclass{article}
\usepackage{amsmath}
\usepackage{amsfonts}
\usepackage{amssymb}
\usepackage{mathrsfs}
\usepackage{dsfont}
\usepackage{cancel}

\usepackage{graphicx}



\setlength\parindent{0pt}

\author{Pranav Tikkawar}
\title{HW1: Math 423}

\begin{document}
\maketitle

\section*{1.1 Problem 11}
Verify that $u(x,y) = f(x)g(y)$ is a solution of the PDE $u u_{xy} = u_x u_y$ for all pairs of functions $f$ and $g$ of one variable.\\
\textbf{Solution:}\\
\begin{align*}
    u_{x} &= f'(x)g(y)\\
    u_{y} &= f(x)g'(y)\\
    u_{xy} &= f'(x)g'(y)\\
    u_{x}u_{y} &= f'(x)g(y)f(x)g'(y)\\
    u u_{xy} &= f(x)g(y)f'(x)g'(y)\\
    u u_{xy} &= u_{x}u_{y}
\end{align*}
Thus clearly $u(x,y) = f(x)g(y)$ is a solution of the PDE $u u_{xy} = u_x u_y$ for all pairs of functions $f$ and $g$ of one variable.

\section*{1.1 Problem 12}
Verify by direct substitution that the function $u(x,y) = sin(nx)sinh(ny)$ is a solution of the PDE $u_{xx} + u_{yy} = 0$ for all positive integers $n$.\\
\textbf{Solution:}\\
\begin{align*}
    u_{x} &= ncos(nx)sinh(ny)\\
    u_{y} &= nsin(nx)cosh(ny)\\
    u_{xx} &= -n^2sin(nx)sinh(ny)\\
    u_{yy} &= n^2sin(nx)sinh(ny)\\
    u_{xx} + u_{yy} &= -n^2sin(nx)sinh(ny) + n^2sin(nx)sinh(ny)\\
    u_{xx} + u_{yy} &= 0
\end{align*}
Thus clearly $u(x,y) = sin(nx)sinh(ny)$ is a solution of the PDE $u_{xx} + u_{yy} = 0$ for all positive integers $n$.

\section*{1.2 Problem 2}
Solve the equation $3u_y + u_{xy} = 0$\\
\textbf{Solution:}\\
Let $v = u_y$\\
Then $v_x = u_{xy}$\\
Thus the equation becomes $3v + v_x = 0$\\
This is a first order linear ODE.\\
The integrating factor is $e^{3x}$\\
Multiplying both sides by the integrating factor we get\\
\begin{align*}
    3e^{3x}v + e^{3x}v_x &= 0\\
    \frac{d}{dx}(e^{3x}v) &= 0\\
    e^{3x}v &= C\\
    v &= C_1e^{-3x}
\end{align*}
Where $C$ is a constant $\in \mathds{R}$.\\
Now we have $v = u_y$\\
Integrating $v$ with respect to $y$ we get\\
\begin{align*}
    u &= \int v dy\\
    u &= \int Ce^{-3x} dy\\
    u &= C_1e^{-3x}y + C_2(x)
\end{align*}
Where $C_1$ is a constant s $\in \mathds{R}$ and $C_2$ is function of $x$.\\


\section*{1.2 Problem 6}
Solve the equation $\sqrt{1 - x^2}u_x + u_y = 0$ with the condition that $u(0,y) = y$\\
\textbf{Solution:}\\
Noticing that the directional derivative in the direction of $(\sqrt{1 - x^2}, 1)$ is 0. The function is constant along these curves. We can find the characteristic curves by solving the ODE $\frac{dy}{dx} = \frac{1}{\sqrt{1 - x^2}}$\\

$$\frac{dy}{dx} = \frac{1}{\sqrt{1-x^2}}$$
$$y = arcsin(x) + C$$
Now we have $u(x, y(x)) = u(x, arcsin(x) + C)$\\
Let $x=0$ as an arbitrary value so we have $u(0, y(0)) = u(0, arcsin(0) + C) = u(0, C) = f(C)$\\
Thus $C = y - arcsin(x)$\\ 
Thus $u(x, y) = f(y - arcsin(x))$\\
Now we can apply the initial condition $u(0, y) = y$\\
$u(0, y) = f(y - arcsin(0)) = f(y) = y$\\
Thus $f(y) = y$\\
Thus $u(x, y) = y - arcsin(x)$\\

\section*{1.2 Problem 10}
Solve $u_x + u_y + u = e^{x+2y}$ with $u(x,0) = 0$ \\
\textbf{Solution:}\\
$$x' = x+y, y' = x - y$$
$$x = \frac{x' + y'}{2}, y = \frac{x' - y'}{2}$$
$$u_x = u_{x'} + u_{y'}$$
$$u_y = u_{x'} - u_{y'}$$
$$2u_{x'} + u(x',y') = e^{\frac{3x' -y'}{2}}$$
This is now a first order linear ODE.\\
The integrating factor is $e^{2x'}$\\
Multiplying both sides by the integrating factor we get
$$2e^{2x'}u_{x'} + e^{2x'}u = e^{\frac{7x' -y'}{2}}$$
$$\frac{d}{dx'}(e^{2x'}u) = e^{\frac{7x' -y'}{2}}$$
$$e^{2x'}u = \frac{2}{7}e^{\frac{7x' -y'}{2}} + C(y)$$
$$u = \frac{2}{7}e^{-\frac{3x' -y'}{2}} + C(y')e^{-2x'}$$
Now we have $u(x', y')$ Now we need to convert back into $u(x,y)$\\
$$u(x, y) = \frac{2}{7}e^{x+2y} + C(x-y)e^{-2x-2y}$$
\section*{1.3 Problem 6}
Consider heat flow in a long circular cylinder where the tempurature only depends on t and r. From the 3d heat equation derivae the equation $u_t = k(u_{rr} + \frac{u_r}{r})$\\
\textbf{Solution:}\\
The heat equation in 3D is given by\\
$$c \rho \frac{\partial u}{\partial t} = \nabla \cdot (k \nabla u)$$
In cylindrical coordinates we can derive the Laplacian by rewretiting the divergence in cylindrical coordinates.\\
Consider that for cylindrical coordinates we have\\
\begin{align*}
    r = \sqrt{x^2 + y^2}\\
    \theta = tan^{-1}(\frac{y}{x})\\
\end{align*} 
Since we know the Laplacian is in the form of:\\
$$\nabla^2 = \frac{\partial^2}{\partial x^2} + \frac{\partial^2}{\partial y^2}  + \frac{\partial^2}{\partial z^2}$$
We can write each of these terms in cylindrical coordinates.\\
Thus we have\\
$$\frac{\partial^2}{\partial x^2} = \frac{\partial^2 r}{\partial x^2}\frac{\partial}{\partial r} + (\frac{\partial r}{\partial x})^2\frac{\partial^2}{\partial r^2} + \frac{\partial^2 \theta}{\partial x^2}\frac{\partial}{\partial \theta} + (\frac{\partial \theta}{\partial x})^2\frac{\partial^2}{\partial \theta^2} $$
$$ \frac{\partial^2}{\partial y^2} = \frac{\partial^2 r}{\partial y^2}\frac{\partial}{\partial r} + (\frac{\partial r}{\partial y})^2\frac{\partial^2}{\partial r^2} + \frac{\partial^2 \theta}{\partial y^2}\frac{\partial}{\partial \theta} + (\frac{\partial \theta}{\partial y})^2\frac{\partial^2}{\partial \theta^2} $$
Thus writing out all the partial derivatives we have\\
\begin{align*}
    \frac{\partial r}{\partial x} &= \frac{x}{r}\\
    \frac{\partial r}{\partial y} &= \frac{y}{r}\\
    \frac{\partial \theta}{\partial x} &= -\frac{y}{r^2}\\
    \frac{\partial \theta}{\partial y} &= \frac{x}{r^2}\\
    \frac{\partial^2 r}{\partial x^2} &= \frac{y^2}{r^3}\\
    \frac{\partial^2 r}{\partial y^2} &= \frac{x^2}{r^3} \\
    \frac{\partial^2 \theta}{\partial x^2} &= \frac{2xy}{r^4}\\
    \frac{\partial^2 \theta}{\partial y^2} &= -\frac{2xy}{r^4}
\end{align*}
Notice that $\frac{\partial^2}{\partial z^2} = \frac{\partial^2}{\partial z^2}$\\
Also not that many terms cancel out like the $\frac{\partial^2 \theta}{\partial x^2}$ and $\frac{\partial^2 \theta}{\partial y^2}$ terms.\\ As well as many other terms sum to nice things\
Thus we can write the Laplacian in cylindrical coordinates as\\
$$\nabla^2 = \frac{\partial^2}{\partial r^2} + \frac{1}{r}\frac{\partial}{\partial r} + \frac{1}{r^2}\frac{\partial^2}{\partial \theta^2} + \frac{\partial^2}{\partial z^2}$$
Thus we can note that since we are only interested in the radial and time dependence we can ignore the angular and z dependence.\\
Thus it follows that we have\\
$$\nabla^2 = \frac{\partial^2}{\partial r^2} + \frac{1}{r}\frac{\partial}{\partial r}$$
Thus we can write the heat equation as\\
$$u_t = k u_{rr} + \frac{u_r}{r}$$
with $k = \frac{k}{c \rho}$\\

\section*{1.3 Problem 10}
If $f(x)$ is continus and $|f(x)| \leq \frac{1}{|x|^3 + 1}$ for all $x$ show that $$\int_{D} \nabla \cdot f dx = 0$$
\textbf{Solution:}\\
Let $D$ be a larger ball with boundary $\partial D$ We can use the divergence theorem to write\\
$$|\int_{D} \nabla \cdot f dx| = |\int_{\partial D} f \cdot n d(\partial D)|$$
By cauchy schwartz we have\\
that $|f \cdot n| \leq |f| |n|$\\
Since $|n| = 1$ we have\\
$$ |\int_{\partial D} f \cdot n d(\partial D)| \leq \int_{\partial D} |f| d(\partial D)$$
It hs been given that $|f(x)| \leq \frac{1}{|x|^3 + 1}$.\\
since we have taken $D$ to be a ball of radius $R$ we can do a change of variable to polar, spherical coordinates with $d(\partial D)  = R^2sin(\theta) d\theta d\phi$, with integrating from $\theta \in (0,2\pi), \phi \in (0, \pi)$\\
We can also note that $|x| = R$ on the boundary of the ball.\\
Thus we have\\
$$\int_{\partial D} |f| \cdot n d(\partial D)| \leq \int_{\partial D} \frac{1}{R^3 + 1} d(\partial D) = \int_{0}^{\pi}\int_{0}^{2\pi} \frac{R^2}{R^3+1} sin(\theta)d\theta d\phi$$
This integral evaluates to\\
$$\frac{R^2}{R^3 + 1} (4 \pi)$$
We have now shown that\\
$$|\int_{D} \nabla \cdot f dx| \leq \frac{4 \pi R^2}{R^3 + 1}$$
Now we can take the limit as $R \to \infty$\\
$$\lim_{R \to \infty} \frac{4 \pi R^2}{R^3 + 1} = 0$$
Note that it goes to 0 similar to $\frac{1}{R}$\\
Therefore we have shown that\\
$$\int_{D} \nabla \cdot f dx = 0$$
Where $D$ is all of space

\end{document}
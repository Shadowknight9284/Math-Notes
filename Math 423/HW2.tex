\documentclass{article}
\usepackage{amsmath}
\usepackage{amsfonts}
\usepackage{amssymb}
\usepackage{mathrsfs}
\usepackage{dsfont}
\usepackage{cancel}

\usepackage{graphicx}


\setlength\parindent{0pt}

\author{Pranav Tikkawar}
\title{TODO}

\begin{document}
\maketitle

\section*{1.4 Problem 4 ***}
A rod occupying the interval $0 \leq x \leq l$ is subject to the heat source
$f(x) = 0$ for $0 < x < l/2$ , and $f(x) = H$ for $l/2 < x < l$ where $H > 0$. The rod has physical constants $c = \rho = k = 1$, and its ends are kept at zero temperature.
\subsection*{a} 
Find the steady-state temperature of the rod.\\
\textbf{Solution:} \\
$u(0, t) = u(l, t) = 0, \lim_{t \rightarrow \infty} u_t = 0$ and $u(l/2, t)$ is equal on both sides with $u(l/2,t)_x$ are also equal on both sides\\
The steady-state temperature $u(x)$ satisfies the equation
$$ u_{xx} + f(x) = 0 $$
We can then integrate this to get:
\begin{align*}
u_{xx} &= -f(x) \\
u_{x} &= -\int f(x) dx \\
u_{x} &= \begin{cases}
0 + C_1(t) & \text{for } 0 < x < l/2 \\
-Hx + C_2(t) & \text{for } l/2 < x < l
\end{cases}\\
u &= -\begin{cases}
C_1(t)x + C_3(t) & \text{for } 0 < x < l/2 \\
-\frac{H}{2}x^2 + C_2(t)x + C_4(t) & \text{for } l/2 < x < l
\end{cases}
\end{align*}

Now we can apply the boundary conditions.\\

At $x = 0$:
$$u(0,t) = C_3(t) = 0$$
At $x = l$:
$$u(l,t) = -\frac{H}{2}l^2 + C_2(t)l + C_4(t) = 0$$
At $x = l/2$:
$$u(l/2,t) = -\frac{H}{2}\left(\frac{l}{2}\right)^2 + C_2(t)\left(\frac{l}{2}\right) + C_4(t) = C_1(t)\frac{l}{2}$$
And the partial derivative of x at $x = l/2$:
$$u_x(l/2,t) = -H\left(\frac{l}{2}\right) + C_2(t) = C_1(t)$$


\subsection*{b} 
Which point is the hottest, and what is the temperature there?





\section*{1.4 Problem 6 ***}
Two homogeneous rods have the same cross section, specific heat c, and
density $\rho$ but different heat conductivities $\kappa_1$ and $\kappa_2$ and lengths $L_1$ and $L_2$. Let $k_j = \kappa_j/c\rho$ be their diffusion constants. They are welded together
so that the temperature $u$ and the heat flux $\kappa u_x$ at the weld are continuous.
The left-hand rod has its left end maintained at temperature zero. The
right-hand rod has its right end maintained at temperature T degrees.\\
\subsection*{a} Find the equilibrium temperature distribution in the composite rod.

\subsection*{b} Sketch it as a function of x in case $k_1 = 2, k_2 = 1, L_1 = 3, L_2 = 2, \text{ and } T = 10$. (This exercise requires a lot of elementary algebra, but
it's worth it.)


\section*{1.5 Problem 2 ***}
Consider the problem 
$$ u''(x) + u'(x) = f(x)$$
$$ u'(0) = u(0) = \frac{1}{2}[u'(l) + u(l)]$$
With $f(x)$ a given function.\\
\subsection*{a} Is the solution unique? Explain.\\
We know from our Ordinary Differential Equations class that the solution to a second order linear inhomogeneous differential equation is unique.\\
We can solve the homogeneous equation to get the general solution and then add the particular solution to get the unique solution.\\
The boundary conditions are also unique as there are 2 conditions, when 2 are needed, so the solution is unique.\\
Solving the homogeneous equation we get:
$$ u_h(x) = C_1e^{-x} + C_2$$





\subsection*{b} Does a solution necessarily exist or is there a constion that $f(x)$ must satisfy for a solution to exist?\\

\section*{1.5 Problem 6 ***}
Solve the equation 
$$ u_x + 2xy^2 u_y = 0$$
\textbf{Solution: }\\
We solve this by noticing the directional derivative in the direction of the vector $(1, 2xy^2)$ is 0. This means that the solution is constant along the lines $y^2 = x^2 + C$ for some constant $C$.\\
Thus we can also say that 
$$ \frac{dy}{dx} = \frac{2xy^2}{1} = 2xy^2$$
This is a separable differential equation and we can solve it by separating the variables and integrating.\\
\begin{align*}
    \frac{dy}{dx} &= 2xy^2\\
    \int y^{-2} dy &= \int 2x dx\\
    -y^{-1} &= x^2 + C\\
\end{align*}


\section*{1.6 Problem 4 ***}
What is the type of the equation:
$$ u_{xx} - 4u_{xy} + 4u_{yy} = 0$$
Show that by direct substition that $u(x,y) = f(y+2x) + xg(y+2x)$ is a solution for arbitrary functions $f$ and $g$. 

\section*{1.6 Problem 6 ***}
Consider the equation
$$ 3u_y + u_{xy} = 0$$
\subsection*{a}
What is its type?\\
Elliptic

\subsection*{b}
Find the general solution. Hint ($v = u_y$)\\

\subsection*{c}
With auxiliary conditions $u(x,0) = e^{-3x}$ and $u_y(x,0) = 0$, does a solution exist? Is it unique?\\


\end{document}
\documentclass[answers,12pt,addpoints]{exam}
\usepackage{import}

\import{C:/Users/prana/OneDrive/Desktop/MathNotes}{style.tex}

% Header
\newcommand{\name}{Pranav Tikkawar}
\newcommand{\course}{01:640:423}
\newcommand{\assignment}{Homework n}
\author{\name}
\title{\course \ - \assignment}

\begin{document}
\maketitle


\newpage
\begin{questions}
    \question 4.2 2
    Consider the equation \( u_{tt} = c^2 u_{xx} \) for \( 0 < x < l \), with the boundary conditions \( u_x(0,t) = 0 \), \( u(l, t) = 0 \) (Neumann at the left, Dirichlet at the right).
    \begin{parts}
        \item Show that the eigenfunctions are \( \cos[(n+\frac{1}{2})\pi x/l] \) 
        \item Write the series expansion of the solution \( u(x,t) \)
    \end{parts}
    \begin{solution}
        \textbf{Part a} \\
        Since the wave equation and BC are linear homogenous, we can use seperation of variables. Let \( u(x,t) = X(x)T(t) \). Then, we have
        \[ \frac{T''}{c^2T} = \frac{X''}{X} = \alpha \]
        Then, we have
        \begin{align*}
            T'' - c^2\alpha T &= 0 \\
            X'' - \alpha X &= 0
        \end{align*}
        The boundary conditions are \( X'(0) = 0 \) and \( X(l) = 0 \). \\
        We can then consider the ODE of \( X \). Clearly we must have $\alpha < 0 $ for non trivial solutions. Thus with $\alpha = -\lambda^2$, we have
        \begin{align*}
            X'' - \alpha X &= 0 \\
            X'(0) &= 0 \\
            X(l) &= 0
        \end{align*}
        \[ X(x) = c_1 \cos(\lambda x) + c_2 \sin(\lambda x) \] 
        With the boundary conditions, we have
        \begin{align*}
            X'(0) &= -c_1\lambda\sin(0) + c_2\lambda\cos(0) = 0 \\
            X(l) &= c_1\cos(\lambda l) + c_2\sin(\lambda l) = 0
        \end{align*}
        Clealry $c_2 = 0$ by the first BC and $cos(\lambda l) = 0$ by the second BC. Thus, we have
        \[ \lambda = \frac{(2x+1)\pi}{2l} \]
        Thus, the eigenfunction are \( \cos[(n+\frac{1}{2})\pi x/l] \)\\
        \textbf{Part b} \\
        Now we need to consider the ODE for \( T \). We have
        \[ T'' + c^2\lambda^2T = 0 \]
        The general solution is
        \[ T(t) = c_3\cos(c\lambda t) + c_4\sin(c\lambda t) \]
        Thus, the general solution is
        \begin{align*}
            u(x,t) &= \sum_{n=0}^{\infty}X(x)T(t)\\
            &= \sum_{n=0}^{\infty} \left[ c_3\cos(c\lambda t) + c_4\sin(c\lambda t) \right] \cos\left[ \lambda x \right]\\
            &= \sum_{n=0}^{\infty} \left[ A_n \cos\left( \frac{(2n+1)\pi c t}{2l} \right) + B_n \sin\left( \frac{(2n+1)\pi c t}{2l} \right) \right] \cos\left[ \frac{(2n+1)\pi x}{2l} \right]
        \end{align*}
    \end{solution}
    \question 4.3 2
    Consider the eigenvalue problem with Robin BC at both ends:
    \begin{align*}
        -X'' &= \lambda X \\
        X'(0) - a_0X(0) = 0, &\quad X'(l) + a_1X(l) = 0
    \end{align*}
    \begin{parts}
        \item Show that $\lambda = 0$ iff $a_0 + a_1 = -a_0a_1l$
        \item Find the eigenfunctions corresponding to the zero eigenvalue
    \end{parts}
    \begin{solution}
        \textbf{Part a} \\
        If $\lambda = 0$, then we have
        \begin{align*}
            -X'' &= 0 \\
            X'(0) - a_0X(0) &= 0 \\
            X'(l) + a_1X(l) &= 0
        \end{align*}
        The general solution is
        \[ X(x) = c_1X + c_2 \]
        Then, we have
        \begin{align*}
            X'(0) - a_0X(0) &= c_1 - a_0c_2 = 0 \\
            X'(l) + a_1X(l) &= c_1 + a_1(c_1 l + c_2) = 0
        \end{align*}
        Solving for $c_1$ , we have
        \begin{align*}
            c_1 &= a_0c_2 \\
        \end{align*}
        Substituting back in, we have
        \begin{align*}
            a_0c_2 + a_1(a_0c_2 l + c_2) &= 0 \\
            c_2(a_0 + a_0a_1l + a_1) &= 0\\
            a_0 + a_1 &= -a_0a_1l 
        \end{align*}
        \textbf{Part b} \\
        If $\lambda = 0$, then we have
        \begin{align*}
            X(x) &= c_1x + c_2\\
            &= a_0x +1
        \end{align*}

    \end{solution}

    \question 4.3 15
    Find the equation for the eigenvalues $\lambda$ of the problem
    \[ (\kappa(x) X' )' + \lambda \rho X = 0 \quad \text{for } 0 < x < l \text{ with } X(0) = X(l) = 0 \]
    where $\kappa(x) = \kappa_1^2$ for $x<a$ and $\kappa(x) = \kappa_2^2$ for $x>a$. and $\rho(x) = \rho_1^2$ for $x<a$ and $\rho(x) = \rho_2^2$ for $x>a$. where all these constants are positive and $0 < a < l$.
    \begin{solution}
        We have 
        $$\begin{cases}
            \kappa_1^2 X'' +\lambda \rho_1^2 X = 0 \quad x < a \\
            \kappa_2^2 X'' +\lambda \rho_2^2 X = 0 \quad x > a
        \end{cases}$$
        $$\begin{cases}
            X'' = -\frac{\lambda \rho_1^2}{\kappa_1^2} X  \quad x < a \\
            X'' = -\frac{\lambda \rho_2^2}{\kappa_2^2} X \quad x > a
        \end{cases}$$
        We can now consider the cases of the sign of $\lambda$.
        \textbf{Case 1: $\lambda = \mu^2$} \\
        Then we have
        $$\begin{cases}
            X = C_1 cos(\mu\frac{\rho_1}{\kappa_1}x) + C_2 sin(\mu\frac{\rho_1}{\kappa_1}x) \quad x < a \\
            X = C_3 cos(\mu\frac{\rho_2}{\kappa_2}x) + C_4 sin(\mu\frac{\rho_2}{\kappa_2}x) \quad x > a
        \end{cases}$$
        Now applying the BC: (for sake of ease consider $\alpha_1 = \mu\frac{\rho_1}{\kappa_1}$ and $\alpha_2 = \mu\frac{\rho_2}{\kappa_2}$)
        \begin{align*}
            X(0) &= 0 = C_1 \\
            X(l) &= 0 = C_3 \cos(\alpha_2 l) + C_4 \sin(\alpha_2 l) \implies C_3 = -C_4 \frac{\sin(\alpha_2 l)}{\cos(\alpha_2 l)} = -C_4 \tan(\alpha_2 l)
        \end{align*}
        Now our solution is
        $$\begin{cases}
            X = C_2 \sin(\alpha_1 x) \quad x < a \\
            X = C_4 (\sin(\alpha_2 x) - \tan(\alpha_2 )\cos(\alpha_2 x)) \quad x > a
        \end{cases}$$
        for a new $C_5 = C_4 \cos(\alpha_2 l)$
        $$ \begin{cases}
            X = C_2 \sin(\alpha_1 x) \quad x < a \\
            X = C_5 \sin(\alpha_2 (x-l))  \quad x > a
        \end{cases}$$
        To solve for the last 2 constants we need to satify the following equations:
        \begin{align*}
            X(a^-) &= X(a^+) \implies C_2 \sin(\alpha_1 a) = C_5 \sin(\alpha_2 (a-l)) \\
            X'(a^-) &= X'(a^+) \implies C_2 \alpha_1 \cos(\alpha_1 a) = C_5 \alpha_2 \cos(\alpha_2 (a-l))
        \end{align*}
        We can see that $$ \frac{\kappa_1}{\rho_1} \tan(\alpha_1 a) + \frac{\kappa_2}{\rho_2} \tan(\alpha_2 (l-a)) = 0$$ 
        The zeros of this equation are the eigenvalues.
        \textbf{Case 2: $\lambda = 0$} \\
        Then we have $$\begin{cases}
            X'' = 0 \quad x < a \\
            X'' = 0 \quad x > a
        \end{cases}$$
        The general solution is
        $$\begin{cases}
            X = C_1 x + C_2 \quad x < a \\
            X = C_3 x + C_4 \quad x > a
        \end{cases}$$
        Applying the BC, we have
        \begin{align*}
            X(0) &= 0 = C_2 \\
            X(l) &= 0 = C_3 l + C_4 \implies C_4 = -C_3 l
        \end{align*}
        And applying the continourity condition, we have 
        \begin{align*}
            X(a^-) &= X(a^+) \implies C_1 a &= C_4 (a-l)\\
            X'(a^-) &= X'(a^+) \implies C_1 &= C_4
        \end{align*}
        Thus, we have $C_1 = 0$ and $C_4 = 0$. Thus, the eigenfunctions are trivial and $\lambda = 0$ is not an eigenvalue.
        \textbf{Case 3: $\lambda = -\mu^2$} \\
        Then we have
        $$\begin{cases}
            X'' = \frac{\mu^2 \rho_1^2}{\kappa_1^2} X \quad x < a \\
            X'' = \frac{\mu^2 \rho_2^2}{\kappa_2^2} X \quad x > a
        \end{cases}$$
        The general solution is
        $$\begin{cases}
            X = C_1 cosh(\mu\frac{\rho_1}{\kappa_1}x) + C_2 sinh(\mu\frac{\rho_1}{\kappa_1}x) \quad x < a \\
            X = C_3 cosh(\mu\frac{\rho_2}{\kappa_2}x) + C_4 sinh(\mu\frac{\rho_2}{\kappa_2}x) \quad x > a
        \end{cases}$$
        Applying the bc we have
        \begin{align*}
            X(0) &= 0 = C_1 \\
            X(l) &= 0 = C_3 \cosh(\mu\frac{\rho_2}{\kappa_2}l) + C_4 \sinh(\mu\frac{\rho_2}{\kappa_2}l) \implies C_3 = -C_4 \frac{\sinh(\mu\frac{\rho_2}{\kappa_2}l)}{\cosh(\mu\frac{\rho_2}{\kappa_2}l)} = -C_4 \tanh(\mu\frac{\rho_2}{\kappa_2}l)
        \end{align*}
        following similar steps as case 1, we have
        $$\begin{cases}
            C_2 \sinh(\mu\frac{\rho_1}{\kappa_1}x) \quad x < a \\
            C_5 \sinh(\mu\frac{\rho_2}{\kappa_2}(x-l)) \quad x > a
        \end{cases}$$
        Thus we get the equation
        $$ \frac{\kappa_1}{\rho_1} \tanh(\mu\frac{\rho_1}{\kappa_1} a) + \frac{\kappa_2}{\rho_2} \tanh(\mu\frac{\rho_2}{\kappa_2} (l-a)) = 0$$
        Thus the zeros of the above equation are the eigenvalues but clearly there are none for this case.\\\\
        Thus the eigenvalues are $\lambda = \mu^2$ where $\mu$ is the solution to the equation
        $$ \frac{\kappa_1}{\rho_1} \tan(\alpha_1 a) + \frac{\kappa_2}{\rho_2} \tan(\alpha_2 (l-a)) = 0$$
    \end{solution}
    \question 5.1 2
    Let $\phi(x) \equiv x^2$ for $0 \leq x \leq 1 = l$. 
    \begin{parts}
        \item Calculate the Fourier sine series of $\phi(x)$
        \item Calculate the Fourier cosine series of $\phi(x)$
    \end{parts}
    \begin{solution}
        \textbf{Part a} \\
        We have
        \begin{align*}
            \phi(x) &= x^2 \\
            &= \sum_{n=1}^{\infty} B_n \sin(n\pi x)
        \end{align*}
        We can calculate the coefficients as
        \begin{align*}
            x^2 &= \sum_{n=1}^{\infty} B_n \sin(n\pi x) \\
            \int_{0}^{1} x^2 \sin(m \pi x) dx &= \int_{0}^{1} \sum_{n=1}^{\infty} B_n \sin(n\pi x) \sin(m\pi x) dx \\
            \int_{0}^{1} x^2 \sin(m \pi x) dx &= \sum_{n=1}^{\infty} B_n \int_{0}^{1} \sin(n\pi x) \sin(m\pi x) dx \\
            \int_{0}^{1} x^2 \sin(n \pi x) dx &= \frac{1}{2} B_n \\
            B_n &= 2 \int_{0}^{1} x^2 \sin(n\pi x) dx\\
            B_n &= 2 \frac{-2 + (-1)^n(2-n^2\pi^2)}{n^3\pi^3}
        \end{align*}
        Therefore the Fourier sine series of $\phi(x)$ is
        \[ \sum_{n=1}^{\infty} 2 \frac{-2 + (-1)^n(2-n^2\pi^2)}{n^3\pi^3} \sin(n\pi x) \]
        \textbf{Part b} \\
        We have
        \begin{align*}
            \phi(x) &= x^2 \\
            &= \frac{a_0}{2} \sum_{n=1}^{\infty} A_n \cos(n\pi x)
        \end{align*}
        We can first calculate $a_0$ as
        \begin{align*}
            a_0 &= 2 \int_{0}^{1} x^2 dx \\
            &= \frac{2}{3}
        \end{align*}
        Then we can calculate the coefficients as
        \begin{align*}
            x^2 &= \frac{1}{3} + \sum_{n=1}^{\infty} A_n \cos(n\pi x) \\
            \int_{0}^{1} x^2 \cos(m \pi x) dx &= \int_{0}^{1} \frac{1}{3}\cos(m \pi x) + \sum_{n=1}^{\infty} A_n \cos(n\pi x) \cos(m\pi x) dx \\
            \int_{0}^{1} x^2 \cos(m \pi x) dx &= \sum_{n=1}^{\infty} A_n \int_{0}^{1} \cos(n\pi x) \cos(m\pi x) dx \\
            \int_{0}^{1} x^2 \cos(n \pi x) dx &= \frac{1}{2} A_n \\
            A_n &= 2 \int_{0}^{1} x^2 \cos(n\pi x) dx\\
            A_n &= \frac{4(-1)^n}{n^2 \pi^2}
        \end{align*}
        Therefore the Fourier cosine series of $\phi(x)$ is
        \[ \frac{1}{3} + \sum_{n=1}^{\infty} \frac{4(-1)^n}{n^2 \pi^2} \cos(n\pi x) \]
    \end{solution}
    \question 5.1 9
    Solve $u_{tt} = c^2 u_{xx}$ for $0 < x < \pi$ with the BC $u_x(0,t) = u_x(\pi,t) = 0$ and the IC $u(x,0) = 0$ and $u_t(x,0) = \cos^2(x)$.
    \begin{solution}
        Since BC and PDE are linear homogenous, we can use seperation of variables. Let $u(x,t) = X(x)T(t)$. Then we have
        \[ \frac{T''}{c^2T} = \frac{X''}{X} = \alpha \]
        Then we have
        \begin{align*}
            T'' - c^2\alpha T &= 0 \\
            X'' - \alpha X &= 0
        \end{align*}
        The boundary conditions are $X'(0) = 0$ and $X(\pi) = 0$. \\
        We can then consider the ODE of $X$. Clearly we must have $\alpha =< 0 $ for non trivial solutions. Thus with $\alpha = -\lambda^2$, we have
        \begin{align*}
            X'' + \lambda^2 X &= 0 \\
            X'(0) &= 0 \\
            X(\pi) &= 0
        \end{align*}
        \[ X(x) = c_1 \cos(\lambda x) + c_2 \sin(\lambda x) \]
        With the boundary conditions, we have
        \begin{align*}
            X'(0) &= -c_1\lambda\sin(0) + c_2\lambda\cos(0) = 0 \implies c_2 = 0 \\
            X'(\pi) &= -c_1\lambda\sin(\lambda \pi) + c_2\lambda\cos(\lambda \pi) = 0 \implies \lambda = n
        \end{align*}
        Thus, the eigenfunction are $ \cos(n x) $. for the eigenvalues $\alpha = -\lambda^2 $\\
        Now we need to consider the ODE for $T$. We have
        \[ T'' + c^2\lambda^2T = 0 \]
        The general solution is
        \[ T(t) = c_1 \cos(c\lambda t) + c_2 \sin(c\lambda t) \]
        Additionally for $\alpha = 0$ we have
        \[ X(x) = c_1 x + c_2 \]
        Applying the BC, we have
        \begin{align*}
            X'(0) &= 0 = c_1 \\
            X'(\pi) &= 0 = c_1 
        \end{align*}
        Thus, the eigenfunctions are $X_0(x) = C_2 $ and for $T$ we have $T_0(t) = C_5 t + C_6$. 
        Thus the general solution is
        \[ u(x,t) = \sum_{n=1}^{\infty} \left[ A_n \cos(n x) \cos(c n t) + B_n \cos(n x) \sin(c n t) \right] + C t + D \] 
        Now we need to consider the IC. We have
        \begin{align*}
            u(x,0) &= 0 = \sum_{n=1}^{\infty} A_n \cos(n x) + D \\
            \implies A_n &= 0 \quad \text{and} \quad D = 0
        \end{align*}
        \begin{align*}
            u_t(x,0) &= \cos^2(x) = \sum_{n=1}^{\infty} c n B_n \cos(n x) + C
        \end{align*}
        Clealry $C = \frac{1}{2}$ and $B_n = \frac{1}{4c} $ for $n =2$ and $0$ otherwise. Thus the solution is
        \[ u(x,t) = \frac{t}{2} + \frac{1}{4c} \cos(2x) \sin(2ct) \]
    \end{solution}
    \question 6.1 5
    Solve $u_{xx} + u_{yy} = 1$ in $r < a$ with $u(x, y)$ vanishing on $r = a$
    \begin{solution}
        The PDE we want to solve is $\Delta u = 1$ \\
        We can convert to polar since we have $r < a$. We have
        $$ \Delta u = \frac{\partial^2 u}{\partial r^2} + \frac{1}{r} \frac{\partial u}{\partial r} + \frac{1}{r^2} \frac{\partial^2 u}{\partial \theta^2} = 1$$
        Since $u$ vanishes on $r = a$, we can consdider the solution are radially symetric. Thus we have
        $$ \frac{d^2 u}{d r^2} + \frac{1}{r} \frac{d u}{d r} = 1$$
        We can solve this ODE as
        $$ r\frac{d^2 u}{d r^2} + \frac{d u}{d r} = r$$
        $$ \frac{d}{d r}(r\frac{d u}{d r}) = r$$
        $$ u(r) = \frac{r^2}{4} + C_1ln(r) + C_2$$
        Applying the BC, we have
        $$ u(0) = 0 = C_1$$
        $$ u(a) = 0 = \frac{a^2}{4} + C_2$$
        Thus, the solution is
        $$ u(r) = \frac{r^2}{4} - \frac{a^2}{4}$$
        $$ u(x,y) = \frac{x^2 + y^2 - a^2}{4}$$
    \end{solution}
    \question 6.1 11
    Show that there is no solution of 
    $$ \Delta u = f \quad \text{in } D , \quad \frac{\partial u}{\partial n} = g \quad \text{on } \partial D$$
    in three dimensions, unless 
    $$ \int_{D} f dV = \int_{\partial D} g dS$$
    \begin{solution}
        We can use the divergence theorem to show this. We have
        $$ \int_{D} \Delta u dV = \int_{\partial D} \frac{\partial u}{\partial n} dS \implies \int_{D} f dV = \int_{\partial D} g dS$$
        Clealry the only way the above equation can hold is if the two integrals are equal. \\\\
        For 2D we can utilize Green's theorem to show this. We have
        $$ \int_{D} \Delta u dA = \int_{\partial D} \frac{\partial u}{\partial n} ds \implies \int_{D} f dA = \int_{\partial D} g ds$$
        Clealry the only way the above equation can hold is if the two integrals are equal.\\\\
        For 1D, we can utilize the fundamental theorem of calculus to show this. We have
        $$ \int_{D} \Delta u dx = \int_{\partial D} \frac{\partial u}{\partial n} dx \implies \int_{D} f dx = B - A$$
    \end{solution}
    \question 6.2 2
    Prove that the eigenfunctions $\{sin(my) sin(nz) \}$ are orthogonal on the square $0 < y < \pi$, $0 < z < \pi$ 
    \begin{solution}
        To show that the eigenfunctions are orthogonal, we need to show that
        $$ \int_{0}^{\pi} \int_{0}^{\pi} \sin(my) \sin(nz)  \sin(py) \sin(qz) dy dz = 0 \quad \text{for } m \neq p \text{ or } n \neq q$$
        We can first seperate the integrals as
        $$ \int_{0}^{\pi} \sin(my) \sin(py) dy \int_{0}^{\pi} \sin(nz) \sin(qz) dz$$
        Then we can use the trig identity
        $$ \sin(a) \sin(b) = \frac{1}{2} \left[ \cos(a-b) - \cos(a+b) \right]$$
        Thus we have
        $$ \int_{0}^{\pi} \sin(my) \sin(py) dy = \frac{1}{2} \int_{0}^{\pi} \cos((m-p)y) - \cos((m+p)y) dy$$
        $$ \int_{0}^{\pi} \sin(nz) \sin(qz) dz = \frac{1}{2} \int_{0}^{\pi} \cos((n-q)z) - \cos((n+q)z) dz$$
        Once we evaluate the integrals we can see that the integrals are zero for $m \neq p$ or $n \neq q$
        Therefore the eigenfunctions are orthogonal.
    \end{solution}
\end{questions}

\end{document}
\documentclass[answers,12pt,addpoints]{exam}
\usepackage{import}

\import{C:/Users/prana/OneDrive/Desktop/MathNotes}{style.tex}

% Header
\newcommand{\name}{Pranav Tikkawar}
\newcommand{\course}{01:XXX:XXX}
\newcommand{\assignment}{Homework n}
\author{\name}
\title{\course \ - \assignment}

\begin{document}
\maketitle

\tableofcontents

\newpage

\section{Concepts}
\subsection{Study Guide Concepts}
\begin{itemize}
    \item 2.3
    \item 2.4
    \item 2.5
    \item 4 - Boundary Value Problems
    \item 6
    \begin{itemize}
        \item Laplacian in polar
        \item Seperation of Variables
        \item Rectagles (6.2)
        \item Circles Wedges and Annuli (6.4)
        \item NO max principle, MVT, Poisson's formula
    \end{itemize}
    \item 5.1
    \begin{itemize}
        \item Fourier Series, full, sin and cos
        \item No convergence
    \end{itemize}
\end{itemize}
\subsubsection{2.3 - The Diffusion Equation}
\begin{definition}[Max Principle (weak)]
    If $u(x,t)$ is a solution to the Diffusion Equation in a rectangle $0 \leq x \leq L$, $0 \leq t \leq T$, then the maximum of $u(x,t)$ occurs on the boundary of the rectangle. In other words on $x = 0, x = L, t = 0$. \\
    The minimum is similar as we can show that $-u(x,t)$ satisfies the same equation.\\
    The natrual interpretaion of this is that if you have a rod with no internal heat sourse, the hottest or coldest spot can only occour at $t =0$ or on the edges. 
\end{definition}
\begin{definition}[Uniqueness]
    There is uniqueness for the Dirichlet problem for the Diffusion Equation.
    That means there is at most one solution of 
    $$ \begin{cases}
        u_t - ku_{xx} = f(x,t) \text{ for } 0 < x < L, t > 0\\
        u(x,0) = \phi(x) \\
        u(0,t) = g(t) \\
        u(L,t) = h(t)
    \end{cases}$$
    For any given $f(x,t), \phi(x), g(t), h(t)$\\
    We can do proof by max principle.
    \begin{proof}
        We want to show that for all $u_1, u_2$ that satisfy the above conditions, $u_1 = u_2$.\\
        Let $w = u_1 - u_2$. Then $w$ satisfies the following:
        $$ \begin{cases}
            w_t - kw_{xx} = 0 \text{ for } 0 < x < L, t > 0\\
            w(x,0) = 0 \\
            w(0,t) = 0 \\
            w(L,t) = 0
        \end{cases}$$
        By max prinicple $w(x,t)$ has a maximum on its boundary. Also it must have a minimum on its boundary. Since $w(x,0) = 0$, the minimum and the maximum must be 0. Thus $w(x,t) = 0$ for all $x,t$.\\
        Thus $u_1 = u_2$.
    \end{proof}
    Now we can do a proof by energy.
    \begin{proof}
        We know that $w = u_1 - u_2$\\
        \begin{align}
            0 &= 0 \cdot w\\
            &= (w_t - kw_{xx})w\\
            &= (1/2w^2)_t + (-kw w_x)_x + kw_x^2
        \end{align}
        We can now integrat about $0 < x < L$ \\
        \begin{align}
            0 &= \int_0^L (1/2w^2)_t dx {- kw_x w |_0^L}_{goes to 0} + k \int_0^L w_x^2 dx\\
            \frac{d}{dt} \int_0^L 1/2w^2 dx &= -k  \int_0^L w_x^2 dx\\
        \end{align}
        Clealrly the derivative of $\int_0^L w^2 dx$ is decreasing 
        $$ \int_0^L w^2 dx \leq \int_0^L w(x,0)^2 dx $$
        The RHS is 0, so the LHS is 0. Thus $w = 0$. 
    \end{proof}
\end{definition}
\begin{definition}[Stablitity]
    The solution to the Diffusion Equation is stable. That means that if you have a small perturbation in the initial conditions, the solution will not change much.\\
    In other words they functions are "bounded" by initial conditions.\\
    This is in a $L_2$ sense.
    $$ \int_0^l [u_1(x,t) - u_2(x,t)]^2 dx \leq \int_0^l [u_1(x,0) - u_2(x,0)]^2 dx$$
\end{definition}
\subsubsection{2.4 - Diffusion on the Whole Line}
\begin{definition}[Invariance Properties]
    We have 5 basic invariance properties of the Diffusion Equation.
    \begin{itemize}
        \item Translation $u(x - y, t)$ is a solution if $u(x,t)$ is a solution.
        \item Any derivative of $u(x,t)$ is a solution. 
        \item A linear combination of solutions is a solution.
        \item An integral of a solution is a solution. Thus if $S(x,t)$ is a solution then so is $S(x - y, t)$ and so is $v(x,t) = \int_{-\infty}^x S(x - y, t) g(y) dy$. for any $g(y)$.
        \item Dilation. If $u(x,t)$ is a solution then so is $u(\sqrt{a}x, at)$ for any $a > 0$.
    \end{itemize}
\end{definition}
\begin{definition}[Fundamental Solution to the Diffusion Equation]
    The fundamental solution to the Diffusion Equation is 
    $$ S(x,t) = \frac{1}{\sqrt{4\pi kt}} e^{-x^2/4kt}$$
    This is a solution to the Diffusion Equation with $f(x,t) = 0$ and $u(x,0) = \delta(x)$.
    We can derive this by utilizing the invariance properties.
\end{definition}
\subsubsection{2.5 - Comparison of Waves and Diffusion}
\begin{table}[h!]
\centering
\begin{tabular}{|c|c|c|}
\hline
\textbf{Property} & \textbf{Waves} & \textbf{Diffusion} \\ \hline
Speed of Propogation & $c$ & Infinite \\ \hline
Singulatities for $t > 0$ & Transported along characteristics with speed c & Lost immediately \\ \hline
Well posed for $t > 0$ & Yes & Yes for bounded \\ \hline
Well posed for $t < 0$ & Yes & No \\ \hline
Max Principle & No & Yes \\ \hline
Behavior at infinity & Energy is constant so it doesn't decay & Decays to zero \\ \hline
Information & Transported & Lost gradually \\ \hline
\end{tabular}
\caption{Comparison of Waves and Diffusion}
\label{tab:comparison}
\end{table}
\subsubsection{4.1 - Seperation of Variables}



\newpage

\section{Problems}
\begin{questions}
    \question Question 1.
\end{questions}






\end{document}
\documentclass[answers,12pt,addpoints]{exam}
\usepackage{import}

\import{C:/Users/prana/OneDrive/Desktop/MathNotes}{style.tex}

% Header
\newcommand{\name}{Pranav Tikkawar}
\newcommand{\course}{01:XXX:XXX}
\newcommand{\assignment}{Homework n}
\author{\name}
\title{\course \ - \assignment}

\begin{document}
\maketitle

\tableofcontents

\newpage

\section{Concepts}
\subsection{Study Guide Concepts}
\begin{itemize}
    \item 2.3
    \item 2.4
    \item 2.5
    \item 4 - Boundary Value Problems
    \item 6
    \begin{itemize}
        \item Laplacian in polar
        \item Seperation of Variables
        \item Rectagles (6.2)
        \item Circles Wedges and Annuli (6.4)
        \item NO max principle, MVT, Poisson's formula
    \end{itemize}
    \item 5.1
    \begin{itemize}
        \item Fourier Series, full, sin and cos
        \item No convergence
    \end{itemize}
\end{itemize}
\subsubsection{2.3 - The Diffusion Equation}
\begin{definition}[Max Principle (weak)]
    If $u(x,t)$ is a solution to the Diffusion Equation in a rectangle $0 \leq x \leq L$, $0 \leq t \leq T$, then the maximum of $u(x,t)$ occurs on the boundary of the rectangle. In other words on $x = 0, x = L, t = 0$. \\
    The minimum is similar as we can show that $-u(x,t)$ satisfies the same equation.\\
    The natrual interpretaion of this is that if you have a rod with no internal heat sourse, the hottest or coldest spot can only occour at $t =0$ or on the edges. 
\end{definition}
\begin{definition}[Uniqueness]
    There is uniqueness for the Dirichlet problem for the Diffusion Equation.
    That means there is at most one solution of 
    $$ \begin{cases}
        u_t - ku_{xx} = f(x,t) \text{ for } 0 < x < L, t > 0\\
        u(x,0) = \phi(x) \\
        u(0,t) = g(t) \\
        u(L,t) = h(t)
    \end{cases}$$
    For any given $f(x,t), \phi(x), g(t), h(t)$\\
    We can do proof by max principle.
    \begin{proof}
        We want to show that for all $u_1, u_2$ that satisfy the above conditions, $u_1 = u_2$.\\
        Let $w = u_1 - u_2$. Then $w$ satisfies the following:
        $$ \begin{cases}
            w_t - kw_{xx} = 0 \text{ for } 0 < x < L, t > 0\\
            w(x,0) = 0 \\
            w(0,t) = 0 \\
            w(L,t) = 0
        \end{cases}$$
        By max prinicple $w(x,t)$ has a maximum on its boundary. Also it must have a minimum on its boundary. Since $w(x,0) = 0$, the minimum and the maximum must be 0. Thus $w(x,t) = 0$ for all $x,t$.\\
        Thus $u_1 = u_2$.
    \end{proof}
    Now we can do a proof by energy.
    \begin{proof}
        We know that $w = u_1 - u_2$\\
        \begin{align}
            0 &= 0 \cdot w\\
            &= (w_t - kw_{xx})w\\
            &= (1/2w^2)_t + (-kw w_x)_x + kw_x^2
        \end{align}
        We can now integrat about $0 < x < L$ \\
        \begin{align}
            0 &= \int_0^L (1/2w^2)_t dx {- kw_x w |_0^L}_{goes to 0} + k \int_0^L w_x^2 dx\\
            \frac{d}{dt} \int_0^L 1/2w^2 dx &= -k  \int_0^L w_x^2 dx\\
        \end{align}
        Clealrly the derivative of $\int_0^L w^2 dx$ is decreasing 
        $$ \int_0^L w^2 dx \leq \int_0^L w(x,0)^2 dx $$
        The RHS is 0, so the LHS is 0. Thus $w = 0$. 
    \end{proof}
\end{definition}
\begin{definition}[Stablitity]
    The solution to the Diffusion Equation is stable. That means that if you have a small perturbation in the initial conditions, the solution will not change much.\\
    In other words they functions are "bounded" by initial conditions.\\
    This is in a $L_2$ sense.
    $$ \int_0^l [u_1(x,t) - u_2(x,t)]^2 dx \leq \int_0^l [u_1(x,0) - u_2(x,0)]^2 dx$$
\end{definition}
\subsubsection{2.4 - Diffusion on the Whole Line}
\begin{definition}[Invariance Properties]
    We have 5 basic invariance properties of the Diffusion Equation.
    \begin{itemize}
        \item Translation $u(x - y, t)$ is a solution if $u(x,t)$ is a solution.
        \item Any derivative of $u(x,t)$ is a solution. 
        \item A linear combination of solutions is a solution.
        \item An integral of a solution is a solution. Thus if $S(x,t)$ is a solution then so is $S(x - y, t)$ and so is $v(x,t) = \int_{-\infty}^\infty S(x - y, t) g(y) dy$. for any $g(y)$.
        \item Dilation. If $u(x,t)$ is a solution then so is $u(\sqrt{a}x, at)$ for any $a > 0$.
    \end{itemize}
\end{definition}
\begin{definition}[Fundamental Solution to the Diffusion Equation]
    The fundamental solution to the Diffusion Equation is 
    $$ S(x,t) = \frac{1}{\sqrt{4\pi kt}} e^{-x^2/4kt}$$
    This is a solution to the Diffusion Equation with $f(x,t) = 0$ and $u(x,0) = \delta(x)$.
    We can derive this by utilizing the invariance properties.
\end{definition}
\subsubsection{2.5 - Comparison of Waves and Diffusion}
\begin{table}[h!]
\centering
\begin{tabular}{|c|c|c|}
\hline
\textbf{Property} & \textbf{Waves} & \textbf{Diffusion} \\ \hline
Speed of Propogation & $c$ & Infinite \\ \hline
Singulatities for $t > 0$ & Transported along characteristics with speed c & Lost immediately \\ \hline
Well posed for $t > 0$ & Yes & Yes for bounded \\ \hline
Well posed for $t < 0$ & Yes & No \\ \hline
Max Principle & No & Yes \\ \hline
Behavior at infinity & Energy is constant so it doesn't decay & Decays to zero \\ \hline
Information & Transported & Lost gradually \\ \hline
\end{tabular}
\caption{Comparison of Waves and Diffusion}
\label{tab:comparison}
\end{table}
\subsubsection{4.1 - Seperation of Variables}
\begin{definition}[Seperation Solution Process for waves]
    We can consider the homogeneous Dirichlet problem for wave Equation.
    Due to linearity we can see that we have a seperated solution of the form $u(x,t) = X(x)T(t)$.\\
    $$ \begin{cases}
        u_{tt} = c^2 u_{xx} \\
        u(0,t) = u(L,t) = 0 \\
        u(x,0) = \phi(x) \\
        u_t(x,0) = \psi(x)
    \end{cases}$$
    Thus we can see the following ratios:
    $$ -\frac{T''}{cT} = -\frac{X''}{X} = \lambda$$
    We know this must be a consant since it doesnt depend on $x$ or $t$.\\
    We can now do our test cases: \\
    $$\begin{cases}
        \lambda = \beta^2 \\
        \lambda = 0 \\
        \lambda = -\beta^2
    \end{cases}
    $$
    We can Clearly see that this doesn't make sense for $\lambda = 0$ and $\lambda = -\beta^2$. they lead to solutions that are trivial and solutions that do not follow the boundary conditions.\\
    Thus for $\lambda = \beta^2$ we have the following:
    $$\begin{cases}
        X'' + \beta^2 X = 0 \\
        T'' + c^2\beta^2 T = 0
    \end{cases}
    \begin{cases}
        X(x) = A \cos(\beta x) + B \sin(\beta x) \\
        T(t) = C \cos(\beta ct) + D \sin(\beta ct)
    \end{cases}
    $$
    Thus by imposing the BC we see that $A = 0$ and $B sin(\beta l) = 0$. Thus for non trivial solutions we have $\beta = n\pi/l$.
    Thus our $\lambda = (n\pi/l)^2$ and our particualr eigenfunction correspodning to this eigenvalue is $X_n(x) = \sin(n\pi x/l)$ and $T_n(t) = \cos(n\pi ct/l) + \sin(n \pi ct/t)$.\\
    When we take our particualr solutions $u_n(x,t) = (A_n \cos(n\pi ct/l) + B_n \sin(n\pi ct/l)) \sin(n\pi x/l)$ we can see that we can form any solution as a linear combination of these solutions.
    $$ u(x,t) = \sum_{n=1}^\infty (A_n \cos(n\pi ct/l) + B_n \sin(n\pi ct/l)) \sin(n\pi x/l)$$
    We also require our IC to be satisfied. 
    $$ \phi(x) = \sum_{n=1}^\infty A_n \sin(n\pi x/l)$$
    $$ \psi(x) = \sum_{n=1}^\infty B_n \sin(n\pi x/l)$$
\end{definition}
\begin{definition}[Seperation Solution Process for Diffusion]
    We can consider the homogeneous Dirichlet problem for Diffusion Equation.
    $$ \begin{cases}
        u_t = ku_{xx} \\
        u(0,t) = u(L,t) = 0 \\
        u(x,0) = \phi(x)
    \end{cases}$$
    We can see that we have a seperated solution of the form $u(x,t) = X(x)T(t)$.\\
    $$ \begin{cases}
        -\frac{T'}{kT} = -\frac{X''}{X} = \lambda
    \end{cases}
    $$
    We can now do our test cases: \\
    $$\begin{cases}
        \lambda = \beta^2 \\
        \lambda = 0 \\
        \lambda = -\beta^2
    \end{cases}$$
    We can see that this is the same $X''/X$ as the wave equation. Thus we have the same solutions.
    Thus our $\lambda = (n\pi/l)^2$ For $n \in \mathbb{Z}$ thus our $T_n(t) = e^{-k(n\pi/l)^2 t}$ and $X_n(x) = \sin(n\pi x/l)$.\\
    Thus our particualr solutions $u_n(x,t) = A_n e^{-k(n\pi/l)^2 t} \sin(n\pi x/l)$ and we can form any solution as a linear combination of these solutions.
    $$ u(x,t) = \sum_{n=1}^\infty A_n e^{-k(n\pi/l)^2 t} \sin(n\pi x/l)$$
    We also require our IC to be satisfied.
    $$ \phi(x) = \sum_{n=1}^\infty A_n \sin(n\pi x/l)$$
\end{definition}
\subsubsection{4.2 - The Neumann Condition}
\begin{definition}[Neumann Condition for Diffusion]
    The Neumann Condition is the following:
    $$ \begin{cases}
        u_t = ku_{xx} \\
        u_x(0,t) = u_x(L,t) = 0 \\
        u(x,0) = \phi(x)
    \end{cases}$$
    We can see that we can reach the same eigenvalues but we need to check the eigenfunctions for solving the IC. \\
    Thus we can see that $X'(l) = 0 = -C \beta \sin(\beta l)$. Thus $\beta = n\pi/l$ and $X_n(x) = \cos(n\pi x/l)$ and $T_n(t) = e^{-k(n\pi/l)^2 t}$.\\
    Additionally $0$ is an eigenvalue and $X_0(x) = 1$ and $T_0(t) = 1$. (which makes it a consntant)\\
    Thus our particualr solutions $u_n(x,t) = A_n e^{-k(n\pi/l)^2 t} \cos(n\pi x/l)$ and we can form any solution as a linear combination of these solutions.
    $$ u(x,t) = A_0/2 + \sum_{n=1}^\infty A_n e^{-k(n\pi/l)^2 t} \cos(n\pi x/l)$$
    We also require our IC to be satisfied.
    $$ \phi(x) = A_0/2 + \sum_{n=1}^\infty A_n \cos(n\pi x/l)$$
\end{definition}
\begin{definition}[Neumann Condition for Waves]
    $$\begin{cases}
        u_{tt} = c^2 u_{xx} \\
        u_x(0,t) = u_x(L,t) = 0 \\
        u(x,0) = \phi(x) \\
        u_t(x,0) = \psi(x)
    \end{cases}$$
    We can see that we can reach the same eigenvalues but we also see that $0$ is an eigenvalue with $X_0(x) = 1$ and $T_0(t) = A + Bt$.\\
    Thus we can say the LC of our particulars 
    $$ u(x,t) = A_0/2 + B_0 t/2 + \sum_{n=1}^\infty (A_n \cos(n\pi ct/l) + B_n \sin(n\pi ct/l)) \cos(n\pi x/l)$$
    We also require our IC to be satisfied.
    $$ \phi(x) = A_0/2 + \sum_{n=1}^\infty A_n \cos(n\pi x/l)$$
    $$ \psi(x) = B_0/2 + \sum_{n=1}^\infty (n \pi c/l) B_n \cos(n\pi x/l)$$ 
\end{definition}
\subsubsection{4.3 - The Robin Condition}
\begin{definition}[Robin Condition for Diffusion]
    The Robin Condition is the following:
    $$ \begin{cases}
        u_t = ku_{xx} \\
        u_x(0,t) - a_0 u(0,t) = 0\\
        u_x(L,t) + a_L u(L,t) = 0 \\
        u(x,0) = \phi(x)
    \end{cases}$$
    \textbf{THIS IS EXCESSIVE JUST DO ALGEBRA}

\end{definition}

\subsubsection{6.1 - Laplace's Equation}
\begin{definition}[Laplace's Equation]
    We define Laplace's Equation (homogeneous) as the following:
    $$ \begin{cases}
        \Delta u = 0
    \end{cases}$$
    And the inhomogeneous Laplace's Equation as the following:
    $$ \begin{cases}
        \Delta u = f(x) 
    \end{cases}$$
\end{definition}
\begin{definition}[Max Principle]
    The Max is on the boundary of the region.
    The max can be inside the region if the region if the solution is constant.
\end{definition}
\begin{definition}[Invariance Properties]
    We say Laplace's Equation is invariant under all rigid motions.\\
    A rigid motion in the plane consists of tranlations and rotations.\\
    IE 
    $$ x' = x + a, y' = y + b$$
    and
    $$ x' = x \cos(\theta) - y \sin(\theta), y' = x \sin(\theta) + y \cos(\theta)$$ 
\end{definition}
\begin{definition}[Laplacian in Polar]
    We can define the 2-D Laplacian in polar coordinates as the following:
    $$ \Delta_2 = \frac{\partial^2 }{\partial x^2} + \frac{\partial^2 }{\partial y^2} $$
    We can prove that if we take the following change of variables:
    $$ x = r \cos(\theta), y = r \sin(\theta)$$
    That the Laplacian in polar coordinates is the following:
    $$ \Delta_2 = \frac{\partial^2 }{\partial r^2} + \frac{1}{r} \frac{\partial }{\partial r} + \frac{1}{r^2} \frac{\partial^2 }{\partial \theta^2}$$
    \textbf{Note:} $log(r)$ is the fundamental solution to the Laplacian in 2-D as it solves the rotationally invariant polar Laplacian.
\end{definition}

\subsubsection{6.2 - Rectangles and Cubes}
\textbf{This is a bitch to do, just know that we seperate it into a LC of each BC and solve each one in order of homogeneous then inhomogeneous}

\subsubsection{6.4 - Circles, Wedges and Annuli}
\begin{definition}[Wedge]
    We can take a wegde being 
    $$\setof{0 < \theta < \theta_0, 0 < r < a}$$
    Where our BC are 
    $$ u(r,0) =  u(r,\beta) = 0, u_r(a, \theta) = h(\theta)$$
    We can seperate the variables into 
    \begin{align*}
        u(r,\theta) &= R(r)\Theta(\theta) \\
        \Theta'' + \lambda \Theta &= 0 \\
        r^2 R'' + r R' - \lambda R &= 0
    \end{align*}
    \begin{align*}
        \Theta(\theta) &= \sin\frac{n \pi \theta}{\beta} \\
        R(r) &= r^{\alpha} \text{ for }  \alpha = \pm \sqrt{\lambda} = \pm \frac{n \pi }{\beta}
    \end{align*}
    Thus the LC of our solution is 
    $$ u(r,\theta) = \sum_{n=1}^{\infty} A_n r^{n \pi / \beta} \sin \frac{n \pi \theta}{\beta}$$
    We can solve the inhomogeneous BC
    $$ h(\theta) = \sum_{n=1}^{\infty} A_n \frac{n \pi}{\beta} a^{n \pi /B -1} \sin \frac{n \pi \theta}{\beta}$$
    We can sove for $A_n$ by recognizing that the RHS is the Fourier Sine Series of $h(\theta)$.
    $$ A_n = a^{1 - n \pi / \beta} \frac{2}{n \pi} \int_0^{\beta} h(\theta) \sin \frac{n \pi \theta}{\beta} d\theta$$
\end{definition}
\begin{definition}[Annulus]
    
\end{definition}
\begin{definition}[Exterior of a Circle]
    
\end{definition}

\subsubsection{5.1 - The Fourier Coefficients}
\begin{definition}[Fourier Sine Series]
    These following integrals show the orthogonality of $sin$ and $cos$ functions.
    $$ \int_0^l \sin(n\pi x/l) \sin(m\pi x/l) dx = \begin{cases}
        0 & \text{ if } n \neq m \\
        l/2 & \text{ if } n = m
    \end{cases}$$
    Thus if we consider the following:
    \begin{align*}
        \phi(x) &= \sum_{n=1}^\infty A_n \sin(n\pi x/l)\\
        \phi(x) \sin(m\pi x/l) &= \sum_{n=1}^\infty A_n \sin(n\pi x/l) \sin(m\pi x/l)\\
        \int_0^l \phi(x) \sin(m\pi x/l) dx& = \sum_{n=1}^\infty A_n \int_0^l \sin(n\pi x/l) \sin(m\pi x/l) dx\\
        \int_0^l \phi(x) \sin(m\pi x/l) dx &= A_m l/2\\
        A_m &= \frac{2}{l} \int_0^l \phi(x) \sin(m\pi x/l) dx
    \end{align*}
    Thus we can see that the $A_n$ are the Fourier Sine Coefficients.
    We can continue for all values of $n$ to get the entire series.
\end{definition}
\begin{definition}[Fourier Cosine Series]
    These following integrals show the orthogonality of $sin$ and $cos$ functions.
    $$ \int_0^l \cos(n\pi x/l) \cos(m\pi x/l) dx = \begin{cases}
        0 & \text{ if } n \neq m \\
        l/2 & \text{ if } n = m
    \end{cases}$$
    Thus if we consider the following:
    \begin{align*}
        \phi(x) &= A_0/2 + \sum_{n=1}^\infty A_n \cos(n\pi x/l)\\
        \phi(x) \cos(m\pi x/l) &= A_0/2 \cos(m\pi x/l) + \sum_{n=1}^\infty A_n \cos(n\pi x/l) \cos(m\pi x/l)\\
        \int_0^l \phi(x) \cos(m\pi x/l) dx& = A_0/2 \int_0^l \cos(m\pi x/l) dx + \sum_{n=1}^\infty A_n \int_0^l \cos(n\pi x/l) \cos(m\pi x/l) dx\\
        \int_0^l \phi(x) \cos(m\pi x/l) dx &= A_m l/2\\
        A_m &= \frac{2}{l} \int_0^l \phi(x) \cos(m\pi x/l) dx
    \end{align*}
    We can see that the $A_0$ has $n=0$ and the $cos$ term is 1. Thus $A_0 = 2/l \int_0^l \phi(x) dx$.\\
    Thus we can see that the $A_n$ are the Fourier Cosine Coefficients.
    We can continue for all values of $n$ to get the entire series.
\end{definition}
\begin{definition}[Full Fourier Series]
    We can see that the full Fourier Series is the sum of the Sine and Cosine Series.
    $$ \phi(x) = \frac{a_0}{2} + \sum_{n=1}^\infty a_n \cos(n\pi x/l) + \sum_{n=1}^\infty b_n \sin(n\pi x/l)$$
    We can see that the interval is $[-l,l]$ and our eigenfunctions are $\setof{1, \cos(n\pi x/l), \sin(n\pi x/l)}$.Thus we can mulitply any two of these and integrate to get the orthogonality.\\
    Thus we can see that 
    $$ \begin{cases}
        A_n = \frac{1}{l} \int_{-l}^l \phi(x) \cos(n\pi x/l) dx\\
        B_n = \frac{1}{l} \int_{-l}^l \phi(x) \sin(n\pi x/l) dx
    \end{cases}$$
\end{definition}
\begin{definition}[$c_n, a_n, b_n$]
   .\\
    $a_n = c_n + c_{-n}$\\
    $b_n = i(c_n - c_{-n})$
\end{definition}


\newpage
\section{Content to Review}
4.1 - Seperation of Variables for waves\\
4.3 - Robin Bounding Coundition
5.1 - Recognizing Fourier Series

\newpage
\section{Problems}
\begin{questions}
    \question Pg(110): \\
    Solve the following problem:
    \begin{align*}
        u_tt &= c^2 u_xx \\
        u(0,t) &= u(L,t) = 0 \\
        u(x,0) &= x \\
        u_t(x,0) &= 0
    \end{align*}
    We know that for the wave equation \\
    $$u(x,t) = \sum_1^\infty (A_n \cos(n\pi ct/l) + B_n \sin(n\pi ct/l)) \sin(n\pi x/l)$$
    $$ u_t(x,t) = \sum_1^\infty (-A_n \sin(n\pi ct/l) + B_n \cos(n\pi ct/l)) \frac{n\pi c}{l} \sin(n\pi x/l)$$
    Thus from our IC
    $$ 0 = \sum_1^\infty \frac{n\pi c}{l} B_n \sin(n\pi x/l)$$
    Thus $B_n = 0$ for all $n$.\\
    Now for our other IC
    $$ x = \sum_1^\infty A_n \sin(n\pi x/l)$$
    We can see that $\setof{A_i}$ is the Fourier Sine Coefficients of $x$.\\





\end{questions}

\newpage
\section{Review after exam}
\begin{questions}
    \question 3\\
    $$\begin{cases}
        \Delta u = 0 \text{ in } x^2+y^2 > 1\\
        u = y^2 \text{ on } x^2 + y^2 = 1
        u \text{ is bounded as } x^2 + y^2 \to \infty 
    \end{cases}$$
    \begin{solution}
        Sol in form of 
    $$ u(r, \theta) = \frac{a_0}{2} + \sum_{n=1}^\infty r^{-n} (a_n \cos(n\theta) + b_n \sin(n\theta))$$
    We can see that the BC is $y^2 = r^2 \sin^2(\theta)$
    $$ \frac{a_0}{2} + \sum_{n=1}^\infty r^{-n} (a_n \cos(n\theta) + b_n \sin(n\theta)) = sin^2(\theta) $$
    $$ sin^2(\theta) = \frac{1}{2} - \frac{1}{2} \cos(2\theta)$$
    $$\begin{cases}
        a_0 = 1\\
        b_n = 0 \\
        a_2 = -1/2
    \end{cases}
    $$
    So our solution is 
    $$u(r, \theta) = 1/2 - \frac{1}{2r^2}cos(2\theta)$$
    \end{solution}
    

    \question 4\\
    $$\begin{cases}
        \Delta u = \lambda u \text{ in } D\\
        u = 0 \text{ on } \partial D
    \end{cases}$$
    \begin{solution}
        Separating of variables\\
        $$ u(x,y) = X(x)Y(y)$$
        $$ X''Y + XY'' = \lambda XY$$
        $$ \frac{X''}{X} = - \frac{Y''}{Y} + \lambda = \alpha$$
        We solve in $X$
        $$ \begin{cases}
            X'' - \alpha X = 0\\
            X(0) = X(\pi) = 0
        \end{cases}$$
        $\alpha = -n^2$ and $X_n = \sin(nx)$\\
        We solve in $Y$
        $$ \begin{cases}
            Y'' - (\lambda - \alpha)Y = 0\\
            Y(0) = Y(\pi) = 0
        \end{cases}$$
        $\lambda - \alpha = -m^2$ and $Y_m = \sin(mx)$\\
        Thus $\lambda = -m^2 - n^2$. 
        $$ u_{n,m}(x,y) = \sin(nx) \sin(my)$$
        for $n,m \in \mathbb{N}$
        Notice that each lamda is not uniquely detemrined 
    \end{solution}

    \question 5\\
    $$ \partial_{xx}u + \partial_{yy}u + \partial_{xy}u = 0$$
    $u$ is in form of $u(x,y) = X(x)Y(y)$
    \begin{solution}
        \begin{align*}
            X''Y + XY'' + X'Y' &= 0 \\
            \frac{X''}{X} + \frac{X'Y'}{XY} &= - \frac{Y''}{Y}\\
            \text{Take partial x}\\
            \frac{\partial}{\partial x} \left[ (\frac{X''}{X})' + \frac{X'}{X}' \frac{Y'}{Y}\right] &= 0\\
            \frac{Y'}{Y} = -\frac{\frac{X''}{X}'}{\frac{X'}{X}'} &= \alpha\\
            Y' = \alpha Y &\implies Y(y) = e^{\alpha y}\\
            \frac{X''}{X} + \frac{X'}{X} \alpha = - \alpha^2 &\implies X'' + \alpha X' + \alpha^2 X = 0\\
            \lambda = \frac{-1\pm i\sqrt{3}}{2}\alpha\\
            X_1(x) = e^{\lambda_0 \alpha x}\\
            X_2(x) = e^{\overline{\lambda_0} \alpha x}\\
        \end{align*}

        \textbf{Transport method}
        $$ (\partial_{x}^2 + \partial_{y}^2 + \partial_{x}\partial_{y} )u= 0$$
        Factor as 
        $$ (x-\lambda y )( x - \overline{\lambda} y)$$
        Thus our operator is 
        $$ (\partial_x - \lambda \partial_y)(\partial_x - \overline{\lambda} \partial_y)u = 0$$
        $$u(x,y) = f(\lambda x +y) + g(\overline{\lambda} x + y)$$
        If we take $f, g$ to be exponential then we can see that the solution is sepeble\\
        $$ e^{\lambda x} e^{y} + e^{\overline{\lambda} x} e^{y}$$
        Take $f(z) = e^{\alpha z} = g(z)$


    \end{solution}
\end{questions}








\end{document}
\documentclass[answers,12pt,addpoints]{exam}
\usepackage{import}

\import{C:/Users/prana/OneDrive/Desktop/MathNotes}{style.tex}

% Header
\newcommand{\name}{Pranav Tikkawar}
\newcommand{\course}{01:640:423}
\newcommand{\assignment}{Chapter 6}
\author{\name}
\title{\course \ - \assignment}

\begin{document}
\maketitle
\section*{Chapter 6}
\begin{align*}
    u &= u(x,y)\\
    \Delta u &= u_{xx} + u_{yy}\\
\end{align*}
\begin{definition}[Weak Max Principle]
    $D \subset \mathbb{R}^2$ is open, bounded, connected. $\Delta u = 0$ in $D$. $u$ is continuous on $\bar{D}$.\\
    Then $\max_{\bar{D}} u = \max_{\partial D} u$.
\end{definition}
\begin{remark}
    Also hold for min \\
    Analogous to max principle for the heat equation. and the proof is similar.\\
    Strong max principle says max (or min) cannot be achieved in the interior of $D$ unless $u$ is constant\\
    $\partial D$ and $\bar{D}$ are closed and bounded sets; $u$ is continuous here. max and min are achieved here. (by extreme value theorem)\\
\end{remark}
\textbf{Idea:} If max is attained at $p \in D$ then $u_{xx}(p) \leq 0$, $u_{yy}(p) \leq 0$\\
Typically inequality is strict. $\implies \Delta u(p) < 0$ which is a contradiction \\
What to do when $\Delta u = 0$?\\ 
if wwe had $\Delta u > 0$ in $D$ we will det the desired contradiction.\\
\begin{proof}
    \begin{align*}
        v(x,y) &= u(x,y) + \epsilon(x^2 + y^2)\\
        \Delta v &= \Delta u + 4\epsilon\\
        \Delta v &= 0 + 4\epsilon > 0 \text{ in } D \text{ for } \epsilon > 0\\
    \end{align*}
    $$\max_{\bar{D}} v \text{ is achieved on } \partial D \text{ at } p_0 = (x_0,y_0) \in \partial D$$
    $$ u \leq v \leq v(p_0) = u(p_0) + \epsilon(x_0^2 + y_0^2) \leq \max_{\partial D} u + \epsilon$$
    $\bar{D} $ is bounded. $\exists R > 0$ such that $|(x,y)| \leq R$ for all $(x,y) \in \bar{D}$\\
    Thus for any $(x,y) \in \bar{D}$, $u(x,y) \leq \max_{\partial D} u + \epsilon R$\\
    This is true for all $\epsilon > 0$\\
    Thus as $\epsilon \to \epsilon$\\
    \begin{align*}
        \max(\bar{D}) u &\leq \max(\partial D) u\\
        \max(\bar{D}) u &\geq \max(\partial D) u\\
    \end{align*}
\end{proof}
\begin{corollary}[Uniqueness]
    $$\begin{cases}
        \Delta u = f \text{ in } D\\
        u = g \text{ on } \partial D\\
    \end{cases}$$
    Where f, g are continuous. and $u$ is continous in $D$. This Dirichilet problem has at most one solution.\\
    $D$ is open, bounded, connected.\\
    \begin{proof}
        suppose $u_1$ and $u_2$ are two solutions.\\
        Let $u = u_1 - u_2$\\
        $$\begin{cases}
            \Delta u = 0 \text{ in } D\\
            u = 0 \text{ on } \partial D\\
        \end{cases}$$
        by max principle, $\max(\bar{D}) u = \max(\partial D) u = 0$\\
        by min principle, $\min(\bar{D}) u = \min(\partial D) u = 0$\\
    \end{proof}
\end{corollary}
\begin{definition}[Invariance]
    $\Delta$ is invariant under rigid motions: translations and rotation:\\
    IE 
    Do a change of variables $x' = x + a$, $y' = y + b$\\
    $$u_{xx} + u_{yy} = 0 \implies u_{x'x'} + u_{y'y'} = 0$$
    $$ \Delta = \Delta' $$
    $$\begin{bmatrix}
        x'\\
        y'\\
    \end{bmatrix} = \begin{bmatrix}
        \cos(\theta) & -\sin(\theta)\\
        \sin(\theta) & \cos(\theta)\\
    \end{bmatrix} \begin{bmatrix}
        x\\
        y\\
    \end{bmatrix}$$
    $$u_{xx} + u_{yy} = 0 \implies u_{x'x'} + u_{y'y'} = 0$$
    $$ \Delta = \Delta' $$
    $$\Delta u = u_{rr} + \frac{1}{r} u_r + \frac{1}{r^2} u_{\theta\theta}$$
    This is clearly invariant under rotation.\\
\end{definition}
In physical problems, $\Delta$ is used to model isotropic phenomena.\\
Isoptropic meaning the same in all directions.\\
Conductivity equation: \\
$u$ is electric potential, \\
$-\nabla u = E$ is the electric field.\\
$div(\sigma \nabla u ) = 0$ in $D$ where $\sigma$ is a matrix function.\\
Search for soluntion to $\Delta u = 0$ in $D$ that is rotationally invariant.\\
\begin{align*}
    u &= u(r)\\
    \Delta u &= u_{rr} + \frac{1}{r} u_r\\
    \Delta u &= 0 \implies u_{rr} + \frac{1}{r} u_r = 0\\
    u_r + \frac{1}{r} u = C_1\\
    u = C_1 \ln(r) + C_2\\
    u = \frac{1}{2} ln(x^2 + y^2) \\
\end{align*} 
The last line is called the fundamental solution to $\Delta u = 0$ in $\mathbb{R}^2$\\
In $\mathbb{R}^3$, we have the ODE $\frac{1}{r} \frac{\partial }{\partial r} (r^2 \frac{\partial u}{\partial r}) = 0$\\
\begin{align*}
    u &= u(r)\\
    \Delta u &= u_{rr} + \frac{2}{r} u_r\\
    \Delta u &= 0 \implies u_{rr} + \frac{2}{r} u_r = 0\\
    u_r + \frac{2}{r} u = C_1\\
    u = C_1 r + C_2\\
    u = \frac{1}{r} \\
    u = \frac{1}{\sqrt{x^2 + y^2 + z^2}}
\end{align*}
This is the fundamental solution to $\Delta u = 0$ in $\mathbb{R}^3$\\
In $3D$ u is the electric potential of a unit charge a the origin. 
\section{Poission's Formula}
In 2D $D = \setof{r < 1 \text{ and } -\pi \leq \theta < \pi}$\\
$$\begin{cases}
    \Delta u = 0 \text{ in } D\\
    u = f \text{ on } \partial D\\
\end{cases}$$
$$ u = \frac{a_0}{2} + \sum_{n=1}^{\infty} r^n (a_n \cos(n\theta) + b_n \sin(n\theta))$$
$$f(\theta) = u(1,\theta) = \frac{a_0}{2} + \sum_{n=1}^{\infty} a_n \cos(n\theta) + b_n \sin(n\theta)$$
$$a_n = \frac{1}{\pi} \int_{-\pi}^{\pi} f(\theta) \cos(n\theta) d\theta$$
$$b_n = \frac{1}{\pi} \int_{-\pi}^{\pi} f(\theta) \sin(n\theta) d\theta$$
We can now calulate this series explicitly.\\
\begin{align*}
    u(r,\theta) &= \frac{1}{\pi} \int_{-\pi}^{\pi} r^n f(\theta) \left(\cos(n\theta)\cos(n\phi) + \sin(n\theta)\sin(n\phi)\right)d\phi + \frac{1}{2} \int_{-\pi}^{\pi} f(\phi)d\phi\\
    &= \frac{1}{\pi} \int_{-\pi}^{\pi} f(\theta) \cos(n(\theta - \phi))d\phi + \frac{1}{2} \int_{-\pi}^{\pi} f(\phi)d\phi\\
    &= \frac{1}{\pi} \int_{-\pi}^{\pi} f(\theta) \left[\frac{1}{2} + \sum_{n=1}^\infty r^n \cos(n(\theta - \phi))\right] d\phi
\end{align*}
Now consider the complex exponential $e^{in\theta}$\\
\begin{align*}
    z &= re^{i(\theta -\phi)}\\
    z^n &= r^n e^{in(\theta - \phi)}\\
    Re(z^n) &= r^n \cos(n(\theta - \phi))\\
    \frac{1}{2} + Re\sum_{n=1}^\infty z^n &= \frac{1}{2} + Re\left(\frac{z}{1-z}\right)\\
    &= Re\left(\frac{1-z +2z}{2(1-z)}\right)\\
    &= \frac{1}{2}Re\left(\frac{1+z}{1-z}\right)\\
    &= \frac{1}{2} \left( \frac{1- |z|^2}{|1-z|^2}\right)\\
    &= \frac{1}{2} \frac{1 - r^2}{1 - 2Re(z) + |z|^2}\\
    &= \frac{1}{2} \frac{1 - r^2}{1 - 2r\cos(\theta - \phi) + r^2}\\
\end{align*}
$$ u(r,\theta) = \frac{1}{2\pi} \int_{-\pi}^{\pi} f(\phi) P(r,\theta - \phi) d\phi$$
Where $P(r,\theta) = \frac{1 - r^2}{1 - 2r\cos(\theta) + r^2}$ is the Poission Kernel.\\





\end{document}

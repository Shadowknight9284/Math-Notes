\documentclass[answers, 12pts,addpoints]{exam}
\usepackage{import}


\import{C:/Users/prana/OneDrive/Desktop/MathNotes}{style.tex}

% Header
\newcommand{\name}{Pranav Tikkawar}
\newcommand{\course}{01:640:423}
\newcommand{\assignment}{Chapter 6}
\author{\name}
\title{\course \ - \assignment}

\begin{document}
\maketitle

\section*{Chapter 6}
\begin{align*}
    u &= u(x,y)\\
    \Delta u &= u_{xx} + u_{yy}\\
\end{align*}
\begin{definition}[Weak Max Principle]
    $D \subset \mathbb{R}^2$ is open, bounded, connected. $\Delta u = 0$ in $D$. $u$ is continuous on $\bar{D}$.\\
    Then $\max_{\bar{D}} u = \max_{\partial D} u$.
\end{definition}
\begin{remark}
    Also hold for min \\
    Analogous to max principle for the heat equation. and the proof is similar.\\
    Strong max principle says max (or min) cannot be achieved in the interior of $D$ unless $u$ is constant\\
    $\partial D$ and $\bar{D}$ are closed and bounded sets; $u$ is continuous here. max and min are achieved here. (by extreme value theorem)\\
\end{remark}
\textbf{Idea:} If max is attained at $p \in D$ then $u_{xx}(p) \leq 0$, $u_{yy}(p) \leq 0$\\
Typically inequality is strict. $\implies \Delta u(p) < 0$ which is a contradiction \\
What to do when $\Delta u = 0$?\\ 
if wwe had $\Delta u > 0$ in $D$ we will det the desired contradiction.\\
\begin{proof}
    \begin{align*}
        v(x,y) &= u(x,y) + \epsilon(x^2 + y^2)\\
        \Delta v &= \Delta u + 4\epsilon\\
        \Delta v &= 0 + 4\epsilon > 0 \text{ in } D \text{ for } \epsilon > 0\\
    \end{align*}
    $$\max_{\bar{D}} v \text{ is achieved on } \partial D \text{ at } p_0 = (x_0,y_0) \in \partial D$$
    $$ u \leq v \leq v(p_0) = u(p_0) + \epsilon(x_0^2 + y_0^2) \leq \max_{\partial D} u + \epsilon$$
    $\bar{D} $ is bounded. $\exists R > 0$ such that $|(x,y)| \leq R$ for all $(x,y) \in \bar{D}$\\
    Thus for any $(x,y) \in \bar{D}$, $u(x,y) \leq \max_{\partial D} u + \epsilon R$\\
    This is true for all $\epsilon > 0$\\
    Thus as $\epsilon \to \epsilon$\\
    \begin{align*}
        \max(\bar{D}) u &\leq \max(\partial D) u\\
        \max(\bar{D}) u &\geq \max(\partial D) u\\
    \end{align*}
\end{proof}
\begin{corollary}[Uniqueness]
    $$\begin{cases}
        \Delta u = f \text{ in } D\\
        u = g \text{ on } \partial D\\
    \end{cases}$$
    Where f, g are continuous. and $u$ is continous in $D$. This Dirichilet problem has at most one solution.\\
    $D$ is open, bounded, connected.\\
    \begin{proof}
        suppose $u_1$ and $u_2$ are two solutions.\\
        Let $u = u_1 - u_2$\\
        $$\begin{cases}
            \Delta u = 0 \text{ in } D\\
            u = 0 \text{ on } \partial D\\
        \end{cases}$$
        by max principle, $\max(\bar{D}) u = \max(\partial D) u = 0$\\
        by min principle, $\min(\bar{D}) u = \min(\partial D) u = 0$\\
    \end{proof}
\end{corollary}
\begin{definition}[Invariance]
    $\Delta$ is invariant under rigid motions: translations and rotation:\\
    IE 
    Do a change of variables $x' = x + a$, $y' = y + b$\\
    $$u_{xx} + u_{yy} = 0 \implies u_{x'x'} + u_{y'y'} = 0$$
    $$ \Delta = \Delta' $$
    $$\begin{bmatrix}
        x'\\
        y'\\
    \end{bmatrix} = \begin{bmatrix}
        \cos(\theta) & -\sin(\theta)\\
        \sin(\theta) & \cos(\theta)\\
    \end{bmatrix} \begin{bmatrix}
        x\\
        y\\
    \end{bmatrix}$$
    $$u_{xx} + u_{yy} = 0 \implies u_{x'x'} + u_{y'y'} = 0$$
    $$ \Delta = \Delta' $$
    $$\Delta u = u_{rr} + \frac{1}{r} u_r + \frac{1}{r^2} u_{\theta\theta}$$
    This is clearly invariant under rotation.\\
\end{definition}
In physical problems, $\Delta$ is used to model isotropic phenomena.\\
Isoptropic meaning the same in all directions.\\
Conductivity equation: \\
$u$ is electric potential, \\
$-\nabla u = E$ is the electric field.\\
$div(\sigma \nabla u ) = 0$ in $D$ where $\sigma$ is a matrix function.\\
Search for soluntion to $\Delta u = 0$ in $D$ that is rotationally invariant.\\
\begin{align*}
    u &= u(r)\\
    \Delta u &= u_{rr} + \frac{1}{r} u_r\\
    \Delta u &= 0 \implies u_{rr} + \frac{1}{r} u_r = 0\\
    u_r + \frac{1}{r} u = C_1\\
    u = C_1 \ln(r) + C_2\\
    u = \frac{1}{2} ln(x^2 + y^2) \\
\end{align*} 
The last line is called the fundamental solution to $\Delta u = 0$ in $\mathbb{R}^2$\\
In $\mathbb{R}^3$, we have the ODE $\frac{1}{r} \frac{\partial }{\partial r} (r^2 \frac{\partial u}{\partial r}) = 0$\\
\begin{align*}
    u &= u(r)\\
    \Delta u &= u_{rr} + \frac{2}{r} u_r\\
    \Delta u &= 0 \implies u_{rr} + \frac{2}{r} u_r = 0\\
    u_r + \frac{2}{r} u = C_1\\
    u = C_1 r + C_2\\
    u = \frac{1}{r} \\
    u = \frac{1}{\sqrt{x^2 + y^2 + z^2}}
\end{align*}
This is the fundamental solution to $\Delta u = 0$ in $\mathbb{R}^3$\\
In $3D$ u is the electric potential of a unit charge a the origin. 
\section{Poission's Formula}
In 2D $D = \setof{r < 1 \text{ and } -\pi \leq \theta < \pi}$\\
$$\begin{cases}
    \Delta u = 0 \text{ in } D\\
    u = f \text{ on } \partial D\\
\end{cases}$$
$$ u = \frac{a_0}{2} + \sum_{n=1}^{\infty} r^n (a_n \cos(n\theta) + b_n \sin(n\theta))$$
$$f(\theta) = u(1,\theta) = \frac{a_0}{2} + \sum_{n=1}^{\infty} a_n \cos(n\theta) + b_n \sin(n\theta)$$
$$a_n = \frac{1}{\pi} \int_{-\pi}^{\pi} f(\theta) \cos(n\theta) d\theta$$
$$b_n = \frac{1}{\pi} \int_{-\pi}^{\pi} f(\theta) \sin(n\theta) d\theta$$
We can now calulate this series explicitly.\\
\begin{align*}
    u(r,\theta) &= \frac{1}{\pi} \int_{-\pi}^{\pi} r^n f(\theta) \left(\cos(n\theta)\cos(n\phi) + \sin(n\theta)\sin(n\phi)\right)d\phi + \frac{1}{2} \int_{-\pi}^{\pi} f(\phi)d\phi\\
    &= \frac{1}{\pi} \int_{-\pi}^{\pi} f(\theta) \cos(n(\theta - \phi))d\phi + \frac{1}{2} \int_{-\pi}^{\pi} f(\phi)d\phi\\
    &= \frac{1}{\pi} \int_{-\pi}^{\pi} f(\theta) \left[\frac{1}{2} + \sum_{n=1}^\infty r^n \cos(n(\theta - \phi))\right] d\phi
\end{align*}
Now consider the complex exponential $e^{in\theta}$\\
\begin{align*}
    z &= re^{i(\theta -\phi)}\\
    z^n &= r^n e^{in(\theta - \phi)}\\
    Re(z^n) &= r^n \cos(n(\theta - \phi))\\
    \frac{1}{2} + Re\sum_{n=1}^\infty z^n &= \frac{1}{2} + Re\left(\frac{z}{1-z}\right)\\
    &= Re\left(\frac{1-z +2z}{2(1-z)}\right)\\
    &= \frac{1}{2}Re\left(\frac{1+z}{1-z}\right)\\
    &= \frac{1}{2} \left( \frac{1- |z|^2}{|1-z|^2}\right)\\
    &= \frac{1}{2} \frac{1 - r^2}{1 - 2Re(z) + |z|^2}\\
    &= \frac{1}{2} \frac{1 - r^2}{1 - 2r\cos(\theta - \phi) + r^2}\\
\end{align*}
$$ u(r,\theta) = \frac{1}{2\pi} \int_{-\pi}^{\pi} f(\phi) P(r,\theta - \phi) d\phi$$
Where $P(r,\theta) = \frac{1 - r^2}{1 - 2r\cos(\theta) + r^2}$ is the Poission Kernel.\\
$$P(z) = \frac{1 - |z|^2}{|1-z|^2}$$
Poisson kernal 
$$ P(r, \theta) = \frac{1 - r^2}{1 - 2r\cos(\theta) + r^2}$$
$$u(r, \theta) = \frac{1}{2\pi} \int_{-\pi}^{\pi} f(\phi) P(r, \theta - \phi) d\phi$$
\begin{remark}
    \begin{align*}
        u(1, \theta) = 0 
        P(1, \theta - \phi) = \frac{1 - 1}{1 - 2\cos(\theta - \phi) + 1} = \frac{0}{0} \text{ at} \theta = \phi\\
    \end{align*}
    We cannot interchange limits and integrals.\\
    \begin{align*}
        u(1,\theta) = \lim_{r \to 1} u(r,\theta) \neq \int_{-\pi}^{\pi} \lim_{r \to 1} f(\phi) P(r, \theta - \phi) d\phi\\
    \end{align*}
    $P(r, \theta - \phi)$ in the limit $r \to 1$ becomes $2\pi \delta(\theta - \phi)$ so that $\lim_{r \to 1} u(r,\theta) = f(\theta)$\\
    Look at book to reprouce the proof o $u(r,\theta) - f(\theta)$
\end{remark}
Now let us consider as an integration over a circle:
D is the unit disk in $\mathbb{R}^2$\\
$\bar{x} = (x,y) \leftrightarrow (r, \theta)$ on the inside\\
$\bar{y} = (1, \phi)$ which is the boundary\\
$$r = |\bar{x}| = \sqrt{x^2 + y^2} = r^2 -2\bar{x}\bar{y} + 1$$
$$P(r, \theta - \phi) = \frac{1 - |\bar{x}|^2}{|\bar{x} - \bar{y}|^2}$$
$$u(\bar{x}) = \frac{1 - |\bar{x}|^2}{2\pi} \int_{\partial D} \frac{f(\bar{y})}{|\bar{x} - \bar{y}|^2} dS(\bar{y})$$
This is a arc length integral.\\
$$ u(0) = \frac{1}{2\pi} \int_{-\pi}^{\pi} f(\bar{y}) dS(\bar{y}) = \frac{1}{2\pi} \int_{-\pi}^{\pi} u(1,\phi) dS(\bar{y})$$
$u(0)$ is the average of $u(1,\phi)$ over the boundary.\\
\textbf{Excercise:} rewrite Poission's formula in a open disk\\
\begin{theorem}[Mean Value Property.]
    $u$ is harmonic in $D_r(\bar{x})$ and continous in the closed disk, then \\
    (1) hollow: 
    $$ u(\bar{x}) = \frac{1}{2\pi r} \int_{\partial D_r(\bar{x})} u(\bar{y}) dS(\bar{y})$$
    $$ = (ave)\int_{\partial D_r(\bar{x})} u(\bar{y}) dS(\bar{y})$$
    (2) solid:
    $$ u(\bar{x}) = \frac{1}{\pi r^2} \int_{D_r(\bar{x})} u(\bar{y}) d(\bar{y})$$
    $$ = (ave)\int_{D_r(\bar{x})} u(\bar{y}) d(\bar{y})$$
    \begin{proof}
        (1) Done in the previous section. We can translate to the origin.\\
        (2) Polar integral\\
        $$\int_{D_r(\bar{x})} u(\bar{y}) d(\bar{y}) = \int_0^r \int_{\partial D_s(\bar{x})} u(\bar{y}) dS(\bar{y}) ds$$
        $$ = \int_0^r u(\bar{x} \cdot 2\pi \rho) dS$$
        $$ \pi r^2 u(\bar{x})$$
    \end{proof}
    \begin{remark}
        In 2D. If we have a harmonic function. IE\\
        $$\begin{cases}
            \Delta u = 0 \text{ in } D = D_1(0)\\
            u = f \text{ on } \partial D\\
        \end{cases}$$
        We can find a funciton $f = u + iv$ (complex analytic)\\
        $v$ is the harmonic conjugate of $u$\\
        Cauchy's integral formula\\
        $$ f(z) = \frac{1}{2\pi i} \int_{\partial D} \frac{f(w)}{w - z} dw$$
        We can see this as using the Poission kernel.\\
        in reality this is using the boundary to find the interior.\\
        $f(0) = \frac{1}{2\pi i} \int_{\partial D} \frac{f(w)}{w} dw = \frac{1}{2\pi} \int_{-\pi}^{\pi} f(e^{i\theta}) d\theta $\\
        $$u(0) = \frac{1}{2\pi} \int_{-\pi}^{\pi} u(e^{i\theta}) d\theta$$
    \end{remark}
\end{theorem}
\begin{theorem}[Strong Max Principle]
    $D$ is open, bounded, connected. $\Delta u = 0$ in $D$. $u$ is continous on $\bar{D}$.\\
    Then max or min of $u$ is achieved on $\bar{D}$\\
    It cannot be attained in the interior of $D$ unless $u$ is constant.\\  
    \begin{proof}
        Suppose $M = \max_{\bar{D}}(u)$ is attained in $D$. ie there exists $\bar{x}_0 \in D$ such that $u(\bar{x}_0) = M$\\
        Then there exists $r > 0$ such that $D_r(\bar{x}_0) \subset D$\\
        $$M = u(\bar{x}_0) = \frac{1}{2\pi r} \int_{\partial D_r(\bar{x}_0)} u(\bar{y}) dS(\bar{y})$$
        $$\int_{\partial D_r(\bar{x}_0)} M - u(\bar{y}) dS(\bar{y}) = M \cdot \pi r^2 -\int_{\partial D_r(\bar{x}_0)} u(\bar{y}) dS(\bar{y}) = 0$$
        $M - u \geq 0 \text{ in } D_r(\bar{x})$ and has $0$ integral hence $M - u = 0$ in $D_r(\bar{x}_0)$\\
        $u(\bar{y}) = M$ for any $\bar{y} \in D_r(\bar{x}_0)$\\
        Remains to show that $u$ is $M$ throughout $D$\\
        Pick any $\bar{x} \in D$\\
        Take center of disks that remain inside the region.\\
        The center of each next disk is in the previous disk.\\
        Have a fixed step/ radius.\\
        This implies $u$ is constant in $D$\\
    \end{proof}
\end{theorem}
\begin{theorem}[Smoothness]
    $D$ is open, $u$ has second order partials that are all continous in $D$.\\
    $u \in C^2(D)$ and $\Delta u = 0$ in $D$\\
    Then $u \in C^\infty(D)$\\
    \begin{proof}
        Pick $\bar{x}_0 \in D$\\
        Consider a disk $D_r(\bar{x}_0) \subset D$\\
        Let $\bar{x}_0 = 0$ and $r = 1$\\
        To prove that $u$ has all partials at $0$, $u(\bar{x}) = \frac{1-|x|^2}{2\pi}\int_C \frac{u(\bar{y})}{|\bar{x} - \bar{y}|^2} dS(\bar{y})$\\
        As long as we stay away form the boundary we are chilling
    \end{proof}
\end{theorem}
Recall: 
$$ u \in C^2(D) \& \delta u = 0 \implies u \in C^\infty(D)$$
\begin{remark}
    $u$ may even be discontinous on $\partial D$\\
    No conclusion can be drawn about smoothness on the boundary.\\
\end{remark}
\section*{6.4 }
Consider wedges and annuli.\\
These become rectangles on the $r, \theta$ plane.\\\\
$$ \begin{cases}
    \Delta u = 0 \text{ in }  r>1\\
    u = f \text{ on } r = 1\\
    u \text{ is bounded as } r \to \infty\\
\end{cases}$$
We take all have the eigenfunctions being $1, ln(r), r^n, r^{-n}$\\
We only keep $1, r^{-n}$ because of BC. 
$$ u(r,\theta) = a_0/2 + \sum_{n=1}^\infty r^{-n} (a_n \cos(n\theta) + b_n \sin(n\theta))$$
Summing we get 
$$ u(r,\theta) = \frac{1}{2\pi} \int_{-\pi}^{\pi} f(\phi) P(1/r, \theta - \phi) d\phi$$
Where $P(r,\theta) = \frac{1 - r^2}{1 - 2r\cos(\theta) + r^2}$ is the Poission Kernel.\\
Notice that $P(1/r, \theta ) = -P(r, \theta)$\\
\begin{example}
    $$V(\rho, \theta) = u(1/\rho, \theta), \rho < 1$$
    $$r = 1/\rho > 1$$
    Now have infinity mapped to the origin, and we can change our BC
    $$\begin{cases}
        \Delta V = 0 \text{ in } 0 < \rho < 1\\
        V = f \text{ on } \rho = 1\\
        V \text{ is bounded as } \rho \to 0\\
    \end{cases}$$
\end{example}
$$ u(r, \theta) = a_0/2 + \tilde{a_0}ln(r)/2 + \sum_{n=1}^\infty r^{n} (a_n \cos(n\theta) + b_n \sin(n\theta)) + r^{-n}(\tilde{a_n}cos(n\theta) + \tilde{b_n}sin(n\theta))$$


\end{document}

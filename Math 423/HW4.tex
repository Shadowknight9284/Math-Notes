\documentclass[answers,12pt,addpoints]{exam}
\usepackage{import}

\import{C:/Users/prana/OneDrive/Desktop/MathNotes}{style.tex}

% Header
\newcommand{\name}{Pranav Tikkawar}
\newcommand{\course}{01:640:423}
\newcommand{\assignment}{Homework 4}
\author{\name}
\title{\course \ - \assignment}

\begin{document}
\maketitle

\begin{questions}
    \question Question 2.4 11
    \begin{parts}
        \part Consider the diffusion equation on the whole line with the usual
        initial condition $u(x, 0) = \phi(x)$. If $\phi(x)$ is an odd function, show
        that the solution $u(x, t)$ is also an odd function of $x$. (Hint: Consider
        $u(-x, t) + u(x, t)$ and use the uniqueness.)
        \part Show that the same is true if “odd” is replaced by “even.”
        \part Show that the analogous statements are true for the wave equation.
    \end{parts}
    \textbf{Solution:}\\
    a) \\
    Let $u(x, t)$ be the solution to the diffusion equation on the whole line with the initial condition $u(x, 0) = \phi(x)$. Let $\phi(x)$ be an odd function. We want to show that $u(x, t)$ is also an odd function of $x$.\\
    More formally: $\phi(-x) = -\phi(x) \implies u(-x,t) = -u(x,t)$ 
    We can consider $u(-x,t)$ as it has the associated initial values $\phi(-x) = -\phi(x)$.\\
    We can then consider $v(x,t) = u(-x,t) + u(x,t)$.\\ 
    We know that $v$ must solve the diffusion equation through linearity.\\
    We also know that $v(x,0) = u(-x,0) + u(x,0) = -\phi(x) + \phi(x) = 0$.\\
    By uniqueness, we know that $v(x,t) = 0$.\\
    Thus, $u(-x,t) + u(x,t) = 0 \implies u(-x,t) = -u(x,t)$.\\
    Thus, $u(x,t)$ is an odd function of $x$.\\
    b) \\
    We can use the same logic as in part a) to show that if $\phi(x)$ is an even function, then $u(x,t)$ is also an even function of $x$.\\
    More formally: $\phi(-x) = \phi(x) \implies u(-x,t) = u(x,t)$\\
    We can consider $u(-x,t)$ as it has the associated initial values $\phi(-x) = \phi(x)$.\\
    We can then consider $v(x,t) = u(-x,t) - u(x,t)$.\\
    We know that $v$ must solve the diffusion equation through linearity.\\
    We also know that $v(x,0) = u(-x,0) - u(x,0) = \phi(x) - \phi(x) = 0$.\\
    By uniqueness, we know that $v(x,t) = 0$.\\
    Thus, $u(-x,t) - u(x,t) = 0 \implies u(-x,t) = u(x,t)$.\\
    Thus, $u(x,t)$ is an even function of $x$.\\
    c) \\
    We can use similar logic to show that the analogous statements are true for the wave equation.\\
    For the wave equation with odd initial conditions, we can show that the solution is odd.\\
    We have initial conditions $u(x,0) = \phi(x)$ and $u_t(x,0) = \psi(x)$.\\
    We have $\phi(-x) = -\phi(x)$ and $\psi(-x) = -\psi(x)$.\\
    We can consider $u(-x,t)$ as it has the associated initial values $\phi(-x) = -\phi(x)$ and $\psi(-x) = -\psi(x)$.\\
    We can then consider $v(x,t) = u(-x,t) + u(x,t)$.\\
    We know that $v$ must solve the wave equation through linearity.\\
    We also know that $v(x,0) = u(-x,0) + u(x,0) = -\phi(x) + \phi(x) = 0$ and $v_t(x,0) = u_t(-x,0) + u_t(x,0) = -\psi(x) + \psi(x) = 0$.\\
    By uniqueness, we know that $v(x,t) = 0$.\\
    Thus, $u(-x,t) + u(x,t) = 0 \implies u(-x,t) = -u(x,t)$.\\
    Thus, $u(x,t)$ is an odd function of $x$.\\
    For the wave equation with even initial conditions, we can show that the solution is even.\\
    We have initial conditions $u(x,0) = \phi(x)$ and $u_t(x,0) = \psi(x)$.\\
    We have $\phi(-x) = \phi(x)$ and $\psi(-x) = \psi(x)$.\\
    We can consider $u(-x,t)$ as it has the associated initial values $\phi(-x) = \phi(x)$ and $\psi(-x) = \psi(x)$.\\
    We can then consider $v(x,t) = u(-x,t) - u(x,t)$.\\
    We know that $v$ must solve the wave equation through linearity.\\
    We also know that $v(x,0) = u(-x,0) - u(x,0) = \phi(x) - \phi(x) = 0$ and $v_t(x,0) = u_t(-x,0) - u_t(x,0) = \psi(x) - \psi(x) = 0$.\\
    By uniqueness, we know that $v(x,t) = 0$.\\
    Thus, $u(-x,t) - u(x,t) = 0 \implies u(-x,t) = u(x,t)$.\\
    Thus, $u(x,t)$ is an even function of $x$.\\
    \question Question 2.4 16
    Solve the diffusion equation with constant dissipation $$u_t - ku_{xx} + bu = 0 \text{ for } -\infty < x< \infty \text{ with } u(x,0) = \phi(x) $$ 
    Where $b$ is constant.\\
    \textbf{Solution:}\\
    Following the hin we can make the change of variables that
    $$ u(x,t) = e^{-bt}v(x,t) $$
    Substituting this into the diffusion equation we get
    \begin{align*}
        u_t = -be^{-bt}v + e^{-bt}v_t\\
        u_{xx} = e^{-bt}v_{xx}
    \end{align*}
    Now substituting these into the diffusion equation we get:
    $$ -be^{-bt}v + e^{-bt}v_t - e^{-bt}v_{xx} + be^{-bt}v$$
    $$ e^{-bt}v_t - ke^{-bt}v_{xx} = 0$$
    $$ v_t = kv_{xx}$$
    We can see that $v(x,t)$ is the solution to the diffusion equation with initial condition $v(x,0) = \phi(x)$.\\
    We know the solution for this is 
    $$ v(x,t) = \frac{1}{\sqrt{4\pi kt}}\int_{-\infty}^{\infty}e^{-\frac{(x-s)^2}{4kt}}\phi(s)ds$$
    $$ u(x,t) = \frac{e^{-bt} }{\sqrt{4\pi kt}}\int_{-\infty}^{\infty}e^{-\frac{(x-s)^2}{4kt}}\phi(s)ds$$

    \question Question 2.5 2\\
    Consider a traveling wave $u(x,t) = f(x-at)$ where f is a given function of one variable
    \begin{parts}
        \part If it is a solution of the wave equation, show that the speed must be $a \pm c$ (unless f is a linear function).
        \part If it is a solution of the diffusion equation, find f and show that the speed a is arbitrary.
    \end{parts}
    \textbf{Solution:}\\
    a)\\
    If $u(x,t) = f(x-at)$ is a solution of the wave equation, then we have
    $$ u_{tt} = a^2f''(x-at)$$
    $$ u_{xx} = f''(x-at)$$
    Substituting these into the wave equation we get
    $$ a^2f''(x-at) = c^2f''(x-at)$$
    $$ a^2 = c^2$$
    $$ a = \pm c$$
    Unless f is a linear function.\\ 
    Then $f'' = 0$ and $a \neq c$.\\
    b)\\
    If $u(x,t) = f(x-at)$ is a solution of the diffusion equation, then we have
    $$ u_t = -af'(x-at)$$
    $$ u_{xx} = f''(x-at)$$
    Substituting these into the diffusion equation we get
    $$ -af'(x-at) = kf''(x-at)$$
    We can then thus we can rewrite this as
    $$ \frac{-a}{k} = \frac{f''(x-at)}{f'(x-at)}$$
    If we consider $\lambda = x-at$ then we can rewrite this as
    $$ \frac{f''(\lambda)}{f'(\lambda)} = \frac{d}{d \lambda} ln(f')$$
    Thus we can solve this ODE as 
    \begin{align*}
    ln(f') &= \frac{-a}{k}\lambda + C\\
    f' &= C_1 e^{\frac{-a}{k}\lambda}
    \end{align*}
    Integrating a second time we get 
    \begin{align*}
    f &= C_2 + C_1 \int e^{\frac{-a}{k}\lambda} d\lambda\\
    f &= C_2 + C_3 e^{\frac{-a}{k}\lambda}
    \end{align*}
    Now puting it back in terms of $x-at$ we get
    $$ f(x-at) = C_2 + C_3 e^{\frac{-a}{k}(x-at)}$$
    This clearly solves the diffusion equation and the speed $a$ is arbitrary.\\

    \question Question 2.5 3
    Let $u$ satisfy the diffusion question $u_t = \frac{1}{2}u_{xx}$ Let
    $$v(x,t) = \frac{1}{\sqrt{t}}e^{\frac{x^2}{2t}}u(\frac{x}{t}, \frac{1}{t}) $$
    Show that $v$ the "backwards" diffusion equation $v_t = -\frac{1}{2}v_{xx}$
    \textbf{Solution:}\\
    We can do this by computing the partial derivatives of $v$ and substituting them into the backwards diffusion equation.\\
    We have
    \begin{align*}
        v_t &= \left[\frac{-1}{2}t^{-3/2} e^{x^2 /2t} - t^{-1/2}\left( \frac{x^2}{2t^2}\right)e^{x^2 / 2t} \right]u + t^{-1/2}e^{x^2 / 2t}\left[ \frac{-x}{t^2}u_x - \frac{1}{t^2}u_t \right]\\
        v_x &= \frac{1}{\sqrt{t}} \left[ \frac{x}{t}e^{x^2 / 2t}u + \frac{1}{t} e^{x^2 / 2t}u_x \right] = t^{-3/2}e^{x^2 / 2t}(xu + u_x)\\
        v_{xx} &= t^{-3/2} \left\{ \frac{x}{t} e^{x^2 / 2t} (xu + u_x) + e^{x^2 / 2t}\left[u + \frac{x}{t}u_x + \frac{1}{t}u_xx \right]  \right\}
    \end{align*}
    We can then simplify the expression for $v_t$ and $v_{xx}$ 
    \begin{align*}
        v_t &= \frac{-1}{2}t^{-5/2}e^{x^2 / 2t}\left[ (x^2 + t)u + 2xu_x + 2u_t \right] 
        v_xx &= t^{-5/2} e^{x^2 / 2t} \left[ (x^2 + t)u + 2xu_x + 2u_t \right]
    \end{align*}
    We can then substitute these into the backwards diffusion and clearly see that it satisfies the equation.\\

    \question Question 4.1 2
    Consider a metal rod \(0 < x < l\), insulated along its sides but not at
    its ends, which is initially at temperature \(1\). Suddenly both ends are
    plunged into a bath of temperature \(0\). Write the differential equation,
    boundary conditions, and initial condition. Write the formula for the
    temperature \(u(x,t)\) at later times. In this problem, assume the infinite
    series expansion
    $$1 = \frac{4}{\pi}\left( \sin(\frac{\pi x}{l})+ \frac{1}{3}\sin(\frac{3\pi x}{l}) + \frac{1}{5}\sin(\frac{5\pi x}{l}) + \dots \right) $$
    \textbf{Solution:}\\
    We can utilize the PDE $u_t = ku_{xx}$ with boundary conditions $u(0,t) = 0$ and $u(l,t) = 0$ and initial condition $u(x,0) = 1$.\\
    Since the heat equation and the boundary conditions are linear and homogenous, we can use the method of seperation of variables and assume the solution is of the form $u(x,t) = X(x)T(t)$.\\
    We can then substitute this into the heat equation to get
    $$ u_t = ku_xx \implies XT' = kX''T$$
    Seperate the variables to get 
    $$ \frac{T'}{kT} = \frac{X''}{X} = \alpha$$
    We can then solve the ODEs for the cases of $\alpha = 0, -\lambda^2$ and $\lambda^2$ to get the general solution.
    \textbf{If $\alpha = 0$}\\
    We get $X(x) = c_1 + c_2x$\\
    We can then apply the boundary conditions to get $c_1 = c_2 = 0$\\
    Thus we get the trivial solution $X(x) = 0$\\
    \textbf{If $\alpha = \lambda^2$}\\
    We get $X(x) = c_1\cosh(\lambda x) + c_2\sinh(\lambda x)$\\
    We can then apply the boundary conditions to get $c_1 = 0$ and $c_2 = 0$\\
    Thus we get the trivial solution $X(x) = 0$ \\
    \textbf{If $\alpha = -\lambda^2$}\\
    We get $X(x) = c_1\cos(\lambda x) + c_2\sin(\lambda x)$\\
    We can then apply the boundary conditions to get $c_1 = 0$ and $c_1\cos(\lambda l) + c_2 \sin(\lambda l) = 0$\\
    We then get $ \sin(\lambda l) = 0$\\
    Solving this we get $\lambda = \frac{n\pi}{l}$ for $n \in \mathbb{N}$\\
    The eigenfunctions corresponding to these eigenvalues are $X_n(x) = \sin(\lambda_n x)$\\
    Now we can consider the differential equation for $T(t)$\\
    We get $T'(t) = -k\lambda^2 T(t)$\\
    Thus we can solve to get 
    $$T(t) = C e^{-k \lambda^2 t} $$
    Thus our eigenfunctions for $T(t)$ are 
    $$ T_n(t) = e^{-k \lambda_n^2 t}$$
    Thus our general solution is
    \begin{align*}
        u(x,t) &= \sum_{n=1}^{\infty} C_n e^{-k \lambda_n^2 t} \sin(\lambda_n x)\\
        &= \sum_{n=1}^{\infty} C_n e^{-k \left(\frac{n\pi}{l}\right)^2 t} \sin\left(\frac{n\pi x}{l}\right)
    \end{align*}
    Now we must satisfy the initial condition $u(x,0) = 1$\\
    We can do this by setting $t = 0$ in the general solution to get
    $$ u(x,0) = \sum_{n=1}^{\infty} C_n \sin\left(\frac{n\pi x}{l}\right) = 1$$
    By the hint we can see that $1 = \frac{4}{\pi}\left( \sin(\frac{\pi x}{l})+ \frac{1}{3}\sin(\frac{3\pi x}{l}) + \frac{1}{5}\sin(\frac{5\pi x}{l}) + \dots \right) $\\
    Thus we can see that $C_n = \frac{4}{(2n-1)\pi}$\\
    Thus our final solution is
    $$ u(x,t) = \sum_{n=0}^\infty \frac{4}{(2n-1)\pi} e^{-k \left(\frac{(2n-1)\pi}{l}\right)^2 t} \sin\left(\frac{(2n-1)\pi x}{l}\right)$$


    \question Question 4.1 3
    A quantum-mechanical particle on the line with an infinite potential outside the interval $(0,l)$ (“particle in a box”) is given by Schrödinger's equation $u_t = i u_{xx}$ on $(0,l)$ with Dirichlet conditions at the ends. Separate the variables and use (8) to find its representation as a series.
    \textbf{Solution:}\\
    We can utilize the PDE $u_t = iu_{xx}$ with boundary conditions $u(0,t) = 0$ and $u(l,t) = 0$.\\
    We can assume an arbitrary initial condition $u(x,0) = \phi(x)$.\\
    Since the heat equation and the boundary conditions are linear and homogenous, we can use the method of seperation of variables and assume the solution is of the form $u(x,t) = X(x)T(t)$.\\
    We can seperate the variables to get
    $$ \frac{T'}{iT} = \frac{X''}{X} = \alpha$$
    For the $X(x)$ ODE we can solve the cases of $\alpha = 0, -\lambda^2$ and $\lambda^2$ to get the general solution.\\
    As we saw from the prior question we must take $\alpha = -\lambda^2$\\
    And the eigenvaules are $\lambda = \frac{n\pi}{l}$ for $n \in \mathbb{N}$\\
    The eigenfunctions corresponding to these eigenvalues are $X_n(x) = \sin(\lambda_n x)$\\
    Now we can consider the differential equation for $T(t)$\\
    We get $T'(t) = -i\lambda^2 T(t)$\\
    Thus we can solve to get
    $$T(t) = C e^{-i\lambda^2 t} $$
    Thus our eigenfunctions for $T(t)$ are
    $$ T_n(t) = e^{-i\lambda_n^2 t}$$
    Thus our general solution is
    \begin{align*}
        u(x,t) &= \sum_{n=1}^{\infty} C_n e^{-i \lambda_n^2 t} \sin(\lambda_n x)\\
        &= \sum_{n=1}^{\infty} C_n e^{-i \left(\frac{n\pi}{l}\right)^2 t} \sin\left(\frac{n\pi x}{l}\right)
    \end{align*}

\end{questions}

\end{document}
\documentclass[answers,12pt,addpoints]{exam}
\usepackage{import}

\import{C:/Users/prana/OneDrive/Desktop/MathNotes}{style.tex}

% Header
\newcommand{\name}{Pranav Tikkawar}
\newcommand{\course}{01:640:423}
\newcommand{\assignment}{Chapter 7}
\author{\name}
\title{\course \ - \assignment}

\begin{document}
\maketitle
\section*{Greeen's identities}
Works in $2D$ and $3D$\\
We can consider a body $D$ and an $\vec{n}$ which is the normal to the boundary of $D$.\\
$$\partial_n u = \nabla u \cdot n$$

$u,v$ are nice functions on $\bar{D} = Dv \partial D$\\
Now consider integration by parts and divergence theorm:
\begin{align*}
    \int_D v \Delta u d\bar{x} &= \int_D \nabla \cdot (v \nabla u) d\bar{x} - \int_D \nabla v \cdot \nabla u d\bar{x}\\
\end{align*}
\begin{theorem}[Green's First Identity]
    Since $div(v \nabla u) = div(<v u_x, v u_y>) = \partial_x(v u_x) + \partial_y(v u_y)$\\
    $ = v_x u_x + v u_{xx} + v_y u_y + v u_{yy}$\\
    $ v \Delta u = div(v \nabla u) - \nabla v \cdot \nabla u$\\
\begin{align*}
    \int_D v \Delta u d\bar{x} &= \int_{\partial D} v \nabla u \cdot n d\bar{s} - \int_D \nabla v \cdot \nabla u d\bar{x}
\end{align*}
We can then consider that $\int_{\partial D} v \nabla u \cdot n d\bar{s} = \int_{D} v \nabla u + \nabla v \cdot \nabla u d\bar{x}$\\
This gives us Green's first identity.
\end{theorem}
\begin{theorem}[Green's second Theorem]
    We can do similar stuff by swithcing u and v\\
    $$\int_{\partial D} u \partial_n v d\bar{s} = \int_D u \Delta v  + \nabla u \cdot \nabla v d\bar{x}$$
    We can take the difference between the above and Green's first identity to get:
    $$\int_{\partial D} (u \partial_n v - v \partial_n u) d\bar{x} = \int_{D} u \Delta v - v \Delta u d\bar{x}$$
    This is Green's second identity.
\end{theorem}
\begin{example}
    Take $v= 1$\\
    $$\int_{\partial D} \partial_n u ds = \int_{D} \Delta u d\bar{x}$$
    This is just the divergence theorem.
\end{example}
\begin{example}[Neumann problem]
    $$(1) \text{ is } \begin{cases}
        \Delta u = 0 & \text{in } D\\
        \partial_n u = g & \text{on } \partial D
    \end{cases}$$
    Need $\int_{\partial D} g ds = 0$\\
    for the solvability of (1)\\ 
\end{example}
\begin{remark}
    $$\begin{cases}
        \Delta u = F & \text{in } D\\
        u = g & \text{on } \partial D
    \end{cases}$$
    We need 
    $$\int_{D} F d\bar{x} = \int_{\partial D} g ds$$
    This is called the compatibility condition.
\end{remark}
\begin{remark}
    If $u$ solves (1) then $u + c$ also solves (1)\\
    Thus no uniqueness.\\
    This is the only obstruction to uniqueness\\
    IE if $u,v$ solve (1) then $u-v$ = constant.
    This is because $w = u-v$ solves $\Delta w = 0$ and $\partial_n w = 0$\\
    The only way to have uniqueness is to have a normalization condition ie 
    $$\int_{\partial D} u ds = 1$$  
\end{remark}
\begin{theorem}[Mean Value Property]
    This was prved in 2D using the Poisson kernel.\\
    But the following is an alternative way to prove it.\\\\
    $B_r = \{\bar{x} : |\bar{x}| < r\}$\\ 
    $u \in C^2(\bar{B_r})$ and $\Delta u = 0$ in $B_r$\\
    We want to prove that average of the sphere is the value at the center.\\
    $$u(0) = (ave) \int_{\partial B_r} u ds$$
    We can consider $f(r) = (ave) \int_{\partial B_r} u ds = \frac{1}{4\pi r^2}\int_{\partial B_r} u(\bar{x}) ds$ \\
    We want to rescale, since we know $|x| = r$\\
    $$ |\bar{y}| = \frac{|\bar{x}|}{r} = 1$$
    Thus we get 
    $$ f(r) = \frac{1}{4\pi r^2} \int_{\partial B_1} u(r\bar{y}) r^2 ds(\bar{y}) = \frac{1}{4\pi} \int_{\partial B_1} u(r\bar{y}) ds(\bar{y})$$
    Now we can differntiate $f(r)$ with respect to $r$
    $$f'(r) = \frac{1}{4\pi} \int_{\partial B_1} \nabla u(r\bar{y}) \cdot \bar{y} ds(\bar{y})$$
    Clealry $\bar{y}$ is the norm to the sphere.\\
    We can also first scale back\\
    $$f'(r) = \frac{1}{4\pi} \int_{\partial B_r} \nabla u(\bar{x}) \cdot \bar{x}/r ds(\bar{x}) / r^2= $$ 
    $$ = \frac{1}{4\pi r^2} \int_{\partial B_r} \partial_n u ds$$
    We can now apply Green's identity to get
    $$f'(r) = \frac{1}{4\pi r^2} \int_{B_r} \Delta u d\bar{x} = 0$$
    Thus $f(r)$ is constant. for any $r \leq 1$\\
    We can now take $$f(1) = \lim_{r \to 0} f(r) = \lim_{r \to 0} \frac{1}{4\pi r^2} \int_{\partial B_r} u ds$$
    We can see that this is $0/0$ but we can use some dirac delta type moment to get $u(0)$
    \begin{align*}
        f(1) &= \lim_{r \to 0} \frac{1}{4\pi r^2} \int_{\partial B_r} u ds\\
        &= \lim_{r \to 0} \frac{1}{4\pi } \int_{\partial B_r} u(r\bar{y}) ds(\bar{y})\\
        &= u(0)
    \end{align*}
\end{theorem}
\begin{theorem}[Uniqueness for Dirichlet Problems]
    $$\begin{cases}
        \Delta u = 0 & \text{in } D\\
        u = 0 & \text{on } \partial D
    \end{cases} \implies u = 0$$
    We can prove this by maximum principle.\\
    Alternative proof by green's first identity.\\
    If we take $v = u$ then we get
    $$ \int_{\partial D} u \partial_n n ds = \int_D u \Delta u + |\nabla u|^2 d\bar{x}$$
    We know $\Delta u = 0$ and $u = 0$ on $\partial D$\\
    $$\int_{\partial D} 0 \partial_n n ds = \int_D u 0 + |\nabla u|^2 d\bar{x}$$ 
    $$ 0 = \int_D |\nabla u|^2 d\bar{x}$$
    This implies that $|\nabla u| = 0$ an thus $u = $ constant.\\
    Thus by the boundary condition $u = 0$.
\end{theorem}
\begin{theorem}[Dirichlet principle]
    $$\begin{cases}
        \Delta u = 0 & \text{in } D\\
        u = f & \text{on } \partial D
    \end{cases}$$
    Admissible set $A$: is the set of all functions $\in C^2(D)$ with the same BC \\
    $$A = \setof{w \in C^2(\bar{D}): w = f \text{ on } \partial D}$$
    We can introduce the energy functional
    $$E(w) = \frac{1}{2} \int_D |\nabla w|^2 d\bar{x}$$
    This is called the energy functional.\\
    This is kinda like potential energy.\\
    Assume $u \in A$ then \\
    $$ \Delta u = 0 \text{ in } D \leftrightarrow E[u] = \min_{w \in A} E[w]$$
    Ie the minimum energy is harmonic in $D$\\
\end{theorem}
\begin{theorem}[Green's Functions]
    $\Phi(\bar{x}) \begin{cases}
        \frac{1}{2\pi} ln(|\bar{x}|) & \text{in } R^2\\
        \frac{1}{4\pi |\bar{x}|} & \text{in } R^3
    \end{cases}$
\end{theorem}
\begin{lemma}
    For any $\bar{x}_0 \in R^d$
    $$ \Delta \Phi(\bar{x} - \bar{x}_0) = \delta(\bar{x} - \bar{x}_0)$$
    Distributional weak $\Delta$  
\end{lemma}
\begin{definition}
    $D = R^d$
    (G2) = 
    $$ \int_D u \Delta v - v \Delta u d\bar{x} = \int_{\partial D} (u \partial_n v - v \partial_n u) d\bar{s}$$
    suppose 
    $v \in \mathcal{D} = C_c^{\infty}(R^d)$\\
    Then the partials derivative of normal on v and v are 0.\\
    Thus we get
    $$ \int_D \Delta u v dx = \int_D u \Delta v dx$$
    We can use this to define the weak laplacian which is used for "rough" functions $u \in L^2(D)$
\end{definition}
\begin{definition}[Weak Laplacian]
    $ \Delta u$ to be a distribution such that $\mathcal{D} \to R$\\
    $$ \langle \Delta u, v \rangle = \langle u, \Delta v \rangle$$
    where $\langle \cdot, \cdot \rangle$ is $\int_D u v dx$\\
    This is the weak laplacian.
    $\langle \delta_0 , v \rangle = v(0)$
\end{definition}
\begin{lemma}
    For any $\bar{x}_0 \in R^d$
    $$ \Delta \Phi(\bar{x} - \bar{x}_0) = \delta(\bar{x} - \bar{x}_0)$$
\end{lemma}
\begin{definition}[Equality of Distributions]
    $$\langle \Delta \Phi(x-x_0), v\rangle = - \langle \delta(x-x_0), v \rangle$$
    $$\langle \Delta \Phi(x-x_0), v\rangle = - v(x_0)$$
    $$ \int_{R^d} \Phi(x-x_0) \Delta v dx = - v(x_0)$$
    \begin{proof}
        Say $x_0 = 0$
        $$ \int_{R^d} \Phi(x) \Delta v dx = \lim_{\epsilon \to 0} \int_{R^d} \Phi(x) \Delta v dx$$
        $$ = \lim_{\epsilon \to 0} \int_{|x| = \epsilon} \Phi(x) \partial_n v - v \partial_n \Phi ds$$ 
        Say $d =3$\\
        $$ | \int_{|x| = \epsilon} \Phi(x) \partial_n ds| \leq C \cdot \frac{1}{\epsilon} \cdot \epsilon^2 = C \epsilon \to 0$$
        Now consider $\partial_n \Phi v = \nabla \Phi(x) \cdot n(x) $
        We know that $n(\bar{x}) = \frac{\bar{x}}{\epsilon}$\\
        $$ \Phi(x) = \frac{1}{4\pi \sqrt{x^2 +y^2 + z^2}}$$
        $$ \nabla \Phi(x) = (\Phi_x, \Phi_y, \Phi_z) = - \frac{1}{4\pi} \frac{1}{2} (x^2 + y^2 + z^2)^{-3/2} ( 2x, 2y, 2z)$$ 
        $$ - \frac{1}{4\pi} (x^2 + y^2 + z^2)^{-3/2} (x, y, z)$$
        $$ - \frac{1}{4\pi} \frac{\bar{x}}{|\bar{x}|^3}$$
        $$ \partial_n \Phi(\bar{x}) = \frac{-\bar{x}}{4 \pi |\bar{x}|^3} \cdot \frac{\bar{x}}{\epsilon} = \frac{-1}{4\pi \epsilon^2}$$
        $$ \lim_{\epsilon \to 0} \frac{1}{4\pi \epsilon^2} \int_{|x| = \epsilon} v(\bar{x}) ds = - v(0)$$
        We show this by change of variable with $y = \frac{x}{\epsilon}$\\
        Now $y$ has size 1 and 
        $$ \int_{|x| = \epsilon} v(\bar{x}) ds = \int_{|y| = 1} v(\epsilon \bar{y}) \epsilon^2 dy $$
        $$ = \frac{1}{4\pi} \int_{|y| = 1} \lim_{\epsilon \to 0} v(\epsilon \bar{y}) ds(\bar{y}) = - v(0)$$
    \end{proof}
\end{definition}
\begin{lemma}[2: to solve possion Problems]
    Let $f \in C^2_c (R^d)$ and let 
    $$ u(x) = \int_{R^d} \Phi(x-y) f(y) dy$$
    Then $\Delta u = -f$ in $R^d$\\
    $$ \Delta u =\int_{R^d} \Delta_x \Phi(x-y) f(y) dy$$
    When we use lemma one ($\Delta \Phi(x-y) = \delta(x-y)$)\\
    $$ = \int_{R^d} \delta(x-y) f(y) dy = -f(x)$$
    $$f(x) = \int_{R^d} \delta(x-y) f(y) dy$$
    $$u(x) = \int_{R^d} \Phi(x-y) f(y) dy$$
    $$L = -\Delta$$
    $$k f(x) = \int_{R^d} \Phi(x-y) f(y) dy$$
\end{lemma}
Thus lemma 1 is saying $K L v = v$\\
An Lemma 2 says $L K v = v$\\
In otherwords $K$ is the inverse of $L$\\
Goal: $Lu = f \implies u = L^{-1}f$\\
$G_0 (x,y)$ the greens function for  the free space $R^d$\\
When we consider the inverse, it is not unique 
\begin{remark}
    The solution to $Lu = f$ in $R^d$ is not unique.\\
    As for any $\Delta \Phi = 0$\\
    $u + \Phi$ is also a solution.\\
    For uniqnuess we need conditions at infinite and $u \to 0$ as $|x| \to \infty$\\
\end{remark}
Bounded domain $D$\\
$$ \begin{cases}
    \Delta u = - f & \text{in } D\\
    u = 0 & \text{on } \partial D
\end{cases}$$
\begin{definition}
    Let $G(x,y)$ solbe for each 
    $$ \begin{cases}
        \Delta_x G(x,y) = \delta(x-y) & \text{in } D\\
        G(x,y) = 0 & \text{on } \partial D
    \end{cases}$$
    this is called Green's functions for $D$ (with Dirichlet BC)\\
    To find a representation formula for u, use Green's identity with u and G\\
    $$\int_D u(x) \Delta_x G(x,y) - G(x,y) \Delta u(x) dx = \int_{\partial D} (u \partial_n G - G \partial_n u) ds$$
    We can use some key facts to simplyfy this
    \begin{align*}
        \Delta_x G(x,y) &= \delta(x-y)\\
        \Delta_x u(x) &= -f(x)\\
        u(x) &= 0 \text{ on } \partial D\\
        G(x,y) &= 0 \text{ on } \partial D
    \end{align*} 
    Now we can fix $y$ and use Greens for $u(x)$ and $G(x,y)$
    $$ u(y) = \int_D G(x,y) f(x) dx$$
\end{definition}
We now have 2 natural questions
\begin{enumerate}
    \item Existence of $G(x,y)$
    \item How to find $G(x,y)$
\end{enumerate}
\begin{remark}
    if $\begin{cases}
        \Delta u = f & \text{in } D\\
        u = g & \text{on } \partial D
    \end{cases}$
    $u(x) = - \int_{\partial D} g(y) \partial_n G(x,y) ds(y) + \int_D G(x,y) f(y) dy$
\end{remark}
We know that 
$$G_0(x,y) = \Phi(x-y)$$
search for $G$ in the form of 
$$G(x,y) = G_0(x,y) + \Phi^{x}(y)$$
Fix $x$ and then 
$$\Delta_y G(x,y) = \Delta_y G_0(x,y) + \Delta_y \phi^{x}(y) $$
$$ = -\delta(x-y) + \Delta_y \phi^{x}(y)$$
$$G(x,y) = 0 \implies \phi^{x}(y) = -G_0(x,y)$$
Fixed $x \in D$\\
$$\begin{cases}
    \Delta \phi^{x}(y) = 0 & \text{in } D\\
    \phi^{x}(y) = -\Phi(x,y) & \text{on } \partial D
\end{cases}$$
Where $\phi^{x}(y)$ is the corrector funciton \\
$$G(x,y) = \Phi(x-y) + \phi^{x}(y)$$

In 3D we have $G_0(x,y)$ is \\
- the electrostatic potential ar y create by point charge at x\\
- the gravitational potential at y created by point mass at x\\
Now consider 
$$ - \Delta u = f$$
in $R^3$\\
$$ u(x) = \int_{R^3} G_0(x,y) f(y) dy$$
- the electrostatic potential at x due to the charge distribution defined by $f$\\
- the gravitational potential at x due to the mass distribution defined by $f$\\

This is also called newtonian potential\\

\textbf{History!}\\
Lagrange in 1774, moon motion\\
$G_0$ and $u$ appeared
Laplace in 1782, also in 1787, rings of saturn\\
observed that $\Delta u =0$ outside the support of $f$\\
Poisson in 1813\\
$- \Delta u = f$ solves poisson problems\\

For $D$ \\
Grounded counductor\\
$G_0$ is the electrostatic potnetial at y due to the charge at x in the presence of a grounded conductor on $\partial D$ (where $u = 0$ on $\partial D$)\\

Conductive plane\\
Place a charge at point near a conucting sheet\\
Sicne the sheet is conducting, the things are free to move.\\
We want to consider the eq state.\\
We claim that this is the same as considering the charge with no conducting sheet, and then adding a point of oppisote charge on the other side of the sheet (where it would be).\\

BC on a perfect conductor, $E$ must be normal to the conducting surface.\\
This is due the fact that the rgad is perpendicualr to the level curves\\
$L$ is a level curve of $u$ $\implies$ $ u $ is constant on $L$\\

$$D = \setof{x \in R^3: x_3 > 0}$$
$$\partial D = \setof{x \in R^3: x_3 = 0}$$
Method of images \\
$$\begin{cases}
    \Delta \phi^{x}(y) = 0 & \text{in } D\\
    \phi^{x}(y) = -\Phi(x,y) & \text{on } \partial D
\end{cases}$$
$x = (x_1, x_2, x_3)$\\
$x* = (x_1, x_2, -x_3)$\\
$\phi^{x}(y) = -\Phi(y-x*)$\\
We can see that the conditions are met.\\
Let $y \in \partial D$, $y = (y_1, y_2, 0)$\\
Then $|y - x| = |y - x*|$\\
Therfore $\Phi(y-x) = \Phi(y-x*)$\\
$$G(x,y) = \Phi(y-x) - \Phi(y-x*)$$
$$G(x,y) = \frac{1}{4\pi |y-x|} - \frac{1}{4\pi |y-x*|}$$
Now using this we can cover the possion probelm in the half space\\
Greens functions for D: fix $x \in D$\\
$$\begin{cases}
    -\Delta_y G(x,y) = \delta(x-y) & \text{in } D\\
    G(x,y) = 0 & \text{on } \partial D
\end{cases}$$
$$G(x,y) = \Phi(y-x) - \phi^{x}(y)$$
$$\begin{cases}
    \Delta_y \phi^{x}(y) = 0 & \text{in } D\\
    \phi^{x}(y) = \Phi(y-x) & \text{on } \partial D
\end{cases}$$
$$ \Phi(x) = \begin{cases}
    \frac{1}{4\pi |x|} & \text{in } R^3\\
    \frac{1}{2\pi} ln(|x|) & \text{in } R^2
\end{cases}$$
\textbf{3D Half Space}\\
$$D = R^3_+ = \setof{x \in R^3: x_3 > 0}$$
When reflecting the point $x, (x_1,x_2,x_3)$ over the plane $x_3 = 0$ aka $\partial D$ we get $x* = (x_1, x_2, -x_3)$\\
$$\phi(y) = \Phi(y-x*)$$
$$\begin{cases}
    \Delta \phi= 0 & \text{in } D\\
    \phi = \Phi(y-x) & \text{on } \partial D
\end{cases}$$
\begin{align*}
    G(x,y) &= \Phi(y-x) - \Phi(y-x*)\\
    &= \frac{1}{4\pi |y-x|} - \frac{1}{4\pi |y-x*|}\\
    &= \frac{1}{4\pi \sqrt{(y_1-x_1)^2 + (y_2-x_2)^2 + (y_3-x_3)^2}} - \frac{1}{4\pi \sqrt{(y_1-x_1)^2 + (y_2-x_2)^2 + (y_3+x_3)^2}}
\end{align*}
$$\begin{cases}
    \Delta u = 0 & \text{in } D\\
    u = g & \text{on } \partial D
\end{cases}$$
$$u(x) = \int_{\partial D} g(y) \partial_n G(x,y) ds(y)$$
\begin{align*}
    \partial_{n_y} G(x,y) &= \nabla_y G(x,y) \cdot n(y)\\
    &= \nabla_{y_3} G(x,y)
\end{align*}
\begin{align*}
    4\pi \partial_{y_3} \phi &= \partial_{y_3}\left(\left[(y_1-x_1)^2 +  (y_2-x_2)^2 + (y_3-x_3)^2\right]^{-1/2}\right) \\
    &= \frac{-1}{2} |y-x|^{-3} \cdot 2 (y_3-x_3)\\
    &= -\frac{y_3 - x_3}{|y-x|^3}\\
    -\partial_{y_3} G(x,y) &= \frac{1}{4\pi} \frac{(y_3 - x_3)}{|x-y|^3} - \frac{1}{4\pi} \frac{(y_3 + x_3)}{|x*-y|^3}\\
    &= -\frac{x_3}{2\pi |x-y|^3}\\
    u(x) &= - \int_{\partial D} g(y) \frac{x_3}{2\pi |x-y|^3} ds(y)\\
    &= \frac{x_3}{2\pi} \int_{\partial D} \frac{g(y)}{|x-y|^3} ds(y)\\
    u(x_1,x_2,x_3) &= \frac{x_3}{2\pi} \iint_{R^2} \frac{g(y_1,y_2)}{\sqrt{(x_1-y_1)^2 + (x_2-y_2)^2 + x_3^2}} dy_1 dy_2
\end{align*}
\textbf{2D Half Space}\\
$$D = R^2_+ = \setof{x \in R^2: x_2 > 0}$$
\begin{align*}
    G(x,y) &= \Phi(y-x) - \Phi(y-x*)\\
    &= -\frac{1}{2\pi} ln(|y-x|) + \frac{1}{2\pi} ln(|y-x*|)\\
    u(x) &= \frac{x_2}{\pi} \int_{\partial D} \frac{g(y)}{|x-y|^2} ds(y)\\
    &= \frac{x_2}{\pi} \int_{R} \frac{g(t)}{(t-x_1)^2 + x_2^2} dt
\end{align*}
$G$ is the poission kernal for the half space\\
\textbf{Quarter space}\\
We make multiple refection (3 in 2D) to get the quarter space\\
\textbf{3D ball}\\
$$D = B_1(0) = \setof{x = (x_1,x_2,x_3) \in R^3: |x| < 1}$$
$$\partial D = \setof{x \in R^3: |x| = 1}$$
Recall: For a point charge at 0, the e-field at $x$, $E= kq \frac{\hat{x}}{|x|^3}$\\
Where $\hat{x} = \frac{x}{|x|}$\\
$k = \frac{1}{4\pi \epsilon_0}$\\
We know that $E = -\nabla v$ where v is the potential function\\
$$v(x) = kq \frac{1}{|x|}$$
Goal: given charge 1 ar $\bar{x}$ in $D$; find an image charge such that the potential by the these two charge system is 0 on $\partial D$\\
$\bar{X} = c\bar{x}$ where we fix a point inside and outisde the ball that is colinear to the origin. \\
Goal is to choose $c$ and $Q$ such that $v(\bar{y}) = 0$ for $\bar{y} \in \partial D$\\
$$V = k q\frac{1}{|\bar{y} - \bar{x}|} - kQ \frac{1}{|\bar{y} - \bar{X}|}$$
at $A$ (other side of points): $\frac{kq}{1+|x|} + \frac{kQ}{1+|c\bar{x}|} = 0$\\
at $B$ (in betweeen points): $\frac{kq}{1-|x|} + \frac{kQ}{1-|c\bar{x}|} = 0$\\
For ease take $r = |x|$
$$\begin{cases}
    \frac{1}{1+r} + \frac{Q}{1+cr} = 0\\
    \frac{1}{1-r} + \frac{Q}{1-cr} = 0
    \implies c = \frac{1}{r^2}
\end{cases}$$
$$Q = -\frac{1}{r}$$
$$ \bar{X} = \frac{\bar{x}}{|x|^2}$$
So if we put the charge at $\bar{X}$ of a charge of $-\frac{1}{r}$ then the potential is 0 on the boundary.\\
\begin{definition}
    $x^* = \frac{x}{|x|^2}$ is called the dual/ image point of $x$ in $B_1(0)$ with respect to $\partial B_1(0)$
\end{definition}
$G(x,y) = \frac{1}{4\pi} \left(\frac{1}{|y-x|} - \frac{1}{|x||y-x*|}\right)$\\
Claim $G(x,y) = 0$ for $|y| = 1$\\
\begin{proof}
    $$|y-x| = |x| \cdot |y-x*|$$\\
    $$ r^2 |y-x|^2 = r^2 (1-2yx \frac{1}{r^2} + \frac{1}{r^2})$$
    $$ = r^2 - 2xy +1 = |x-y|^2$$
\end{proof}
at $x = 0$\\
$x* = \frac{x}{|x|^2} = \infty???$\\
\begin{align*}
    \frac{1}{|x||y-x*|} &= \frac{1}{ry - \frac{x}{r}}\\
    &= \frac{1}{ry - \hat{x}}\\
    &= \lim_{r \to \infty} \frac{1}{|\hat{x}|} = 1
\end{align*}
$$G(0,y) = \frac{1}{4\pi} \frac{1}{|y|} - \frac{1}{4\pi} $$
$$ \begin{cases}
    \Delta u = 0 & \text{in } |x| < 1\\
    u = g & \text{on } |x| = 1
\end{cases}$$
$$u(x) = \frac{1-|x|^2}{4\pi} \int_{|x| = 1} \frac{g(y)}{|y-x|^3} ds(y)$$
this is for 3d.
$$u(x) = \frac{1-|x|^2}{2\pi} \int_{|x| = 1} \frac{g(y)}{|y-x|^2} ds(y)$$
2D with conformal mappings:\\
We can use the half plane to find the disk.\\
half plane: z $H^+$, and disk: w $D$\\
$$w = \frac{z-i}{z+i}$$
$$z = i\frac{1+w}{1-w}$$
take any 2 points $w_1, w_2$ which map to $z_1, z_2$ in $D$, $G_D(w_1, w_2) = G_{H^+}(z_1, z_2)$\\
where $f(w_1) = z_1$ and $f(w_2) = z_2$\\

\end{document}
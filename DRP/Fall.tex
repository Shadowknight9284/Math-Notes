\documentclass[answers,12pt,addpoints]{exam}
\usepackage{import}

\import{C:/Users/prana/OneDrive/Desktop/MathNotes}{style.tex}

% Header
\newcommand{\name}{Pranav Tikkawar}
\newcommand{\course}{01:XXX:XXX}
\newcommand{\assignment}{Homework n}
\author{\name}
\title{\course \ - \assignment}

\begin{document}
\maketitle


\newpage
\section{2/19}
$$ d B_{f(t)} = \sqrt{f'(t)} dB_t $$
Loo into hornstin olbeck process
Understand Stationatiy (strong and weak)
Resad about ARMA and ARIMA what they mean, what they do, and differences
\subsection{Reading 2/19-2/26}
\begin{definition}[Ergodic Property with a constant limit]
    also known as EPCL
    $$ \exists \mu \in \R \suchthat \Prob( \lim_{n \to \infty} \bar{x} = \mu) = 1 $$
    where $\bar{x} = \frac{1}{n} \sum_{i=1}^n x_i$
\end{definition}
\begin{definition}[$L^2$ - Ergodic propetty with a constant limit]
    also known as $L^2$-EPCL
    $$ \exists \mu \in \R \suchthat \lim_{n \to \infty} \E((\bar{x} - \mu)^2) = 0 $$
\end{definition}
EPLC doesnt hold in a lack of stablilty, high variabilty of marginal distributions, and absobsing states.\\
\begin{definition}[Strict Stationarity]
    $X_t \in \R (t \in \Z)$ is said to be strictly stationary if 
    \begin{align*}
        \forall k \in \Z, \forall m \in \N, \forall t_1, t_2, \ldots, t_m \in \Z\\
        (X)_{t_1}, (X)_{t_2}, \ldots, (X)_{t_m} =_d (X)_{t_1 + k}, (X)_{t_2 + k}, \ldots, (X)_{t_m + k}
    \end{align*}
    In other words, the colllection of distributions of the random variables $(X)_{t_1}, (X)_{t_2}, \ldots, (X)_{t_m}$ is the same as the collection of distributions of the random variables shifted over $(X)_{t_1 + k}, (X)_{t_2 + k}, \ldots, (X)_{t_m + k}$
    \begin{example}[Moverage Average Process of order q]
        $X_t = \sum_{j=0}^{q} \psi_j \epsilon_{t-1} (t \in \Z)$
        where $\epsilon_t \in \R$ iid, $\psi_j \in \R$ and $j = 0, 1, \ldots, q$.
    \end{example}
\end{definition}
\begin{definition}[Weak Stationarity]
    Let $X_t \in \R (t \in \Z)$ s.t. $\forall t \in \Z$, $\E(|X_t|) < \infty$ \\
    Then $\mu_t = \E(X_t)$ is called the expected balue funtion or mean function of the process.\\
    IF $E(x_t^2) < \infty$ then $\gamma: \Z \times \Z \to \R$ with $\gamma(s,t) = \cov(X_s, X_t)$ is called the autocovariance function of the process. (acf)\\
    Also $\E{(X_s - \mu_s)(X_t - \mu_t)} = \cov(X_s, X_t)$\\
    Additionally 
    $$ \rho(s,t) = \text{corr}(X_s, X_t) = \frac{\cov(X_s, X_t)}{\sqrt{\var(X_s) \var(X_t)}}$$
    is called the autocorrelation function of the process. (acf)\\
    We can also say something is weakly stationary if\\
    \begin{align*}
        \E{X_t^2} < \infty \\
        \exists \mu \in \R \suchthat \E(X_t) = \mu \\
        \exists \gamma: \Z \to \R \suchthat \forall s,t \in \Z: \cov(X_s, X_t) = \gamma(t-s)
    \end{align*}
\end{definition}
\begin{definition}[k-step prediction mean squared error]
    Let $\mathscr{F}_{\leq t} = \sigma(X_s, s \leq t)$ \\
    $\mathscr{X}_t = \{Y | Y \in L^2(\Omega), \mathscr{F}_{\leq t}-\text{measurable}\}$\\
    Then the $k$-step prediction mean squared error is defined for $k \geq 1$ and $Y \in \mathscr{X}_t$ as\\
    $$\sigma_k^2(Y) = \E((X_{t+k} - Y)^2 $$ 
    
\end{definition}
\textbf{ASK to explain thm 2.1 (pg 23) and thm 2.2 (pg 24)}\\
\begin{definition}[Orthogonal Complement]
    Let $(H,\langle \cdot, \cdot \rangle)$ be a Hilbert space and $A \subseteq H$ be a closed subspace of $H$.\\
    The orthogonal complement of $A$ is defined as
    $$A^{\perp} = \{x: x\in H \suchthat \forall y \in A: \langle x,y \rangle = 0\}$$
    We also write $A \perp B \iff \forall x \in A, y \in B: \langle x,y \rangle = 0$\\
    Note that $A^{\perp}$ is a closed linear subspace of $H$ 
\end{definition}
\textbf{Dont understand the implications of corollary 2.3 pg 26}
\begin{definition}[Optimal Forecast]
    Let $X_t \in \R (t \in \Z)$ be a weakly stationary process. \\
    \begin{align*}
        \mathscr{F}_{\leq t} = \sigma(X_s: s \leq t)\\
        \mathscr{X}_t = \{ Y | Y \in L^2(\Omega), \mathscr{F}_{\leq t}-\text{measurable} \}\\
        k \in \N, \hat{X}_{t+k} \in \mathscr{X}_t
    \end{align*}
    Then $\hat{X}_{t+k}$ is the optimal forecast of $X_{t+k}$ given the information up to time $t$ (i.e. $\mathscr{F}_{\leq t}$)
    $$\iff$$
    $$\sigma_k^2(\hat{X}_{t+k}) = \inf_{Y \in \mathscr{X}_t} \sigma^2(Y)$$
\end{definition}
\begin{definition}[Conditional Expectation]
    A $\mathscr{G}$-measurable random variable $Y$ is said to be the conditional expectation of $X$ given $\mathscr{G}$ if
    \begin{align*}
        \E(X | \mathscr{G}) = Y \\
        \iff \forall A \in \mathscr{G}: \int_A X dP = \int_A Y dP
    \end{align*}
    \textbf{Note:} $\E(X | \mathscr{G})$ is a $\mathscr{G}$-measurable random variable.\\
\end{definition}
\begin{corollary}
    \begin{align*}
        X_t \in \R (t \in \Z \text{ is a weakly stationary process})\\
        \hat{X}_{t+k} \in \mathscr{X}_t\\
        \sigma_k^2(\hat{X}_{t+k}) = \inf_{Y \in \mathscr{X}_t} \sigma^2(Y) \iff \hat{X}_{t+k} = \E(X_{t+k} | \mathscr{F}_{\leq t})\\
    \end{align*}
    In other words the optimal forecast of $X_{t+k}$ given the information up to time $t$ is the conditional expectation of $X_{t+k}$ given $\mathscr{F}_{\leq t}$
\end{corollary}
\begin{definition}[Infinte linear past of $X_t$]
    \begin{align*}
        L_t^0 = \left\{Y | Y = \sum_{j=1}^k a_j X_{t_j}, k \in \N, a_j \in \R, t_j \in \Z, t_j \leq t \right\}\\
        L_t = \overline{L_t^0} = \left\{Y | \exists Y \in L_t^0 (n \in \N) \suchthat \lim_{n\to \infty} || Y - Y_t||^2_{L^2(\Omega)} = 0\right\}\\
        L_{-\infty} = \bigcap_{t = -\infty}^{\infty} L_t = \text{infinite linear past of $X_t$}
    \end{align*}
\end{definition}
\begin{definition}[Optimal Linear Forcast of $X_{t+k}$ given $\mathscr{F}_{\leq t}$]
    Let $X_t \in \R (t \in \Z)$ be a weakly stationary process with $k \geq 1$ and $\hat{X}_{t+k} \in L_t$\\
    Then
    \begin{align*}
        \hat{X}_{t+k} = \text{optimal linear forecast of $X_{t+k}$ given $\mathscr{F}_{\leq t}$} \\
        \iff \\
        \sigma_k^2(\hat{X}_{t+k}) = \inf_{Y \in \mathscr{X}_t} \sigma^2(Y)\\
    \end{align*}
\end{definition}
\begin{definition}[Deterministic Stochastic]
    Let $X_t \in \R (t \in \Z)$ be a weakly stationary process.\\
    It is called deterministic if $\sigma_k^2(X_{t+k}) = 0$ \\
    More generally:
    \begin{align*}
        \forall t \in \Z: \inf_{Y \in L_t} \mathbb{E}\left[(Z_{t+1} - Y)^2\right] =0
    \end{align*}
\end{definition}
\begin{theorem}[Wold Decomposition Theorem]
    Let $X_t \in \R (t \in \Z)$ be a weakly stationary process.\\
    Then 
    \begin{align*}
        \exists a_0, a_1, a_2, \ldots \text{ s.t. } a_0 = 1, \sum_{j=1}^{\infty} a_j^2 < \infty
    \end{align*}
    and 
    \begin{align*}
        \exists \epsilon_t, \mu_t (t \in \Z) \text{ s.t. } \forall s,t \in \Z:\\
        \epsilon_t \in L_t, \mu_t \in L_{-\infty}\\
        E(\epsilon_t) = 0, \cov(\epsilon_s, \epsilon_t) \sigma_{\epsilon}^2 \delta_{s,t} \leq \infty, \cov(\epsilon_s, \mu_t) = 0\\
        X_{t_{a.s., L^2(\Omega)}} = \mu_t + \sum_{j=0}^{\infty} a_j \epsilon_{t-j} (t \in \Z)
    \end{align*}
    pg(26/34)
    This is literally fourier series for time series.\\
    https://math.stackexchange.com/questions/703246/i-have-trouble-understanding-the-proof-of-the-wold-decomposition-theorem
\end{theorem}
\begin{definition}[Purely Stochastic or Regular]
    We say if $X_t$ has wold decomposition with $\mu_t \equiv \mu \in \R, \sigma_{\epsilon}^2 > 0$\\
    Then $X_t$ is called purely stochastic or regular.\\
\end{definition}
\begin{definition}
    Let $a = \{a_j\}_{j\in \N} \in \R^{\N}$ , $0 < \beta < \infty$\\
    Then 
    \begin{align*}
        || a ||_{\ell^{\beta}} = \left( \sum_{j=1}^{\infty} |a_j|^{\beta} \right)^{\frac{1}{\beta}} < \infty\\
    \end{align*}
\end{definition}
\begin{definition}[Future and Asymptotic events]
    \begin{align*}
        \mathscr{F}_t(t \in \Z) = \text{ sequence of $\sigma$-algebras on } \Omega\\
        \mathscr{F}_{>t} = \sigma\left( \bigcup_{s=t+1}^{\infty} \mathscr{F}_s \right) \text{future events}\\
        \mathscr{F}^{\infty} = \bigcap_{t=1}^{\infty} \mathscr{F}_{>t} \text{ asymptotic events}\\
    \end{align*}
\end{definition}
\begin{definition}[Ergodic Processes]
    Let $X_t(t \in \Z)$ be $\mathscr{F}_t$-measurable.\\
    \begin{gather*}
        X_t \text{ is an Ergodic process} \\
        \iff \\
        \forall B \in \mathscr{F}^{\infty}: \Prob^2(B) = \Prob(B) \\
        \iff \\
        \forall B \in \mathscr{F}^{\infty}: \Prob(B) \in \{0,1\}
    \end{gather*}
\end{definition}
\begin{theorem}[Kolmogorov's 0-1 Law]
    \begin{gather*}
        X_t (t \in \Z) \text{ iid } \implies X_t \text{ is ergodic} \\ 
    \end{gather*}
\end{theorem}
\begin{theorem}[Birkhoff's Ergodic Theorem]
    \begin{gather*}
        X_t (t\in \Z) \text{ strictly stationary, ergodic}, \mathbb{E}(|X_t|) < \infty \\
        \implies \\
        \mu = E(X_t) \in \R \text{ and } \overline{x} \to \mu \text{ a.s.} 
    \end{gather*}
    
\end{theorem}
\section{2/26}
\begin{definition}[Fockker Plank]
    \begin{gather*}
        \frac{d p(t,x)}{dt} = \frac{d}{dx} (\mu(t,x)p(t,x)) + \frac{d^2}{dx^2} (\sigma(t,x)p(t,x))\\  
    \end{gather*}
    where $p(t,x)$ is the probability density function of the process $X_t$\\
\end{definition}
\begin{definition}[$B_{f(t)}$]
    $d B_{f(t)} = \sqrt{f'(t)} dB_t$\\
    \begin{gather*}
        X_t = \frac{B_{e^t}}{\sqrt{e^t}} \\
        dX_t = \frac{dB_{e^t}}{e^{t/2}} + \frac{B_{e^t}}de^{-t/2} + \lbrace 
    \end{gather*}
\end{definition}
Look through example of Random walk +1,-1 and the linear past of it. 


\subsection{Reading 2/26-3/5}

\begin{definition}[A1-A4]
    \begin{align}
        \forall t \in \Z: \mu_t = \E(X_t) \in \R\\
        \forall t \in \Z: \sigma^2_t = \var(X_t) < \infty\\
        \exists \mu \in \R \suchthat \lim_{t \to \infty} \mu_t = \mu\\
        \lim_{t \to \infty} \text{cov}(\bar{x_n}, X_n)= 0
    \end{align}
    Weak stationarity implies A1-A3\\
    \begin{align*}
        \lim_{k \to \infty} \gamma_X(k) = 0 \\
        \gamma_{X}(k) \text{ Cesaro summable with limit 0}\\
        \lim_{k \to \infty} \cov(\bar{x_n} , X_n) = \lim_{n \to \infty} \frac{1}{n} \sum_{k=1}^{n-1} \gamma_X(k) = 0
        A4
    \end{align*}
    This is an implication from top to bottom for A4
\end{definition}
\begin{definition}[Backshift/ Lag Operator]
    \begin{align*}
        B: \R^{\Z} \to \R^{\Z}\\
        x = \{x_t\}_{t \in \Z} \to y = Bx \\
        \text{with } y_t = x_{t-1} \\
    \end{align*}
    This shifts the sequence to the left by 1.
\end{definition}
\begin{definition}[Linear Process]
    $X_t \in \R (t \in \Z), \mathbb{E} (X_t) = \mu \in \R$ is a linear cprocess in $L^2(\Omega)$ if 
    $$ \exists \epsilon_t \in \R (t \in \Z) \text{ iid } \mathbb{E}(\epsilon_t) = 0, \sigma^2_{\epsilon} = \text{var}(\epsilon_t) < \infty$$
    and 
    $$ \exists a_j \in \R (j \in \Z), \sum_{j=-\infty}^{\infty} a_j^2 < \infty$$
    s.t.
    $$ \forall t \in \Z: \lim_{N \to \infty} || X_t - X_{t,N}||^2_{L^2(\Omega)} = 0$$
    We then write 
    $$ X_t = \mu + \sum_{j=-\infty}^{\infty} a_j \epsilon_{t-j} (t \in \Z) = \mu + \left( \sum_{j = -\infty}^\infty a_j B^j \right) \epsilon_t$$
    If $a_j = 0$ for $j < 0$ then we call it a causal linear process.\\
    By definition if $X_t$ is a linear process then it is weakly and strictly stationary.\\
\end{definition}
\textbf{NOTE: I am skipping most of the rest of the specific assumtions in favor of Reading more Probability Measures for Time Series}


\section{upto 3/25}

\begin{theorem}[Kolmogotow consistency Theorem]
    *** WHAT IS THIS\\
    From what I understand it is a existence theorem of the Probability measures for time series.\\
\end{theorem}
\begin{definition}[Transformations of Time Series]
    Some transformations of time series are
   \begin{itemize}
    \item Linear processes as a transformation of iid $\epsilon_t$
    \item Random walk as a transformation with $X_t = X_t + \epsilon_t$
    \item Transformations in the spectral domain: $X_t = \sum_{j=-\infty}^{\infty} e^{it\lambda} dZ_X(\lambda)$
   \end{itemize}
\end{definition}
\begin{definition}[Section 3.3]
    Very Confusing,as to why we care about conditions for the expected value of the time series.\\
\end{definition}

\textbf{Moment Generating function}
\begin{definition}[ARMA and GF]
    MA is the $1 - \sum_{j=1}^{q} \theta_j B^j$\\
    AR is the $\frac{1}{1 - \sum_{j=1}^{p} \phi_j B^j}$\\
    The arma is combo, being a rational function of B.\\
\end{definition}
\begin{definition}[Generating Function]
    
\end{definition}

% http://en.wikipedia.org/wiki/Kosambi%E2%80%93Karhunen%E2%80%93Lo%C3%A8ve_theorem

Kosambi–Karhunen–Loève_theorem

Repreoducing the kernel 


\section{_}
Do the computaton for KL decomp for a brownian motion. 
Write ARMA as generating function

PCA <-> Mercer's Theorem


\end{document}
\documentclass[answers,12pt,addpoints]{exam}
\usepackage{import}

\import{C:/Users/prana/OneDrive/Desktop/MathNotes}{style.tex}

% Header
\newcommand{\name}{Pranav Tikkawar}
\newcommand{\course}{01:XXX:XXX}
\newcommand{\assignment}{Homework n}
\author{\name}
\title{\course \ - \assignment}

\begin{document}
\maketitle


\newpage

\section{Fair Pricing of a random payment}
\textbf{Theme} Exchange a fixed payment for a random payment.\\
Q: What criterion can we apply to justify fairness in the exchange?\\
Point: There are more than one criteria. Under which condition the criterion apply, and which do not?\\
1) Suppose $X$ is a random payment, then $E(X)$ is the fair price of $X$.\\
Justified by the LLN.\\
So the criteria must be that the payment is repeated many times and independent.\\
\textbf{Call Option}
on a fixed day the holder can choose to buy a share of stock for fixed price $K$.\\
Let $S$ be the price of the stock on that day and our profit is $X = (S-K)_+$.\\
Clealry this cannot follow LLN because we only have one day.\\
\textbf{Pricing by Risk Prefrence}: Utility function: $U:$ cash $\to$ happiness \\
\textbf{Indifference pricing}: $\phi$ is our utility function, then our indeffrence price is $\phi^{-1}(\mathbb{E}(\phi(X)))$\\
\textbf{Jensen's Inequality}: $\phi$ convex $\implies \mathbb{E}(\phi(X)) \geq \phi(\mathbb{E}(X))$\\
\textbf{Risk Aversion}: $\phi$ concave $\implies \mathbb{E}(\phi(X)) \leq \phi(\mathbb{E}(X))$\\

\end{document}
\documentclass{article}
\usepackage{amsmath}
\usepackage{amsfonts}
\usepackage{amssymb}
\usepackage{cancel}

\usepackage{graphicx}


\setlength\parindent{0pt}

\author{Pranav Tikkawar}
\title{TODO}

\begin{document}
\maketitle
\section*{Question 1}
Consider $\begin{bmatrix}
    x \\
    y
\end{bmatrix}'(t) = v(x,y,t), x(0)=x_0, y(0) = y_0$ \\
Apply Picard iteration scheme starting with inital function the constant function $X(t) = \begin{bmatrix}
    x_0 \\
    y_0
\end{bmatrix}$ Then compare the examples
\subsection*{a}
$v(x,y,t) = \begin{bmatrix}
    t\\
    x^2
\end{bmatrix}$\\
The Picard iteration scheme is given 
$$x_0(t) = x_0$$
$$x_1(t) = x_0 + \int_0^t \begin{bmatrix}
    s\\
    x_0^2
\end{bmatrix} ds = \begin{bmatrix}
    x_0 + \frac{t^2}{2} \\
    y_0 + x_0^2t
\end{bmatrix}$$
$$x_2(t) = x_0 + \int_0^t \begin{bmatrix}
    s \\
    (x_0 + \frac{s^2}{2})^2
\end{bmatrix} ds = \begin{bmatrix}
    x_0 + \frac{t^2}{2} \\
    y_0 + x_0^2t + x_0\frac{t^3}{3} + \frac{x^5}{20}
\end{bmatrix}$$
$$x_3(t) = x_0 + \int_0^t \begin{bmatrix}
    s \\
    (x_0 + \frac{s^2}{2})^2
\end{bmatrix} ds = \begin{bmatrix}
    x_0 + \frac{t^2}{2} \\
    y_0 + x_0^2t + x_0\frac{t^3}{3} + \frac{x^5}{20}
\end{bmatrix}$$
We can see from here that for all $X_n$ for $n \geq 2$ they will be eqaul meaning that $$X(t) = \begin{bmatrix}
    x_0 + \frac{t^2}{2} \\
    y_0 + x_0^2t + x_0\frac{t^3}{3} + \frac{x^5}{20}
\end{bmatrix}$$
\subsection*{b}
$v(x,y,t) = \begin{bmatrix}
    -y\\
    x
\end{bmatrix}$ with specific intial condition $x(0) = 1, y(0) = 0$
The Picard iteration scheme is given
$$x_0(t) = \begin{bmatrix}
    x_0 \\
    y_0
\end{bmatrix}$$
$$x_1(t) = \begin{bmatrix}
    x_0 \\
    y_0
\end{bmatrix} + \int_0^t \begin{bmatrix}
    -y_0 \\
    x_0
\end{bmatrix} ds = \begin{bmatrix}
    x_0 - y_0t \\
    y_0 + x_0t
\end{bmatrix}$$
$$x_2(t) = \begin{bmatrix}
    x_0 \\
    y_0
\end{bmatrix} + \int_0^t \begin{bmatrix}
    -y_0 - x_0t \\
    x_0 - y_0t
\end{bmatrix} ds = \begin{bmatrix}
    x_0 - y_0t - x_0\frac{t^2}{2} \\
    y_0 + x_0t - y_0\frac{t^2}{2}
\end{bmatrix}$$
$$x_3(t) = \begin{bmatrix}
    x_0 \\
    y_0
\end{bmatrix} + \int_0^t \begin{bmatrix}
    -y_0 - x_0t + y_0\frac{t^2}{2} \\
    x_0 - y_0t - x_0\frac{t^2}{2}
\end{bmatrix} ds = \begin{bmatrix}
    x_0 - y_0t - x_0\frac{t^2}{2} + y_0\frac{t^3}{6} \\
    y_0 + x_0t - y_0\frac{t^2}{2} - x_0\frac{t^3}{6}
\end{bmatrix}$$
After plugging in for $x_0 = 1, y_0 = 0$ we get
$$x_3(t) = \begin{bmatrix}
    1 - \frac{t^2}{2} \\
    t - \frac{t^3}{6}
\end{bmatrix}$$
As we continue to iterate we will notice a common taylor series expansion of $sin(t)$ and $cos(t)$ \\
Thus:
$$X(t) = \begin{bmatrix}
    cos(t) \\
    sin(t)
\end{bmatrix}$$
\section*{Question 2}
A function that obeys the properties of a contraction maps must obey 2 properties.\\
The function decreases by a constant factor $0 < k < 1$ for all $x,y \in \mathbb{R}^n$ and $f(x) - f(y) \leq k(x-y)$\\
This means that a function in the form of $f(x) = kx$ will satisfy this property.\\
For example $f(x) = .5x$ will satisfy this property as at each natural number $n$ the function will decrease by a factor of .5.\\
We can continue this to that it will go to 0 as $n \to \infty$\\
Thus the function $f(x) = .5x$ is a contraction map.
\end{document}
\documentclass{article}
\usepackage{amsmath}
\usepackage{amsfonts}
\usepackage{amssymb}
\usepackage{mathrsfs}
\usepackage{cancel}

\usepackage{graphicx}


\setlength\parindent{0pt}

\author{Pranav Tikkawar}
\title{Workshop 11}

\begin{document}
\maketitle
\section{Question 1}
Suppose $p(t), q(t)$ are continuous functions on $(a,b)$ Consider $x'' + px' + qx = 0$\\
Given any two differentiable functions $x_1(t), x_2(t)$, define the function $W(t) = x_1(t)x_2'(t) - x_1'(t)x_2(t)$ Called the Wronskian \\
Assumimg these are two solutions of the DE for $t$ in $(a,b)$ show that $W'(t) = -p(t)W(t)$. Use this to show $W(t) = 0 or W(t) \neq 0$ for all $t$ in $(a,b)$\\
\textbf{Solution:}\\
$$W(t) = x_1(t)x_2'(t) - x_1'(t)x_2(t)$$
$$W'(t) = x_1' x_2' + x_1 x_2'' - x_1'x_2' - x_1''x_2$$
$$W'(t) = x_1 x_2'' - x_1''x_2$$
$$W'(t) = x_1(- px_2' - qx_2) - x_2(-px_1' - qx_1)$$
$$W'(t) = -px_1x_2' - qx_1x_2 + px_1x_2' + qx_1x_2$$
$$W'(t) = -p(x_1x_2' - x_1'x_2)$$
$$W'(t) = -pW(t)$$
Now, we know that $W'(t) = -pW(t)$ we can "solve" for $W(t)$ by integrating both sides. \\
$$\int \frac{dW(t)}{W(t)} = \int -p(t)dt$$
$$\ln|W(t)| = -\int p(t)dt + C$$
$$W(t) = e^{-\int p(t)dt}C$$
We see from prior experience that this solution is existant and unique for all time $t$ in $(a,b)$\\ Thus, $W(t) = 0$(an equilibrium solution) or $W(t) \neq 0$ for all $t$ in $(a,b)$

\section{Question 2}
Consider the DE $x'' + px' + qx = r$ where $p(t), q(t), r(t)$ are continuous functions on $(a,b)$\\
Let $x_1(t), x_2(t)$ be two solutions of the DE for $t$ in $(a,b)$ such that $W(t) = \neq 0$ for all $t$ in $(a,b)$\\
Let $Y(t) = c_1(t)x_1(t) + c_2(t)x_2(t)$\\
Assume $c_1'x_1 + c_2'x_2 = 0$. Show $c_1'x_1' + c_2'x_2' = r(t)$\\
Conclude that we get a formula for $c_1'(t), c_2'(t)$\\
\textbf{Solution:}\\
$$Y(t) = c_1(t)x_1(t) + c_2(t)x_2(t)$$
$$Y'(t) = c_1x_1' + c_2x_2'$$
$$Y''(t) = c_1'x_1' + c_1x_1'' + c_2'x_2' + c_2x_2'' $$
$$Y''(t) + pY'(t) + qY(t) = r$$
$$c_1'x_1' + c_1x_1'' + c_2'x_2' + c_2x_2'' + pc_1x_1' + pc_2x_2' + qc_1x_1 + qc_2x_2 = r$$
$$c_1'x_1' + c_2'x_2' + c_1(x_1'' + px_1' + qx_1) + c_2(x_2'' + px_2' + qx_2) = r$$
$$c_1'x_1' + c_2'x_2' = r$$ 
As desired.\\
Now, we can solve for $c_1'(t), c_2'(t)$ by considering the matrix representaion of the above equations. \\
$$\begin{bmatrix}
x_1 & x_2\\
x_1' & x_2'
\end{bmatrix} \begin{bmatrix}
c_1'\\
c_2'
\end{bmatrix} = \begin{bmatrix}
0\\
r
\end{bmatrix}$$
$$\begin{bmatrix}
c_1'\\
c_2'
\end{bmatrix} = \begin{bmatrix}
    x_2' & -x_2\\
    -x_1' & x_1
\end{bmatrix} \begin{bmatrix}
0\\
r
\end{bmatrix} \frac{1}{W(t)}$$
$$\begin{bmatrix}
c_1'\\
c_2'
\end{bmatrix} = \begin{bmatrix}
    -x_2r\\
    x_1r
\end{bmatrix} \frac{1}{W(t)}$$
As desired.\\

\section{Question 3}
Set up the ODE as $\begin{bmatrix}
    x \\
    y
\end{bmatrix}' = \begin{bmatrix}
    0 & 1\\
    -q & -p
\end{bmatrix} \begin{bmatrix}
    x\\
    y
\end{bmatrix} + \begin{bmatrix}
    0\\
    r
\end{bmatrix}$\\
Notice $\begin{bmatrix}
    x_1\\
    x_1'
\end{bmatrix}$ and $\begin{bmatrix}
    x_2\\
    x_2'
\end{bmatrix}$ are two LI solution of the homogenous part. Do the integral in the Duhamels fomular and extract first component. Compare to the Previous Question.\\
\textbf{Solution:}\\
From Duhamels formula in terms of flow we get:\\
$$X(t) = \Phi_{t,0}X_0 + \int_0^t \Phi_{t,s}r(s)ds$$
$$X(t) = \Phi_{t,0}X_0 + M(t)\int_0^t M(s)^{-1}\begin{bmatrix}
    0\\
    r(s)
\end{bmatrix}ds $$
$$X(t) = \Phi_{t,0}X_0 + \begin{bmatrix}
    x_1(t) & x_2(t)\\
    x_1'(t) & x_2'(t)
\end{bmatrix} \int_0^t \begin{bmatrix}
    x_2(s) & -x_2(s)\\
    -x_1(s) & x_1(s)
\end{bmatrix} \begin{bmatrix}
    0\\
    r(s)
\end{bmatrix}ds$$
$$X(t) = \Phi_{t,0}X_0 + \begin{bmatrix}
    x_1(t) & x_2(t)\\
    x_1'(t) & x_2'(t)
\end{bmatrix} \int_0^t \begin{bmatrix}
    c_1'(s)\\
    c_2'(s)
\end{bmatrix}ds$$
$$X(t) = \Phi_{t,0}X_0 + \begin{bmatrix}
    x_1(t) & x_2(t)\\
    x_1'(t) & x_2'(t)
\end{bmatrix} \begin{bmatrix}
    c_1(t)\\
    c_2(t)
\end{bmatrix}$$
$$X(t) = \Phi_{t,0}X_0 + \begin{bmatrix}
    c_1(t)x_1(t) + c_2(t)x_2(t)\\
    c_1(t)x_1'(t) + c_2(t)x_2'(t)
\end{bmatrix}$$
$$X(t) = \Phi_{t,0}X_0 + Y(t)$$



\end{document}
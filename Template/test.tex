\documentclass{article}
\usepackage{amsmath}
\usepackage{amsfonts}
\usepackage{amssymb}
\usepackage{mathrsfs}
\usepackage{cancel}

\usepackage{graphicx}


\setlength\parindent{0pt}

\author{Pranav Tikkawar}
\title{TODO}

\begin{document}
\maketitle

The complete characterization of the pairs of exponents $(p,q)$ for which the \emph{Fourier extension estimate}



holds for a certain dimension $d$  is one of the big open questions in Harmonic Analysis. 

Here $\mathcal{M}$ is a hypersurface in $\mathbb{R}^d$, $\sigma$ is a surface measure carried by $\mathcal{M}$, and $\widehat{f\sigma}$ is the Fourier transform of the measure $f\sigma$.  Such extension estimate admits an equivalent dual formulation, a so-called \emph{Fourier restriction estimate}, which holds with the very same constant.

The curvature of the hypersurface $\mathcal{M}$ plays a key role in the theory of Fourier restriction. This is due to the decay properties of the Fourier transform of measures which are supported on surfaces that possess some degree of curvature. 


We will review these facts and, if time allows, we will recall some classical results in the area such as the Stein-Tomas theorem.


We will then focus on the subarea of \emph{sharp} Fourier restriction (equivalently, extension) theory.

For a triple $(p,q,d)$ for which inequality \eqref{inequ_extension} holds we will consider questions like: What is the value of the optimal constant? 

If maximizers (namely, functions that attain the optimal constant) exist, what are they? 

We will focus on the case of spheres. 

A major result in the area of sharp spherical Fourier restriction is due to D. Foschi who showed that constant functions are maximizers for the $L^2(\mathbb{S}^{2})$ to $L^4(\mathbb{R}^3)$ Fourier extension estimate. 


We will review this result of Foschi and we will conclude by discussing several open problems in the area of sharp Fourier restriction on spheres.



\end{document}
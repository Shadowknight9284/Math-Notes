\documentclass[answers,12pt,addpoints]{exam}
\usepackage{import}

\import{C:/Users/prana/OneDrive/Desktop/MathNotes}{style.tex}

% Header
\newcommand{\name}{Pranav Tikkawar}
\newcommand{\course}{Intro to Quantum Computating}
\newcommand{\assignment}{Homework 1}
\author{\name}
\title{\course \ - \assignment}

\begin{document}
\maketitle
\begin{questions}
    \question In the class, we showed that the eigenvectors of the Pauli Z operator are given by \(|0\rangle\), \(|1\rangle\) with eigenvalues 1, \(-1\) respectively:
    \[ Z|0\rangle = |0\rangle, \quad Z|1\rangle = -|1\rangle, \]
    where
    \[ |0\rangle = \begin{pmatrix} 1 \\ 0 \end{pmatrix}, \quad |1\rangle = \begin{pmatrix} 0 \\ 1 \end{pmatrix}. \]
    Find the eigenvectors of the Pauli X operator and the corresponding eigenvalues. Find the \(2 \times 2\) matrix that relates the eigenvectors of the X operator to the eigenvectors of the Z operator.
    \begin{solution}
        We can do this by solving the eigenvalue equation
        \[ X|v\rangle = \lambda |v\rangle, \]
        where \(X\) is the Pauli X operator given by
        \[ X = \begin{pmatrix} 0 & 1 \\ 1 & 0 \end{pmatrix}. \]
        The eigenvalues of the Pauli X operator are \(\lambda = 1\) and \(\lambda = -1\). The eigenvectors corresponding to these eigenvalues are
        \[ |+\rangle = \frac{1}{\sqrt{2}} \begin{pmatrix} 1 \\ 1 \end{pmatrix}, \quad |-\rangle = \frac{1}{\sqrt{2}} \begin{pmatrix} 1 \\ -1 \end{pmatrix}. \]
        The 2x2 matrix that relates the eigenvectors of the X operator to the eigenvectors of the Z operator is given by
        \[ U = \frac{1}{\sqrt{2}}\begin{pmatrix} 1 & 1 \\ 1 & -1 \end{pmatrix}. \]
    \end{solution}
    \question  Recall that the commutator of two operators is given by \([A, B] = AB - BA\). Show that (below the symbol \(\dagger\) indicates Hermitian conjugation)
    \begin{parts}
        \part \([A, B]^\dagger = [B^\dagger, A^\dagger]\)
        \part \([A, B] = -[B, A]\)
        \part If \(A, B\) are Hermitian, then \(i[A, B]\) is also Hermitian
    \end{parts}
    \begin{solution}
        \textbf{a} 
        \begin{align*}
            [A, B]^\dagger &= (AB - BA)^\dagger \\
            &= (AB)^\dagger - (BA)^\dagger \\
            &= B^\dagger A^\dagger - A^\dagger B^\dagger \\
            &= [B^\dagger, A^\dagger]
        \end{align*}
        \textbf{b}
        \begin{align*}
            [A, B] &= AB - BA \\
            &= -BA + AB \\
            &= -[B, A]
        \end{align*}
        \textbf{c}
        Suppose \(A, B\) are Hermitian, then $A = A^\dagger$ and \(B = B^\dagger\). Then we need that $i[A, B] = i[A, B]^\dagger$. We have
        \begin{align*}
            (i[A, B])^\dagger &= -i[B^\dagger, A^\dagger] \\
            &= i[A^\dagger, B^\dagger] \\
            &= i[A, B] \\
            &= i[A, B]
        \end{align*}
        Thus \(i[A, B]\) is Hermitian.
    \end{solution}
    \question An operator \(A\) is called anti-Hermitian if \(A^\dagger = -A\).
    \begin{parts}
        \part Show that eigenvalues of anti-Hermitian operators are purely imaginary.
        \part Show that expectation values of anti-Hermitian operators are purely imaginary for any given state \(|\psi\rangle\).
    \end{parts}
    \begin{solution}
        \textbf{a} 
        Suppose \(A\) is an anti-Hermitian operator with eigenvalue $a$ and eigenvector \(|a \rangle\). Then we have
        \begin{align*}
            A|a\rangle &= a |a\rangle \\
            A^\dagger |a\rangle &= a^* |a\rangle \\
            -A |a\rangle &= a^* |a\rangle \\
            -a |a\rangle &= a^* |a\rangle \\
            -a &= a^*
        \end{align*}
        Thus \(a\) is purely imaginary.
        \textbf{b}
        Suppose \(A\) is an anti-Hermitian operator and \(|\psi\rangle\) is any state. We want to show that \(\langle \psi | A | \psi \rangle^* = -\langle \psi | A | \psi \rangle\). We have
        \begin{align*}
            \langle \psi | A | \psi \rangle^* &= \langle \psi | A^\dagger | \psi \rangle \\
            &= -\langle \psi | A | \psi \rangle
        \end{align*} 
        Thus \(\langle \psi | A | \psi \rangle\) is purely imaginary.
    \end{solution}
\end{questions}

\end{document}
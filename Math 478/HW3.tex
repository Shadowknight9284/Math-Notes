\documentclass[answers,12pt,addpoints]{exam}
\usepackage{import}

\import{C:/Users/prana/OneDrive/Desktop/MathNotes}{style.tex}

% Header
\newcommand{\name}{Pranav Tikkawar}
\newcommand{\course}{01:640:478}
\newcommand{\assignment}{Homework 3}
\author{\name}
\title{\course \ - \assignment}

\begin{document}
\maketitle


\newpage
\begin{questions}
    \question Question 1.\\
    In the class we showed that, if \( S \) and \( T \) are independent exponential random variables, having rates \( \lambda \) and \( \mu \), then \(\min\{S, T\} \approx \text{exponential}(\lambda + \mu)\), and \( P(S < T) = \frac{\lambda}{\lambda + \mu} \). Extend these results to show that, if \( T_1, \ldots, T_n \) are independent exponential \((\lambda_i)\) distributed random variables, then \(\min\{T_1, \ldots, T_n\} \sim \text{exponential} (\lambda_1 + \cdots + \lambda_n)\), and \( P(T_i = \min (T_1, \ldots, T_n)) = \frac{\lambda_i}{\lambda_1 + \cdots + \lambda_n} \).
    \begin{solution}
        To show that \(\min\{T_1, \ldots, T_n\} \sim \text{exponential} (\lambda_1 + \cdots + \lambda_n)\), we can use the memoryless property of the exponential distribution. We can see for the 2 element case that 
        \begin{align*}
            P(\min\{T_1, T_2\} > t) &= P(T_1 > t, T_2 > t)\\
            &= P(T_1 > t)P(T_2 > t) \text{ (independence)}\\
            &= e^{-\lambda_1 t}e^{-\lambda_2 t}\\
            &= e^{-(\lambda_1 + \lambda_2)t}
        \end{align*}
        And clealry the only way that this is possible is if the minimum of the two is exponential with rate \(\lambda_1 + \lambda_2\). We can extend this to the \(n\) element case. \\
        \begin{align*}
            P(\min\{T_1, \ldots, T_n\} > t) &= P(T_1 > t, \ldots, T_n > t)\\
            &= P(T_1 > t) \cdots P(T_n > t) \text{ (independence)}\\
            &= e^{-\lambda_1 t} \cdots e^{-\lambda_n t}\\
            &= e^{-(\lambda_1 + \cdots + \lambda_n)t}
        \end{align*}
        Clearly this is the CDF of an exponential distribution with rate \(\lambda_1 + \cdots + \lambda_n\).\\
        Thus we have shown that \(\min\{T_1, \ldots, T_n\} \sim \text{exponential} (\lambda_1 + \cdots + \lambda_n)\).\\\\
        Now to show that \( P(T_i = \min (T_1, \ldots, T_n)) = \frac{\lambda_i}{\lambda_1 + \cdots + \lambda_n} \), we can use the same logic as above. We can see that for the two element case that the probability that \(S < T\) is the proportion of the rate of \(S\) to the sum of the rates of \(S\) and \(T\). We can extend this to the \(n\) element case.\\
        \begin{align*}
            P(T_i = \min (T_1, \ldots, T_n)) &= P(T_i < T_1, \ldots, T_i < T_{i-1}, T_i < T_{i+1}, \ldots, T_i < T_n)\\
            &= P(T_i < T_1) \cdots P(T_i < T_{i-1})P(T_i < T_{i+1}) \cdots P(T_i < T_n)\\
            &= \frac{\lambda_i}{\lambda_1 + \cdots + \lambda_n}
        \end{align*}
    \end{solution}
    \newpage

    \question Question 2.\\
    A spacecraft can keep traveling if at least two of its three engines are working. Suppose that the failure times of the engines are exponential with means 1 year, 1.5 years, and 3 years. What is the average length of time the spacecraft can travel? Hint: Use the results stated in problem 1.
    \begin{solution}
        We can let \(T_1, T_2, T_3\) be the failure times of the engines. Each $T_i$ is exponentially distributed with \(\lambda = 1, 2/3, 1/3\) respectively.
        By the memoryless property of the exponential distribution, the time until the first engine fails is exponentially distributed with rate \(\lambda_1 + \lambda_2 + \lambda_3 = 2\). Thus the mean time of the first engine failing is 0.5 years.\\ 
        Now for the average time for the second failure of the engine would be each of the remaining engines failing in the following cases: \\\\
        If the first engine fails at time \(t\) it will happen with probability \(\frac{1}{2}\). Then the remaining two engines have a failure rate of \(\lambda_2 + \lambda_3 = 1\). Thus the mean time for the second engine to fail is \(1\) years.\\
        If the second engine fails at time \(t\), it will happen with probability \(\frac{1}{3}\). Then the remaining two engines have a failure rate of \(\lambda_1 + \lambda_3 = 4/3\). Thus the mean time for the second engine to fail is \(3/4\) years.\\
        If the third engine fails at time \(t\), it will happen with probability \(\frac{1}{6}\) Then the remaining two engines have a failure rate of \(\lambda_1 + \lambda_2  = 5/3 \) Thus the mean time for the second engine to fail is \(3/5\) years.\\
        Now we can calculate the average time the spacecraft can travel by adding the mean times of each of the engines failing.
        \begin{align*}
            \text{Average time spacecraft can travel} &= 0.5 + \frac{1}{2} \cdot 1 + \frac{1}{3} \cdot \frac{3}{4} + \frac{1}{6} \cdot \frac{3}{5}\\
            &= 0.5 + 0.5 + 0.25 + 0.10\\
            &= 1.35 \text{ years}
        \end{align*}
        Thus the average time the spacecraft can travel is 1.35 years.
    \end{solution}

    \question Question 3.\\
    In good years, storms occur according to a Poisson process with rate 3 per unit time, while in other years they occur according to a Poisson process with rate 5 per unit time. Suppose next year will be a good year with probability 0.3. Let \( N(t) \) denote the number of storms during the first \( t \) time units of next year.
    \begin{parts}
        \part Find \(P\{N(t) = n\}\)
        \part Is \(N(t)\) a Poisson process? 
        \part Does \(N(t)\) have stationary increments? Why or why not?
        \part Does \(N(t)\) have independent increments? Why or why not?
        \part If next year starts off with 3 storms by time t=1 what is the conditional probabilty next year will be a good year?
    \end{parts}
\begin{solution}
    \begin{parts}
        \part We can use law of total probability to find \(P\{N(t) = n\}\). We can see that
        \begin{align*}
            P\{N(t) = n\} &= P\{N(t) = n | \text{good year}\}P\{\text{good year}\} + P\{N(t) = n | \text{bad year}\}P\{\text{bad year}\}\\
            &= \frac{(3t)^n}{n!}0.3 + e^{-5t}\frac{(5t)^n}{n!}0.7
        \end{align*}
        \part \(N(t)\) is not a Poisson process because the rate of the process changes with time since there are good days and bad days 
        \part \(N(t)\) 
    \end{parts}
\end{solution}





\end{questions}

\end{document}
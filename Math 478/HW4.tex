\documentclass[answers,12pt,addpoints]{exam}
\usepackage{import}

\import{C:/Users/prana/OneDrive/Desktop/MathNotes}{style.tex}

% Header
\newcommand{\name}{Pranav Tikkawar}
\newcommand{\course}{01:640:478}
\newcommand{\assignment}{Homework 4}
\author{\name}
\title{\course \ - \assignment}

\begin{document}
\maketitle


\newpage
\begin{questions}
    \question 1
    Let \(\{N_1(t), t \geq 0\}\) and \(\{N_2(t), t \geq 0\}\) be independent renewal processes. Let \(N(t) = N_1(t) + N_2(t)\)
    \begin{parts}
        \part Are the interarrival times of \(\{N(t), t \geq 0\}\) independent?
        \part Are they identically distributed?
        \part Is \(\{N(t), t \geq 0\}\) a renewal process?
    \end{parts}
    \begin{solution}
        (i) No, the interarrival times of \(\{N(t), t \geq 0\}\) are not independent.\\
        This is due to the fact that the interarrival times of \(\{N_1(t), t \geq 0\}\) and \(\{N_2(t), t \geq 0\}\) are independent, but the sum of two independent random variables is not independent but is given by the min of the two random variables.\\

        (ii) No, the interarrival times of \(\{N(t), t \geq 0\}\) are not identically distributed.\\
        This is due to the fact that the interarrival times of \(\{N_1(t), t \geq 0\}\) and \(\{N_2(t), t \geq 0\}\) are independent, but the sum of two independent random variables is not identically distributed but is given by the sum of the two random variables.\\

        (iii) No, since the interarrival times of \(\{N(t), t \geq 0\}\) are not independent nor identically distributed, \(\{N(t), t \geq 0\}\) is not a renewal process.\\

    \end{solution}

    \question 2
    A worker sequentially works on jobs. Each time a job is completed, a new one is begun. Each
    job, independently, takes a random amount of time having distribution \(F\) to complete. However,
    independently of this, shocks occur according to a Poisson process with rate \(\lambda\). Whenever a shock
    occurs, the worker discontinues working on the present job and starts a new one. In the long run,
    at what rate are jobs completed?\\
    Hint: Let \(T\) be the time it takes to complete a job. Let \(W\) be the time it would take to complete
    the first job attempted. Let \(S\) be the time of the first shock. To compute \(\mathbb{E}[T]\), develop an
    equation for it, by conditioning on the possible outcomes of \(W\); i.e., compute \(\mathbb{E}[T]\) by computing
    \(\mathbb{E}[\mathbb{E}[T \mid W]]\). To compute \(\mathbb{E}[T \mid W = w]\), compute \(\mathbb{E}[T \mid W = w, S = x]\), multiply by the density
    \(f_S(x)\) and integrate over \(x\).
    \begin{solution}
        We can take $T$ to be the time it takes to complete a job.\\
        We can take $W$ to be the time it would take to complete the first job attempted.\\
        We can take $S$ to be the time of the first shock.\\
        To compute $E[T]$, we can develop an equation for it, by conditioning on the possible outcomes of $W$; i.e., compute $E[T]$ by computing $E[E[T | W]]$.\\
        To compute $E[T | W = w]$, we can compute $E[T | W = w, S = x]$, multiply by the density $f_S(x)$ and integrate over $x$.\\
        We can see that $E[T | W = w] = E[T | W = w, S < w]P(S < w) + E[T | W = w, S > w]P(S > w)$\\
        \begin{align*}
            P(S < w) &= 1 - e^{-\lambda w}\\
            P(S \geq w) &= e^{-\lambda w}\\
            E[T | W = w, S \geq w] &= w \quad \text{Job comlpleted before shock}\\
            E[T | W = w, S < w] &= w + E[T] \quad \text{Job not comlpleted before shock an restart}
        \end{align*}
        Thus we have that
        \begin{align*}
            E[T | W = w] &= (w + E[T])(1 - e^{-\lambda w}) + w e^{-\lambda w}\\
            &= w + E[T] - E[T] e^{-\lambda w} 
        \end{align*}
        Thus we have that
        \begin{align*}
            E[T] &= E[E[T | W]]\\
            &= E[w + E[T] - E[T] e^{-\lambda w}]\\
            &= E[w] + E[E[T]] - E[E[T] E[e^{-\lambda w}]]\\
            &= E[w] + E[T] - E[T] E[e^{-\lambda w}]\\
            E[T] &= \frac{E[W]}{E[e^{-\lambda w}]}
        \end{align*}
        Thus be solving for $1/ E[T]$, which is the rate at which jobs are completed, we have that
        $$ \text{Rate at which jobs are completed} = \frac{1}{E[T]} = \frac{E[e^{-\lambda W}]}{E[W]}$$

        

    
    \end{solution}

    \question 3
    Machines in a factory break down at an exponential rate of six per hour. There is a single repairman
    who fixes machines at an exponential rate of eight per hour. The cost incurred in lost production
    when machines are out of service is \$10 per hour per machine. What is the average cost rate
    incurred due to failed machines?\\
    Hint: Model this as an M/M/1 queue.
    \begin{solution}
        We know that the rate at which machines break down is $\lambda = 6$ and the rate at which the repairman fixes machines is $\mu = 8$.\\
        The cost incurred in lost production when machines are out of service is \$10 per hour per machine.\\
        We can model this as an M/M/1 queue.\\
        The average cost rate incurred due to failed machines is given by:
        \begin{align*}
            \text{Average cost rate} &= \text{Cost per hour per machine} \times \text{Average number of machines in the system}\\
            &= 10 \times \text{Average number of machines in the system}
        \end{align*}
        We know that the average number of machines in the system is given by:
        \begin{align*}
            L = \frac{\lambda}{\mu - \lambda}
        \end{align*}
        Thus the average cost rate incurred due to failed machines is given by:
        \begin{align*}
            \text{Average cost rate} &= 10 \times \frac{6}{8 - 6}\\
            &= 10 \times 3\\
            &= 30
        \end{align*}
        Thus the average cost rate incurred due to failed machines is \$30 per hour.
    \end{solution}

    \question 4
    For an M/M/1 queue with capacity \(N < \infty\), show that the average number of customers being
    served (under steady state conditions) is approximately \(\rho\) when \(\rho\) is very small, and that it is
    approximately \(1 - \rho^{-N}\) when \(\rho\) is very large.
    Hint: See what we did in class on analysis of the finite capacity queues
    \begin{solution}
        We know that the average number of customers being served is given by:
        \begin{align*}
            L = \frac{\rho(1-(N+1)\rho^N + N\rho^{N+1})}{(1-\rho)1-\rho^{N+1}}
        \end{align*}
        For very small $\rho$, we can see that $\rho^N$ is very small and thus we can ignore the term as well as $1-\rho \approx 1$
        \begin{align*}
            L &\approx \frac{\rho}{1}\\
            &= \rho
        \end{align*}
        For very large $\rho$, we can consider the probability of having n customers in the system to be
        $$ P_n = \frac{(1-\rho)\rho^n}{1-\rho^{N+1}}$$
       The average number of customers beign served equicalent to 1 minus the probability of having 0 customers in the system.
       $$1 - P_0 = 1 - \frac{(1-\rho)\rho^0}{1-\rho^{N+1}} = 1 - \frac{1-\rho}{1-\rho^{N+1}} = \frac{\rho - \rho^{N+1}}{1-\rho^{N+1}} =  1 - \rho^{-N}$$
    \end{solution}

    \question 
    A bank plans to install an ATM. From past experience they know that a user spends 3 minutes on
average doing a transaction. They also want the average number of users at the facility (in the line
and at the machine) at a given time to be 3. Assume Poisson arrivals, exponential service times,
and steady state conditions.
1
(i) What will be the maximum average number of users per hour the ATM will serve?
(ii) What will be the average queue time for a customer?
(iii) If the average number of users double, i.e. L = 6, how would the answers to the previous
parts change?
Hint: This description fits the model of a M/M/1 queue with infinite capacity

\begin{solution}
    \textbf{(i)} We know that the average number of users at the facility is given by:
    \begin{align*}
        L = \frac{\rho}{1 - \rho}
    \end{align*}
    Thus $L = 3$ implies that $\rho = \frac{3}{4}$.\\
    Since we have that $\mu = \frac{1}{3}$, then $\lambda = \frac{3}{4} \times \frac{1}{3} = \frac{1}{4}$.\\
    Converting to per hour, we have that the maximum average number of users per hour the ATM will serve is given by:
    \begin{align*}
        \lambda = \frac{1}{4} \times 60 = 15
    \end{align*}
    Thus the maximum average number of users per hour the ATM will serve is 15.\\

    \textbf{(ii)} The average queue time for a customer is given by:
    $$W_q = \frac{L_q}{\lambda} = \frac{\rho^2}{\lambda(1-\rho)}$$
    Thus the average queue time for a customer is given by:
    \begin{align*}
        W_q = \frac{(.75)^2}{(.25)(1-.75)} = 9
    \end{align*}
    Thus the average queue time for a customer is 9 minutes.\\

    \textbf{(iii)} If the average number of users double, i.e. $L = 6$, then $\rho = \frac{6}{7}$.\\
    Then with $\mu = 1/3$, we have that $\lambda = \frac{6}{7} \times \frac{1}{3} = \frac{2}{7}$.\\
    Converting to per hour, we have that the maximum average number of users per hour the ATM will serve is given by:
    \begin{align*}
        \lambda = \frac{2}{7} \times 60 = 17.14
    \end{align*}

    The average queue time for a customer is given by:
    $$W_q = \frac{L_q}{\lambda} = \frac{\rho^2}{\lambda(1-\rho)}$$
    Thus the average queue time for a customer is given by:
    \begin{align*}
        W_q = \frac{(6/7)^2}{(2/7)(1-6/7)} = 18
    \end{align*}

    Thus the average queue time for a customer is 18 minutes.\\


\end{solution}

\end{questions}

\end{document}
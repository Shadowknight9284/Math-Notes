\documentclass{article}
\usepackage{amsmath}
\usepackage{amsfonts}
\usepackage{amssymb}
\usepackage{mathrsfs}
\usepackage{cancel}

\usepackage{graphicx}


\setlength\parindent{0pt}

\author{Pranav Tikkawar}
\title{Vocab 300}

\begin{document}
\maketitle

\section{Sets}

\section{Functions}
\subsubsection*{Function}
A function is a special relation such that each input has exactly one output.\\
$f : A \rightarrow B$ means that $f$ is a function from $A$ to $B$ where $A$ and $B$ are sets.\\
$f(x) = y$ means that $f$ maps $x$ to $y$ where $x\in A$ and $y \in B$ .\\
$f := \{(x,y) \in f : (\forall x \in A)(\exists! y \in B)[f(x) = y]\}$
\subsubsection*{Composition}
The composition of two functions $f$ and $g$ is a function $f \circ g$ such that $(f \circ g)(x) = f(g(x))$\\
$g \circ f: A \rightarrow C$ where $f : A \rightarrow B$ and $g : B \rightarrow C$\\
$(g \circ f)(x) = f(g(x))$ for all $x \in A$\\
$g \circ f := (\forall x \in A)(\exists! z \in C)(g(f(x) = z) )$
\subsubsection*{Domain}
The domain of a function is the set of all possible inputs.\\
$dom(f) := \{x \in A : (\exists y \in B)[f(x) = y]\}$ \\ 
$dom(f) = A$ basically for every function $f$.
\subsubsection*{Codomain}
The codomain of a function is the set of all possible outputs.\\
$codom(f) := B$
\subsubsection*{Range}
The range of a function is the set of all outputs such that there is an input that maps to it.\\
$range(f) := \{y \in B : (\exists x \in A)[f(x) = y]\}$\\
$range(f) \subseteq codom(f)$
\subsubsection*{Injective/ One-to-One}
A function is injective if each output has at most one input.\\
$f$ is injective if $(\forall x_1,x_2 \in A)[f(x_1) = f(x_2) \Rightarrow x_1 = x_2]$\\
$f$ is injective if $(\forall x_1,x_2 \in A)[x_1 \neq x_2 \Rightarrow f(x_1) \neq f(x_2)]$
\subsubsection*{Surjective/ Onto}
A function is surjective if each output has at least one input.\\
$f$ is surjective if $(\forall y \in B)(\exists x \in A)[f(x) = y]$
\subsubsection*{Bijective}
A function is bijective if it is both injective and surjective.
\subsection*{Goldilocks}
A function is bijective if $(\forall y \in B)(\exists! x \in A)[f(x) = y]$\\
A function is one to one if $(\forall y \in B)(\exists$ at most $x \in A)[f(x) = y]$\\
A function is onto if $(\forall y \in B)(\exists$ at least $x \in A)[f(x) = y]$
\subsubsection*{Left Invertable}
A function is left invertable if there exists a function $g$ such that $g \circ f = id_A$\\
$g : B \rightarrow A$ and $g \circ f = id_A$\\
$g(f(x)) = x$ for all $x \in A$
\subsubsection*{Right Invertable}
A function is right invertable if there exists a function $g$ such that $f \circ g = id_B$\\
$g : B \rightarrow A$ and $f \circ g = id_B$\\
$f(g(y)) = y$ for all $y \in B$
\subsubsection*{Invertable}
A function is invertable if it is both left and right invertable.\\
$f$ is invertable if there exists a function $g$ such that $g \circ f = id_A$ and $f \circ g = id_B$\\
$g : B \rightarrow A$ and $g \circ f = id_A$ and $f \circ g = id_B$\\
$g(f(x)) = x$ and $f(g(y)) = y$ for all $x \in A$ and $y \in B$
$g$ is unique if a function is invertable, but it is not nessisarily unique for left and right inverse
\subsubsection*{Left Inverse}
A function $g$ is a left inverse of another function $f$ if $g \circ f = id_A$\\
$g : B \rightarrow A$ and $g \circ f = id_A$\\
\subsection*{Right Inverse}
A function $g$ is a right inverse of another function $f$ if $f \circ g = id_B$\\
$g : B \rightarrow A$ and $f \circ g = id_B$\\
\subsubsection*{Image}
The Image of a function is the set of all outputs given a set of inputs.\\
$X \subseteq A, Im_f(X) = \{f(x) : x \in X \}$
\subsubsection*{Preimage}
The preimage of a function is the set of all inputs that map to a given output or set of outputs.\\
$Y \subseteq B, PreIm(Y) = \{x \in A : f(x) \in Y\}$
\subsubsection*{Caterpillar Lemma}
For any function $f : A \rightarrow B$, the preimage sets of distinct elements are pairwaise disjoint.

\section{Relations}
\subsubsection*{Relation}
A relation is a set of ordered pairs.\\
$R \subseteq A \times B$\\
$R := \{(x,y): x\in A, y \in B, xRy\}$\\
Usually the domain and codomain of a relation are the same, and thus we can look at other properties 
\subsubsection*{Composition}
The composition of two relations $R$ and $S$ is a relation $R \circ S$ such that $(x,z) \in R \circ S$ if $(\exists y)[(x,y) \in S \wedge (y,z) \in R]$\\

\subsubsection*{Reflexive}
A relation is reflexive if every element is related to itself.\\
$R$ is reflexive if $(\forall x \in A)[xRx]$
\subsubsection*{Symmetric}
A relation is symmetric if for every pair of elements, if one is related to the other, then the other is related to the first.\\
$R$ is symmetric if $(\forall x,y \in A)[xRy \Rightarrow yRx]$
\subsubsection*{Antisymmetric}
A relation is antisymmetric if for every pair of elements, if one is related to the other, then the other is not related to the first.\\
$R$ is antisymmetric if $(\forall x,y \in A)[xRy \wedge yRx \Rightarrow x = y]$
\subsubsection*{Transitive}
A relation is transitive if for every pair of elements, if one is related to the other, and the other is related to a third, then the first is related to the third.\\
$R$ is transitve if $(\forall x,y,z \in A)[xRy \wedge yRz \Rightarrow xRz]$
\subsubsection*{Equivalence Relation}
A relation is an equivalence relation if it is reflexive, symmetric, and transitive.\\
\subsubsection*{Equivalence Class}
The equivalence class of an element $x$ is the set of all elements related to $x$.\\
$[x]_R := \{y \in A : xRy\}$\\
They are usually represented as $[x]$ where $x$ is a member of the class.\\
\subsubsection*{Partition}
A partition of a set $A$ is a set of nonempty subsets of $A$ such that every element of $A$ is in exactly one subset.\\
$\mathscr{P} = \{A_i : i \in I\}$ is a partition of $A$ if $\begin{cases}
    A_i \neq \emptyset\\
    A_i \cap A_j = \emptyset \text{ for } i \neq j\\
    \bigcup_{i \in I} A_i = A
\end{cases}$\\
In other words: \begin{enumerate}
    \item Each set is nonempty
    \item Each pair of sets is disjoint
    \item The union of all sets is the original set 
    \item Each element is in exactly one set
\end{enumerate}
\subsubsection*{Partial Order}
A relation is a partial order if it is reflexive, antisymmetric, and transitive.
\subsubsection*{Total Order}
A relation is a total order if it is a partial order and for every pair of elements, one is related to the other.\\
Thus there is a trichotomy between any two elements.
$\begin{cases}
    xRy\\
    x = y\\
    yRx
\end{cases}$
\subsection*{Relation Table}
\subsubsection*{Equality Relation}
The equality relation is an equivalence relation.\\
It is also a partial order 
\subsubsection*{Inequality Relation}
Symmetric. 
\subsubsection*{< on $\mathbb{R}$}
Antisymmetric and Transitive.
\subsubsection*{$\leq$ on $\mathbb{R}$}
Reflexive, Antisymmetric, and Transitive.\\
Thus it is a partial order.\\
In fact it is a total order
\subsubsection*{Divdes}
$a|b$ iff $(\exists k \in \mathbb{Z})[b = ak]$\\
This is a partial order on $\mathbb{N}$ but not on $\mathbb{Z}$.



\end{document}
\documentclass{article}
\usepackage{amsmath}
\usepackage{amsfonts}
\usepackage{amssymb}
\usepackage{ mathrsfs }
\usepackage{cancel}

\usepackage{graphicx}


\setlength\parindent{0pt}

\author{Pranav Tikkawar}
\title{Math 300: Midterm 3 Review}

\begin{document}
\maketitle
\section*{Question 1}
Let $A = \{1,2,3\}$. Give a relation on $A$ that is
For all these relations, consider that $R \subset A \times A$.
\subsection*{a}
Reflexive, symmetric, and transitive.\\
\textbf{Solution:}\\
Let $R = \{(1,1), (2,2), (3,3), (1,2), (2,1), (2,3), (3,2), (1,3), (3,1)\}$.
\subsection*{b}
Reflexive, symmetric, but not transitive. \\
\textbf{Solution:}\\
Let $R = \{(1,1), (2,2), (3,3), (1,2),(2,1),(2,3),(3,2)\}$.
\subsection*{c}
Reflexive, not symmetric, and transitive.\\
\textbf{Solution:}\\
Let $R = \{(1,1), (2,2), (3,3), (1,2)\}$.
\subsection*{d}
Reflexive, not symmetric, and not transitive.\\
\textbf{Solution:}\\
Let $R = \{(1,1), (2,2), (3,3), (1,2), (2,3)\}$.
\subsection*{e}
Not reflexive, symmetric, and transitive.\\
\textbf{Solution:}\\
Let $R = \emptyset $.
\subsection*{f}
Not reflexive, symmetric, and not transitive.\\
\textbf{Solution:}\\
Let $R = \{(1,2), (2,1)\}$.
\subsection*{g}
Not reflexive, not symmetric, and transitive.\\
\textbf{Solution:}\\
Let $R = \{(1,2), (2,3), (1,3)\}$.
\subsection*{h}
Not reflexive, not symmetric, and not transitive.\\
\textbf{Solution:}\\
Let $R = \{(1,2), (2,3)\}$.

\section*{Question 2}
\subsection*{a}
Let $A= \{ 1,2\}$. All the relations on A which are symmetric and transitive, but not reflexive
\textbf{Solution:}\\
$R = \emptyset, \{(1,1) \} \{(2,2) \}$
\subsection*{b}
Let $A= \{ 1,2,3,4,5\}$. How many relations which are both symmetric and antisymmetric \\
\textbf{Solution:}\\ 
There are 32 such relations. If we consider the powerset of $A$ then see that every single subset of $A$ can be a relation that is symetric and antisymmetric if the relation is the identy relation. So there are $2^5 = 32$ such relations.
\section*{Question 3}
Let $A = \{1,2,3\}$ For each of the following relations on A, determine whether it is reflexive, symmetric, antisymmetric, and/or transitive.
\subsection*{a}
$R = \{(1,2)\}$\\
\textbf{Solution:}\\
Reflexive: No. $(1,1)$ is not in $R$.\\
Symmetric: No. $(2,1)$ is not in $R$.\\
Antisymmetric: Yes. \\
Transitive: Yes. 
\subsection*{b}
$S = \{(1,2),(1,3)\}$
\textbf{Solution:}\\
Reflexive: No. $(1,1)$ is not in $S$.\\
Symmetric: No. $(2,1)$ is not in $S$.\\
Antisymmetric: Yes. \\
Transitive: Yes.
\subsection*{c}
$T = \{(1,2),(2,1),(1,1)\}$
\textbf{Solution:}\\
Reflexive: No. $(2,2)$ is not in $T$.\\
Symmetric: Yes
Antisymmetric: No. $(1,2)$ and $(2,1)$ are in $T$ but $1 \neq 2$.\\
Transitive: No. $(1,2)$ and $(2,1)$ are in $T$ but $(2,2)$ is not in $T$.\\
\section*{Question 4}
Let $A = \{1,2,3\}$. Size of relations:
\begin{itemize}
    \item Min Reflexive: $\{(1,1),(2,2),(3,3) \}$
    \item Min symmetric: $\emptyset$
    \item Min antisymmetric: $\emptyset$
    \item Min transitive: $\emptyset$
    \item Min equivalence: $\{(1,1),(2,2),(3,3) \}$
    \item Min partial order: $\{(1,1),(2,2),(3,3) \}$
    \item Max symmetric: $\{(1,1),(2,2),(3,3),(1,2),(1,3),(2,1),(2,3),(3,1),(3,2)\} $
    \item Max antisymmetric: $\{(1,1),(2,2),(3,3),(1,2),(1,3),(2,3)\}$
    \item Max equivalence: $A \times A$
    \item Max partial: $\{(1,1),(1,2),(1,3),(2,2),(2,3),(3,3)\} $
\end{itemize}
\section*{Question 5}
Let S be the relation on $\mathbb{R}$ defined by $xSy$ : $ x<y+1$. Determine whether S is reflexive, symmetric, antisymmetric, transitive. \\
\textbf{Reflexive:}\\
Need $xSx$ : $x < x+1$. This is true for all $x \in \mathbb{R}$. So S is reflexive.\\
\textbf{Symmetric:}\\
Need $xSy \Rightarrow ySx$ Counterexample: $x=1$, $y=100$. $1 < 100+1$ but $100 \cancel{<} 1+1$. So S is not symmetric.\\
\textbf{Antisymmetric:}\\
Need $xSy \land ySx \Rightarrow x = y$ Counterexample: $x=1$, $y=1.5$ $1 < 1.5+1$ and $1.5 < 1+1$ but $1 \neq 1.5$. So S is not antisymmetric.\\
\textbf{Transitive:}\\
Need $xSy \land ySz \Rightarrow xSz$ Counterexample: $x=5$, $y=4.3$, and $z=3.5$. $5 < 4.3+1$ and $4.3 < 3.5+1$ but $5 \cancel{<} 3.5+1$. So S is not transitive.\\
\section*{Question 6}
Let $E \subset \mathbb{N} \times \mathbb{N}$ be the relation defined as $xEy$ : $xy \leq x+y$. Determine whether E is reflexive, symmetric, antisymmetric, transitive.\\
\textbf{Reflexive:}\\
$xEx : x \cdot x \leq x + x$. This is not true for values of 3 or greater. So E is not reflexive.\\
\textbf{Symmetric:}\\
if $xEy : xy \leq x+y$ then $yEx : yx \leq y+x$. This is true as multiplication and addition is commutative. So E is symmetric.\\
\textbf{Antisymmetric:}\\
if $xEy : xy \leq x+y$ and $yEx : yx \leq y+x$ then $x=y$. This is not true as $x=2$ and $y=3$ is a counterexample. So E is not antisymmetric.\\
\textbf{Transitive:}\\
if $xEy : xy \leq x+y$ and $yEz : yz \leq y+z$ then $xEz$ would be $xz \leq x+z$. This is not true for $x=2$, $y=1$, and $z=3$. So E is not transitive.\\
\section*{Problem 7}
Let $D$ be the relation on $\mathbb{N}$ defined as: $x D y$ iff $x^2 | y$. Determine whether $D$ is reflexive, symmetric, antisymmetric, and transitive.
\subsection*{Reflexive}
Need: $x D x$ \\
$x^2 | x$ \\
Counter: 2. $2^2$ does not divide $2$. So $D$ is not reflexive.
\subsection*{Symmetric}
Need: $x D y \Rightarrow y D x$ \\
$x^2 | y \Rightarrow y^2 | x$ \\
This is not true in general. For example, $2^2 | 4$ but $4^2 \cancel{|} 2$. So $D$ is not symmetric.
\subsection*{Antisymmetric}
Need: $x D y \land y D x \Rightarrow x = y$ \\
$x^2 | y \land y^2 | x \Rightarrow x = y$ \\
$y = kx^2$ and $x = qy^2$ for some $k, q \in \mathbb{N}$. \\
Substitute $y = kx^2$ into $x = qy^2$ to get $x = q(kx^2)^2 = qk^2x^4$. \\
Divide by $x$ (as it is $\neq 0$) to get $1 = qk^2x^3$. \\
This is only true if $x = 1$ and $q = k = 1$. So $x = y$ and $D$ is antisymmetric
\subsection*{Transitive}
Need: $x D y \land y D z \Rightarrow x D z$ \\
$x^2 | y \land y^2 | z \Rightarrow x^2 | z$ \\
Suppose  $k, q \in \mathbb{N}$ such that $y = kx^2 \land z = qy^2$ \\
Then $z = q(kx^2)^2 = qk^2x^4$. \\
Since $k, q, x \in \mathbb{N}$, $qk^2x^2 \in \mathbb{N}$. We can call this $r$. \\
Thus $z = rx^2$ and $x^2 | z$. So $D$ is transitive.
\section*{Problem 8}
Let $S$ be the relation on $\mathbb{N}$ defined as: $x S y$ iff $x|y^2$ Determine whether $S$ is reflexive, symmetric, antisymmetric, and transitive.
\subsection*{Reflexive}
Need: $x S x$ \\
$x | x^2$ \\
$(\exists q \in \mathbb{N} )x^2 = qx$ \\
This is true for all $x \in \mathbb{N}$. as the "q" will be x to satisfy the equation So $S$ is reflexive.
\subsection*{Symmetric}
Need: $x S y \Rightarrow y S x$ \\
$x | y^2 \Rightarrow y | x^2$ \\
$(\exists k,q \in \mathbb{N} )y^2 = kx \Rightarrow x^2 = qy$ \\
This is not true in general. For example, $x=3$ and $y=6$. $3|36$ but $6 \cancel{|} 9$. So $S$ is not symmetric.
\subsection*{Antisymmetric}
Need: $x S y \land y S x \Rightarrow x = y$ \\
$x | y^2 \land y | x^2 \Rightarrow x = y$ \\
$(\exists k,q \in \mathbb{N} )y^2 = kx \land x^2 = qy \rightarrow x = y$\\
Counter: $x = 2$ and $y = 4$. $2|16$ and $4|4$ but $2 \neq 4$. So $S$ is not antisymmetric
\subsection*{Transitive}
Need: $x S y \land y S z \Rightarrow x S z$ \\
$x | y^2 \land y | z^2 \Rightarrow x | z^2$ \\
$(\exists k,q,r \in \mathbb{N} )y^2 = kx \land z^2 = qy \rightarrow z^2 = rx$\\
Counter example is $x = 8$, $y = 4$, and $z = 2$. $8|16$ and $4|4$ but $8 \cancel{|} 4$. So $S$ is not transitive.
\section*{Problem 9}
Let $S = \mathbb{R} \times \mathbb{R}$ be define as follows: for $(x_1, y_1) \in S$ and $(x_2, y_2) \in S$. We have $(x_1, y_1) P (x_2, y_2)$ iff $x_1 \leq x_2$ and $y_1 > y_2$. Determine whether $P$ is reflexive, symmetric, antisymmetric, and transitive.
\subsection*{Reflexive}
Need: $(x, y) P (x, y)$ \\
$x \leq x$ and $y > y$ \\
This is not true for all $(x, y) \in S$. So $P$ is not reflexive. Counter: $(1, 2)$
\subsection*{Symmetric}
Need: $(x_1, y_1) P (x_2, y_2) \Rightarrow (x_2, y_2) P (x_1, y_1)$ \\
$x_1 \leq x_2$ and $y_1 > y_2 \Rightarrow x_2 \leq x_1$ and $y_2 > y_1$ \\
This is not true in general. For example, $(1, 2) P (2, 1)$ but $(2, 1) \cancel{P} (1, 2)$. So $P$ is not symmetric.
\subsection*{Antisymmetric}
Need: $(x_1, y_1) P (x_2, y_2) \land (x_2, y_2) P (x_1, y_1) \Rightarrow (x_1, y_1) = (x_2, y_2)$ \\
$x_1 \leq x_2$ and $y_1 > y_2 \land x_2 \leq x_1$ and $y_2 > y_1 \Rightarrow (x_1, y_1) = (x_2, y_2)$ \\
This is vacuously true as there is no $(x_1, y_1) and (x_2, y_2)$ that satisfy the conditions. So $P$ is antisymmetric
\subsection*{Transitive}
Need: $(x_1, y_1) P (x_2, y_2) \land (x_2, y_2) P (x_3, y_3) \Rightarrow (x_1, y_1) P (x_3, y_3)$ \\
$x_1 \leq x_2$ and $y_1 > y_2 \land x_2 \leq x_3$ and $y_2 > y_3 \Rightarrow x_1 \leq x_3$ and $y_1 > y_3$ \\
This is true due to the transitive property of the inequalities. So $P$ is transitive.

\section*{Problem 10}
The properties of reflexivity, symmetry, antisymmetry, and transitivity are related to the
identity relation and the operations of inversion and composition. Let $R \subset A \times A$ Prove:
\subsection*{a}
$R$ is reflexive iff $I_A \subset R$
\subsubsection*{Proof}
\subsubsection*{Forward}
Suppose $x \in A$. 
Assume $R$ is reflexive. 
Need $I_A \subseteq R$.\\
Since $R$ is reflexive, $\forall x \in A, xRx$. This means $(x, x) \in R$. Since $I_A = \{(x, x) | x \in A\}$, $I_A \subseteq R$.
\subsubsection*{Backward}
Suppose $x \in A$
Assume $I_A \subseteq R$.
Need $R$ is reflexive.\\
Since $I_A \subseteq R$, $\forall x \in A, (x, x) \in R$. This means that $\forall x \in A, xRx$. So $R$ is reflexive.
\subsection*{b}
$R$ is symmetric iff $R = \overleftarrow{R}$
\subsubsection*{Proof}
\subsubsection*{Forward}
Suppose $a, b \in A$.
Assume $R$ is symmetric.
Need $R = \overleftarrow{R}$.\\
In other words, $\{(a, b) | a, b \in A \land aRb\} = \{(b, a) | a, b \in A \land aRb\}$.\\
Since $R$ is symmetric, $\forall a, b \in A, aRb \Rightarrow bRa$. This means that $(a, b) \in R \Rightarrow (b, a) \in R$. So $R \subseteq \overleftarrow{R}$. Also since $R$ is symmetric $\forall a, b \in A, bRa \Rightarrow aRb$. This means that $(b, a) \in R \Rightarrow (a, b) \in R$. So $\overleftarrow{R} \subseteq R$. So $R = \overleftarrow{R}$.

\subsubsection*{Backward}
Suppose $R = \overleftarrow{R}$. Then $\forall x, y \in A, (x, y) \in R \Rightarrow (y, x) \in R$. This means that $\forall x, y \in A, xRy \Rightarrow yRx$. So $R$ is symmetric.,
\subsection*{c}
$R$ is antisymmetric iff $R \cap \overleftarrow{R} \subset I_A$
\subsubsection*{Proof}
\subsubsection*{Forward}
Suppose $R$ is antisymmetric. Then $\forall x, y \in A, xRy \land yRx \Rightarrow x = y$. This means that $(x, y) \in R \land (y, x) \in R \Rightarrow x = y$. Since $(x,y) \in R$ means $xRy$, $(y,x) \in R$ means $x\overleftarrow{R}y$ and $x=y$ means $(x, y) \in I_A$. Thus $R \cap \overleftarrow{R} \subset I_A$.
\subsubsection*{Backward}
Suppose $R \cap \overleftarrow{R} \subset I_A$. Then $\forall x, y \in A, (x, y) \in R \land (y, x) \in R \Rightarrow (x, y) \in I_A$. This means that $\forall x, y \in A, xRy \land yRx \Rightarrow x = y$. Which is the definiton of antisymmetry, so $R$ is antisymmetric.
\subsection*{d}
$R$ is transitive iff $R \circ R \subset R$
\subsubsection*{Proof}
\subsubsection*{Forward}
Suppose $x,z \in A$.
Assume $R$ is transitive.
Need $R \circ R \subset R$.\\
Let $(x,z) \in R \circ R$ then $\exists y \in A$ such that $xRy \land yRz$. Since $R$ is transitive and $xRz$. So $(x,z) \in R$. So $R \circ R \subseteq R$.

\subsubsection*{Backward}
Suppose $R \circ R \subset R$. Then $\forall x, y, z \in A, (x, y) \in R \land (y, z) \in R \Rightarrow (x, z) \in R$. This means that $\forall x, y, z \in A, xRy \land yRz \Rightarrow xRz$. Which is the definition of transitivity, so $R$ is transitive.
\section*{Problem 11}
A relation $V$ on $\mathbb{R}$ is given by $xVy$ iff $x=y$ or $xy = 1$. 
\subsection*{a}
Prove that $V$ is an equivalence relation.\\
\subsubsection*{Reflexive}
Need: $xVx$ \\
$x = x$ or $x \cdot x = 1$ \\
This is true for all $x \in \mathbb{R}$. So $V$ is reflexive.
\subsubsection*{Symmetric}
Need: $xVy \Rightarrow yVx$ \\
$x = y$ or $xy = 1 \Rightarrow y = x$ or $yx = 1$ \\
This is true for all $x, y \in \mathbb{R}$. So $V$ is symmetric
\subsubsection*{Transitive}
Need: $xVy \land yVz \Rightarrow xVz$ \\
$x = y$ or $xy = 1 \land y = z$ or $yz = 1 \Rightarrow x = z$ or $xz = 1$ \\
{Case 1: $x = y$ and $y = z$}\\
Then $x = z$ and $xVz$. So $V$ is transitive.\\
{Case 2: $x = y$ and $yz = 1$}\\
Then $xz = 1$ and $xVz$. So $V$ is transitive.\\
{Case 3: $xy = 1$ and $y = z$}\\
Then $xz = 1$ and $xVz$. So $V$ is transitive.\\
{Case 4: $xy = 1$ and $yz = 1$}\\
Then $1/x = 1/z$ and $x=z$ and $xVz$. So $V$ is transitive.\\
Thus $V$ is an equivalence relation.
\subsection*{b}
Describe the equivalence classes of $3, -2/3, and 0$.\\
The equivalence class of $3$ is $\{3, 1/3\}$\\
The equivalence class of $-2/3$ is $\{-2/3, -3/2\}$\\
The equivalence class of $0$ is $\{0\}$\\ 
\section*{Problem 12}
Let $T$ be the relation on $\mathbb{Z}$ defined as: $aTb$ iff there exists nonzero integers r and s such that $ar^2 = bs^2$. Prove that $T$ is an equivalence relation.
\subsection*{Reflexive}
Need: $aTa$ \\
There exists $r = s = 1$ such that $a = a$. So $T$ is reflexive.
\subsection*{Symmetric}
Need: $aTb \Rightarrow bTa$ \\
Suppose $aTb$. Then there exists $r, s \neq 0$ such that $ar^2 = bs^2$. This means that $bs^2 = ar^2$. So $bTa$. So $T$ is symmetric
\subsection*{Transitive}
Need: $aTb \land bTc \Rightarrow aTc$ \\
Suppose $aTb$ and $bTc$. Then there exists $r, s, t, u \neq 0$ such that $ar^2 = bs^2$ and $bt^2 = cu^2$. This means that $ar^2 / s^2 = cu^2 / t^2$. $ar^2t^2 = bs^2u^2$. So $aTc$. So $T$ is transitive.\\
\section*{Problem 13}
Let $R$ be the relation on $\mathbb{N}$ defined as: $aRb$ iff there exists odd integers $k$ and $l$ such that $ak =bl$.
\subsection*{a}
$20R12$ is in this relation as $k = 3, l = 5$. to make $20*3 = 60 = 12*5$.\\
\subsection*{b}
$7R10$ is not in this relation as there are no odd integers $k$ to multiply to 7 to make an even number.\\
\subsection*{c}
$20R10$ is not in the relation as there are no odd integers $k$ to multiply to 20 to make an "odd" multiple of 10.\\
\subsection*{d}
\subsubsection*{Reflexive}
Need $aRa$\\
There exists $k = l = 1$ such that $ak = al$. So $R$ is reflexive.
\subsubsection*{Symmetric}
Need $aRb \Rightarrow bRa$\\
Suppose $aRb$. Then there exists $k, l$ such that $ak = bl$. This means that $bl = ak$. So $bRa$. So $R$ is symmetric
\subsubsection*{Transitive}
Need $aRb \land bRc \Rightarrow aRc$\\
Suppose $aRb$ and $bRc$. Then there exists $k, l, m, n$ such that $ak = bl$ and $bm = cn$. This means that $akm = bln$. Since 2 odd integers multiplied is also odd then $aRc$. So $R$ is transitive.
\subsection*{e}
Describe the equivalence claseses:\\
The equivalence class of $2^0$ is $\{2n+1: n \in \mathbb{N}\}$\\
The equivalence class of $2^1$ is $\{2(2n+1): n \in \mathbb{N}\}$\\
The equivalence class of $2^2$ is $\{2^2(2n+1): n \in \mathbb{N}\}$\\
The equivalence class of $2^k$ $\forall k \geq 0$ is $\{2^k(2n+1): n \in \mathbb{N}\}$\\
In other words, The equiliances classes are represented as $2^k$ where $k \in \mathbb{Z}$ and $k \geq 0$. And all the elements of the equivalence class is all the numbers $x \in \mathbb{N}$ which have the same power of 2 in thier prime factorization.\\


\section*{Problem 14}
On $\mathbb{N}$ a relation $P$ is given by $aPb$ iff the prime factorization of $a$  and $b$ have the same power of 2. 
\subsection*{a}
\subsubsection*{Reflexive}
Need $aPa$\\
The prime factorization of $a$ is the same as the prime factorization of $a$. So $P$ is reflexive.
\subsubsection*{Symmetric}
Need $aPb \Rightarrow bPa$\\
Suppose $aPb$. Then the prime factorization of $a$ and $b$ have the same power of 2. This means that the prime factorization of $b$ and $a$ have the same power of 2. So $P$ is symmetric
\subsubsection*{Transitive}
Need $aPb \land bPc \Rightarrow aPc$\\
Suppose $aPb$ and $bPc$. Then the prime factorization of $a$ and $b$ have the same power of 2 and the prime factorization of $b$ and $c$ have the same power of 2. This means that the prime factorization of $a$ and $c$ have the same power
\subsection*{b}
$1/P$: $\{1, 3,5 \}$\\
$4/P$: $\{4, 12,20 \}$\\
$72/P$: $\{8,24,60\}$
\section*{Problem 15}
\subsection*{a}
Let $P$ and $Q$ be equivalence relations on a set $A$. Prove that $R := P \cap Q$ is an equivalence relation on $A$.\\
\subsubsection*{Reflexive}
Suppose $a \in A$.
Need $aRa$\\
Since $P$ and $Q$ are equivalence relations, $aPa$ and $aQa$. This means that $(a,a) \in P$ and $(a,a) \in Q$. So $a \in P \cap Q$. So $aRa$.
\subsubsection*{Symmetric}
Suppose $a,b \in A$ 
Assume $aRb$.
Need $bRa$\\
Since $aRb$, $(a,a) \in P \cap q$ so $(a,b) \in P$ and $(a,b) \in Q$ Since $P$ and $Q$ are both equivalence relations they are symmetric. So $(b,a) \in P$ and $(b,a) \in Q$. So $bRa$. So $R$ is symmetric.

\subsubsection*{Transitive}
Suppose $a,b,c \in A$
Assume $aRb$ and $bRc$.
Need $aRc$\\
Since $aRb$ and $bRc$, $(a,b) \in P \cap Q$ and $(b,c) \in P \cap Q$. This means that $(a,b) \in P$ and $(a,b) \in Q$ and $(b,c) \in P$ and $(b,c) \in Q$. Since $P$ and $Q$ are equivalence relations, they are transitive. So $(a,c) \in P$ and $(a,c) \in Q$. So $aRc$. So $R$ is transitive.\\

\subsection*{b}
Give an exaple of two equivalence relations $P$ and $Q$ on $A = \{ 1,2,3 \} $ such that $T := P \cup Q$ is not an equivalence relation on $A$.\\
\textbf{Solution:}\\
Let $A = \{ 1,2,3\}$, \\
$P = \{ (1,1), (2,2), (3,3), (1,3), (3,1) \}$, \\
$Q = \{ (1,1), (2,2), (3,3), (1,2), (2,1) \}$.\\
Then $T = P \cup Q = \{ (1,1), (2,2), (3,3), (1,3), (3,1), (1,2), (2,1) \}$.\\
$T$ is not an equivalence relation as it is not transitive. For example, $(3,1) \in T$ and $(1,2) \in T$ but $(3,2) \cancel{\in} T$.\\


\section*{Problem 16}
Let $P$ be the relation on $\mathbb{N}$ defined as: $aPb$ iff $b=2^ka$ for some integers $k \geq 0$. Prove that $P$ is a partial order.\\
\subsection*{Reflexive}
Need: $aPa$\\
$a = 2^0a$. So $aPa$.
\subsection*{Antisymmetric}
Need: $aPb \land bPa \Rightarrow a = b$\\
Suppose $aPb$ and $bPa$. Then $b = 2^ka$ and $a = 2^lb$ for some $k, l \geq 0$. This means that $b = 2^k2^lb$. So $b = 2^{k+l}b$. This means $k+l = 0$ and $k = l = 0$. So $a = b$
\subsection*{Transitive}
Need: $aPb \land bPc \Rightarrow aPc$\\
Suppose $aPb$ and $bPc$. Then $b = 2^ka$ and $c = 2^lb$ for some $k, l \geq 0$. This means that $c = 2^k2^la$. So $c = 2^{k+l}a$. So $aPc$.\\
\section*{Problem 17}
Let $A$ an arbitrary nonempty set, and let $P$ be a partial order on $A$. Define a new relation $<$ on $A$ as follows $x<y$ iff $xPy$ and $x \neq y$. 
\subsubsection*{a}
Prove that there are no $x,y \in A$ such that $x<y$ and $y<x$.\\
Suppose there exists $x, y \in A$ such that $x<y$ and $y<x$. Then $xPy$ and $yPx$. Due to $P$ being Partial Order and thus antisymmetric, $x = y$. This is a contradiction. So there are no $x, y \in A$ such that $x<y$ and $y<x$.\\
\subsubsection*{b}
Prove that $<$ is transitive.\\
Assume that $<$ is not transitive. Then there exists $x,y,z\in A$ such that $x<y$ and $y<z$ but $x \cancel{<} z$. If $x<y$ then $xPy$ and if $y<z$ then $yPz$. Since $P$ is a partial order, it is transitive. So $xPz$. But $x \cancel{<} z$ is a contradiction. So $<$ is transitive.\\

\section*{Problem 18}
Let $A$ be a nonempty set with partial order P. for each $t \in A$ define $S_t := \{ x \in A: xPt\}$\\
Let $\mathscr{F} = \{S_t : t \in A \}$ then $\mathscr{F}$ is a subset of $\mathscr{P}(A)$ [since for every $t \in A$, $S_t \subset A$]. and thus can be partially orded by $\subseteq $. Let $a,v \in A$ be arbitrary.
\subsection*{i}
Prove that if $aPb$ then $S_a \subseteq S_b$\\
Suppose $aPb$. Let $x \in S_a$. Then $xPa$. Since $aPb$, $xPb$. So $x \in S_b$. So $S_a \subseteq S_b$.\\ 

Suppose $a,b \in A$. Assume $aPb$. Need $S_a \subseteq S_b$. Let $x \in S_a$. Then $xPa$. Since $aPb$, $xPb$ by the transitivity of $P$. So $x \in S_b$. So $S_a \subseteq S_b$.\\

\subsection*{ii}
Prove that if $S_a \subseteq S_b$ then $aPb$\\
Suppose $a,b \in A$. Assume $S_a \subseteq S_b$. Need $aPb$. Let $x \in S_a$. Then $xPa$. Since $S_a \subseteq S_b$, $x \in S_b$. Since we know that $P$ is a partial order, it is also Reflexive, so $a$ is in the set $S_a$. So there is an element $x \in S_b$ where $x=a$ so $aPb$.\\

\section*{Problem 19}

\subsection*{a}
Let $P$ and $Q$ be partial orders on the same nonempty set $A$. Prove that $P \cap Q$ is a partial order on $A$.\\
For sake of ease: Let $R := P \cap Q$\\


\subsubsection*{Reflexive}
Suppose $a \in A$. 
Need $aRa$\\
Since $P$ and $Q$ are partial orders and thus reflexive, $aPa$ and $aQa$. This means that $(a,a) \in P$ and $(a,a) \in Q$. So $(a,a) \in P \cap Q$. So $aRa$.

\subsubsection*{Antisymmetric}
Suppose $a,b \in A$
Assume $aRb$ and $bRa$.
Need: $a = b$\\
Since $aRb$ and $bRa$, $(a,b) \in P \cap Q$ and $(b,a) \in P \cap Q$. This means that $(a,b) \in P$ and $(a,b) \in Q$ and $(b,a) \in P$ and $(b,a) \in Q$. Since $P$ and $Q$ are partial orders and antisymmetric, $aPb$ and $aQb$ and $bPa$ and $bQa$ implies $a = b$. So $R$ is antisymmetric.


\subsubsection*{Transitive}
Suppose $a,b,c \in A$.
Assume $aRb$ and $bRc$.
Need $aRc$\\
Since $aRb$ and $bRc$, $(a,b) \in P \cap Q$ and $(b,c) \in P \cap Q$. This means that $(a,b) \in P$ and $(a,b) \in Q$ and $(b,c) \in P$ and $(b,c) \in Q$. Thus $aPb$ and $aQb$ and $bPc$ and $bQc$. Since $P$ and $Q$ are partial order and transitive, $aPc$ and $aQc$. Since $aPc$ and $aQc$, $a \in P \cap Q$. So $aRc$. So $R$ is transitive.\\

\subsection*{b}
Give an example of two partial orders $P$ and $Q$ on $A \{1,2,3 \}$ such that $P \cup Q$ is not a partial order on $A$.\\
\subsubsection*{Solution}
Let $A = \{1,2,3\}$\\
Let $P = \{ (1,1), (2,2), (3,3), (1,3) \}$\\
Let $Q = \{ (1,1), (2,2), (3,3), (3,1) \}$\\
Then $P \cup Q = \{ (1,1), (2,2), (3,3), (1,3), (3,1) \}$\\
Which is not antisymmetric, thus not a partial order.\\


\section*{Problem 20}
\subsection*{a}
Let $\leq_1 and\leq_2$ be total orders on the same nonempty set $A$. Let $P$ be the relation on A defined by $aPb$ iff $a\leq_1 b$ and $a\leq_2 b$. Prove that $P$ is a partial order on A.
\subsubsection*{Reflexive}
Suppose $a \in A$.
Need: $aPa$\\
Since $\leq_1$ and $\leq_2$ are total orders, $a\leq_1 a$ and $a\leq_2 a$. So $aPa$.
\subsubsection*{Antisymmetric}
Suppose $a,b \in A$.
Assume $aPb$ and $bPa$.
Need: $a = b$\\
Since $aPb$ and $bPa$, $a\leq_1 b$ and $a\leq_2 b$ and $b\leq_1 a$ and $b\leq_2 a$. Since $\leq_1$ and $\leq_2$ are total orders and antisymmetric, $a = b$ as desired
\subsubsection*{Transitive}
Suppose $a,b,c \in A$.
Assume $aPb$ and $bPc$.
Need $aPc$\\
Since $aPb$ and $bPc$, $a\leq_1 b$ and $a\leq_2 b$ and $b\leq_1 c$ and $b\leq_2 c$. Since $\leq_1$ and $\leq_2$ are total orders and transitive, $a\leq_1 c$ and $a\leq_2 c$. So $aPc$. So $P$ is transitive.\\

\subsection*{b}
Give an example of 2 total orders on the same set $A$ such that the relation $P$ is not a total order on $A$.\\
$aPb$ iff $a\leq_3 b$ and $a\leq_4 b$\\
\subsubsection*{Solution}
Let $A = \{1,2,3\}$\\
Let $\leq_3 := \leq$
Let $\leq_4 := \geq$\\
$\leq_3 = \{ (1,1), (1,2), (1,3), (2,2), (2,3), (3,3) \}$\\
$\leq_4 = \{ (1,1), (2,1), (3,1), (2,2), (3,2), (3,3) \}$\\
$P = \{ (1,1), (2,2), (3,3)\}$\\
Counter to Total order: $1 \cancel{P} 2$ and $2 \cancel{P} 1$\\
$I_A \subseteq P$

\section*{Problem 21}
Let $P$ be a partial order on a set $A$, and let $B \subseteq A$. Prove that if B contains one of its upper bounds $s$ then $s$ is the least upper bound of $B$.
\subsubsection*{Proof}
Suppose $A$, $B$ are sets and $P$ is a partial order on $A$ where $B \subseteq A$\\
Assume $s \in B$ and $s$ is an upper bound of $B$.\\
Need: $s$ is the least upper bound of $B$. In other words $\forall t$ that are upper bounds $sPt$\\
Let $t$ be an upper bound of $B$. Since $s$ is an upper bound of $B$ that is also in $B$, there does not exist another upperbound in $B$ that is less than $s$ since the definition of an upper bound is $\{s: \forall a \in B, aPs\}$. So every upperbound of $B$ other than $s$ must related to $s$ by $P$. In other words $sPt$ thus $s$ is the least upper bound of $B$.\\ 

\section*{Problem 22}
Let $S \subseteq \mathbb{R}$ be a bounded set and let $T$ be an non-empty subset of $S$. Prove that $$inf(s) \leq inf(T) \leq sup(T) \leq sup(S)$$
\subsubsection*{Proof}
\textbf{Proof of existence}
Suppose $S$ is a bounded set and $T$ is a non-empty subset of $S$.\\
Since $S$ is bounded, $inf(S)$ and $sup(S)$ exist. Since $T$ is a non-empty subset of $S$, $T$ is also bounded. So $inf(T)$ and $sup(T)$ exist.\\
\textbf{Proof of $inf(s) \leq inf(T)$}\\
Let $i_s = inf(S)$ and $i_t = inf(T)$.\\
Since $T$ is a subset of $S$, $inf(S)$ is a lower bound of $T$. So by the definition of infimum $inf(S) \leq inf(T)$.\\
\textbf{Proof of $inf(T) \leq sup(T)$}\\
Let $i_t = inf(T)$ and $s_t = sup(T)$.\\
Since $T$ is a bounded set and is non empty $inf(T) \leq sup(T)$.\\
\textbf{Proof of $sup(T) \leq sup(S)$}\\
Let $s_t = sup(T)$ and $s_s = sup(S)$.\\
Since $T$ is a subset of $S$, $sup(S)$ is an upperbound of $T$. So by the definition of supremum $sup(T) \leq sup(S)$.\\ 

\section*{Problem 23}
Let $B$ and $C$ be non-empty sets of real numbers such that $b \leq c$  for all $b \in B$ and $c \in C$. Prove that $sup(B) \leq inf(C)$.
\textbf{Proof}
Suppose $B$ and $C$ are non-empty sets of real numbers such that $b \leq c$ for all $b \in B$ and $c \in C$.\\
Need: $sup(B) \leq inf(C)$.\\
Since $B$ is a non-empty set of real numbers, $sup(B)$ exists. Since $C$ is a non-empty set of real numbers, $inf(C)$ exists due to the axiom of completeness.\\
Let $c \in C$. Since $(\forall b \in B) b \leq c$, $c$ is an upper bound of $B$. So by definition of supremum $sup(B) \leq c$. Since $c$ is an arbitrary element of $C$ and $sup(B) \leq c$, $sup(B)$ is a lower bound of $C$. So by definition of infimum $sup(B) \leq inf(C)$.\\

\section*{Problem 24}
Let $A$ be an arbitrary nonempty set and let 

\section*{Problem 25}
Let $A$  be the set of all closed subintervals of $[0,1]$ with positive length. $A$ is a partially ordered by set inclusion. A set $B \subseteq A$ is a collection of intervals, also note $[0,1] \in A$. \subsubsection*{i}
Does every nonempty subset $B$ of $A$ have an upper bound?\\
Yes as for every set $B$ in $A$, $[0,1]$ will be an element such that $1 \in A$ and $\forall x \in B , x \subseteq [0,1]$. So $[0,1]$ is an upper bound for $B$.\\
\subsubsection*{ii}
Does every nonempty subset $B$ of $A$ have a least upper bound?\\
Yes, as since there exists an upper bound for every nonempty subset $B$ of $A$, the least upper bound will be the interval $s$ such that all of the upperbounds of $B$ contain $s$. In other words $sup(B) := s \in A$ such that $\forall x$ that are upper bounds of $B, s \subseteq x$.\\
Since $B$ is bounded then it has a maximum which is in $B$. This maximum is an upper bound of $B$ and is a subset of all other upper bounds of $B$. So the maximum is the least upper bound of $B$.\\
\textbf{Review for Later}

\subsubsection*{iii}
Does every nonempty subset $B$ of $A$ have a maximum?\\

\subsubsection*{iv}

\subsubsection*{v}

\subsubsection*{vi}

\section*{Problem 26}

\section*{Problem 27}

\section*{Definitions}
\subsection*{Power Set}
Let $A$ be a set. The power set of $A$ is the set of all subsets of $A$.\\
$\mathscr{P}(A) := \{X: X \subseteq A\}$\\

\subsection*{Cartesian Product}
Let $A$ and $B$ be sets. The Cartesian product of $A$ and $B$ is the set of all ordered pairs $(a,b)$ where $a \in A$ and $b \in B$.\\
$A \times B := \{(a,b): a\in A, b \in B \}$

\subsection*{Set Partition}
Let $A$ be a set. A set partition of $A$ is a collection of nonempty subsets of $A$ such that every element of $A$ is in exactly one of the subsets.\\
$P = \{A_1, A_2, \ldots, A_n\}$ is a partition of $A$ if:\\
$\forall x \in A: \exists i \in \{1,2,\ldots,n\} : x \in A_i$\\
$\forall i,j \in \{1,2,\ldots,n\} : i \neq j \Rightarrow A_i \cap A_j = \emptyset$\\

\subsection*{Identity Relation}
Let $A$ be a set. The identity relation on $A$ is the relation that relates every element of $A$ to itself.\\
$I_A := \{(a,a): a \in A\}$
\subsection*{Composition}
Let $P$ and $Q$ be relations. $P:A \rightarrow B$ and $Q:B \rightarrow C$. The composition of $P$ and $Q$ is the relation that relates $a$ to $c$ if there exists $b$ such that $aPb$ and $bQc$.\\
$P \circ Q := \{(a,c): \exists b \in B: aPb \land bQc\}$

\subsection*{Inverse}
Let $P$ be a relation from $A$ to $B$. The inverse of $P$ is the relation that relates $b$ to $a$ if $aPb$.\\
$\overleftarrow{P} := \{(b,a): a\in A, b\in B, aPb\}$

\subsection*{Dom}
Let $P$ be a relation from $A$ to $B$. The domain of $P$ is the set of all elements of $A$ that are related to some element of $B$.\\
$dom(P) := \{a \in A: \exists b \in B: aPb\}$

\subsection*{Range}
Let $P$ be a relation from $A$ to $B$. The range of $P$ is the set of all elements of $B$ that are related to some element of $A$.\\
$ran(P) := \{b \in B: \exists a \in A: aPb\}$

\subsection*{title}


\end{document}
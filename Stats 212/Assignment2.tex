\documentclass{article}
\usepackage{amsmath}
\usepackage{amsfonts}
\usepackage{amssymb}
\usepackage{mathrsfs}
\usepackage{cancel}

\usepackage{graphicx}


\setlength\parindent{0pt}

\author{Pranav Tikkawar}
\title{Assignment 2}



\begin{document}
\maketitle
\section*{1}
The null hypothesis ($H_0$) should be that the mean amount spent per trip to Quartz Emporium ($\mu$) is equal to \$22.50. This is because the supermarket claims that the mean amount spent per trip is \$22.50. \\

The alternative hypothesis ($H_1$) should be that the mean amount spent per trip to Quartz Emporium ($\mu$) is more than \$22.50. This is because the WatchDogs suspect that the supermarket has discreetly hiked up their prices. \\

Null hypothesis ($H_0$): $\mu = 22.50$
Alternative hypothesis ($H_1$): $\mu > 22.50$
\section*{2}
With the sample mean of 25.8364 and a sample standard deviation of 16.15334, the standardized test statistic evaluates to 1.460497

\section*{3}
They would want a test statistics that is greater then 1.28 to be 90\% certain that they do not make a false claim.

\section*{4}
The WatchDogs should reject the null hypothesis, as a test statistic of 1.46 is greater than 1.28. Therefore it is likely that the mean amount spent per trip to Quartz Emporium is more than \$22.50.

\section*{5}
The p-value for the hypothesis test is 0.072145

\section*{6}
The smallest level of significance the hypothesis test could be rejected at is 0.072145. This is (not coincidentally) the p-value of the hypothesis test.

\end{document}
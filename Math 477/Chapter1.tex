\documentclass{article}
\usepackage{amsmath}
\usepackage{amsfonts}
\usepackage{amssymb}
\usepackage{mathrsfs}
\usepackage{cancel}

\usepackage{graphicx}


\setlength\parindent{0pt}

\author{Pranav Tikkawar}
\title{Math Theory of Probability}

\begin{document}
\maketitle
\tableofcontents
\section{Chapter 1: Combinatorial Analysis}
\subsection*{5/28}
\textbf{Basic Principle of Counting.}\\
Suppose that 2 experiments are to be preformed. Then if exp 1 can result in any one of $n_1$ possible outcomes and for each of these outcomes, exp 2 can result in any one of $n_2$ possible outcomes, then the total number of possible outcomes for the 2 experiments is $n_1 \cdot n_2$.\\

\textbf{Permutations.}\\
How many ways are there of arranging $n$ distinct things?\\
There are $n$ ways to choose the first thing, $n-1$ ways to choose the second thing, $n-2$ ways to choose the third thing, and so on.\\
Thus, the total number of ways of arranging $n$ distinct things is $n \cdot (n-1) \cdot (n-2) \cdot \ldots \cdot 2 \cdot 1 = n!$\\

\textbf{Permutations with repeats.}\\
$$\frac{n!}{n_1! \cdot n_2! \cdot \ldots \cdot n_r!} $$
different permutation of $n$ objects which any arbitrary $n_i$ are alike.\\

\textbf{Combinations.}\\
$$ \binom{n}{r} = \frac{n!}{(n-r)!r!}$$
How many ways are there of choosing $r$ things from $n$ distinct things?
\subsection*{5/29}


\section{Chapter 2: Axioms of Probability}

\section{Chapter 3}

\section{Chapter 4}

\section{Chapter 5}



\end{document}
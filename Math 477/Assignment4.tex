\documentclass{article}
\usepackage{amsmath}
\usepackage{amsfonts}
\usepackage{amssymb}
\usepackage{mathrsfs}
\usepackage{cancel}

\usepackage{graphicx}


\setlength\parindent{0pt}

\author{Pranav Tikkawar}
\title{Assignment 4}

\begin{document}
\maketitle
\section*{Question 33}
To find what is the number of papers needed to maximize profit we need to take the Expected value of each number of papers and then find the maximum value.\\
Let $b$ be the number of papers bought, $s$ the number of papers sold, and $r$ the profit per paper sold.\\
With the price of buying a paper being 10 cents and the price of selling a paper being 15 cents, the profit per paper sold is 5 cents minus the left over papers.\\
Thus the formula used to find profit is $r = 0.05s - 0.10(b-s)$.\\ 
Now we need to multiply the probability of each of the number of papers being sold by the profit per paper sold.\\
With a binomial distribution of $n=10$ and $p=\frac{1}{3}$ the Expected value formula is 
$$ E(X) = \sum_{s=0}^{n} r \cdot \binom{n}{s}p^s(1-p)^{n-s}$$
For $b = 0$ the Expected value is 0\\
For $b = 1$ the Expected value is 4.740\\
For $b = 2$ the expected value is 8.185\\
For $b = 3$ the expected value is 8.700\\
For $b = 4$ the expected value is 5.315\\
Clearly the maximum expected value is when $b = 3$.\\

\section*{Question 54}
I would use a poisson distribution to  model the number of cars abandonded on a highway.\\
\subsection*{a}
No abandonded cars: $P(X=0) = \frac{e^{-2}2^0}{0!} = e^{-2}$

\subsection*{b}
At least two abandonded cars: $P(X \geq 2) = 1 - P(X < 2) = 1 - P(X = 0) - P(X = 1) = 1 - e^{-2} - 2e^{-2} = 1 - 3e^{-2}$

\section*{Question 75}
$P(X = x) = \binom{9+x}{x} (1/2)^10(1/2)^x$

\section*{Question 5}
$P(X = 0)$ is 1\\
Since $P(X=0) = 1 - P(X \neq 0) = 1 - P(X = 1)$ Thus $P(X\neq 0) = P(X = 1)$ and $P(X=0) + P(X=1) = 1$\\
Also $E(X) = 3Var(X) = 3E(X^2) - 3E(X)^2$.\\
Thus $1P(X=1) = 3(1P(X=1^2) - (1P(X=1))^2)$\\
Thus $P(X=1) = 0$\\
Therefore $P(X=0) = 1$ and $P(X \neq 0) = 0$\\
\end{document}
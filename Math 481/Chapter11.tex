\documentclass[answers,12pt,addpoints]{exam}
\usepackage{import}

\import{C:/Users/prana/OneDrive/Desktop/MathNotes}{style.tex}

% Header
\newcommand{\name}{Pranav Tikkawar}
\newcommand{\course}{01:XXX:XXX}
\newcommand{\assignment}{Homework n}
\author{\name}
\title{\course \ - \assignment}

\begin{document}
\maketitle
\section{Chapter 11: Confidence Intervals}
Given a $\alpha$ such that $0 < \alpha < 1$, an \underline{Interval Estimation} stratgey provides two statistics (r.v) $L$ and $R$ s.t $P(L < \theta < R) = 1 - \alpha$. The interval $[L, R]$ is called a $(1-\alpha)\%$ confidence interval for $\theta$.\\
We can say that if you repeat the expirment $N$ times and gte $N$ intervals, then $(1-\alpha)\%$ of the intervals will contain the true value of $\theta$.\\
\begin{definition}[Confidence Interval]
    A confidence interval with (1-$\alpha$) confidence level are two statistics $L$ and $R$ such that $P(L < \theta < R) = 1 - \alpha$.
\end{definition}
\begin{remark}
    If someone says after an expirment that $2 < \lambda < 2.1$ with 90\% confidence, it means that on average that 90\% of the intervals will contain the true value of $\lambda$.
\end{remark}
We want CI to be symetric about $\bar{X}$ 




\newpage
CI so far: 
$N(\mu, \sigma^2)$ population
\begin{enumerate}
    \item $\sigma^2$ known: $\bar{X} \pm z_{\alpha/2} \frac{\sigma}{\sqrt{n}}$
    \item $\sigma^2$ unknown: $\bar{X} \pm t_{\alpha/2, n-1} \frac{S}{\sqrt{n}}$
\end{enumerate}
We can think of $z_{\alpha/2}$ in the normal curve as the shaded area in the tails.\\\\
\textbf{New Context:}
Two pops $N(\mu_1, \sigma_1^2)$ and $N(\mu_2, \sigma_2^2)$ and sample from both \\
$n_1$ and $n_2$ from each
$$X_{11} \ldots X_{1n_1} \sim N(\mu_1, \sigma_1^2)$$
$$X_{21} \ldots X_{2n_2} \sim N(\mu_2, \sigma_2^2)$$
These are independent but not necessarily identically distributed.\\
Let $\bar{X}_1$ and $\bar{X}_2$ be the sample means and $S_1^2$ and $S_2^2$ be the sample variances.\\
Want CI for $\mu_1 - \mu_2$\\
\textbf{Case 1: } $\sigma_1^2$ and $\sigma_2^2$ are known.\\
We can use point estimators:
$$\bar{X}_1 - \bar{X}_2 \sim N(\mu_1 - \mu_2 , \frac{\sigma_1^2}{n_1} + \frac{\sigma_2^2}{n_2})$$
$$\bar{X}_1 - \bar{X}_2 \pm z_{\alpha/2} \sqrt{\frac{\sigma_1^2}{n_1} + \frac{\sigma_2^2}{n_2}}$$
\textbf{Case 2: } $\sigma_1^2$ and $\sigma_2^2$ are unknown. but $\sigma_1^2 = \sigma_2^2$\\
we can define a pooled sample variance:
$$S_p^2 = \frac{(n_1 - 1)S_1^2 + (n_2 - 1)S_2^2}{n_1 + n_2 - 2}$$
\begin{remark}
    This is a weighted average of the sample variances. with weights $n_1 - 1$ and $n_2 - 1$
\end{remark}
\begin{remark}
    $$S_p^2 := \frac{\sum^{n_1} (X_{1i} - \bar{X_1})^2 + \sum^{n_2}( X_{2i} + \bar{X_2} )^2}{n_1 + n_2 - 2}$$
    $$ \sim \chi^2_{n_1 + n_2 - 2}$$
\end{remark}










\end{document}
\documentclass{article}
\usepackage{amsmath}
\usepackage{amsfonts}
\usepackage{amssymb}
\usepackage{mathrsfs}
\usepackage{dsfont}
\usepackage{cancel}

\usepackage{graphicx}


\newcommand{\prob}[1]{\mathds{P}(#1)}
\newcommand{\expec}[1]{\mathds{E}(#1)}
\newcommand{\var}[1]{\text{Var}(#1)}
\newcommand{\ex}[1]{\textbf{Example #1}}


\setlength\parindent{0pt}

\author{Pranav Tikkawar}
\title{TODO}

\begin{document}
\maketitle

\section{Probability Review}
\subsection*{Moment Generating Functions}
Suppose $X$ is a random variable. The $r$th moment of $X$ about the origin is defined as
$$\mu_r' := \expec{X^r} = \int x^r f(x) dx$$
where $f(x)$ is the PDF. \\
The first moment is the mean indicated by $\mu$ \\
The $r$th moment about the mean is defined as
$$ \mu_r := \expec{(X - \mu)^r} = \int (x - \mu)^r f(x) dx$$
$\mu_2$ is the variance of $X$ indicated by $\sigma^2$ and is always non-negative \\
$\var{x} = \expec{X^2} - \expec{X}^2$ \\
A random variable $X$ taking values in $\mathds{R}$ is said to be norm with parameter $\mu$ and $\sigma^2$ if its PDF is given by
$$f(x) = \frac{1}{\sqrt{2\pi \sigma^2}} e^{-\frac{(x - \mu)^2}{2\sigma^2}}$$
Case: $\mu = 0$ and $\sigma^2 = 1$ is called the standard normal distribution. \\
\textbf{Moment Generating Function} \\
The moment generating function of a random variable $X$ is defined as 
$$M_X(t) = \expec{e^{tX}} = \int e^{tx} f(x)dx $$
Note $e^{tx} = 1 + tx + \frac{(tx)^2}{2!} + \frac{(tx)^3}{3!} + ...$ \\
Can also be considered as 
$$ M_X(t) = \sum_{n=0}^{\infty} \frac{t^n \mu_n'}{n!}$$
where $\mu_n$ is the $n$th moment of $X$ about the origin. \\
$$M_X(t) = \expec{e^{tX}} = \expec{1 + tX + \frac{(tX)^2}{2!} + \frac{(tX)^3}{3!} + ...}$$
$$M_X(t)' = \expec{Xe^{tX}} = \expec{X} + \expec{X^2}t + \expec{X^3}\frac{t^2}{2!} + ...$$
$$M_X(0)' = \expec{X} $$
$$M_X(0)^{(n)} = \mu_n(x) = \expec{X^n}$$





% Write this later !!!
% $$\int x \frac{e^{(-(x-\mu)^2)/2\sigma^2}}{\sqrt{2\pi \sigma^2}} dx$$
% $$ z = (-(x-\mu)^2)/2\sigma^2 $$
% $$ dz = -(x-\mu)/\sigma^2 dx$$
% $$ -\sigma^2 dz = (x-\mu) dx$$
% $$  \int x \frac{e^{(-(x-\mu)^2)/2\sigma^2}}{\sqrt{2\pi \sigma^2}} - \mu \frac{e^{(-(x-\mu)^2)/2\sigma^2}}{\sqrt{2\pi \sigma^2}}  + \mu \frac{e^{(-(x-\mu)^2)/2\sigma^2}}{\sqrt{2\pi \sigma^2}}  dx $$
% $$ \int (x-\mu)\frac{e^{(-(x-\mu)^2)/2\sigma^2}}{\sqrt{2\pi \sigma^2}}dx  + \int \mu \frac{e^{(-(x-\mu)^2)/2\sigma^2}}{\sqrt{2\pi \sigma^2}} dx $$
% $$\int -\sigma^2 e^{-z} dz + \mu \int \frac{e^{(-(x-\mu)^2)/2\sigma^2}}{\sqrt{2\pi \sigma^2}} dx$$



\end{document}
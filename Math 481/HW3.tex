\documentclass[answers,12pt,addpoints]{exam}
\usepackage{import}

\import{C:/Users/prana/OneDrive/Desktop/MathNotes}{style.tex}

% Header
\newcommand{\name}{Pranav Tikkawar}
\newcommand{\course}{01:640:481}
\newcommand{\assignment}{Homework 3}
\author{\name}
\title{\course \ - \assignment}

\begin{document}
\maketitle


\newpage
\begin{questions}
    \question Question 1. 8.23\\
    Variance of sample variance of a normal family. (Hint: See the text for a theorem that tells you about the distribution of sample variance of a normal family. We discussed the variance of that kind of distributions.) [3O1]\\
Use Theorem 11 to show that, for random samples of size n from a normal
population with the variance $\sigma^2$, the sampling distribution of $S^2$ has the mean $\sigma^2$
and the variance $\frac{2\sigma^4}{n-1}$ . (A general formula for the variance of S2 for
random samples from any population with finite second and fourth moments may be
found in the book by H\@. Cramer listed among the references at the end of ´
this chapter.)\\
\textbf{Solution:}\\
We know that the sampling distribution of $(n-1)S^2/\sigma^2$ is $\chi^2_{n-1}$.\\
We can clearly see that the mean of $S_2 = \sigma^2$ due to properties of expectaion and the chi-square distribution as
\begin{align*}
    E((n-1)S^2 / \sigma^2) &= (n-1) \\
    E(S^2) &= \sigma^2
\end{align*}
Now, we can find the variance of $S^2$ as follows:
\begin{align*}
    Var((n-1)S^2/ \sigma^2) &= 2(n-1)
    Var(S^2) &= \frac{2\sigma^4}{n-1}
\end{align*}
\begin{solution}
    Clearly the mean of $S^2$ is $\sigma^2$ and the variance is $\frac{2\sigma^4}{n-1}$.
\end{solution}
    \question Question 2. 6.7\\
    Prove the important recursive property of the gamma function integral.\\
Use integration by parts to show that \(\Gamma(\alpha) = (\alpha - 1) \cdot \Gamma(\alpha - 1)\) for \(\alpha > 1\).\\
\textbf{Solution:}\\
We know that the gamma function is defined as:
\begin{align*}
    \Gamma(\alpha) = \int_0^\infty x^{\alpha - 1}e^{-x}dx
\end{align*}
Now, we can use integration by parts to prove the recursive property of the gamma function integral.
\begin{align*}
    \Gamma(\alpha) &= \int_0^\infty x^{\alpha - 1}e^{-x}dx\\
    &= \left[-x^{\alpha - 1}e^{-x}\right]_0^\infty + (\alpha - 1)\int_0^\infty x^{\alpha - 2}e^{-x}dx\\
    &= 0 + (\alpha - 1)\int_0^\infty x^{\alpha - 2}e^{-x}dx\\
    &= (\alpha - 1)\Gamma(\alpha - 1)
\end{align*}
\begin{solution}
    We have proved that \(\Gamma(\alpha) = (\alpha - 1) \cdot \Gamma(\alpha - 1)\) for \(\alpha > 1\).
\end{solution}
    \question Question 3. 6.11\\
        Understanding the gamma distribution's pdf.\\
        Show that a gamma distribution with $\alpha > 1$ has a relative maximum at $x = \beta(\alpha - 1)$. What happens when
    $0 < \alpha < 1$ and when $\alpha = 1$?\\
    \textbf{Solution:}\\
    The probability density function (pdf) of a gamma distribution is given by:
    \begin{align*}
        f(x; \alpha, \beta) = \frac{ x^{\alpha-1} e^{-x/\beta}}{\beta^\alpha \Gamma(\alpha)}
    \end{align*}
    To find the relative maximum, we take the derivative of the pdf with respect to $x$ and set it to zero:
    \begin{align*}
        \frac{d}{dx} f(x; \alpha, \beta) &= \frac{d}{dx} \left( \frac{x^{\alpha-1} e^{-x/\beta}}{\beta^\alpha \Gamma(\alpha)} \right) \\
        &= \frac{d}{dx} \left( x^{\alpha-1} e^{- x/\beta} \right) \cdot \frac{1}{\beta^\alpha \Gamma(\alpha)} \\
        &= \left( (\alpha - 1)x^{\alpha-2} e^{- x/\beta} - \frac{1}{\beta} x^{\alpha-1} e^{- x/\beta} \right) \cdot \frac{1}{\beta^\alpha \Gamma(\alpha)} \\
        &= x^{\alpha-2} e^{- x/\beta} \left( (\alpha - 1) -  \frac{x}{\beta} \right) \cdot \frac{1}{\beta^\alpha \Gamma(\alpha)} 
    \end{align*}
    Clearly this is only zero when $\alpha -1 - x/\beta = 0$ \\
    \begin{solution}
            $$x = \beta(\alpha - 1)$$
    \end{solution}
    We know that for $0 < \alpha < 1$ the gamma distribution is increasing, and for $\alpha < 1$ the gamma distribution is decreasing.
    And for $\alpha = 1$, the gamma distribution is an exponential distribution.
    \question Question 4. 8.45\\
    Verify the results of Example 4, that is, the sampling distributions of $Y_1$, $Y_n$, and \(\tilde{X}\) shown there for random samples from an exponential population.\\
    \textbf{Solution:}\\
    The exponential distribution is given by:
    \begin{align*}
        f(x; \lambda) = \frac{1}{\lambda} e^{-x/ \lambda }
    \end{align*}
    The mean and variance of the exponential distribution are given by:
    \begin{align*}
        E(X) &= 1/\lambda \\
        Var(X) &= 1/\lambda^2
    \end{align*}
    We want to verify the sample distribution of $Y_1$, $Y_n$, and $\tilde{X}$ and we can utilize theorem 16 to do so. \\
    Theorem 16 for an exponential distribution states that: 
    $$ g_r(y_r) = \frac{n!}{(r-1)! (n-r)!} [\int_0^{y_r}\frac{1}{\lambda} e^{-x/ \lambda} dx]^{r-1} \frac{1}{\lambda} e^{-x/ \lambda} [\int_{y_r}^\infty \frac{1}{\lambda} e^{-x/ \lambda} dx]^{n-r}$$
    We can use this theorem to verify the results of Example 4.\\
    The pdf of $Y_1$ is given by:
    \begin{align*}
        f(y_1) &= \frac{n}{\lambda} [e^{-y_1 /\lambda} * (e^{-y_1 /\lambda})^{n-1}]\\
        &= \frac{n}{\lambda} [e^{-ny_r /\lambda} ] 
    \end{align*}
    The pdf of $Y_n$ is given by:
    \begin{align*}
        f(y_n) &= \frac{n}{\lambda} [(1-(e^{-y_n /\lambda})^{n-1}] * (e^{-y_n /\lambda})]
    \end{align*}
    The pdf of $\tilde{X}$ (where $n = 2m+1$) is given by:
    \begin{align*}
        f(\tilde{x}) &= \frac{(2m+1)!}{m!m!} \frac{1}{\lambda} e^{-\tilde{x}/ \lambda} \left[ \int_0^{\tilde{x}} \frac{1}{\lambda} e^{-x/ \lambda} dx \right]^m \left[ \int_{\tilde{x}}^\infty \frac{1}{\lambda} e^{-x/ \lambda} dx \right]^m\\
        &= \frac{(2m+1)!}{m!m!} \frac{1}{\lambda} e^{-\tilde{x}/ \lambda} \left[ 1 - e^{-\tilde{x}/ \lambda} \right]^m \left[ e^{-\tilde{x}/ \lambda} \right]^m\\
        &= \frac{(2m+1)!}{m!m!} \frac{1}{\lambda} \left[ 1 - e^{-\tilde{x}/ \lambda} \right]^m \left[ e^{-\tilde{x}/ \lambda} \right]^{m+1}
    \end{align*}
    
    \question Question 5. 8.48\\
    Find the mean and the variance of the sampling distribution of Y1 for random samples of size n from the population of Exercise 46\\
    We are considering the distribution \\
    \textbf{Solution:}\\
    The sample distribution of $Y_1$:
    \begin{align*}
        f(y_1) &= n (1-y_1)^{n-1}
    \end{align*} 
    The mean of $Y_1$ is given by:
    \begin{align*}
        E(Y_1) &= \int_0^1 y_1 n (1-y_1)^{n-1} dy_1\\
        &= \int_0^1 n (1-y_1)^{n-1} dy_1\\
        &= (-y(1-y)^n) \Big|_0^1 + \int_0^1 (1-y)^n dy\\
        &= \frac{1}{n+1}(1-y)^{n+1} \Big|_0^1\\
        &= \frac{1}{n+1}
    \end{align*}
    The variance of $Y_1$ is given by:
    \begin{align*}
        Var(Y_1) &= E(Y_1^2) - E(Y_1)^2\\
        &= \int_0^1 y_1^2 n (1-y_1)^{n-1} dy_1 - \left( \frac{1}{n+1} \right)^2\\
    \end{align*}
    We know that 
    $$ \int_0^1 y_1^2 n (1-y_1)^{n-1} dy_1 = \frac{2}{(n+1)(n+2)}$$
    Therefore, the variance of $Y_1$ is given by:
    \begin{align*}
        Var(Y_1) &= \frac{2}{(n+1)(n+2)} - \left( \frac{1}{n+1} \right)^2\\
        &= \frac{n}{(n+1)^2(n+2)}
    \end{align*}
    \begin{solution}
        The mean of $Y_1$ is $\frac{1}{n+1}$ and the variance of $Y_1$ is $\frac{n}{(n+1)^2(n+2)}$.
    \end{solution}


\end{questions}

\end{document}
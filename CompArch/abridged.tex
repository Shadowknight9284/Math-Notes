`\documentclass[answers,12pt,addpoints]{exam}
\usepackage{import}
\usepackage{multicol}

\import{C:/Users/prana/OneDrive/Desktop/MathNotes}{style.tex}

\usepackage[margin=0in, includehead, includefoot, headheight=0pt]{geometry}
\usepackage{setspace}
\setstretch{0}

% Header
\newcommand{\name}{Pranav Tikkawar}
\newcommand{\course}{01:XXX:XXX}
\newcommand{\assignment}{Homework n}
\author{\name}
\title{\course \ - \assignment}

\begin{document}


% Midterm: Wednesday, October 30, during class

% You may bring one page of notes (single-sided)
%     preferably handwritten, or typed at least 10 points
%     no calculators, computers, smart devices
    
% Midterm topics:

% Trends in computer architecture
%     - major components: CPU, memory, System bus, 
%                         IO bus, storage, IO peripherals
%     - exponential growth of transistor count -> Moore's Law
%     - processor-memory speed gap  -> caches
%     - "power wall"/"multicore crisis"

% C programming
%     - basic C syntax
%         - function and variable declarations
%         - control structures
%             if/else, switch, while, do/while, for
%         - structs
%             bundle together elements, possibly of different types
            
%             void foo(struct example e);  // struct passed by value
%             void foo(struct example *e); // struct passed by reference
%                                 (actually a pointer passed by value)

%         - arrays and pointers
%             arrays are objects in memory
%             array variables are always associated with a specific array
            
%             int a[20]; 
%                 automatically created and destroyed
%                 cannot be reassigned
%                 a = b  (not allowed)
%                 a[i] = n;
                
%             int *p; 
%             p = a;  // now p and a both refer to the same array
%             p[i] = n;
            
%             p = (int *) malloc(20 * sizeof(int));
%                 // now p refers to an array on the heap
%             p[i] = n;   // still allowed
        
%             memset(p, 0, 20 * sizeof(int));
        
%             p = (int *) calloc(20, sizeof(int));
            
%             sizeof(p)  == sizeof(int *)
            
            
%             int m[4][5]; // m is array of 4 arrays of 5 ints
%             m[i][j]
            
%             int **p;  // p points to an array of pointers to arrays of ints
%             p = malloc(4 * sizeof(int *));
%             p[i] = malloc(5 * sizeof(int));
%             p[i][j]  
            
            
%             int n;
%             int *p;
%             int **q;
%             p = &n;
%             q = &p;
            
%             n, *p, **q, p[0], q[0][0]   -- all the same
        
%         - typedef
        
%             typedef struct list node;
%                 "node" is the same type as "struct list"
                
%             typedef double vector[3];
%                 "vector" is the same type as "double [3]"
                
%                 vector x;     double x[3];

%     - memory management
%         - address vs value
        
%             make sure to be clear whether you are talking about
%             the value of an object or its location
            
%             int x;
%                 x = 1;
%                 y = x;
        
%         - using unary & and *
        
%             int x;
            
%                 x  - value in x / data stored in x
%                 &x - location of x / pointer to x
                
                
%             int *p;
%             p = &x;
            
%                 p  - pointer stored in p
%                 *p - value of the integer p points to
%                 &p - location of p / pointer to p
                
                
%                 *&p === p
%                 &*p === p
%                 *&*p === *p
        
%         - malloc and free
        
%             use malloc to create objects on the heap
%             -> not directly associated with any variable
            
%             use free to deallocate heap objects
            
%             int *p = ...
%             int *q = malloc(20 * sizeof(int));
%             q = p;  // whoops, the object we just allocated has been lost
        
%     - C strings
%         - trailing null terminator ('\0'; don't confuse with '0')
%             allows for string size to be different from array size
            
%             "foo"   <- treated as pointer to an array in memory
%                 <- terminator automatically added by compiler
                
%             char s[10] = "foo";   // initializes first 4 elements of s
%             char s[10] = {'f', 'o', 'o', '\0'};
            
            
%             s = "bar";  not allowed
%             strcpy(s, "bar");
            
            
%             char *p = malloc(20);
%             char *q = malloc(20);
%             strcpy(p, "foo");
            
%             q = p;    // same object
%             strcpy(q, p);  // same string, different objects
            
            
%         - strcpy(), strlen(), strcmp(), strcat()
%         - memcpy(), memmove()
        
%     - printf, fprintf, sprintf
%             printf("secret number is %d!\n", secret_number)
            
%             char str[100];
%             sprintf(str, "%d", 1024);
%             // now str contains "1024" (4 chars + terminator)
            
%             snprintf(str, 100, "%d", 1024);
        
%     - scanf, fscanf, sscanf
    
%             scanf("%d", &secret_number);
            
%             scanf("%lf", &matrix[i][j]);
    
%     - fopen, fclose

% Data representation

%         00101100  = 32+8+4

%     - binary (unsigned) integers                
%         - fixed-width arithmetic
%         - differences between sign-magnitude, 1s' complement, 2's complement
%             sign bit, other bits
        
%         - format and range of 2's complement signed integers
        
%             11111111
        
%             10000000 ... 01111111
%             -128 ... 127
%             -2^(b-1)  ... 2^(b-1)-1
        
%     - hexadecimal and octal
%         FF 
%         7F
%         80
        
        
%         10(hex) = 16(decimal)
    
%     - IEEE floating point (general)
%         sign bit, exponent bits, significand (fractional) bits
        
%             1.ffffffffff * 2^(eeeeee - bias)
    
%         - exponent bias
%             bias(E bits) = 2^(E-1)-1
%         - special exponent values
%             all zero: subnormal values and zero
%                 implicit integer bit is 0, not 1
%                 exponent is 1 - bias
%             all ones: infinity and NaNs
            
%         - interpretation of normal, zero, denormal, infinite, and NaN values

% Assembly
%     - high-level ideas
%         - fetch/execute cycle
%         - opcode and operands
%         - operation types: mov, arithmetic and logic, control
%     - IA32 / X86 assembly (AT&T / GAS style)
%         - 2-argument form for add, sub, etc.
%                 addl %eax, %ebx    # ebx = ebx + eax
%                 subl %eax, %ebx    # ebx = ebx - eax
%         - operand types: byte/word/double word
%         - register names
%         - mov instruction
%         - operands: immediate, absolute, direct, indirect
            
%             movl  $100, %eax  ; EAX = 100
%             movl  %ebx, %eax  ; EAX = EBX
%             movl  100, %eax   ; EAX = memory[100 ... 103]
%             movl  (%ebx), %eax ; EAX = memory[EBX ... EBX+3]
            
%             movb  (%ebx), %al  ; EAX[0] = memory[EBX]
%             movzbl (%ebx), %eax ; EAX = memory[EBX]
            
%             movl (%ebx,%ecx), %eax ; EAX = memory[EBX+ECX .. EBX+ECX+3]
%             movl 100(%ebx,%ecx,4), %eax
%                 ; EAX = memory[10+EBX+ECX*4 .. 10+EBX+ECX*4+3]
        
%         - lea
%         - push, pop
        
%             pushl %eax  ; ESP = ESP - 4; Memory[ESP..ESP+3] = EAX

%     - call, ret

\begin{center}
    \Large \textbf{Computer Architecture Cheatsheet}
\end{center}

\begin{multicols}{2}
    \begin{itemize}
        \item \textbf{Trends in Computer Architecture}
        \begin{itemize}
            \item Major components: CPU, memory, System bus, IO bus, storage, IO peripherals
            \item Exponential growth of transistor count $\rightarrow$ Moore's Law
            \item Processor-memory speed gap $\rightarrow$ caches
            \item "Power wall"/"multi-core crisis"
        \end{itemize}
        
        \item \textbf{C Programming}
        \begin{itemize}
            \item Basic C syntax: function and variable declarations, control structures (if/else, switch, while, do/while, for), structures
            \item Arrays and pointers: arrays are objects in memory, array variables are always associated with a specific array
            \item Memory management: address vs value, using unary \& and *, malloc and free
            \item C strings: trailing null terminator, \texttt{strcpy()}, \texttt{strlen()}, \texttt{strcmp()}, \texttt{strcat()}, \texttt{memcpy()}, \texttt{memmove()}
            \item \texttt{printf}, \texttt{fprintf}, \texttt{sprintf}, \texttt{scanf}, \texttt{fscanf}, \texttt{sscanf}, \texttt{fopen}, \texttt{fclose}
        \end{itemize}
        
        \item \textbf{Data Representation}
        \begin{itemize}
            \item Binary (unsigned) integers: fixed-width arithmetic, sign-magnitude, 1's complement, 2's complement
            \item Hexadecimal and octal
            \item IEEE floating point: sign bit, exponent bits, significant (fractional) bits, exponent bias, special exponent values
        \end{itemize}
        
        \item \textbf{Assembly}
        \begin{itemize}
            \item High-level ideas: fetch/execute cycle, opcode and operands, operation types (mov, arithmetic and logic, control)
            \item IA32 / X86 assembly (AT\&T / GAS style): 2-argument form for add, sub, etc., operand types (byte/word/double word), register names, mov instruction, operands (immediate, absolute, direct, indirect)
            \item lea, push, pop, call, ret
        \end{itemize}
    \end{itemize}
\setlength{\baselineskip}{12pt}
\end{multicols}

\end{document}'
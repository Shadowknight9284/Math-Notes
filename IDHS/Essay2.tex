\documentclass[answers,12pt,addpoints]{exam} 
\usepackage{import}

\import{C:/Users/prana/OneDrive/Desktop/MathNotes}{style.tex}

% Header
\newcommand{\name}{Pranav Tikkawar}
\newcommand{\course}{Interdisciplinary Honors Seminar}
\newcommand{\assignment}{Emotional Nash Equilibrium Flawed Assumptions}
\author{\name}
\title{\course \ - \assignment}

\begin{document}
\maketitle


Psychology and Game theory are two fields that work hand in hand to explain human behavior and decision-making. Especially when considering non-cooperative multiplayer games, where a collection of people have limited knowledge about their competitors' decisions, blending these two disciplines proves to be a great theoretical baseline for understanding individual choice. A situation in which each player has a collection of choices that result in their best outcome, given that their competitors also choose the most rational decisions, is called a Nash Equilibrium. There are two major assumptions required for this analysis/equilibrium to hold: one of infinite rationality, where each player can reason infinitely into the future and decide on the best outcome from that, and a detachment from the negative emotions that may arise from the other players' choices. Although these assumptions seem lofty, with careful consideration of what can be inferred from the Nash Equilibrium, we can derive even better models for intrapersonal decision-making.  

While emotions provide a powerful heuristic for decision-making, especially in small steps into the future, they can be detrimental to the longer-term analysis of rational decisions. Computing a sensible choice for a person is a demanding task that usually requires a significant amount of time. However, emotions and heuristics provide computational shortcuts. Computing \textbf{the} rational choice would require infinite recursive reasoning for each party, which is nearly impossible, especially when not every individual has the endless rationality assumption. A prominent example is when a person stumbles into a deal that seems "too good to be true." Although the agreement is obviously profitable, some may be wary of such a deal due to not knowing the intentions of the other party\footnote{Camerer, 1997}. This is a good survival instinct, but it is not always congruent with what the optimal personal choice would be.

Additionally, when making a decision, people don't only consider the utility their actions have for themselves, but also the utility they may bring to the people around them, even if those people are opponents. An example of this is a game where friends play a gambling game, where a player's win means their opponent must lose money. Especially in cases where one player is significantly more skilled than another, the experienced player may lower stakes or intentionally sandbag, as the guilt associated with an unfair game is uncomfortable \footnote{Irlenbusch \& Villeval, 2015}. While it is obviously not optimal to play suboptimally, there is sometimes utility in making your peers and adversaries happy by making yourself lose. Thus showing that solely looking at each actor in a non-cooperative game as an unemotional yet rational actor does not fully encompass the actual utility of certain decisions in real life.

Relaxing these assumptions and actually incorporating them into the analysis yields positive results. When considering the emotional and mental state of a player as a decision in and of itself, the Nash equilibrium analysis can still be computed. This model can help approach non-material incentives and disincentives, such as pride, guilt, and envy, within the framework of an equilibrium equation, similar to the classical model. The emotional state choice also interacts similarly to regular decision choice when it comes to deciding the best payoff, where the best selection of emotional state is chosen to respond to each other's emotional state \footnote{Winter, Méndez-Naya, \& García-Jurado, 2016}. The power of this model lies in its ability to explicitly value things other than material payoff, such as relationships. Although it still has flaws, such as the assumption that each player can choose their mental state and the assumption of accurately assessing another person's mental state, it is a step that incorporates a more analytical view into psychology through the lens of game theory.
\\\\
\textbf{References}
\begin{enumerate}
    \item[1] Camerer, Colin F. (1997). Progress in behavioral game theory. Journal of Economic Perspectives, 11(4), 167–188.
    \item[2] Irlenbusch, Bernd, \& Villeval, Marie Claire. (2015). Behavioral ethics: How psychology influenced economics and how economics might inform psychology? Current Opinion in Psychology, 6, 87–92. https://doi.org/10.1016/j.copsyc.2015.04.004
    \item[3] Eyal Winter, Luciano Méndez-Naya, Ignacio García-Jurado (2016) Mental Equilibrium and Strategic Emotions. Management Science 63(5):1302-1317. \\
    https://doi.org/10.1287/mnsc.2015.2398 
\end{enumerate}



\end{document}
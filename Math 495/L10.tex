\documentclass[answers,12pt,addpoints]{exam}
\usepackage{import}

\import{C:/Users/prana/OneDrive/Desktop/MathNotes}{style.tex}

% Header
\newcommand{\name}{Pranav Tikkawar}
\newcommand{\course}{01:640:495}
\newcommand{\assignment}{Lecture 10}
\author{\name}
\title{\course \ - \assignment}

\begin{document}
\maketitle


\begin{questions}
    \question Following are two figures from the article “A singularly Valuable Decomposition: The SVD of a Matrix” by Dan Kalman. Can you explain the idea the figures are communicating?
    \begin{solution}
        The first figure shows orthogonal projection of a basis of \( \mathbb{R}^3 \) onto a plane. Then it shows the operation of multiplying the projected basis to another basis. \\
        The second figure shows the null space of \( \mathbf{A} \) mapping to the range of \( \mathbf{A}^T \) and the null space of \( \mathbf{A}^T \) mapping to the range of \( \mathbf{A} \).
    \end{solution}
    \question Suppose \( \mathbf{v}_1, \mathbf{v}_2 \) are eigenvectors of \( \mathbf{A}^T \mathbf{A} \) that are orthogonal \( \mathbf{v}_1 \cdot \mathbf{v}_2 = 0 \). Show their images under \( \mathbf{A} \) are orthogonal - that is \( \mathbf{A} \mathbf{v}_1 \cdot \mathbf{A} \mathbf{v}_2 = 0 \).
    \begin{solution}
        We can see that 
        \begin{align*}
            \langle Av_1 , Av_2 \rangle &= (Av_1)^T (Av_2) \\
            &= v_1^T A^T A v_2 \\
            &= v_1^T \lambda_1 v_2 \\
            &= \lambda_1 v_1^T v_2 \\
            &= 0
        \end{align*}
    \end{solution}
    \question Let   
    \(
    \mathbf{A} =
    \begin{bmatrix}
    1 & -1 \\
    -1 & 2 \\
    2 & -1
    \end{bmatrix}
    \)
    For your convenience, you are told that 
    \(
    \mathbf{A}^T \mathbf{A} =
    \begin{bmatrix}
    6 & -5 \\
    -5 & 6
    \end{bmatrix}
    \)
    has eigenvalues \( \lambda = 11, 1 \) with eigenvectors 
    \(
    \begin{bmatrix}
    -1 \\
    1
    \end{bmatrix}
    ,
    \begin{bmatrix}
    1 \\
    1
    \end{bmatrix}
    \)
    respectively a basis for the two eigenspaces (null space (kernel) of \( \mathbf{A}^T \mathbf{A} - \lambda \mathbf{I} \)). Also, 
    \(
    \mathbf{A} \mathbf{A}^T =
    \begin{bmatrix}
    2 & -3 & 3 \\
    -3 & 5 & -4 \\
    3 & -4 & 5
    \end{bmatrix}
    \)
    has eigenvalues \( \lambda = 1, 0, 11 \) with eigenvectors 
    \(
    \begin{bmatrix}
    0 \\
    1 \\
    1
    \end{bmatrix}
    ,
    \begin{bmatrix}
    -3 \\
    -1 \\
    1
    \end{bmatrix}
    ,
    \begin{bmatrix}
    \frac{2}{3} \\
    -1 \\
    1
    \end{bmatrix}
    \)
    respectively a basis for the three eigenspaces (null space (kernel) of \( \mathbf{A} \mathbf{A}^T - \lambda \mathbf{I} \)).

    \begin{parts}
    \part Obtain a singular value decomposition of \( \mathbf{A} \).
    \begin{solution}
        For our SVD we have 
        \begin{align*}
            \Sigma &= \begin{bmatrix}
            \sqrt{11} & 0 \\
            0 & 1 \\
            0 & 0
            \end{bmatrix} \\
            V &= \begin{bmatrix}
            \frac{-1}{\sqrt{2}} & \frac{1}{\sqrt{2}} \\
            \frac{1}{\sqrt{2}} & \frac{1}{\sqrt{2}}
            \end{bmatrix} \\
            U = & \begin{bmatrix}
                \frac{2}{\sqrt{22}} & 0 & \frac{-3}{\sqrt{10}} \\
                \frac{-3}{\sqrt{22}} & \frac{1}{\sqrt{2}} & \frac{-1}{\sqrt{10}} \\
                \frac{3}{\sqrt{22}} & \frac{1}{\sqrt{2}} & \frac{1}{\sqrt{10}}
            \end{bmatrix}
        \end{align*}

    \end{solution}
    \part Use it to write \( \mathbf{A} \) as sum of rank 1 projection matrices.
    \begin{solution}
        We can write out the SVD as
        \begin{align*}
            11 \begin{bmatrix}
                \frac{2}{\sqrt{22}} \\
                \frac{-3}{\sqrt{22}} \\
                \frac{3}{\sqrt{22}}
            \end{bmatrix}
            \begin{bmatrix}
                \frac{-1}{\sqrt{2}} & \frac{1}{\sqrt{2}}
            \end{bmatrix}
            + 1 \begin{bmatrix}
                0 \\
                \frac{1}{\sqrt{2}} \\
                \frac{1}{\sqrt{2}}
            \end{bmatrix}
            \begin{bmatrix}
                \frac{1}{\sqrt{2}} & \frac{1}{\sqrt{2}}
            \end{bmatrix}
        \end{align*}
    \end{solution}
    \part What is the rank of \( \mathbf{A} \)? Is \( \mathbf{A} \) a ‘full rank’ matrix?
    \begin{solution}
        The rank of \( \mathbf{A} \) is 2. It is not a full rank matrix.
    \end{solution}
    \part Determine the spectral norm of \( \mathbf{A} \). (Recall it is the maximum \( \| \mathbf{A} \mathbf{X} \| \) among \( \mathbf{X} \) with \( \| \mathbf{X} \| = 1 \) with \( \| \cdot \| \) denoting the usual Euclidean norm (length).)
    \begin{solution}
        The spectral norm of \( \mathbf{A} \) is \( \sqrt{11} \). It is the value of the largest singular value.
    \end{solution}
    \end{parts}
    \question Matrices \( \mathbf{A}, \mathbf{b} \) are given matrices of sizes \( m \times n \), \( m \times 1 \) respectively. And, \( \mathbf{x} \in \mathbb{R}^n \) variable. Consider the function \( f : \mathbb{R}^n \rightarrow \mathbb{R} \) defined by \( f (\mathbf{x}) = \| \mathbf{A} \mathbf{x} - \mathbf{b} \|_2 \).
    \begin{parts}
        \part Express \( f (\mathbf{x}) \) using matrix operations.
        \begin{solution}
            \begin{align*}
                f(x) &= (Ax - b)^T (Ax - b) \\
                &= x^T A^T A x + b^T b - b^T Ax - x^T A^T b\\
                &= \langle Ax , Ax \rangle + \langle b , b \rangle - \langle b , Ax \rangle - \langle Ax , b \rangle\\
                &= \langle Ax , Ax \rangle + \langle b , b \rangle - 2 \langle b , Ax \rangle\\
                &= || Ax|| + ||b|| - 2 \langle b , Ax \rangle
            \end{align*}
            This is essentially splitting up the norm into its components.
        \end{solution}
        \part Assuming the derivative exists, compute the derivative (gradient) linear map \( \frac{df}{d\mathbf{x}} (\mathbf{x}) : \mathbb{R}^n \rightarrow \mathbb{R} \). You may use \( \mathbf{h} \in \mathbb{R}^n \) as the argument of this linear map (so, for the “direction”). Examining the expression of this linear map, also determine the \( 1 \times n \) matrix of this linear map \( \frac{df}{d\mathbf{x}} (\mathbf{x}) \).
        \textit{Hint: Differentiate a curve passing through \( \mathbf{x} \) in the direction \( \mathbf{h} \) i.e, use \( \frac{df}{d\mathbf{x}} (\mathbf{x})(\mathbf{h}) = \frac{d}{dt} \bigg|_{t=0} f (\mathbf{x} + t\mathbf{h}) \) and previous part.}
        \begin{solution}
            Using $x+ th$ as $x$ in the previous part, we get
            \begin{align*}
                f(x+th) &= ||A(x+th)|| + ||b|| - 2 \langle b , A(x+th) \rangle\\
                &= x^T A^T A x + t x^T A^T A h + t h^T A^T A x + t^2 h^T A^T A h - 2 b^T A x - 2 t b^T A h + b^T b
            \end{align*}
            Taking the derivative of this with respect to $t$ and evaluating at $t=0$ gives us the derivative.
            \begin{align*}
                \frac{d}{dt} \bigg|_{t=0} f (x + th) &= \frac{d}{dt} \bigg|_{t=0} (||A(x+th)|| + ||b|| - 2 \langle b , A(x+th) \rangle)\\
                &= \frac{d}{dt} \bigg|_{t=0} x^T A^T A x + t x^T A^T A h + t h^T A^T A x + t^2 h^T A^T A h - 2 b^T A x - 2 t b^T A h + b^T b\\
                &= x^T A^T A h + h^T A^T A x - 2t h^T A^T A h - 2 b^T A h \bigg|_{t=0}\\
                &= 2 x^t A^T A h - 2b^T A h\\
                &= 2 (Ax - b)^T A h
            \end{align*}
            Thus we get 
            \begin{align*}
                \frac{d}{dx} f(x) = 2 (Ax - b)^T A
            \end{align*}
        \end{solution}
        \part Determine \( \mathbf{x} \) where the linear map \( \frac{df}{d\mathbf{x}} \) is zero. Do you recognize this scenario from earlier?
        \begin{solution}
            The linear map is zero then \( (Ax - b)^T A = 0 \). This is the normal equation for the least squares problem.
            \begin{align*}
                (Ax - b)^T A &= 0\\
                A^T A x &= A^T b\\
                x &= (A^T A)^{-1} A^T b
            \end{align*}
        \end{solution}
    \end{parts}


\end{questions}

\end{document}
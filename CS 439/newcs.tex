\documentclass[10pt,landscape]{article}
\usepackage[margin=0.5cm]{geometry}
\usepackage{multicol}
\setlength{\columnsep}{0.45cm}
\renewcommand{\familydefault}{\sfdefault}
\setlength{\parskip}{0pt}
\setlength{\parindent}{0pt}
\pagestyle{empty}

\begin{document}
\footnotesize
\begin{center}
    \textbf{CS 439 Exam 2 Cheat Sheet --- Machine Learning Fundamentals}
\end{center}
\vspace{-2mm}
\begin{multicols}{3}

%------------------ Linear Algebra, SVD, PCA -----------------
\section*{Linear Algebra \& SVD/PCA}
\textbf{Vectors \& Matrices}: \\
- Vector: $\vec{v} = (v_1, v_2, ..., v_n)$ \\
- Matrix: $A_{m\times n}$ \\
- Transpose: $A^\top$ \\
- Dot product: $\vec{a}\cdot\vec{b} = \sum_i a_i b_i$ \\

\textbf{Eigenvalues/Eigenvectors}: \\
- $A\vec{x} = \lambda \vec{x}$ \\
- Principal directions (PCA): Variance maximization. \\
- SVD: $A = U\Sigma V^\top$ \\
  $U, V$: orthogonal; $\Sigma$: singular values. Reduces dimensionality.

\textbf{PCA Steps}:\\
1. Standardize data. 2. Compute covariance. 3. Eigendecomposition. 4. Select top-$k$ directions.

%-------------------- Probability & Bayes ---------------------
\section*{Probability \& Bayes}
\textbf{Basic Rules}:\\
- $P(A \cap B) = P(A)P(B|A) = P(B)P(A|B)$ \\
- $P(A \cup B) = P(A) + P(B) - P(A \cap B)$ \\
- $P(A|\text{evidence}) = \frac{P(\text{evidence}|A)P(A)}{P(\text{evidence})}$ \\
- Independent: $P(A\cap B) = P(A)P(B)$ \\

\textbf{Random Variables}: Discrete or continuous.\\
\textbf{Expectation}: $E[X]=\sum xP(x)$ or $\int xp(x)dx$\\

\textbf{Bayes’ Theorem Example}:\\
If $P(G)=0.5$, $P(B)=0.5$, $P(C|G)=0.6$, $P(C|B)=0.1$, then\\
$P(G|C) = \frac{P(C|G)P(G)}{P(C|G)P(G)+P(C|B)P(B)}$\\

%--------------------- Naive Bayes -----------------------
\section*{Naive Bayes Classification}
\textbf{Assumes}: features independent conditioned on class.\\
Classify $x$: Compute $P(C_k|x) \propto \prod_j P(x_j|C_k)P(C_k)$\\
Pick class $k$ maximizing $P(C_k|x)$.\\

\textbf{Example}: Classify email as spam/not spam:\\
$P(\text{spam}|x) \propto P(w_1|\text{spam})...P(w_n|\text{spam})P(\text{spam})$

%--------------------- Linear Regression --------------------
\section*{Linear Regression}
\textbf{Model}: $h_\theta(x) = \theta_0 + \theta_1 x_1 + ... + \theta_n x_n$\\
\textbf{Loss}: $L(\theta) = \frac{1}{n}\sum (h_\theta(x^{(i)}) - y^{(i)})^2$\\
\textbf{Normal Equation}: $\theta = (X^\top X)^{-1}X^\top y$\\
\textbf{Assumptions}: Linearity, constant variance, independence.\\

\textbf{Overfitting}: Model fits training noise.\\
\textbf{Underfitting}: Model can't capture patterns.

%------------------ Gradient Descent --------------------
\section*{Gradient Descent}
Iteratively minimize loss $L(\theta)$.\\
Update: $\theta_j \leftarrow \theta_j - \alpha \frac{\partial L}{\partial \theta_j}$\\
\textbf{Stochastic}: Uses single data point per step.\\
\textbf{Batch}: Uses all data per step.\\
\textbf{Convergence}: Choose $\alpha$ (step size) carefully.

%------------------- Model Complexity ---------------------
\section*{Model Complexity \& Regularization}
\textbf{Bias}: Error from overly simplistic assumptions.\\
\textbf{Variance}: Error from model sensitivity to fluctuations.\\
\textbf{Tradeoff}: High bias $\to$ underfit; high variance $\to$ overfit.

\textbf{Regularization}: Penalizes large $\theta$ values.\\
- Ridge (L2): $L(\theta)+\lambda\sum_{j=1}^{n}\theta_j^2$\\
- Lasso (L1): $L(\theta)+\lambda\sum_{j=1}^{n}|\theta_j|$\\

\textbf{Choose $\lambda$ by cross-validation.}\\
Do not regularize $\theta_0$!\\

\textbf{Cross-validation}: Split training data into $k$ folds; ensures better generalization.

%---------------- Logistic Regression ----------------------
\section*{Logistic Regression}
\textbf{Classifies}: $y\in \{0,1\}$\\
Sigmoid activation: $g(z)=\frac{1}{1+e^{-z}}$\\
\textbf{Hypothesis}: $h_\theta(x)=g(\theta^\top x)$\\
\textbf{Probability Interpretation}: $h_\theta(x)=P(y=1|x,\theta)$\\

\textbf{Decision boundary}: $\theta^\top x=0$\\
\textbf{Loss}: $L(\theta) = -\frac{1}{n}\sum[y^{(i)}\log h_\theta(x^{(i)}) + (1-y^{(i)})\log(1-h_\theta(x^{(i)}))]$

%------------------ Classification Metrics ------------------
\section*{Classification Metrics}
\textbf{Confusion Matrix}: TP, FP, TN, FN.\\
\textbf{Precision}: $\frac{TP}{TP+FP}$\\
\textbf{Recall}: $\frac{TP}{TP+FN}$\\
\textbf{Accuracy}: $\frac{TP+TN}{TP+TN+FP+FN}$\\
\textbf{F1 Score}: $2\frac{\text{Precision}\cdot\text{Recall}}{\text{Precision}+\text{Recall}}$\\
Tune threshold for precision/recall tradeoff.

%--------------- Multiclass Classification ------------------
\section*{Multiclass Classification}
\textbf{OvA}: Train $k$ binary classifiers for $k$ classes.\\
Classifier $i$: $h_\theta^{(i)}(x)=P(y=i|x)$\\
Predict: $\arg\max_i h_\theta^{(i)}(x)$\\

%--------------------- K-means Clustering -------------------
\section*{K-means Clustering}
\textbf{Unsupervised learning}: No labels.\\
Input: $x_1,...,x_m\in\mathbb{R}^n$, $k$ clusters.\\
1. Randomly initialize centers $\mu_1,...,\mu_k$\\
2. Assign each point to closest center.\\
3. Update centers to cluster means.\\
4. Repeat until convergence.

\textbf{Objective}: Minimize $J=\sum_{i=1}^{m} ||x^{(i)}-\mu_{c^{(i)}}||^2$\\
\textbf{K-means++}: Smarter initialization picks distant points as centers.\\
\textbf{Elbow method}: Plot loss vs $k$; choose at "elbow".

%---------------- Hierarchical Clustering -------------------
\section*{Hierarchical Clustering}
Agglomerative: Start with each point in its own cluster; iteratively merge closest clusters.\\
Linkage methods:
\begin{itemize}
\item \textbf{Single}: Closest pair.
\item \textbf{Complete}: Furthest pair.
\item \textbf{Average}: Mean of all pairs.
\end{itemize}
Produces dendrogram showing cluster hierarchy.

Cut tree at desired depth to select $k$ clusters.

%---------------- Feature Engineering -------------------
\section*{Feature Engineering}
- \textbf{Selection}: Remove irrelevant/redundant features.
- \textbf{Extraction}: Build new features (e.g. PCA).
- \textbf{Scaling}: Standardize (zero mean, unit variance).
- \textbf{Encoding}: Convert categoricals (one-hot, ordinal).
- \textbf{Handling missing values}: Impute or drop.

%---------------- Miscellaneous Tips ------------------
\section*{Exam Tips \& Common Pitfalls}
- Always split data into train/test sets (never look at test until the end!).
- Use cross-validation for hyperparameter tuning.
- Regularization cures overfitting, not underfitting.
- Don't regularize intercept term.
- When multiclass, use one-vs-all for logistic regression.
- For K-means, rerun multiple times to avoid poor local minima.

%-------------------- End Column 1 ---------------------


%----------------- Advanced Topics --------------------
\textbf{Maximum Likelihood Estimation (MLE):}\\
Choose parameters maximizing probability of observed data:\\
$\theta_{MLE} = \arg\max_\theta P(\mathbf{y}|\mathbf{X},\theta)$

\textbf{Covariance Matrix (PCA):}
$C = \frac{1}{n}\sum_{i=1}^{n}(x^{(i)}-\mu)(x^{(i)}-\mu)^\top$\\

\textbf{Softmax for Multiclass Logistic:}\\
$\text{softmax}(z_k) = \frac{\exp(z_k)}{\sum_{j}\exp(z_j)}$\\
Predict class with highest probability.

\textbf{Regularization in Gradient Descent:}\\
Update: $\theta_j\gets\theta_j-\alpha\left[\frac{\partial L}{\partial\theta_j}+\lambda\cdot\theta_j\right]$\\
Penalizes large parameter values.

\textbf{Loss Functions:}\\
- Linear regression: MSE.\\
- Logistic regression: Cross-entropy.\\
- K-means: Sum of squared errors.

\textbf{Hyperparameters:}\\
- Set before training ($\lambda$, learning rate, $k$, degree, epochs).
- Use grid search, cross-validation to choose.

\textbf{Overfitting symptoms:} Training error $\ll$ test error.\\
\textbf{Underfitting:} High error everywhere.

\textbf{Dimensionality Reduction:}
Reduce features, mitigate curse of dimensionality.

\textbf{Gradient Check:}
Numerically approximate derivative to check implementation.

%----------------- Important Formulas -------------------
\section*{Quick Formula Reference}
\textbf{Gradient:} $\nabla_\theta L(\theta)$\\
\textbf{Sigmoid:} $g(z)=\frac{1}{1+e^{-z}}$\\
\textbf{Softmax:} $\text{softmax}(z_k)=\frac{e^{z_k}}{\sum_j e^{z_j}}$\\
\textbf{Covariance:} $\operatorname{Cov}(X,Y)=E[(X-E[X])(Y-E[Y])]$\\
\textbf{Standardization:} $x_{std}^{(i)}=\frac{x^{(i)}-\mu}{\sigma}$\\
\textbf{Loss:} $L(\theta)$\\

%------------------- End Column 2 ---------------------

%---------------- Worked Examples/Tips -----------------
\section*{Worked Example: Ridge Regression}
Minimize $L(\theta) = \frac{1}{n}\sum(x_i-4\theta)^2 + \lambda(\theta-3)^2$\\
Solution: $\theta = \frac{4\bar{x}+3\lambda n}{16+\lambda n}$\\
As $\lambda \to 0$: normal equation.\\
As $\lambda \to \infty$: $\theta \to 3$.

\section*{Worked Example: Logistic Regression}
Given features $x = [1, 0.7, 0.5]$, $\theta$,\\
$h_\theta(x)=g(\theta^\top x)$.\\
If $h_\theta(x)=0.8$, $P(y=1|x)=0.8$

\section*{Worked Example: K-means}
Points: (0,0),(0,1),(1,0),(2,2),(3,3); $k=2$;\\
1. Random centers.\\
2. Assign clusters.\\
3. Update means.\\
4. Repeat until cluster assignments don't change.

\section*{Worked Example: Hierarchical (Single Linkage)}
Points: A(0,0), B(1,0), C(0,1), D(4,4), E(5,4)\\
1. Compute pairwise distances.\\
2. Merge closest points.\\
3. Repeat until single cluster.
4. Visualize with dendrogram.

\section*{Precision-Recall Example}
Actual positives: 10. Actual negatives: 10.\\
Predicted TP=8, FP=3, TN=7, FN=2.\\
Precision=$8/11 \approx 0.73$\\
Recall=$8/10=0.8$\\
Accuracy=$15/20=0.75$\\

%------------------ Advanced Tricks ----------------------
\section*{Clustering Tips}
- K-means works best for spherical clusters.
- Hierarchical: Use average linkage for balanced clusters.
- Outliers: K-means++ avoids picking them as centers.

\section*{Feature Engineering Tricks}
- Use PCA after scaling.
- One-hot encoding for categoricals.
- Standardize features for regularized models.

\section*{General ML Strategy}
1. Choose model.
2. Split train/test.
3. Feature engineer.
4. Hyperparameter tune (CV).
5. Train/finalize.
6. Evaluate only once on test.

\section*{Common Pitfalls}
- Using test data during training
- Not scaling before regularization (L1/L2)
- Ignoring precision vs. recall when classes imbalanced
- Only checking accuracy (use confusion matrix, F1, etc.)

\end{multicols}

\end{document}

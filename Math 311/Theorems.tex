\documentclass[answers,12pt,addpoints]{exam}
\usepackage{import}

\import{C:/Users/prana/OneDrive/Desktop/MathNotes}{style.tex}

% Header
\newcommand{\name}{Pranav Tikkawar}
\newcommand{\course}{01:640:311}
\newcommand{\assignment}{Homework n}
\author{\name}
\title{\course \ - \assignment}

\begin{document}
\maketitle
This is a set of all of the theorems talked in class and in the book numbered.

\newpage
\begin{theorem}[0.0.0: Theorem Name]
    This is a theorem. and a teplate for theorems..
    \begin{proof}
        This is a proof. \\
        \begin{align*}
            \text{This is a proof.} \quad e=mc^2
        \end{align*}
    \end{proof}
\end{theorem}

\newpage
\begin{theorem}[Nested Interval Property: (s 1.4)]
    If $I_1 \supseteq I_2 \supseteq I_3 \supseteq \dots$ is a sequence of closed intervals in $\mathbb{R}$ then $\bigcap_{n=1}^{\infty} I_n \neq \emptyset$.
    \begin{proof}
        Let $A = \{a_1, a_2, a_3, \dots\}$ be the set of left endpoints of the intervals. \\
        Now since the $I_n$s are nested, $I_n \subseteq I_1$ for all $n$. \\
        Thus each $a_n \in I_1$ for all $n$. \\
        so $a_n \leq b_1.$ \\
        It follows that $b_1$ is an upper bound for $A$ so sup $A$ exists. \\
        Now we need to prove that $x \in \bigcap_{n=1}^{\infty} I_n$. \\
        To do thi we need to how that $x \in I_n$ for all $n$. \\
        This mean that $a_n \leq x \leq b_n$ for all $n$. \\
        \textbf{Step 1} $a_n \leq x$ for all $n$. \\
        Remember that $x = \sup A$. \\
        So $a_n \leq x$ for all $n$. \\
        \textbf{Step 2} $x \leq b_n$ for all $n$. \\
        Since $x = \sup A$, $x$ i less than very upper bound of $A$ so it i enough to show that $b_n$ is an upper bound of $A$. \\
        $b_n \geq a_m$ for all $m$. \\
        \textbf{Case 1} $n > m$. \\
        Then $I_n \subset I_m$ so $b_n \in [a_m, b_m] = I_m$. \\
        \textbf{Case 2} $n \leq m$. \\
        Then $I_m \subset I_n$ so $a_m \in [a_n, b_n] = I_n$. \\
        so $a_m \leq b_n$. \\
        This $b_n$ is an upper bound of $A$. \\
        Thus $x \leq b_n$ for all $n$. \\
        Thus $x \in I_n$ for all $n$. \\
        Thus $x \in \bigcap_{n=1}^{\infty} I_n$. which means the intersection is not empty.
    \end{proof}
\end{theorem}
\begin{theorem}[Archimedan Property]
    The set $\mathbb{N}$ is not bounded above.\\
    \begin{proof}
        Suppose (by contradiction) $\mathbb{N}$ is bounded above. \\
        Then by the least upper bound property, sup $\mathbb{N}$ exists. \\
        Let us call $\alpha = \sup\mathbb{N}$ and it is a real number. \\
        Thus $\alpha - 1 < \alpha$ so $\alpha - 1$ is not an upper bound of $\mathbb{N}$. \\
        So we can fine an $n \in \mathbb{N}$ such that $ \alpha - 1 < n$. \\
        Thus $\alpha < n + 1$. \\
        But $n + 1 \in \mathbb{N}$ so $\alpha$ is not an upper bound of $\mathbb{N}$. \\
    \end{proof}
\end{theorem}
\begin{theorem}[Density of $\mathbb{Q}$ in $\mathbb{R}$]
    $\forall a < b \in \mathbb{R}$ there exists $q \in \mathbb{Q}$ such that $a < q < b$.
\end{theorem}
\newpage

\begin{definition}[Open Set]
    An open set is a set $S$ that for all $x \in S$ for all epsilon neighborhoods $V_\epsilon(x)$ of $x$, $V_\epsilon(x) \subseteq S$.\\
    In other words, any point has a circle that can be drawn around it that is completely contained in the set.\\
    The union of open sets is open.\\
    The intersection of finitely many open sets is open.\\
    A set is open iff its compliment is closed.
\end{definition}
\begin{definition}[Closed Set]
    A closed set is a set $S$ it contains all of its limit points.\\
    A set is closed iff every cauchy sequence contained in $S$ has limit in $S$.\\
    The intersection of closed sets is closed.\\
    The union of finitely many closed sets is closed.\\
    A set is closed iff its compliment is open.
\end{definition}
\begin{definition}[Limit Point]
    A point $x$ is a limit point of a set $S$ if every epsilon neighborhood $V_\epsilon(x)$ of $x$ intersects the set $S$ at a point other than $x$.\\
    In other words $x$ is a limit point if there is a sequence of points in $S$ that converges to $x$ where the sequence does not contain $x$.
\end{definition}
\begin{definition}[Isolated Point]
    An isolated point is a point $x$ in a set $S$ that is not a limit point of $S$.\\
    In other words, there exists an epsilon neighborhood $V_\epsilon(x)$ of $x$ such that $V_\epsilon(x) \cap S = \{x\}$.
\end{definition}
\begin{definition}[Closure and Interior]
    The closure of a set $S$ denoted by $\overline{S}$ is the union of $S$ and all of its limit points.\\
    The interior of a set $S$ denoted by $S^\circ$ is the collection of all points $x \in S$ such that there exists an epsilon neighborhood $V_\epsilon(x)$ of $x$ that is completely contained in $S$.\\
    Let $S$ be a set then:\\
    $S$ is closed iff $S = \overline{S}$.\\
    $S$ is open iff $S = S^\circ$.
\end{definition}
\begin{definition}[Compact Set]
    A compact set is a set that every Sequence in $K$ has a subsequence that converges to a point in $K$.\\
    A set is compact if and only if it is closed and bounded.\\
    A set is compact if and only if every open cover has a finite subcover.
\end{definition}
\begin{definition}[Open Cover]
    An open cover of a set $S$ is a collection of open sets $\{U_\alpha\}$ such that union of all the open sets contains $S$.\\
    In other words, $S \subseteq \bigcup_{\alpha} U_\alpha$.
\end{definition}
\begin{theorem}[Hiene-Borel Theorem]
    A subset of $\mathbb{R}$ is compact if and only if it is closed and bounded.
\end{theorem}
\begin{definition}[Perfect Set]
    A Perfect set is a set $S$ thtat is closed and contains no isolated points.\\
    A nonempty perfect set is uncountable.\\
    Examples are Cantor Set and the set of all real numbers.
\end{definition}
\begin{definition}[Seperated, Disconnected, and Connected Set]
    Two sets $A$ and $B$ are seperated if $\overline{A} \cap B = A \cap \overline{B} = \emptyset$.\\
    A set $S$ is disconnected if it can be written as the union of two nonempty seperated sets.\\
    A set $S$ is connected if it is not disconnected.
\end{definition}
\begin{definition}[$F_\sigma$ set]
    A set $S$ is an $F_\sigma$ set if it is the countable union of closed sets.
    A set is $F_\sigma$ if and only if its compliment is $G_\delta$.
\end{definition}
\begin{definition}[$G_\delta$ set]
    A set $S$ is an $G_\delta$ set if it is the countable intersection of open sets.
    A set is $G_\delta$ if and only if its compliment is $F_\sigma$.
\end{definition}
\begin{definition}[Dense and Nowhere-Dense Set]
    We say a set $S$ is dense in $X$ if $\overline{S} = X$.\\
    A set $S$ is nowhere-dense if $\overline{S}$ contains no open interval. ie $\overline{S}^\circ = \emptyset$.\\
    A set $E$ is nowhere-dense in $R$ iff $\overline{E}^c$ is dense in $R$. 
\end{definition}
\begin{definition}[Baire's Theorem]
    The set of Real numbers $R$ cannot be written as the countable union of nowhere-dense sets.
\end{definition}


\end{document}
\documentclass[answers,12pt,addpoints]{exam} 
\usepackage{import}

\import{C:/Users/prana/OneDrive/Desktop/MathNotes}{style.tex}

% Header
\newcommand{\name}{Pranav Tikkawar}
\newcommand{\course}{01:640:311H}
\newcommand{\assignment}{Homework 9}
\author{\name}
\title{\course \ - \assignment}

\begin{document}
\maketitle

\begin{questions}

\question Define \( h(x) = 
\begin{cases} 
x^3 \sin\left(\frac{1}{x}\right) & x \neq 0 \\ 
0 & x = 0 
\end{cases} \)
\begin{parts}
    \part Show that \( h \) is differentiable everywhere.
    \begin{solution}
    To show that \( h \) is differentiable everywhere, we need to show that the limit $\lim_{x \to c} \frac{h(x) - h(c)}{x - c}$ exists for all \( c \in \mathbb{R} \).
    We can see that for \( c \neq 0 \), the function is differentiable since it is a composition of differentiable functions. For \( c = 0 \), we need to check the limit:
    \[
    h'(0) = \lim_{x \to 0} \frac{h(x) - h(0)}{x - 0} = \lim_{x \to 0} \frac{x^3 \sin\left(\frac{1}{x}\right)}{x} = \lim_{x \to 0} x^2 \sin\left(\frac{1}{x}\right) = 0.
    \]
    Thus, \( h \) is differentiable at \( x = 0 \) and hence differentiable everywhere.
    \end{solution}
    
    \part Show that \( h' \) is continuous everywhere.
    \begin{solution}
    Consider $h' =\begin{cases}
        3x^2 \sin\left(\frac{1}{x}\right) - x\cos\left(\frac{1}{x}\right) & x \neq 0 \\
        0 & x = 0
    \end{cases} $ 
    To show that \( h' \) is continuous everywhere, we need to show that \( \lim_{x \to c} h'(x) = h'(c) \) for all \( c \in \mathbb{R} \).
    For \( c \neq 0 \), \( h' \) is continuous since it is a composition of continuous functions. For \( c = 0 \), we can let $\epsilon > 0$ be given. 
    \begin{align*}
        |h'(x) - h'(0)| &= \left| 3x^2 \sin\left(\frac{1}{x}\right) - x\cos\left(\frac{1}{x}\right) \right| \\
        &\leq |3x^2| + |x|\\
        &\leq 4|x|
    \end{align*}
    If we take \( \delta = \min\left(1, \sqrt{\frac{\epsilon}{4}}\right) \), then for \( |x| < \delta \), we have
    \begin{align*}
        |h'(x) - h'(0)| &\leq 3|x|^2 + |x| \\
        &\leq 4|x| < 4\delta \\
        &< \epsilon.
    \end{align*} 
    Therefore, \( h' \) is continuous at \( x = 0 \) and hence continuous everywhere.
    \end{solution}
    \part Show that \( h' \) is not differentiable at \( x = 0 \).
    \begin{solution}
    To show that \( h' \) is not differentiable at \( x = 0 \), we need to check the limit
    \[
    h''(0) = \lim_{x \to 0} \frac{h'(x) - h'(0)}{x - 0} = \lim_{x \to 0} \frac{3x^2 \sin\left(\frac{1}{x}\right) - x\cos\left(\frac{1}{x}\right)}{x}.
    \]
    This simplifies to
    \begin{align*}
        h''(0) &= \lim_{x \to 0} \left( 3x \sin\left(\frac{1}{x}\right) - \cos\left(\frac{1}{x}\right) \right) \\
        &= \lim_{x \to 0} 3x \sin\left(\frac{1}{x}\right) - \lim_{x \to 0} \cos\left(\frac{1}{x}\right).
    \end{align*}
    This limit does not exist. Since we can take two sequences approaching 0, \(x_n = \frac{1}{n\pi} \) and \( y_n = \frac{1}{n\pi + \frac{\pi}{2}} \), we find that they concerge to different values. Thus, \( h' \) is not differentiable at \( x = 0 \).
    \end{solution}
\end{parts}

\question Suppose \( f : (a, b) \to \mathbb{R} \) is continuous on \( (a, b) \) and differentiable at \( c \in (a, b) \). Prove that the function \( g : (a, b) \to \mathbb{R} \) defined as
\[
g(x) = 
\begin{cases} 
\frac{f(x) - f(c)}{x - c} & x \neq c \\ 
f'(c) & x = c 
\end{cases}
\]
is continuous.
\begin{solution}
To show that \( g \) is continuous at \( c \), we need to show that for all \( \epsilon > 0 \), there exists a \( \delta > 0 \) such that if \( |x - c| < \delta \), then \( |g(x) - g(c)| < \epsilon \).
We know that since \( f \) is differentiable at \( c \), we have that there exists a $\delta > 0$
    such that

    \begin{align*}
        |x-c| < \delta &\implies \left| \frac{f(x) - f(c)}{x - c} - f'(c) \right| < \epsilon 
    \end{align*}
We also know that
\begin{align*}
    |g(x) - g(c)| &= \left| \frac{f(x) - f(c)}{x - c} - f'(c) \right| \\
    &< \epsilon \quad \text{for } |x - c| < \delta.
\end{align*}
Thus, we have shown that \( g \) is continuous at \( c \). Since \( c \) was arbitrary, \( g \) is continuous on \( (a, b) \).
\end{solution}

\question Prove that \( f : I \to \mathbb{R} \) is differentiable at \( c \in I \) with derivative \( f'(c) \) if and only if we can write 
\[
f(x) = f(c) + f'(c)(x - c) + R_c(x)(x - c)
\]
where \( R_c(x) \) is continuous at \( x = c \) and \( R_c(c) = 0 \).
\begin{solution}
\textbf{Forward Direction:} Assume \( f \) is differentiable at \( c \) with derivative \( f'(c) \). \\
Then we can construct $R_c(x)$ as follows:
\begin{align*}
    R_c(x) &= \begin{cases}
        \frac{f(x)-f(c)-f'(c)(x-c)}{x-c} & x \neq c \\
        0 & x = c
    \end{cases}
\end{align*}
Clealry \( R_c(c) = 0 \) and the equation holds for all \( x \in I \). Now, we need to show that \( R_c(x) \) is continuous at \( c \). We have
\begin{align*}
    \lim_{x \to c} R_c(x) &= \lim_{x \to c} \frac{f(x) - f(c) - f'(c)(x - c)}{x - c} \\
    &= \lim_{x \to c} \frac{f(x) - f(c)}{x - c} - f'(c) \\
    &= f'(c) - f'(c) = 0.
\end{align*}
Thus, \( R_c(x) \) is continuous at \( c \) and the forward direction is proved.\\
\textbf{Backward Direction:} Assume that we can write
\[
f(x) = f(c) + f'(c)(x - c) + R_c(x)(x - c)
\]
where \( R_c(x) \) is continuous at \( c \) and \( R_c(c) = 0 \). \\
Then we can rearrange this to get
\begin{align*}
    \frac{f(x) - f(c)}{x - c} - f'(c) &= R_c(x).
\end{align*}
Since \( R_c(x) \) is continuous at \( c \) and \( R_c(c) = 0 \), we have that for arbitrary $\epsilon>0$ there exists a \( \delta > 0 \) such that for all \( |x - c| < \delta \), we have
\begin{align*}
    \left| R_c(x) \right| < \epsilon.
\end{align*}
Thus we can write
\begin{align*}
    \lim_{x \to c } \left| \frac{f(x) - f(c)}{x - c} - f'(c) \right| = \lim_{x \to c} \left| R_c(x) \right| < \epsilon.
\end{align*}
And thus we have shown that \( f \) is differentiable at \( c \) with derivative \( f'(c) \).\\
\end{solution}

\question Given a differentiable function \( f : A \to \mathbb{R} \), we say that \( f \) is uniformly differentiable on \( A \) when for any \( \epsilon > 0 \), there exists a \( \delta > 0 \) such that if \( |x - y| < \delta \), then 
\[
\left| \frac{f(x) - f(y)}{x - y} - f'(y) \right| < \epsilon.
\]
\begin{parts}
    \part Is \( f(x) = x^2 \) uniformly differentiable on \( \mathbb{R} \)?
    \begin{solution}
        Let $\epsilon >0$ then take $\delta = \epsilon$. Then for any \( |x - y| < \delta \), we have
        \begin{align*}
            \left| \frac{f(x) - f(y)}{x - y} - f'(y) \right| &= \left| \frac{x^2 - y^2}{x - y} - 2y \right| \\
            &= \left| (x + y) - 2y \right| \\
            &= |x - y| < \delta = \epsilon.
        \end{align*}
        Thus, \( f(x) = x^2 \) is uniformly differentiable on \( \mathbb{R} \).
    \end{solution}
    
    \part Is \( g(x) = x^3 \) uniformly differentiable on \( \mathbb{R} \)?
    \begin{solution}
    Note that 
    \begin{align*}
        \left| \frac{g(x) - g(y)}{x - y} - g'(y) \right| &= \left| \frac{x^3 - y^3}{x - y} - 3y^2 \right| \\
        &= |x^2 + xy - 2y^2|\\
        &= |(x - y)(x + y) + y(x - y)|\\
        &= |x - y| |x+2y|
    \end{align*}
    Since there is a $|x +2y|$ term, there are issues with the uniformity of the derivative. \\
    Specifically taking $x_n = n$ and $y_n = n + \frac{1}{n}$, we have
    \begin{align*}
        \lim |x_n - y_n| &= 0\\
        \frac{f(x_n) - f(y_n)}{x_n-y_n} &= 3n^2 + 3 + \frac{1}{n^2}
    \end{align*}
    which converges to $3$ as $n \to \infty$. Thus the limit does not converge to $g'(y)$ uniformly. Therefore, \( g(x) = x^3 \) is not uniformly differentiable on \( \mathbb{R} \).
    \end{solution}
    
    \part Show that if \( f \) is uniformly differentiable on an interval \( I \), then \( f' \) must be continuous on \( I \).
    \begin{solution}
        Suppose $f$ is uniformly differentiable on an interval \( I \). By definition, for any \( \epsilon > 0 \), there exists a \( \delta > 0 \) such that if \( |x - y| < \delta \), then
        \[
        \left| \frac{f(x) - f(y)}{x - y} - f'(y) \right| < \epsilon/2
        \]
        and
        \[
        \left| \frac{f(x) - f(y)}{x - y} - f'(x) \right| < \epsilon/2.
        \]
        This implies that
        \begin{align*}
            |f'(x) - f'(y)| &= \left| \frac{f(x) - f(y)}{x - y} - f'(y) + f'(y) - f'(x) \right| \\
            &\leq \left| \frac{f(x) - f(y)}{x - y} - f'(y) \right| + \left| f'(y) - f'(x) \right| \\
            &< \frac{\epsilon}{2} + \frac{\epsilon}{2} = \epsilon.
        \end{align*}
    \end{solution}
    
    \part If \( f \) is differentiable on a closed, bounded interval \([a, b]\), is \( f \) necessarily uniformly differentiable there? Give a proof or a counterexample to support your answer.
    \begin{solution}
        No, We can use the textbooks example of $g(x) = \begin{cases}
            x^2 sin(\frac{1}{x}) & x \neq 0 \\
            0 & x = 0
        \end{cases}$ on the interval of $[-1,1]$
        We know that $g(x)$ is differentiable on $[-1,1]$ but not uniformly differentiable. To see this, we can take the sequences $x_n = \frac{1}{n\pi}$ and $y_n = \frac{1}{n\pi + \frac{\pi}{2}}$ each converging to $0$ and thus their difference converges to $0$ but the limit of the difference quotient does not converge. 
        \begin{align*}
            \lim \frac{g(x_n) - g(y_n)}{x_n - y_n} \to \infty \text{ as } n \to \infty.
        \end{align*}
        and $g'(y_n) = 2y_n \sin(\frac{1}{y_n}) - \cos(\frac{1}{y_n}) \to 0$ 
        Thus the limit does not converge uniformly to $g'(y)$ and thus $g$ is not uniformly differentiable on $[-1,1]$.
    \end{solution}
\end{parts}

\question If \( f \) is twice differentiable on an open interval containing \( c \) and \( f'' \) is continuous at \( c \), prove that 
\[
f''(c) = \lim_{h \to 0} \frac{f(c + h) + f(c - h) - 2f(c)}{h^2}.
\]
\begin{solution}
Consider the follow limits and applying L'Hospital's rule twice where needed
\begin{align*}
    \lim_{h \to 0} \frac{f(c + h) + f(c - h) - 2f(c)}{h^2} &= \lim_{h \to 0} \frac{f'(c + h) - f'(c-h)}{2h} \\
    &= \lim_{h \to 0} \frac{f''(c + h) + f''(c - h)}{2} \\
    &= \frac{f''(c) + f''(c)}{2} \\
    &= f''(c).
\end{align*}
\end{solution}

\question Suppose that \( g : A \to \mathbb{R} \) and \( a \) is a limit point of \( A \). Also assume that \( g(x) \neq 0 \) for any \( x \in A \). Show that if \( \lim_{x \to a} g(x) = \infty \), then \( \lim_{x \to a} \frac{1}{g(x)} = 0 \).
\begin{solution}
Let $\epsilon > 0$. We know that by the archimedian property of the real numbers there exists and $N \in \mathbb{N}$ such that $\frac{1}{N} < \epsilon$. \\
We also know that since \( \lim_{x \to a} g(x) = \infty \), for all \( M \in \N \) there exists a \( \delta > 0 \) such that if \( 0 < |x - a| < \delta \), then \( g(x) > M \). \\
Thus, we can take \( M = N \) then when $0 < |x - a| < \delta$ we have $0 < \frac{1}{g(x)} < \frac{1}{N} < \epsilon$. \\
Thus we have shown that \( \lim_{x \to a} \frac{1}{g(x)} = 0 \).
\end{solution}

\end{questions}

\end{document}
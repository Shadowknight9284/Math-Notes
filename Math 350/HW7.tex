\documentclass[answers,12pt,addpoints]{exam}
\usepackage{import}

\import{C:/Users/prana/OneDrive/Desktop/MathNotes}{style.tex}

% Header
\newcommand{\name}{Pranav Tikkawar}
\newcommand{\course}{01:640:350H}
\newcommand{\assignment}{Homework 7}
\author{\name}
\title{\course \ - \assignment}

\begin{document}
\maketitle


\newpage
\begin{questions}
    \item Question 4.1 9
    Prove that $det(AB) = det(A)det(B)$ for any $A,B \in M_{2x2}(\mathbb{R})$.
    \begin{solution}
        We can consider the matrices $A$ and $B$ in the following cases: (WLOG)
        \begin{parts}
            \item A is not invertible
            \item A is invertible
        \end{parts}
        \textbf{Case 1: A is not invertible}
        If $A$ is not invertible, then $det(A) = 0$. Thus $det(A)det(B) = 0$ and by $A$ being non-invertible, $AB$ is also non-invertible due to the fact that $A$ has a rank less than 2 therefore regardless of the rank of $B$ the matrix $AB$ will have a rank less than 2. Thus $det(AB) = 0$.\\
        therefore $det(AB) = det(A)det(B)$.\\
        \textbf{Case 2: A is invertible}
        If $A$ is invertible, then $det(A) \neq 0$. We know that $A^{-1}$ exists. Thus $ABy =Ay$ for some $x,y \in F$ is equivalent to $By = x$. \\
        We can consider the augemented matrix $[A|I]$ and row reduce by a series of elementary row operation $E_1 ... E_n$ to $[I|A^{-1}]$.  \\
        We can consider the augemented matrix $[AB|I]$ and row reduce by the same series of elementary row operation $E_1 ... E_n$ to $[B|A^{-1}]$\\
        We also know that for for any row additions it will not change the determinant, for row swaps it will change the determinant by a factor of -1, and for row scaling it will change the determinant by a factor of the scalar.\\
        So for $k$ row swaps and $l$ row scaling we can consider 
        $ 1 = det(I) = (-1)^k * c_1 * c_2 * ... * c_l * det(A)$\\  
        Then we can see that $det(B) = \frac{1}{(-1)^k c_1 * c_2 * ... * c_l} det(AB) $\\
        Which implies that $det(AB) = det(A)det(B)$

        \textbf{Alternatively:}
        We can consider the matrices $A$ and $B$ as arbitrary matrices $A = \begin{bmatrix}
            a & b \\ c & d
        \end{bmatrix}$ and $B = \begin{bmatrix}
            e & f \\ g & h
        \end{bmatrix}$. We can see that
        \begin{align*}
            det(AB) &= det\left(\begin{bmatrix}
                a & b \\ c & d
            \end{bmatrix} \begin{bmatrix}
                e & f \\ g & h
            \end{bmatrix}\right)\\
            &= det\left(\begin{bmatrix}
                ae + bg & af + bh \\ ce + dg & cf + dh
            \end{bmatrix}\right)\\
            &= (ae + bg)(cf + dh) - (af + bh)(ce + dg)\\
            &= aecf + aedh + bgcf + bhdh - afce - afdg - bghc - bhgd\\
            &= aecf + aedh + bgcf + bhdh - afce - afdg - bghc - bhgd\\
            &= a(ecf - fdg) + b(gcf - hgd) + c(aed - bhc) + d(bh - af)\\
            &= det(A)det(B)
        \end{align*}

    \end{solution}
    \item Question 4.1 11
    Let $\delta: M_{2x2}(F) \to F$ be a function with the following three properties:
    \begin{enumerate}
        \item $\delta$ is a linear function of each row of the matrix when the other row is fixed.
        \item if the two rows of $A$ are identical, then $\delta(A) = 0$.
        \item $\delta(I) = 1$.
    \end{enumerate} 
    \begin{parts}
        \item Prove that $\delta(E) = det(E)$ for any elementary matrix $E$.
        \item Prove that $\delta(EA) = \delta(E)\delta(A)$ for any elementary matrix $E$ and any $A \in M_{2x2}(F)$.
    \end{parts}
    \begin{solution}
        \textbf{Part 1:}
        We can consider the elementary matrices $E$ in the following cases:
        \begin{enumerate}
            \item $E$ is a row swap matrix
            \item $E$ is a row scaling matrix
            \item $E$ is a row addition matrix
        \end{enumerate}
        \textbf{Case 1:} $E$ is a row swap matrix
        We can consider the matrix $E$ as $E = \begin{bmatrix} 0 & 1 \\ 1 & 0 \end{bmatrix}$. We can see that 
        \begin{align*}
            \delta(E) &= \delta\left(\begin{bmatrix} 0 & 1 \\ 1 & 0 \end{bmatrix}\right)\\
            &= \delta\left(\begin{bmatrix} 1 & 1 \\ 0 & 1 \end{bmatrix}\right) + \delta\left(\begin{bmatrix} 1 & 0 \\ 1 & 0 \end{bmatrix}\right)_{\text{Goes to } 0}\\
            &= \delta\left(\begin{bmatrix} 1 & 1 \\ 1 & 1 \end{bmatrix}\right)_{\text{Goes to } 0} + \delta\left(\begin{bmatrix} 1 & 0 \\ 1 & -1 \end{bmatrix}\right)\\
            &= \delta\left(\begin{bmatrix} 1 & 0 \\ 0 & -1 \end{bmatrix}\right) + \delta\left(\begin{bmatrix} 0 & 1 \\ 0 & -1 \end{bmatrix}\right)\\
            &= \delta\left(\begin{bmatrix} 1 & 0 \\ 0 & 1 \end{bmatrix}\right) - \delta\left(\begin{bmatrix} 0 & 1 \\ 0 & 1 \end{bmatrix}\right)_{\text{Goes to } 0}\\
            &= -1
        \end{align*}
        \textbf{Case 2:} $E$ is a row scaling matrix
        We can consider the matrix $E$ as $E = \begin{bmatrix} k & 0 \\ 0 & 1 \end{bmatrix}$. We can see that by property 1, $\delta(E) = k\delta\left(\begin{bmatrix} 1 & 0 \\ 0 & 1 \end{bmatrix}\right) = k$.\\
        Additionally if we consider the matrix $E$ as $E = \begin{bmatrix} 1 & 0 \\ 0 & k \end{bmatrix}$, we can see that $\delta(E) = k\delta\left(\begin{bmatrix} 1 & 0 \\ 0 & 1 \end{bmatrix}\right) = k$.\\
        \textbf{Case 3:} $E$ is a row addition matrix
        We can consider the matrix $E$ as $E = \begin{bmatrix} 1 & 0 \\ k & 1 \end{bmatrix}$. We can see that 
        \begin{align*}
            \delta(E) &= \delta\left(\begin{bmatrix} 1 & 0 \\ k & 1 \end{bmatrix}\right)\\
            &= \delta\left(\begin{bmatrix} 1 & 0 \\ 0 & 1 \end{bmatrix}\right) + \delta\left(\begin{bmatrix} 0 & 0 \\ k & 1 \end{bmatrix}\right)\\
            &= 1 + k\delta\left(\begin{bmatrix} 0 & 0 \\ 1 & 1 \end{bmatrix}\right)\\
            &= 1
        \end{align*}
        Additionally if we consider the matrix $E$ as $E = \begin{bmatrix} 1 & k \\ 0 & 1 \end{bmatrix}$, we can see that
        \begin{align*}
            \delta(E) &= \delta\left(\begin{bmatrix} 1 & k \\ 0 & 1 \end{bmatrix}\right)\\
            &= \delta\left(\begin{bmatrix} 1 & 0 \\ 0 & 1 \end{bmatrix}\right) + \delta\left(\begin{bmatrix} 0 & k \\ 0 & 1 \end{bmatrix}\right)\\
            &= 1 + k\delta\left(\begin{bmatrix} 0 & 1 \\ 0 & 1 \end{bmatrix}\right)\\
            &= 1
        \end{align*}
        Thus we can see that $\delta(E) = det(E)$ for any elementary matrix $E$.\\
        \textbf{Part 2:}
        We can take an aribtrary matrix $A = \begin{bmatrix}
            a & b \\
            c & d
        \end{bmatrix}$ and notice that $\delta(A) = ad - bc = det(A)$.\\
        Since $E$ is an elementary matrix and we know that $det(EA) = det(E)det(A)$, we can see that $\delta(EA) = \delta(E)\delta(A)$. by the fact that $\delta(E) = det(E)$ and $\delta(A) = det(A)$.
    \end{solution}
    \item Question 4.1 12
    Let $\delta: M_{2x2}(F) \to F$ be a function with properties from the prior question. Prove that $\delta(A) = det(A)$ for any $A \in M_{2x2}(F)$.
    \begin{solution}
        We can consider an arbitrary matrix $A = \begin{bmatrix}
            a & b \\
            c & d
        \end{bmatrix}$ 
        \begin{align*}
            \delta(A) &= \delta\left(\begin{bmatrix}
                a & b \\
                c & d
            \end{bmatrix}\right)\\
            &= \delta\left(\begin{bmatrix}
                a & b \\
                0 & d
            \end{bmatrix}\right) + \delta\left(\begin{bmatrix}
                a & b \\
                c & 0
            \end{bmatrix}\right)\\
            &= \delta\left(\begin{bmatrix}
                a & 0 \\
                0 & d
            \end{bmatrix}\right) + \delta\left(\begin{bmatrix}
                0 & b \\
                c & 0
            \end{bmatrix}\right) + \delta\left(\begin{bmatrix}
                a & 0 \\
                c & 0
            \end{bmatrix}\right) + \delta\left(\begin{bmatrix}
                0 & b \\
                0 & d
            \end{bmatrix}\right)\\
            &= ad - bc
        \end{align*}
        Clealy $\delta(A) = det(A)$ for any $A \in M_{2x2}(F)$.
    \end{solution}
    \item Question 4.2 7
    Cofactor Expansion: $$\begin{bmatrix}
        0 & 1 & 2 \\
        -1 & 0 & -3 \\
        2 & 3 & 0
    \end{bmatrix}$$
    along the second row.
    \begin{solution}
        $$det(A) = (-1)^{(3)} (-1) det\begin{bmatrix}
            1 & 2 \\
            3 & 0
        \end{bmatrix} + (-1)^{(4)} (0) det\begin{bmatrix}
            0 & 2 \\
            2 & 0
        \end{bmatrix} + (-1)^{(5)} (-3) det\begin{bmatrix}
            0 & 1 \\
            2 & 3
        \end{bmatrix}$$
        $$det(A) = -(-1(-6)) + 0 - (-3(-2)) = -12$$
    \end{solution}
    \item Question 4.2 8
    Cofactor Expansion: $$\begin{bmatrix}
        1 & 0 & 2 \\
        0 & 1 & 5 \\
        -1 & 3 & 0
    \end{bmatrix}$$
    along the third row.
    \begin{solution}
        $$det(A) = (-1)^{(6)} (-1) det\begin{bmatrix}
            0 & 2\\
            1 & 5
        \end{bmatrix} + (-1)^{(7)} (3) det\begin{bmatrix}
            1 & 2\\
            0 & 5
        \end{bmatrix} + (-1)^{(8)} (0) det\begin{bmatrix}
            1 & 0\\
            0 & 1
        \end{bmatrix}$$
        $$ det(A) = (-1(-2))  - 3(5) + 0 = -13$$
    \end{solution}
    \item Question 4.2 14
    $$det\left( \begin{bmatrix}
        2 & 3 & 4 \\
        5 & 6 & 0 \\
        7 & 0 & 0
    \end{bmatrix}\right)$$
    \begin{solution}
        We will cofactor expand along the third row.
        $$det(A) = (-1)^{(6)} (7) det\begin{bmatrix}
            3 & 4\\
            6 & 0
        \end{bmatrix} + (-1)^{(7)} (0) det\begin{bmatrix}
            2 & 4\\
            5 & 0
        \end{bmatrix} + (-1)^{(8)} (0) det\begin{bmatrix}
            2 & 3\\
            5 & 6
        \end{bmatrix}$$
        $$det(A) = -7(-24) - 0 + 0 = 168$$
    \end{solution}
    \item Question 4.2 18
    $$det\left( \begin{bmatrix}
        1 & -2 & 3 \\
        -1 & 2 & -5 \\
        3 & -1 & 2
    \end{bmatrix} \right)$$
    \begin{solution}
        We will cofactor expand along the first row.
        $$det(A) = (-1)^{0} (1) det\begin{bmatrix}
            2 & -5\\
            -1 & 2
        \end{bmatrix} + (-1)^{1} (-2) det\begin{bmatrix}
            -1 & -5\\
            3 & 2
        \end{bmatrix} + (-1)^{2} (3) det\begin{bmatrix}
            -1 & 2\\
            3 & -1
        \end{bmatrix}$$
        $$det(A) = 1(4 - 5) + 2(-2 + 15) + 3(1 - 6) = -1 + 26 - 15 = 10$$
    \end{solution}
    \item Question 4.2 23
    Prove that the determinant of an upper triangular matrix is the product of its diagonal entries.
    \begin{solution}
        Let the upper triangular matrix be 
        $$ \begin{bmatrix}
            a_{11} & a_{12} & a_{13} & ... & a_{1n} \\
            0 & a_{22} & a_{23} & ... & a_{2n} \\
            0 & 0 & a_{33} & ... & a_{3n} \\
            \vdots & \vdots & \vdots & \ddots & \vdots \\
            0 & 0 & 0 & ... & a_{nn}
        \end{bmatrix}$$
        We can see that by cofactor expansion along the last row, we can see that
        $$det(A) = (-1)^{(n^2-n)} 0 + (-1)^{(n^2-n+1)} 0 + ... + (-1)^{(n^2-1)} a_{nn} det\begin{bmatrix}
            a_{11} & a_{12} & ... & a_{1n-1} \\
            0 & a_{22} & ... & a_{2n-1} \\
            \vdots & \vdots & \ddots & \vdots \\
            0 & 0 & ... & a_{n-1n-1}  
        \end{bmatrix}$$
        and by continuously cofactor expanding along the last row, We can continue only requiring one non-trivial term. Thus the solution will be $$ \Pi_{i=1}^{n} (-1)^{n^2-1}a_{ii} = a_{11}a_{22}...a_{nn}$$ 
    \end{solution}
    \item Question 4.3 12
    A matrix $Q \in M_{nxn}(\mathbb{C})$ is called orthogonal if $QQ^t = I$. Prove that if $Q$ is orthogonal, then $det(Q) = \pm 1$.
    \begin{solution}
        We can see that $det(QQ^t) = det(I) = 1$. We can also see that $det(QQ^t) = det(Q)det(Q^t) = det(Q)det(Q) = det(Q)^2$. Thus $det(Q)^2 = 1$ and $det(Q) = \pm 1$.
    \end{solution}
    \item Question 4.3 15
    Prove that if $A,B \in M_{nxn}(F)$ are similar, then $det(A) = det(B)$.
    \begin{solution}
        We can see that if $A$ and $B$ are similar, then there exists an invertible matrix $Q$ such that $B = Q^{-1}AQ$. We can see that $det(B) = det(Q^{-1}AQ) = det(Q^{-1})det(A)det(Q) = det(A)$. Thus $det(A) = det(B)$.
    \end{solution}
    \item Question 4.3 24
    Let $A \in M_{nxn}(F)$ have the form 
    $$A = \begin{bmatrix}
        0 & 0 & ... & 0 & a_1 \\
        -1 & 0 & ... & 0 & a_2 \\
        0 & -1 & ... & 0 & a_3 \\
        \vdots & \vdots & \ddots & \vdots & \vdots \\
        0 & 0 & ... & -1 & a_n
    \end{bmatrix}$$
    Compute $det(A-tI)$.
    \begin{solution}
        We can do a cofactor expansion along the first row.
        $$det(A-tI) = (t) det\left(\begin{bmatrix}
            t & 0 & ... & a_1 \\
            -1 & t & ... & a_2 \\
            0 & -1 & ... & a_3 \\
            \vdots & \vdots & \ddots & \vdots \\
            0 & 0 & ... & t+ a_n
        \end{bmatrix} \right) + (-1)^{n-1}a_0 det\left(\begin{bmatrix}
            -1 & t & ... & 0 \\
            0 & -1 & ... & 0 \\
            \vdots & \vdots & \ddots & \vdots \\
            0 & 0 & ... & -1
        \end{bmatrix} \right)$$
        Clealry this will continue to be a series of $t$'s and $a_i$'s. Thus we can see that
        $$det(A-tI) = t (t (t ... (t + a_n) ... + a_2) + a_1  ) + a_0$$
        $$det(A-tI) = t^n + a_{n-1}t^{n-1} + ... + a_1t + a_0$$
        Additionally notice that this is the matrix for an $n$th order linear recurrence relation: where the $i$th element of the vector $x$ when multiplied by $A$ will give the $i+1$th element of the vector $x$ and the "base element" be given by $a_0$. Thus we can see that the characteristic polynomial of the matrix $A$ is $det(A-tI)$.
    \end{solution}
    \item Question 4.4 2(c)
    Evaluate the determinant of the matrix
    $$ A = \begin{bmatrix}
        2 +i & -1 + 3i\\
        1 - 2i & 3 - i
    \end{bmatrix}$$
    \begin{solution}
        $$det(A) = (2+i)(3-i) - (-1 + 3i)(1 - 2i) = 6 - 2i + 3i - i^2 + 1 - 2i - 3i + 6i^2 = 2 - 4i$$
    \end{solution}
    \item Question 4.4 3(c)
    Evaluate the determinant of the matrix
    $$ A = \begin{bmatrix}
        0 & 1 & 2\\
        -1 & 0 & -3\\
        2 & 3 & 0
    \end{bmatrix}$$
    Along the second column.
    \begin{solution}
        $$det(A) = (-1)^{1} (1) det\begin{bmatrix}
            -1 & -3\\
            2 & 0
        \end{bmatrix} + (-1)^{2} (0) det\begin{bmatrix}
            0 & 2\\
            2 & 0
        \end{bmatrix} + (-1)^{3} (3) det\begin{bmatrix}
            0 & 2\\
            -1 & -3
        \end{bmatrix}$$
        $$det(A) = -(6) - 0 - (3(2)) = -12$$
    \end{solution}
    \item Question 4.4 3(e)
    Evaluate the determinant of the matrix
    $$ A = \begin{bmatrix}
        0 & 1 +i & 2\\
        -2i & 0 & 1 -i\\
        3 & 4i & 0
    \end{bmatrix}$$
    Along the third row.
    \begin{solution}
        $$det(A) = (-1)^{6} (3) det\begin{bmatrix}
            1 + i & 2\\
            0 & 1 - i
        \end{bmatrix} + (-1)^{7} (4i) det\begin{bmatrix}
            0 & 2\\
            -2i & 1 - i
        \end{bmatrix} + (-1)^{8} (0) det\begin{bmatrix}
            0 & 1 + i\\
            -2i & 0
        \end{bmatrix}$$
        $$det(A) = 3((1 - i)(1+i) - 0) - 4i(0 - 2(-2i)) + 0 = 6 +16 = 22$$
    \end{solution}
\end{questions}

\end{document}
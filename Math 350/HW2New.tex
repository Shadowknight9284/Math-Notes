\documentclass{article}
\usepackage{amsmath}
\usepackage{amsfonts}
\usepackage{amssymb}
\usepackage{mathrsfs}
\usepackage{dsfont}
\usepackage{cancel}

\usepackage{graphicx}

\makeatletter
\renewcommand*\env@matrix[1][*\c@MaxMatrixCols c]{%
  \hskip -\arraycolsep
  \let\@ifnextchar\new@ifnextchar
  \array{#1}}
\makeatother


\setlength\parindent{0pt}

\author{Pranav Tikkawar}
\title{Linear Algebra HW 2}

\begin{document}
\maketitle
\section*{Section 1.3 19}
Let $W_1$ and $W_2$ be subspace of a vector space $V$. Prove that $W_1 \cap W_2$  is a subspace of $V$ if and only if $W_1 \subseteq W_2$ or $W_2 \subseteq W_1$.\\
\textbf{Proof:}\\
We need to prove that $W_1 \cup W_2$ is a subspace of $V$ if and only if $W_1 \subseteq W_2$ or $W_2 \subseteq W_1$.\\
\textbf{Proof of $W_1 \subseteq W_2$ or $W_2 \subseteq W_1 \Rightarrow W_1 \cup W_2$ is a subspace of $V$:}\\
Let $W_1, W_2$ be subspaces of $V$.\\
(Without loss of generality) Suppose $W_1 \subseteq W_2$. \\
Thus $W_1 \cup W_2 = W_2$\\
Since $W_2$ is a subspace of $V$, $W_1 \cup W_2$ is a subspace of $V$.\\
(Since $W_1, W_2$  play symmetric roles, the proof is similar if $W_2 \subseteq W_1$)\\
\textbf{Proof of $W_1 \cup W_2$ is a subspace of $V \Rightarrow W_1 \subseteq W_2$ or $W_2 \subseteq W_1$:}\\
Let $W_1 \cup W_2$ be a subspace of $V$.\\
We can do this by a proof by contradiction.\\
Suppose $W_1 \not\subseteq W_2$ and $W_2 \not\subseteq W_1$.\\
Let $x \in W_1$ and $x \not\in W_2$.\\
Let $y \in W_2$ and $y \not\in W_1$.\\
Consider $x + y$.\\
Due to the closure rules of a subspace of each $W_1, W_2$, $x + y \notin W_1$ and $x + y \notin W_2$.\\
This can can be shown by contradiction.\\
Suppose $x + y \in W_1$. (Without loss of Generality)\\
Then there exists an element $-x \in W_1$ (VS 4) Since $W_1$ .\\ 
Thus $x + y + (-x) = y \in W_1$.\\
We can do this symmetrically for $W_2$ to show $x \in W_2$.\\
This is a contradiction since we assumed $x \not\in W_2$ and $y \not\in W_1$.\\
$W_1 \cup W_2$ is a subspace of $V \Rightarrow W_1 \subseteq W_2$ or $W_2 \subseteq W_1$

\section*{Section 1.4 4(a)}
Let $p_1 = x^3 - 3x + 5$, $p_2 = x^3 + 2x^2 - x + 1$, and $p_3 = x^3 + 3x^2 - 1$.\\
We want to see if $p_1 = ap_2 + bp_3$ for some $a,b \in \mathds{R}$.\\
Thus we want:
$$ x^3 - 3x + 5 = a(x^3 + 2x^2 - x + 1) + b(x^3 + 3x^2 - 1) $$
This implies a system of 4 equations:
\begin{align*}
    a + b &= 1\\
    2a + 3b &= 0\\
    -a &= -3\\
    a - b &= 5 
\end{align*}
We can thus then view it in matrix form:
$$ \begin{bmatrix} 1 & 1 \\ 2 & 3 \\ -1 & 0 \\ 1 & -1 \end{bmatrix} \begin{bmatrix} a \\ b \end{bmatrix} = \begin{bmatrix} 1 \\ 0 \\ -3 \\ 5 \end{bmatrix} $$
We then can Consider the augmented matrix and solve for RREF:
$$ 
\begin{bmatrix}[cc|c]
    1 & 1 & 1\\
    2 & 3 & 0\\
    -1 & 0 & -3\\
    1 & -1 & 5
\end{bmatrix} \xrightarrow{-2r_1 + r_2 \rightarrow r_2} \begin{bmatrix}[cc|c]
    1 & 1 & 1\\
    0 & 1 & -2\\
    -1 & 0 & -3\\
    1 & -1 & 5
\end{bmatrix} \xrightarrow{r_1 + r_3 \rightarrow r_3} \begin{bmatrix}[cc|c]
    1 & 1 & 1\\
    0 & 1 & -2\\
    0 & 1 & -2\\
    1 & -1 & 5
\end{bmatrix}$$
$$ \xrightarrow{r_3 - r_2 \rightarrow r_3} \begin{bmatrix}[cc|c]
    1 & 1 & 1\\
    0 & 1 & -2\\
    0 & 0 & 0\\
    1 & -1 & 5
\end{bmatrix} \xrightarrow{r_4 - r_1 \rightarrow r_4} \begin{bmatrix}[cc|c]
    1 & 1 & 1\\
    0 & 1 & -2\\
    0 & 0 & 0\\
    0 & -2 & 4
\end{bmatrix} \xrightarrow{r_4 + 2r_2 \rightarrow r_4} \begin{bmatrix}[cc|c]
    1 & 1 & 1\\
    0 & 1 & -2\\
    0 & 0 & 0\\
    0 & 0 & 0
\end{bmatrix}$$
$$\xrightarrow{r_1 - r_2 \rightarrow r_1} \begin{bmatrix}[cc|c]
    1 & 0 & 3\\
    0 & 1 & -2\\
    0 & 0 & 0\\
    0 & 0 & 0
\end{bmatrix}$$
Thus we have $a = 3$ and $b = -2$.\\
Thus $p_1 = 3p_2 - 2p_3$.
\section*{Section 1.4 5(g)}
In each part, determine wheter the given vector is in the span of S.
$$ \begin{bmatrix}
    1 & 2 \\
    -3 & 4
\end{bmatrix}, S = \left\{ \begin{bmatrix}
    1 & 0 \\
    -1 & 0
\end{bmatrix}, \begin{bmatrix}
    0 & 1 \\
    0 & 1
\end{bmatrix}, \begin{bmatrix}
    1 & 1 \\
    0 & 0
\end{bmatrix}
\right\}\\
$$
\textbf{Notes:} \\
Since we are looking to see if a matrix is in the span og the 3 others, we can simply consider the vector in $\mathds{R}^4$ and see if it is in the span of the other 3 vectors.\\
\textbf{Proof:}\\
We want to find $a,b,c \in \mathds{R}$ such that:
$$ \begin{bmatrix}
    1 & 2 \\
    -3 & 4
\end{bmatrix} = a\begin{bmatrix}
    1 & 0 \\
    -1 & 0
\end{bmatrix} + b\begin{bmatrix}
    0 & 1 \\
    0 & 1
\end{bmatrix} + c\begin{bmatrix}
    1 & 1 \\
    0 & 0
\end{bmatrix} $$
This implies the following system of equations:
\begin{align*}
    a + c &= 1\\
    b + c &= 2\\
    -a &= -3\\
    b &= 4
\end{align*}
We can then represent this in matrix form: 
$$ \begin{bmatrix} 
    1 & 0 & 1 \\ 
    0 & 1 & 1 \\ 
    -1 & 0 & 0 \\ 
    0 & 1 & 0 
\end{bmatrix} \begin{bmatrix} 
    a \\ b \\ c 
\end{bmatrix} = 
\begin{bmatrix} 
    1 \\ 2 \\ -3 \\ 4 
\end{bmatrix} $$
We can then consider the augmented matrix and solve for RREF:
$$ \begin{bmatrix}[ccc|c]
    1 & 0 & 1 & 1\\
    0 & 1 & 1 & 2\\
    -1 & 0 & 0 & -3\\
    0 & 1 & 0 & 4
\end{bmatrix} \xrightarrow{r_1 + r_3 \rightarrow r_3} \begin{bmatrix}[ccc|c]
    1 & 0 & 1 & 1\\
    0 & 1 & 1 & 2\\
    0 & 0 & 1 & -2\\
    0 & 1 & 0 & 4
\end{bmatrix} \xrightarrow{r_2 - r_3 \rightarrow r_2} \begin{bmatrix}[ccc|c]
    1 & 0 & 1 & 1\\
    0 & 1 & 0 & 4\\
    0 & 0 & 1 & -2\\
    0 & 1 & 0 & 4
\end{bmatrix}$$
$$ \xrightarrow{r_4 - r_2 \rightarrow r_4} \begin{bmatrix}[ccc|c]
    1 & 0 & 1 & 1\\
    0 & 1 & 0 & 4\\
    0 & 0 & 1 & -2\\
    0 & 0 & 0 & 0
\end{bmatrix} \xrightarrow{r_1 - r_3 \rightarrow r_1} \begin{bmatrix}[ccc|c]
    1 & 0 & 0 & 3\\
    0 & 1 & 0 & 4\\
    0 & 0 & 1 & -2\\
    0 & 0 & 0 & 0
\end{bmatrix}$$
Thus we have $a = 3$, $b = 4$, and $c = -2$.\\
We can then verify that the vector is in the span of the other 3 vectors.\\ 
$$ \begin{bmatrix}
    1 & 2 \\
    -3 & 4
\end{bmatrix} = 3\begin{bmatrix}
    1 & 0 \\
    -1 & 0
\end{bmatrix} + 4\begin{bmatrix}
    0 & 1 \\
    0 & 1
\end{bmatrix} - 2\begin{bmatrix}
    1 & 1 \\
    0 & 0
\end{bmatrix} $$
Thus the first vector is in the span of the other 3 vectors since we can write it as a linear combination of the other 3 vectors.
\section*{Section 1.4 6}
Show that the vectors $(1,1,0), (1,0,1), (0,1,1)$ generate $F^3$.\\
To show that the vectors generate $F^3$, we need to show that any arbitrary vector $(a_1,a_2,a_3) \in F^3$ can be expressed as a linear combination of the given vectors.\\
We can express an arbitrary vector $(a_1,a_2,a_3) \in F^3$ as a linear combination of the given vectors: 
$$ (a_1,a_2,a_3) = a(1,1,0) + b(1,0,1) + c(0,1,1)$$
This gives us the following system of equations:
\begin{align*}
    a + b &= a_1 \\
    a + c &= a_2 \\
    b + c &= a_3
\end{align*}
We can then represent this in matrix form:
$$ \begin{bmatrix}
    1 & 1 & 0 \\
    1 & 0 & 1 \\
    0 & 1 & 1
\end{bmatrix} \begin{bmatrix}
    a \\ b \\ c
\end{bmatrix} = \begin{bmatrix}
    a_1 \\ a_2 \\ a_3
\end{bmatrix} $$
We can then consider the augmented matrix and solve for RREF:
$$ \begin{bmatrix}[ccc|c]
    1 & 1 & 0 & a_1\\
    1 & 0 & 1 & a_2\\
    0 & 1 & 1 & a_3
\end{bmatrix} \xrightarrow{r_1 - r_2 \rightarrow r_2} \begin{bmatrix}[ccc|c]
    1 & 1 & 0 & a_1\\
    0 & -1 & 1 & a_2 - a_1\\
    0 & 1 & 1 & a_3
\end{bmatrix} \xrightarrow{r_2 + r_3 \rightarrow r_3} \begin{bmatrix}[ccc|c]
    1 & 1 & 0 & a_1\\
    0 & -1 & 1 & a_2 - a_1\\
    0 & 0 & 2 & a_3 + a_2 - a_1
\end{bmatrix}$$
$$ \xrightarrow{-\frac{1}{2}r_3 \rightarrow r_3} \begin{bmatrix}[ccc|c]
    1 & 1 & 0 & a_1\\
    0 & -1 & 1 & a_2 - a_1\\
    0 & 0 & 1 & \frac{a_3 + a_2 - a_1}{2}
\end{bmatrix} \xrightarrow{r_1 - r_2 \rightarrow r_2} \begin{bmatrix}[ccc|c]
    1 & 1 & 0 & a_1\\
    0 & -1 & 0 & \frac{-a_3 + a_2 - a_1}{2}\\
    0 & 0 & 1 & \frac{a_3 + a_2 - a_1}{2}
\end{bmatrix}$$
$$ \xrightarrow{r_1 + r_2 \rightarrow r_1} \begin{bmatrix}[ccc|c]
    1 & 0 & 0 & \frac{-a_3 + a_2 + a_1}{2} \\
    0 & -1 & 0 &  \frac{-a_3 + a_2 - a_1}{2}\\
    0 & 0 & 1 & \frac{a_3 + a_2 - a_1}{2}
\end{bmatrix} \xrightarrow{-r_2 \rightarrow r_2} \begin{bmatrix}[ccc|c]
    1 & 0 & 0 & \frac{-a_3 + a_2 + a_1}{2} \\
    0 & 1 & 0 &  \frac{a_3 - a_2 + a_1}{2}\\
    0 & 0 & 1 & \frac{a_3 + a_2 - a_1}{2}
\end{bmatrix}$$
Thus we have $a = \frac{-a_3 + a_2 + a_1}{2}$, $b = \frac{a_3 - a_2 + a_1}{2}$, and $c = \frac{a_3 + a_2 - a_1}{2}$.\\
Thus we have shown that any arbitrary vector $(a_1,a_2,a_3) \in F^3$ can be expressed as a linear combination of the given vectors.\\
\textbf{Note: }\\
Note that this does not work for every field! This is due to the fact there needs to exists an element of "$2$" in the fields as well as the operation of division by $2$ on the feild must be well defined.\\ 
And example of this is the field $F = \mathds{Z}_2$ where the operation of division by $2$ is not well defined.\\
\section*{Section 1.5  2(d)}
Determine whether the following sets are linearly dependant or linearly independent.
$$ \left\{x^3 - x, 2x^2 + 4, -2x^3 + 3x^2 + 2x+ 6   \right\} \text{ in } P_3(R)$$
\textbf{Answer:} \\
To see if this set is linearly dependant or independent, we can consider if there is a non-trivial set of scalars $a,b,c$ such that:
$$ a(x^3 - x) + b(2x^2 + 4) + c(-2x^3 + 3x^2 + 2x + 6) = 0$$
This implies the following system of equations:
\begin{align*}
    a - 2c &= 0\\
    2b + 3c &= 0\\
    -a + 2c &= 0\\
    4b + 6c &= 0
\end{align*}
We can then represent this in matrix form:  
$$ \begin{bmatrix}
    1 & 0 & -2 \\
    0 & 2 & 3 \\
    -1 & 0 & 2 \\
    0 & 4 & 6
\end{bmatrix} \begin{bmatrix}
    a \\ b \\ c
\end{bmatrix} = \begin{bmatrix}
    0 \\ 0 \\ 0 \\ 0
\end{bmatrix} $$
We can then consider the augmented matrix and solve for RREF:
$$ \begin{bmatrix}[ccc|c]
    1 & 0 & -2 & 0\\
    0 & 2 & 3 & 0\\
    -1 & 0 & 2 & 0\\
    0 & 4 & 6 & 0
\end{bmatrix} \xrightarrow{r_1 + r_3 \rightarrow r_3} \begin{bmatrix}[ccc|c]
    1 & 0 & -2 & 0\\
    0 & 2 & 3 & 0\\
    0 & 0 & 0 & 0\\
    0 & 4 & 6 & 0
\end{bmatrix}$$
$$ \xrightarrow{r_4 - 2r_2 \rightarrow r_4} \begin{bmatrix}[ccc|c]
    1 & 0 & -2 & 0\\
    0 & 2 & 3 & 0\\
    0 & 0 & 0 & 0\\
    0 & 0 & 0 & 0
\end{bmatrix} \xrightarrow{\frac{1}{2}r_2 \rightarrow r_2} \begin{bmatrix}[ccc|c]
    1 & 0 & -2 & 0\\
    0 & 1 & \frac{3}{2} & 0\\
    0 & 0 & 0 & 0\\
    0 & 0 & 0 & 0
\end{bmatrix}$$
Thus we can see that the system of equations has a non-trivial solution where $a = 2c$ and $b = -\frac{3}{2}c$.\\
We can verify that this is a non-trivial solution by plugging in the values of $a,b,c$ into the original equation.\\
For simplicity, we can choose $c = 2$.\\
Thus we have $a = 4$ and $b = -3$.\\
Thus the set is linearly dependant since there exists a non-trivial solution to the equation $a(x^3 - x) + b(2x^2 + 4) + c(-2x^3 + 3x^2 + 2x + 6) = 0$.

\section*{Section 1.5 6}
In $M_{m \times n}(F) $ let $E^{ij}$ denote the matrix whose only nonzero entry is 1 in the $i$-th row and $j$-th column. Prove that $\left\{E^{ij}: 1 \leq i \leq m, 1 \leq j \leq n \right\}$ is linearly independent.\\
\textbf{Answer:} \\
To show that the set is linearly independent we can show that there are no nontrivial linear combinations of the matrices that can generate the zero matrix.\\
Let $c_{ij}E^{ij} = 0$ for some scalars $c_{ij} \in F$.\\
$$ \sum_{i=1}^{m} \sum_{j=1}^{n} c_{ij}E^{ij} = 0$$
This means that the only nonzero entry in the matrix is at the $i$-th row and $j$-th column.\\
We can see that for each $E^{ij}$, the only nonzero entry is at the $i$-th row and $j$-th column.\\
For the matrix to be equal to the zero matrix, we must have $c_{ij} = 0$ for all $i,j$.\\
Thus the only solution is the trivial solution. Thus the set is linearly independent.\\

\section*{Section 1.6 2(b)
Determine which of the following sets are bases for $R^3$ 
$$ \left\{ (2,-4,1), (0,3,-1), (6, 0, -1)\right\}$$
\textbf{Answer:} \\}
We know that $\left\{(1,0,0),(0,1,0),(0,0,1) \right\}$ is a basis for $\mathds{R}^3$ as it is the standard basis for $\mathds{R}^3$\\
From the theorems and corollaries from the section it would be sufficient to show that any set of 3 linearly independent vectors in $R^3$ is a basis for $R^3$.\\
Thus we simply need to show that the set is linearly independent.\\
We can do this by showing that there are no nontrivial solutions to the equation:
$$ a(2,-4,1) + b(0,3,-1) + c(6,0,-1) = 0$$
This implies the following system of equations:
\begin{align*}
    2a + 6c &= 0\\
    -4a + 3b &= 0\\
    a - b - c &= 0
\end{align*}
We can then represent this in matrix form:
$$ \begin{bmatrix}
    2 & 0 & 6 \\
    -4 & 3 & 0 \\
    1 & -1 & -1
\end{bmatrix} \begin{bmatrix}
    a \\ b \\ c
\end{bmatrix} = \begin{bmatrix}
    0 \\ 0 \\ 0
\end{bmatrix} $$
We can then consider the augmented matrix and solve for RREF:
$$ \begin{bmatrix}[ccc|c]
    2 & 0 & 6 & 0\\
    -4 & 3 & 0 & 0\\
    1 & -1 & -1 & 0
\end{bmatrix} \xrightarrow{\frac{1}{2} r_1 \rightarrow r_1} \begin{bmatrix}[ccc|c]
    1 & 0 & 3 & 0\\
    -4 & 3 & 0 & 0\\
    1 & -1 & -1 & 0
\end{bmatrix} \xrightarrow{r_2 + 4r_1 \rightarrow r_2} \begin{bmatrix}[ccc|c]
    1 & 0 & 3 & 0\\
    0 & 3 & 12 & 0\\
    1 & -1 & -1 & 0
\end{bmatrix}$$
$$ \xrightarrow{r_3 - r_1 \rightarrow r_3} \begin{bmatrix}[ccc|c]
    1 & 0 & 3 & 0\\
    0 & 3 & 12 & 0\\
    0 & -1 & -4 & 0
\end{bmatrix} \xrightarrow{r_3 + \frac{1}{3}r_2 \rightarrow r_3} \begin{bmatrix}[ccc|c]
    1 & 0 & 3 & 0\\
    0 & 3 & 12 & 0\\
    0 & 0 & 0 & 0
\end{bmatrix} \xrightarrow{\frac{1}{3}r_2 \rightarrow r_2} \begin{bmatrix}[ccc|c]
    1 & 0 & 3 & 0\\
    0 & 1 & 4 & 0\\
    0 & 0 & 0 & 0
\end{bmatrix} $$
Thus we can see that the system of equations has a non-trivial solution where $a = -3c$ and $b = -4c$ where $c$ is a free variable.\\
Thus the set is linearly dependent and not a basis for $R^3$.

\section*{Section 1.6 3(b)}
Determine which of the following sets are bases for $P_2(R)$
$$ \left\{ 1+2x+x^2,3+x^2, x+x^2 \right\}$$
\textbf{Answer:}\\
We know that the set $\left\{ 1, x, x^2 \right\}$ is a basis for $P_2(R)$.\\
We can see that $P_2(R)$ has dimension 3.\\
To show that this set is a basis for $P_2(R)$, we need to show that the set is linearly independent since we are armed with the same theorems and corollaries from the prior question\\
We can do this by showing that there are no nontrivial solutions to the equation:
$$ a(1+2x+x^2) + b(3+x^2) + c(x+x^2) = 0$$
This implies the following system of equations:
\begin{align*}
    a + 3b &= 0\\
    2a + c &= 0\\
    a + b + c &= 0
\end{align*}
We can then represent this in matrix form:
$$ \begin{bmatrix}
    1 & 3 & 0 \\
    2 & 0 & 1 \\
    1 & 1 & 1
\end{bmatrix} \begin{bmatrix}
    a \\ b \\ c
\end{bmatrix} = \begin{bmatrix}
    0 \\ 0 \\ 0
\end{bmatrix} $$
We can then consider the augmented matrix and solve for RREF:
$$ \begin{bmatrix}[ccc|c]
    1 & 3 & 0 & 0\\
    2 & 0 & 1 & 0\\
    1 & 1 & 1 & 0
\end{bmatrix} \xrightarrow{r_3 - r_1 \rightarrow r_3} \begin{bmatrix}[ccc|c]
    1 & 3 & 0 & 0\\
    2 & 0 & 1 & 0\\
    0 & -2 & 1 & 0
\end{bmatrix} \xrightarrow{r_2 - 2r_1 \rightarrow r_2} \begin{bmatrix}[ccc|c]
    1 & 3 & 0 & 0\\
    0 & -6 & 1 & 0\\
    0 & -2 & 1 & 0
\end{bmatrix}$$
$$ \xrightarrow{-\frac{1}{6}r_2 \rightarrow r_2} \begin{bmatrix}[ccc|c]
    1 & 3 & 0 & 0\\
    0 & 1 & -\frac{1}{6} & 0\\
    0 & -2 & 1 & 0
\end{bmatrix} \xrightarrow{r_3 + 2r_2 \rightarrow r_3} \begin{bmatrix}[ccc|c]
    1 & 3 & 0 & 0\\
    0 & 1 & -\frac{1}{6} & 0\\
    0 & 0 & \frac{2}{3} & 0
\end{bmatrix}
\xrightarrow{\frac{3}{2}r_3 \rightarrow r_3} \begin{bmatrix}[ccc|c]
    1 & 3 & 0 & 0\\
    0 & 1 & -\frac{1}{6} & 0\\
    0 & 0 & 1 & 0
\end{bmatrix}$$ $$\xrightarrow{r_2 + 6r_3 \rightarrow r_2} \begin{bmatrix}[ccc|c]
    1 & 3 & 0 & 0\\
    0 & 1 & 0 & 0\\
    0 & 0 & 1 & 0
\end{bmatrix} \xrightarrow{r_1 - 3r_2 \rightarrow r_1} \begin{bmatrix}[ccc|c]
    1 & 0 & 0 & 0\\
    0 & 1 & 0 & 0\\
    0 & 0 & 1 & 0
\end{bmatrix}$$
Thus we can see that the system of equations has no non-trivial solutions as the only solution is the trivial solution of $a = b = c = 0$.\\
Thus the set is linearly independent and is a basis for $P_2(R)$.

\section*{Section 1.6 9}
The vectors $u_1 = (1,1,1,1), u_2 = (0,1,1,1), u_3 = (0,0,1,1), u_4 = (0,0,0,1)$ form a basis for $F^4$. Find the unique representation of an arbitrary vector $v = (a_1,a_2,a_3,a_4)$ in $F^4$ as a linear combination of $u_1 , u_2 , u_3 , u_4$.\\
\textbf{Answer:}\\
We can express an arbitrary vector $v = (a_1,a_2,a_3,a_4)$ as a linear combination of the given vectors:
$$ (a_1,a_2,a_3,a_4) = a(1,1,1,1) + b(0,1,1,1) + c(0,0,1,1) + d(0,0,0,1)$$
This gives us the following system of equations:
\begin{align*}
    a &= a_1\\
    a + b &= a_2\\
    a + b + c &= a_3\\
    a + b + c + d &= a_4
\end{align*}
We can then represent this in matrix form:
$$ \begin{bmatrix}
    1 & 0 & 0 & 0 \\
    1 & 1 & 0 & 0 \\
    1 & 1 & 1 & 0 \\
    1 & 1 & 1 & 1
\end{bmatrix} \begin{bmatrix}
    a \\ b \\ c \\ d
\end{bmatrix} = \begin{bmatrix}
    a_1 \\ a_2 \\ a_3 \\ a_4
\end{bmatrix} $$
We can then consider the augmented matrix and solve for RREF:
$$ \begin{bmatrix}[cccc|c]
    1 & 0 & 0 & 0 & a_1\\
    1 & 1 & 0 & 0 & a_2\\
    1 & 1 & 1 & 0 & a_3\\
    1 & 1 & 1 & 1 & a_4
\end{bmatrix} \xrightarrow{r_2 - r_1 \rightarrow r_2} \begin{bmatrix}[cccc|c]
    1 & 0 & 0 & 0 & a_1\\
    0 & 1 & 0 & 0 & a_2 - a_1\\
    1 & 1 & 1 & 0 & a_3\\
    1 & 1 & 1 & 1 & a_4
\end{bmatrix} \xrightarrow{r_3 - r_1 \rightarrow r_3} \begin{bmatrix}[cccc|c]
    1 & 0 & 0 & 0 & a_1\\
    0 & 1 & 0 & 0 & a_2 - a_1\\
    0 & 1 & 1 & 0 & a_3 - a_1\\
    1 & 1 & 1 & 1 & a_4
\end{bmatrix}$$
$$ \xrightarrow{r_4 - r_1 \rightarrow r_4} \begin{bmatrix}[cccc|c]
    1 & 0 & 0 & 0 & a_1\\
    0 & 1 & 0 & 0 & a_2 - a_1\\
    0 & 1 & 1 & 0 & a_3 - a_1\\
    0 & 1 & 1 & 1 & a_4 - a_1
\end{bmatrix} \xrightarrow{r_3 - r_2 \rightarrow r_3} \begin{bmatrix}[cccc|c]
    1 & 0 & 0 & 0 & a_1\\
    0 & 1 & 0 & 0 & a_2 - a_1\\
    0 & 0 & 1 & 0 & a_3 - a_2\\
    0 & 1 & 1 & 1 & a_4 - a_1
\end{bmatrix}$$
$$ \xrightarrow{r_4 - r_2 \rightarrow r_4} \begin{bmatrix}[cccc|c]
    1 & 0 & 0 & 0 & a_1\\
    0 & 1 & 0 & 0 & a_2 - a_1\\
    0 & 0 & 1 & 0 & a_3 - a_2\\
    0 & 0 & 1 & 1 & a_4 - a_2
\end{bmatrix} \xrightarrow{r_4 - r_3 \rightarrow r_4} \begin{bmatrix}[cccc|c]
    1 & 0 & 0 & 0 & a_1\\
    0 & 1 & 0 & 0 & a_2 - a_1\\
    0 & 0 & 1 & 0 & a_3 - a_2\\
    0 & 0 & 0 & 1 & a_4 - a_3
\end{bmatrix}$$
Thus we have $a = a_1$, $b = a_2 - a_1$, $c = a_3 - a_2$, and $d = a_4 - a_3$.\\
Thus we have found the unique representation of an arbitrary vector $v = (a_1,a_2,a_3,a_4)$ in $F^4$ as a linear combination of $u_1 , u_2 , u_3 , u_4$.



    

\end{document}
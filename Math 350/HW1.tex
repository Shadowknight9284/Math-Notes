\documentclass{article}
\usepackage{amsmath}
\usepackage{amsfonts}
\usepackage{amssymb}
\usepackage{mathrsfs}
\usepackage{dsfont}
\usepackage{cancel}

\usepackage{graphicx}



\setlength\parindent{0pt}

\author{Pranav Tikkawar}
\title{HW Math 350H}

\begin{document}
\maketitle
\section*{Section 1.2}
\section*{Question 7}
\subsection*{Let $S = \{ 0,1 \}$ and $F = R$. In $\mathcal{F}(S,F)$, show that $f = g$ and $f +g =h$ where $f(t) = 2t + 1$, $g(t) = 1 + 4t - 2t^2$, and $h(t) = 5^t + 1$}
We can prove this by proving each case separately.\\
\textbf{Proof of f = g}
\begin{align*}
    \text{Let } t = 0 \implies f(0) &= g(0) \implies 2(0)+1 = 1 + 4(0) - 2(0)^2 \implies 1 = 1\\
    \text{Let } t = 1 \implies f(1) &= g(1) \implies  = 2(1)+1 = 1 + 4(1) - 2(1)^2 \implies 3 = 3
\end{align*}
\textbf{Proof of f + g = h}
\begin{align*}
    \text{Let } t= 0 \implies f(0) + g(0) &= h(0) \implies 2(0)+1 + 1 + 4(0) - 2(0)^2 = 5^0 + 1 \implies 2 = 2\\
    \text{Let } t= 1 \implies f(1) + g(1) &= h(1) \implies 2(1)+1 + 1 + 4(1) - 2(1)^2 = 5^1 + 1 \implies 6 = 6
\end{align*}
\section*{Question 8}
\subsection*{In any vector space $V$, show that $(a+b)(x+y) = ax + ay + bx + by$  for any $x,y \in V$ and $a,b \in F$}
We can initially treat $x+y$ as a single vector and thus use (VS 7) to distribute the scalars then use (VS 8) on each of the resulting vectors to lead to 4 vectors.\\
\begin{align*}
    & (a+b)(x+y) \\
    &= a(x+y) + b(x+y) , (\text{VS } 7 )\\
    &= (ax + ay) + (bx + by), (\text{VS } 8)
\end{align*}
Note that the there are multiple parentesis representaions for this sum as follows:
\begin{align*}
    &= ((ax + ay) + bx) + by\\
    &= (ax + ay) + (bx + by)\\
    &= (ax + (ay + bx)) + by\\
    &= ax + (ay + (bx + by))\\
    &= ax + ((ay + bx) + by)\\
\end{align*}
These solutions are equivalent as by (VS 2) we can rearrange the order of parentesis along vector addition:
\begin{align*}
    ((ax + ay) + bx) + by = (ax + (ay + bx)) + by (VS 2)\\
    (ax + (ay + bx)) + by = ax + ((ay + bx) + by) (VS 2)\\
    ax + ((ay + bx) + by) = ax + (ay + (bx + by)) (VS 2)\\
    ax + (ay + (bx + by)) = (ax + ay) + (bx + by) (VS 2)
\end{align*}
Finally we can see that $((ax + ay) + (bx + by))$  is equivalent to $ax + ay + bx + by$\\
By the transitive property of equality, we can conclude that $(a+b)(x+y) = ax + ay + bx + by$ regardless of the parenthesis representations we choose to distribute the term out by.\\



\section*{Question 9}
\subsection*{Prove Corollaries 1 and 2 of Theorem 1.1 and Theorem 1.2 (c)}
\textbf{Corollary 1:} The vector \underline{0} in (VS 3)is unique.\\ 
\textbf{Proof:} Assume there are two zero vectors $\underline{0}$ and $\underline{0}'$ in $V$.\\
Thus $\forall x \in V$ we have $x + \underline{0} = x$ and $x + \underline{0}' = x$\\
Thus by transitiviy we have $x + \underline{0} = x + \underline{0}'$\\
By theorem 1.1 we have $\underline{0}  = \underline{0}'$\\
This is a contradiction as we assumed there were two distinct zero vectors.\\
Thus the zero vector is unique.\\
\textbf{Corollary 2:} The vector y in (VS 4) is unique.\\
\textbf{Proof:} Assume there are two vectors $y$ and $y'$ in $V$ such that $x + y = 0$ and $x + y' = 0$\\
Thus by transitiviy we have $x + y = x + y'$\\
Thus by theorem 1.1 we have $y = y'$\\
This is a contradiction as we assumed there were two distinct vectors.\\
Thus the vector y is unique.\\
\textbf{Theorem 1.2 (c):} $a\underline{0} = \underline{0}, \forall a \in F$\\
\textbf{Proof:} Let $a \in F$, Need $a \underline{0} = \underline{0}$\\
Consider $a\underline{0} + a \underline{0}$\\
Thus by (VS 7) we have $a\underline{0} + a\underline{0}= a(\underline{0} + \underline{0})$\\
Thus by (VS 3) we have $a(\underline{0} + a\underline{0}) = a\underline{0}$\\
By (VS 4) We let $y = -a \underline{0}$ and add y to both sided\\
Thus we have $a\underline{0} = \underline{0}$ as desired\\



\section*{Question 11}
\subsection*{Let $V = \{0\}$ consit of a single vector 0 an define $0+0 = 0$ and $c0 = 0$ for each $c \in F$. Prove that $V$ is a vector space over $F$}
Note that $\forall x \in V \implies x = 0$ Since V has only one element.\\
Thus for the following proofs I will simply state what needs to be proven and then implicitly let $x, y, z $ be arbitrary elements of $V$ which implies $x,y,z = 0$\\
\textbf{Proof of VS 1:}\\
Let $x,y \in V$ Nee $x+y = y+x \in V$\\
$0 + 0 = 0 + 0$\\ 
$0 = 0$\\
\textbf{Proof of VS 2:}\\
Let $x,y,z \in V$ Need $(x+y)+z = x+(y+z)$\\
$(0 + 0) + 0 = 0 + (0 + 0)$\\
$0 + 0 = 0 + 0$\\
$0 = 0$\\
\textbf{Proof of VS 3:}\\
Let $x \in V$ Need $\exists \underline{0} \in V, x + \underline{0} = x$\\
$ 0 + \underline{0} = 0$\\
Since $0$ is the only element in $V$, $\underline{0} = 0$\\
Thus $0+0=0$\\
$0=0$\\
\textbf{Proof of VS 4:}\\
Let $x \in V$ Need $\exists y \in V, x + y = 0$\\
Let $y = 0$\\
$x +y = 0 + 0 = 0$ as desired\\
\textbf{Proof of VS 5:}\\
Let $x \in V$ Need $1x = x$\\
$1*0 = 0$\\
$0 = 0$\\
\textbf{Proof of VS 6:}\\
Let $a,b \in F, x \in V$ Need $(ab)x = a(bx)$\\
$ab0 = a(b0)$\\
$0 = a0$\\
$0 = 0$\\
\textbf{Proof of VS 7:}\\
Let $a \in F, x,y \in V$ Need $a(x+y) = ax + ay$\\
$a(0+0) = a0 + a0$\\
$0 = 0$\\
\textbf{Proof of VS 8:}\\
Let $a,b \in F, x \in V$ Need $(a+b)x = ax + bx$\\
$(a+b)0 = a0 + b0$\\
$0 = 0$\\

\section*{Question 12}
\subsection*{A real-valued function $f$ is defined on the set of all real numbers is called and even function if $f(-t) = f(t)$ for all $t \in \mathds{R}$. Prove that the set of even functions defined on the real line with the operations of addition and scalar multiplication defined by $(f+g)(t) = f(t) + g(t)$ and $(cf)(t) = c[f(t)]$ is a vector space over $\mathds{R}$}
Note that any for 2 even functions $f,g$ their sum $f+g$ is also even as:
\begin{align*}
    (f+g)(-t) &= f(-t) + g(-t)\\
    &= f(t) + g(t)\\
    &= (f+g)(t)
\end{align*}
As well as for any scalar $c$:
\begin{align*}
    (cf)(-t) &= c[f(-t)]\\
    &= c[f(t)]\\
    &= (cf)(t)
\end{align*}
\textbf{Proof of VS 1:}\\
Let $f,g \in V$, Need $f+g = g+f$\\
$(f+g)(t) = f(t) + g(t) = g(t) + f(t)$\\
$f(t) + g(t) = f(t) + g(t)$\\
$f+g = g+f$ as desired\\
\textbf{Proof of VS 2:}\\
Let $f,g,h \in V$ Need $(f+g)+h = f+(g+h)$\\
$((f+g)+h)(t) = (f+g)(t) + h(t) = f(t) + g(t) + h(t)$\\
$(f+(g+h))(t) = f(t) + (g+h)(t) = f(t) + g(t) + h(t)$\\
Thus $(f+g)+h = f+(g+h)$ as desired\\
\textbf{Proof of VS 3:}\\
$\exists \underline{0} \in V, \forall f \in V$ Need $f + \underline{0} = f$\\
$(f+\underline{0})(t) = f(t) + \underline{0}(t) = f(t)$\\
Let $\underline{0}(t) = 0$\\
$f(t) + \underline{0} = f(t)$ as desired \\
Note that $\underline{0} \in V$ as $0$ is even since $0(t) = 0(-t) = 0$\\
\textbf{Proof of VS 4:}\\
Let $f \in V$ Need $\exists g \in V$ s.t. $f+g = \underline{0}$\\
Let $g = -1f$\\
$(f+ -1f)(t) = f(t) + -1f(t) = \underline{0}$ as desired\\
Note that $g$ is even as $g(t) = -1f(t) = -1f(-t) = g(-t)$\\
\textbf{Proof of VS 5:}\\
Let $f \in V$ Need $1f = f$\\
$(1f)(t) = 1f(t)$\\
$1f(t) = f(t)$ \\
Thus $1f = f$ as desired\\
\textbf{Proof of VS 6:}\\
Let $a,b \in \mathds{R}, f \in V$ Need $(ab)f = a(bf)$\\
$(ab)f(t) = a(bf(t))$\\
$abf(t) = abf(t)$\\
$(ab)f = a(bf)$ as desired\\
\textbf{Proof of VS 7:}\\
Let $a \in \mathds{R}, f,g \in V$ Need $a(f+g) = af + ag$\\
$a(f+g)(t) = a(f(t) + g(t)) = af(t) + ag(t) = af +ag$ as desired\\
\textbf{Proof of VS 8:}\\
$\forall a,b \in \mathds{R}, \forall f \in V, (a+b)f = af + bf$\\
$(a+b)f(t) = af(t) + bf(t) = af +bf$ as desired\\

\section*{Question 17}
\subsection*{Let $V = \{(a_1,a_2): a_1, a_2 \in F\}$.  where F is a field. Define addition of elements of V coordinatewise, and for $c \in F$ and $(a_1,a_2) \in V$ define $c(a_1,a_2) = (a_1,0)$ Is V a vector space over F with operations?}
V is not a vector space as (VS 5) does not hold.\\
Let $x = (0,1), c = 1$.\\
Thus $1x = 1(0,1) = (0,0)$\\
Clearly $(0,0) \neq (0,1)$

\section*{Section 1.3}
\section*{Question 5}
\subsection*{Prove that $A + A^{t}$ is symmetric for any square matrix A }
Let $A$ be an $n$ by $n$ matrix with each entry $a_{i,j}$ corresponding to entry in the ith row and jth column.\\
Clearly $A^t$ has the values of $a_{ij}$ in the entries in the ith column and jth row. In other words, its values of $a_{ji}$ in the ith row and jth column\\
Thus $A + A^{t}$ has entries of $a_{ij}+a_{ji}$ in the ith row and jth column.\\
This would be symmetric as for every symmetric matrix the for each entry $a_{ij} = a_{ji}$\\
Clearly $a_{ij}+a_{ji} = a_{ji}+a_{ij}$\\
Thus $A + A^{t}$ is symmetric

\section*{Question 8a}
\subsection*{Determine if the following sets are subspaces of $\mathds{R}^3$ under the operation of adition and scalar multiplication defined on $\mathds{R}^3$ Justify your answers: $W_1 = \{(a_1, a_2,a_3) \in \mathds{R}^3: a_1 = 3a_2 \text{ and } a_3 = -a_2\}$}
This is a subspace as the since $W$ is a subset of a vector space ($\mathds{R}^3$) it must satisfy the 8 properties of a vector space.\\
It also satisfies $0 \in W$ as for $a_2 = 0, (0,0,0) \in W$\\
It satisfies the closure property of addition as:\\
Let $x = (a_1,a_2,a_3)$ and $y = (b_1,b_2,b_3)$\\
Thus $x = (3a_2,a_2,-a_2)$ and $y = (3b_2,b_2,-b_2)$\\
$x+y = (3a_2+3b_2,a_2+b_2,-a_2-b_2)$\\
$x+y = (3(a_2+b_2),a_2+b_2,-(a_2+b_2))$\\
Let $c = a_2+b_2$\\
$x+y = (3c,c,-c)$\\
This clearly is also in $W$ thus it satisfies closure.\\
It satisfies the closure property of scalar multiplication as:\\
Let $x = (a_1,a_2,a_3)$\\
$x = (3a_2,a_2,-a_2)$\\
$cx = (3ca_2,ca_2,-ca_2)$\\
$cx = (3c(a_2),c(a_2),-c(a_2))$\\
Let $d = ca_2$\\
$cx = (3d,d,-d)$\\
This clearly is also in $W$ thus it satisfies closure.\\

\section*{Question 8b}
\subsection*{Determine if the following sets are subspaces of $\mathds{R}^3$ under the operation of adition and scalar multiplication defined on $\mathds{R}^3$ Justify your answers: $W_1 = \{(a_1, a_2,a_3) \in \mathds{R}^3: a_1 = a_3 + 2\}$}
This is not a subspace as there is no element in the set that satisfies the condition $0 \in W$\\
Let $x = (a_1,a_2,a_3)$\\
Thus $x = (a_3 + 2,a_2,a_3)$\\
Clearly $0 \notin W $ since $a_3 + 2 \neq a_3, \forall a_3 \in \mathds{R}$\\

\section*{Question 11}
\subsection*{Is the set $W = \{f \in P(F): f(x) = 0 \text{ or } f(x) \text{ has degree } n \}$ a subspace of $P(F)$ if $n \geq 1$? Justify your answer.}
Yes, this is a subspece of $P(F)$ as it satisfies the closure properties of addition and scalar multiplication as well has the 0 polynomial.
Firstly: $0 \in W$ as $f(x) = 0$\\
Thus the zero polynomial is in $W$\\
Secondly: Let $f,g \in W$\\
Then $f(x) = 0$ or $f(x)$ has degree $n$ and can be represented as $\sum_{i=0}^{n} a_i x^i$\\
Then $g(x) = 0$ or $g(x)$ has degree $n$ and can be represented as $\sum_{i=0}^{n} b_i x^i$\\
Thus $f + g = \sum_{i=0}^{n} (a_i + b_i)x^i$\\
If $f(x) = 0$ and $g(x) = 0$ then $f + g = 0$\\
If $f(x)$ or $g(x)$ has degree $n$ and the other is 0, then $f + g$ is either $f$ or $g$ (whichever one is of degree $n$) has degree $n$\\
If both $f(x)$ and $g(x)$ have degree $n$, then $f + g = \sum_{i=0}^{n} (a_i + b_i)x^i$ which also has degree $n$ \\
Thus it is closed under addition\\

Thirdly: Let $f \in W, c \in \mathds{R}$\\
Then $f(x) = 0$ or $f(x)$ has degree $n$ and be represented as $\sum_{i=0}^{n} a_i x^i$\\
Then $cf = c\sum_{i=0}^{n} a_i x^i$\\
Thus $cf \in W$ as $cf = 0$ or $cf = c\sum_{i=0}^{n} a_i x^i$ which has degree $n$\\
Thus it is closed under scalar multiplication\\



\end{document}
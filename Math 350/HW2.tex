\documentclass{article}
\usepackage{amsmath}
\usepackage{amsfonts}
\usepackage{amssymb}
\usepackage{mathrsfs}
\usepackage{dsfont}
\usepackage{cancel}

\usepackage{graphicx}


\setlength\parindent{0pt}

\author{Pranav Tikkawar}
\title{Linear Algebra Homework 2}

\begin{document}
\maketitle

\section*{Section 1.3: 19}
Let $V = \{ \{ (a_1,a_2): a_1, a_2 \in \mathds{R}\} \}$ Define addition of elements of $V$ coordinates and for $(a_1,a_2) \in V$ and $c \in \mathds{R}$, define:
$$c(a_1, a_2)=
\begin{cases}
    (0,0) \text{if } c = 0 \\
    (ca_1, a_2/c) \text{if } c \neq 0
\end{cases}
$$
Is $V$ a vector space over $\mathds{R}$ with these operations?\\
\textbf{Answer:} \\
No, this is not a vector space.\\
To show that $V$ is not a vector space, we need to show that it fails to satisfy one of the vector space axioms. Specifically, we will show that it fails to satisfy axiom VS 8.\\
Let $x = (1,1) \in V$ and let $a = 1$ and $b = 1$. Then $(a+b)x = (1+1)(1,1) = (2)(1,1) = (2,1/2)$.\\
And $ax + bx = 1(1,1) + 1(1,1) = (1,1) + (1,1) = (1+1, 1+1) = (2,2)$.\\
These two are not equal thus $(a+b)x \neq ax + bx$.\\
Thus $V$ is not a vector space.\\
\section*{Section 1.4: 4(a)}
For each list of polynomials in $P_3(R)$ determine whether the first polynomial can be expressed as a linear combination of the other two.\\
$$ x^3 - 3x + 5, \quad x^3 + 2x^2 - x +1, \quad x^3 +3 x^2 - 1 $$\\
\textbf{Answer:}\\
Let $p_1 = x^3 - 3x + 5$, $p_2 = x^3 + 2x^2 - x + 1$, and $p_3 = x^3 + 3x^2 - 1$.\\

We want to see if $p_1 = ap_2 + bp_3$ for some $a,b \in \mathds{R}$.\\
Thus we want:
$$ x^3 - 3x + 5 = a(x^3 + 2x^2 - x + 1) + b(x^3 + 3x^2 - 1) $$
Let $a = 3, b = -2$. Then we have:
\begin{align*}
    3(x^3 + 2x^2 - x + 1) - 2(x^3 + 3x^2 - 1) &= 3x^3 + 6x^2 - 3x + 3 - 2x^3 - 6x^2 + 2 \\
    &= (3-2)x^3 + (6-6)x^2 + (-3)x + (3+2)\\ 
    &= x^3 - 3x + 5
\end{align*}
Thus $p_1$ can be expressed as a linear combination of $p_2$ and $p_3$.\\

\section*{Section 1.4: 5(g)}
In each part, determine wheter the given vector is in the span of S.
$$ \begin{bmatrix}
    1 & 2 \\
    -3 & 4
\end{bmatrix}, S = \left\{ \begin{bmatrix}
    1 & 0 \\
    -1 & 0
\end{bmatrix}, \begin{bmatrix}
    0 & 1 \\
    0 & 1
\end{bmatrix}, \begin{bmatrix}
    1 & 1 \\
    0 & 0
\end{bmatrix}
\right\}\\
$$
\textbf{Answer:} \\
We want to see if we can express the vector as a linear combination of the vectors in $S$.\\
For this to be possible we need to find $a,b,c \in \mathds{R}$ such that:
$$ a \begin{bmatrix}
    1 & 0 \\
    -1 & 0
\end{bmatrix} + b \begin{bmatrix}
    0 & 1 \\
    0 & 1
\end{bmatrix} + c \begin{bmatrix}
    1 & 1 \\
    0 & 0
\end{bmatrix} = \begin{bmatrix}
    1 & 2 \\
    -3 & 4
\end{bmatrix}$$
This gives us the following system of equations:
\begin{align*}
    a + c &= 1 \\
    b + c &= 2 \\
    -a &= -3 \\
    b &= 4
\end{align*}
From the third equation we have $a = 3$.\\
From the fourth equation we have $b = 4$.\\
Substituting these values into the first equation gives us:
$$ 3 + c = 1 \implies c = -2$$
Now substituting these values into the second equation gives us:
$$ 4 + c = 2 \implies c = -2$$
Thus we have $a = 3, b = 4, c = -2$.\\
Thus the vector is in the span of $S$ since there exists a combination of $a,b,c \in \mathds{R}$ that can generate a linear combination.\\

\section*{Section 1.4: 6}
Show that the vectors $(1,1,0), (1,0,1), (0,1,1)$ generate $F^3$.\\

 \textbf{Answer:} \\
To show that the vectors generate $F^3$, we need to show that any arbitrary vector $(a_1,a_2,a_3) \in F^3$ can be expressed as a linear combination of the given vectors.\\
We can express an arbitrary vector $(a_1,a_2,a_3) \in F^3$ as a linear combination of the given vectors: 
$$ (a_1,a_2,a_3) = a(1,1,0) + b(1,0,1) + c(0,1,1)$$
This gives us the following system of equations:
\begin{align*}
    a + b &= a_1 \\
    a + c &= a_2 \\
    b + c &= a_3
\end{align*}
We can solve this system of equations for $a,b,c$ in terms of $a_1,a_2,a_3$.\\
Subtracting the first equation from the second gives us:
$$ c - b = a_2 - a_1 \implies c = a_2 - a_1 + b$$
Substituting this into the third equation gives us:
$$ b + (a_2 - a_1 + b) = a_3 \implies 2b + a_2 - a_1 = a_3 \implies 2b = a_3 - a_2 + a_1 \implies b = \frac{a_3 - a_2 + a_1}{2}$$
Following similar steps we can find $a$ and $c$ in terms of $a_1,a_2,a_3$.\\
$$ a = a_1 - b = a_1 + \frac{-a_3 + a_2 - a_1}{2} = \frac{2a_1 - a_3 + a_2 - a_1}{2} = \frac{a_1 + a_2 - a_3}{2}$$
$$ c = a_3 - b = a_3 - \frac{a_3 - a_2 + a_1}{2} = \frac{2a_3 - a_3 + a_2 - a_1}{2} = \frac{a_3 + a_2 - a_1}{2}$$
Thus we can see that any arbitrary vector $(a_1,a_2,a_3)$ can be expressed as a linear combination of the given vectors. With the scalars $a,b,c$ being defined as above. Thus the vectors generate $F^3$.\\

\textbf{Note:}
Since there are 3 vector and we want to generate $F^3$, we can also check if the vectors are linearly independent. If they are linearly independent then they will generate $F^3$.\\
To check for linear independence we can set up the following equation:
$$ a(1,1,0) + b(1,0,1) + c(0,1,1) = (0,0,0)$$
and prove that the only solution is $a = b = c = 0$.\\
This gives us the following system of equations:
\begin{align*}
    a + b &= 0 \\
    a + c &= 0 \\
    b + c &= 0
\end{align*}
From the first equation we have $b = -a$.\\
Substituting this into the second equation gives us:
$$ a - a + c = 0 \implies c = 0$$
Substituting this into the third equation gives us:
$$ -a + 0 = 0 \implies a = 0$$
Thus $b = 0$.\\
Thus the only solution is $a = b = c = 0$. Thus the vectors are linearly independent and generate $F^3$.\\





\section*{Section 1.5: 2(d)}
Determine whether the following sets are linearly dependant or linearly independent.
$$ \left\{x^3 - x, 2x^2 + 4, -2x^3 + 3x^2 + 2x+ 6   \right\} \text{ in } P_3(R)$$
\textbf{Answer:} \\
This set is linearly dependent.\\
This is due to the fact that there exists a non trivial linear combination of the vectors that can generate the zero vector.\\
Let $a = -1, b = 3/2, c = 1$ Then we have:
$$ -1(x^3 - x) + \frac{3}{2}(2x^2 + 4) + c(-2x^3 + 3x^2 + 2x + 6) = 0$$
Thus the set is linearly dependent.\\


\section*{Section 1.5: 6}
In $M_{m \times n}(F) $ let $E^{ij}$ denote the matrix whose only nonzero entry is 1 in the $i$-th row and $j$-th column. Prove that $\left\{E^{ij}: 1 \leq i \leq m, 1 \leq j \leq n \right\}$ is linearly independent.\\
\textbf{Answer:} \\
To show that the set is linearly independent we can show that there are no nontrivial linear combinations of the matrices that can generate the zero matrix.\\
Let $c_{ij}E^{ij} = 0$ for some scalars $c_{ij} \in F$.\\
$$ \sum_{i=1}^{m} \sum_{j=1}^{n} c_{ij}E^{ij} = 0$$
This means that the only nonzero entry in the matrix is at the $i$-th row and $j$-th column.\\
We can see that for each $E^{ij}$, the only nonzero entry is at the $i$-th row and $j$-th column.\\
For the matrix to be equal to the zero matrix, we must have $c_{ij} = 0$ for all $i,j$.\\
Thus the only solution is the trivial solution. Thus the set is linearly independent.\\



\section*{Section 1.6: 2(b)}
Determine which of the following sets are bases for $R^3$ 
$$ \left\{ (2,-4,1), (0,3,-1), (6, 0, -1)\right\}$$
\textbf{Answer:} \\
To determine if the set is a basis for $R^3$ for a set of 3, 3-dimensional vectors, we need to check if the vectors are linearly independent.\\
We can do this by showing there is no nontrivial linear combination of the vectors that can generate the zero vector.\\
Let $a(2,-4,1) + b(0,3,-1) + c(6,0,-1) = (0,0,0)$ for some scalars $a,b,c \in R$.\\
This gives us the following system of equations:
\begin{align*}
    2a + 6c &= 0 \\
    -4a + 3b &= 0 \\
    a - b - c &= 0
\end{align*}
We can put this into an augmented matrix and row reduce:
$$ \begin{bmatrix}
    2 & 0 & 6 & | & 0 \\
    -4 & 3 & 0 & | & 0 \\
    1 & -1 & -1 & | & 0
\end{bmatrix}$$
Row reducing gives us:
$$ \begin{bmatrix}
    1 & 0 & 3 & | & 0 \\
    0 & 1 & 4 & | & 0 \\
    0 & 0 & 0 & | & 0
\end{bmatrix}$$
We can see that there are 2 leading 1's and 1 free variable. Thus there exists a nontrivial solution. \\
We can not that we can let $a = 3, b = 4, c = -1$. To satisfy the equation $a(2,-4,1) + b(0,3,-1) + c(6,0,-1) = (0,0,0)$.\\
Thus the set is not a basis for $R^3$.\\

\section*{Section 1.6: 3(b)}
Determine which of the following sets are bases for $P_2(R)$
$$ \left\{ 1+2x+x^2,3+x^2, x+x^2 \right\}$$
\textbf{Answer:} \\
To determine if the set is a basis for $P_2(R)$, we need to check if the vectors are linearly independent.\\
We can view each of these polynomial as vectors in $R^3$.\\
Let $p_1 = 1 + 2x + x^2$, $p_2 = 3 + x^2$, $p_3 = x + x^2$.\\
We can express these as vectors:
$$ p_1 = \begin{bmatrix}
    1 \\
    2 \\
    1
\end{bmatrix}, p_2 = \begin{bmatrix}
    3 \\
    0 \\
    1
\end{bmatrix}, p_3 = \begin{bmatrix}
    0 \\
    1 \\
    1
\end{bmatrix}$$
We can set up the following system of equations:
$$ a(1 + 2x + x^2) + b(3 + x^2) + c(x + x^2) = 0$$
This gives us the following system of equations:
\begin{align*}
    a + 3b &= 0 \\
    2a + c &= 0 \\
    a + b + c &= 0
\end{align*}
We can put this into an augmented matrix and row reduce:
$$ \begin{bmatrix}
    1 & 3 & 0 & | & 0 \\
    2 & 0 & 1 & | & 0 \\
    1 & 1 & 1 & | & 0
\end{bmatrix}$$
Row reducing gives us:
$$ \begin{bmatrix}
    1 & 0 & 0 & | & 0 \\
    0 & 1 & 0 & | & 0 \\
    0 & 0 & 1 & | & 0
\end{bmatrix}$$
We can see that there are 3 leading 1's and no free variables. Thus the only solution is the trivial solution. Thus the set is linearly independent.\\

\section*{Section 1.6: 9}
The vectors $u_1 = (1,1,1,1), u_2 = (0,1,1,1), u_3 = (0,0,1,1), u_4 = (0,0,0,1)$ form a basis for $F^4$. Find the unique representation of an arbitrary vector $v = (a_1,a_2,a_3,a_4)$ in $F^4$ as a linear combination of $u_1 , u_2 , u_3 , u_4$.\\
\textbf{Answer:}\\
An arbitrary vector $v = (a_1,a_2,a_3,a_4)$ can be expressed as a linear combination of the basis vectors $u_1, u_2, u_3, u_4$ as follows:
Let $v = c_1 u_1 + c_2 u_2 + c_3 u_3 + c_4 u_4$ for some scalars $c_1, c_2, c_3, c_4 \in F$.\\
This gives us the following system of equations:
$$ v = \begin{bmatrix}
    c_1 \\
    c_1 + c_2 \\
    c_1 + c_2 + c_3 \\
    c_1 + c_2 + c_3 + c_4
\end{bmatrix} = \begin{bmatrix}
    a_1 \\
    a_2 \\
    a_3 \\
    a_4
\end{bmatrix}
$$

\end{document}
\documentclass[answers,12pt,addpoints]{exam}
\usepackage{import}

\import{C:/Users/prana/OneDrive/Desktop/MathNotes}{style.tex}

% Header
\newcommand{\name}{Pranav Tikkawar}
\newcommand{\course}{01:640:350H}
\newcommand{\assignment}{Homework 9}
\author{\name}
\title{\course \ - \assignment}

\begin{document}
\maketitle


\begin{questions}
    \question Sec. 5.4 Problem 2(b)\\
    For each of the following linear operators $T$ on the vector space $V$, and if $V$ is the determine wheter the given subspace $W$ is a $T$-invariant subspace of $V$.\\
    $$V = P(R), \quad T(f(x)) =xf(x), \quad W = P_2(R)$$
    \begin{solution}
        Clearly $W$ is not a $T$-invariant subspace of $V$ since if we take $f(x) = x^2$, then $T(f(x)) = x^3$ which is not in $W$.
    \end{solution}
    \question Sec. 5.4 Problem 3\\
    Let $T$ be a linear operator on a finite dimensional bector space $V$. Prove that the following subspaces are $T$-invariant subspaces of $V$.\\
    \begin{parts}
        \part $\setof{0}$ and $V$\\
        \part $N(T)$ and $R(T)$\\
        \part $E_{\lambda}$ for any eigenvalue $\lambda$ of $T$\\
    \end{parts}
    \begin{solution}
        \textbf{Case: $\setof{0}$}\\
        This is trivial since $T(0) = 0$ and $0 \in \setof{0}$.\\
        \textbf{Case: $V$}\\
        This is also trivial since $T(v) = w$ for any $v \in V$ and $w \in V$.\\
        \textbf{Case: $N(T)$}\\
        Let $v \in N(T)$, then $T(v) = 0$. Since $T$ is a linear operator, $T(0) = 0$ and $0 \in N(T)$. Thus $N(T)$ is a $T$-invariant subspace of $V$.\\
        \textbf{Case: $R(T)$}\\
        Let $v \in R(T)$, then $T(v) = w$. $w \in R(t)$ by definition of $R(T)$. Thus $R(T)$ is a $T$-invariant subspace of $V$.\\
        \textbf{Case: $E_{\lambda}$}\\
        Let $v \in E_{\lambda}$, then $T(v) = \lambda v$. Since $T$ is a linear operator, $T(\lambda v) = \lambda T(v) = \lambda^2 v$. Thus $E_{\lambda}$ is a $T$-invariant subspace of $V$.
    \end{solution}
    \question Sec. 5.4 Problem 6(a)
    For each of the linear operator $T$ on the vector space $V$, find an ordered basis for the $T$-cyclic subspace generated by the vector $z$.\\
    $$V = R^4 \quad T(a,b,c,d) = (a+b, b-c,a+c,a+d) \quad z = e_1$$
    \begin{solution}
        Since $T$ is a linear operator we can say $T = L_A$ for 
        $$A = \begin{bmatrix}
            1 & 1 & 0 & 0\\
            0 & 1 & -1 & 0\\
            1 & 0 & 1 & 0\\
            1 & 0 & 0 & 1
        \end{bmatrix}$$
        Let $W$ be the $T$-cyclic subspace generated by $z$ and $\gamma$ be the basis of $W$. We know that the generating set of $W$ is $\setof{z, T(z), T^2(z), \ldots, T^{n-1}(z)}$. Thus we need the longest LI set of vectors from this set.\\
        Thus 
        \begin{align*}
            T(z) &= (1,0,1,1)\\
            T^2(z) &= (1, -1, 2, 2)\\
            T^3(z) &= (0, -3, 3, 3)
        \end{align*}
        Clearly $T^3(z) = -3T(z) + 3T^2(z)$. Thus $\gamma = \setof{z, T(z), T^2(z)}$ is a basis for $W$.\\
        Now we can see that for any $v \in W$, $v = a_1z + a_2T(z) + a_3T^2(z)$. Which implies $T(v) = a_1T(z) + a_2T^2(z) + a_3T^3(z) = (a_1-3a_3)T(z) + (a_2+3a_3)T^2(z)$ which is in $W$. \\
        Thus we can see that $W$ is a $T$-invariant subspace of $V$. 
    \end{solution}
    \question Sec. 5.4 Problem 6(b)
    For each of the linear operator $T$ on the vector space $V$, find an ordered basis for the $T$-cyclic subspace generated by the vector $z$.\\
    $$ V = P_3(R) \quad T(f(x)) = f''(x) \quad z = x^3$$
    \begin{solution}
        We can see that $T = L_A$ for
        $$A = \begin{bmatrix}
            0 & 0 & 2 & 0\\
            0 & 0 & 0 & 6\\
            0 & 0 & 0 & 0\\
            0 & 0 & 0 & 0
        \end{bmatrix}$$
        Let $W$ be the $T$-cyclic subspace generated by $z$ and $\gamma$ be the basis of $W$. We know that the generating set of $W$ is $\setof{z, T(z), T^2(z), \ldots, T^{n-1}(z)}$. Thus we need the longest LI set of vectors from this set to be the basis for $W$\\
        Thus
        \begin{align*}
            T(z) &= 6x\\
            T^2(z) &= 0
        \end{align*}
        We can see that for any $k > 2$, $T^k(z) = 0$. Thus $\gamma = \setof{z, T(z)} = \setof{x^3, x}$ is a basis for $W$.
    \end{solution}
    \question Sec. 5.4 Problem 9 (for 6(a),(b))
    For each Linear operator $T$ and cyclic subspace $W$ in Exercise 6, compute the characteristic polynomial of $T_W$ in two ways as in Example $6$.
    \begin{solution}
        \textbf{Case: 6(a)}\\
        By means of Theorem 5.21 we can see that $T^3(z) = -3T(z) + 3T^2(z)$. Hence
        $$ 0z + 3T(z) - 3T^2(z) + T^3(z) = 0$$
        Therefore by Theorem 5.21, the characteristic polynomial of $T_W$ is
        $$f(t) = (-1)^3 (0+3t-3t^2+t^3) = -t^3 + 3t^2 - 3t$$
        By means of determinants we can see that $\gamma = \setof{z, T(z), T^2(z)}$ is a basis for $W$. $T(z) = (1,0,1,1)$, $T^2(z) = (1,-1,2,2)$, $T^3(z) = (0,-3,3,3)$. 
        \begin{align*}
            T(z) &= (1,0,1,1) \implies [(0,1,0)]_{\gamma}\\
            T^2(z) &= (1,-1,2,2) \implies [(0,0,1)]_{\gamma}\\
            T^3(z) &= (0,-3,3,3) \implies [(0,-3,3)]_{\gamma}
        \end{align*}
        Thus $[T_W]_{\gamma} = \begin{bmatrix}
            0 & 0 &0\\
            1 & 0 &-3\\
            0 & 1 &3
        \end{bmatrix}$. Thus the characteristic polynomial of $T_W$ is
        $$f(t) = det(A - tI) = \begin{vmatrix}
            -t & 0 & 0\\
            1 & -t & -3\\
            0 & 1 & 3-t
        \end{vmatrix}$$
        $$ = -t\begin{vmatrix}
            -t & -3\\
            1 & 3-t
        \end{vmatrix} = -t(t^2 - 3t + 3) = -t^3 + 3t^2 - 3t$$
        \textbf{Case: 6(b)}\\
        By means of Theorem 5.21: we can see that $T^2(z) = 0$. Hence
        $$ 0z + 0T(z) + T^2(z) = 0$$
        Therefore by Theorem 5.21, the characteristic polynomial of $T_W$ is
        $$f(t) = (-1)^2 (0+0t+t^2) = t^2$$
        By means of determinants we can see that $\gamma = \setof{z, T(z)} = \setof{x^3, x}$ is a basis for $W$. $T(z) = 6x$, $T^2(z) = 0$.
        \begin{align*}
            T(z) &= 6x \implies [(0,6)]_{\gamma}\\
            T^2(z) &= 0 \implies [(0,0)]_{\gamma}
        \end{align*}
        Thus $[T_W]_{\gamma} = \begin{bmatrix}
            0 & 0\\
            6 & 0
        \end{bmatrix}$. Thus the characteristic polynomial of $T_W$ is
        $$f(t) = det(A - tI) = \begin{vmatrix}
            -t & 0\\
            6 & -t
        \end{vmatrix} = t^2$$
    \end{solution}
    \question Sec. 5.4 Problem 10 (for 6(a),(b))\\
    For each linear operator in Exercise 6, find the characteristic polynomial $f(t)$ of $T$, and verify that the characteristic polynomial of $T_W$ divides $f(t)$.\\
    \begin{solution}
        \textbf{Case: 6(a)}\\
        We can see that $T = L_A$ for
        $$A = \begin{bmatrix}
            1 & 1 & 0 & 0\\
            0 & 1 & -1 & 0\\
            1 & 0 & 1 & 0\\
            1 & 0 & 0 & 1
        \end{bmatrix}$$
        The characteristic polynomial of $T$ is
        $$f(t) = det(A - tI) = \begin{vmatrix}
            1-t & 1 & 0 & 0\\
            0 & 1-t & -1 & 0\\
            1 & 0 & 1-t & 0\\
            1 & 0 & 0 & 1-t
        \end{vmatrix} = t^4 - 4t^3 + 6t^2 - 3t$$
        The characteristic polynomial of $T_W$ is
        $$f(t) = -t^3 + 3t^2 - 3t$$
        We can see that $f(t) = (1-t)(t^3 - 3t^2 + 3t)$. Thus $f_{W}(t)$ divides $f(t)$.\\
        \textbf{Case: 6(b)}\\
        We can see that $T = L_A$ for
        $$A = \begin{bmatrix}
            0 & 0 & 2 & 0\\
            0 & 0 & 0 & 6\\
            0 & 0 & 0 & 0\\
            0 & 0 & 0 & 0
        \end{bmatrix}$$
        The characteristic polynomial of $T$ is
        $$f(t) = det(A - tI) = \begin{vmatrix}
            -t & 0 & 2 & 0\\
            0 & -t & 0 & 6\\
            0 & 0 & -t & 0\\
            0 & 0 & 0 & -t
        \end{vmatrix} = t^4$$
        The characteristic polynomial of $T_W$
        $$f(t) = t^2$$
        We can see that $f(t) = t^2(t^2)$. Thus $f_{W}(t)$ divides $f(t)$.
    \end{solution}
    \question Sec. 5.4 Problem 16\\
    Let $T$ be a linear operator on a finite-dimensional vector space $V$ 
    \begin{parts}
        \part Prove that if the characteristic polynomial of $T$ splits, then so oes the characteristic polynomial of the restriction of $T$ to any $T$-invariant subspace of $V$.
        \part Deduce if the characteristic polynomial of $T$ splits, then any non trivial $T$-invariant subspace of $V$ contains an eigenvector of $T$.
    \end{parts}
    \begin{solution}
        \textbf{Part (a)}\\
        Assume the characteristic polynomial of $T$ splits. Let $W$ be a $T$-invariant subspace of $V$. Let $T_W$ be the restriction of $T$ to $W$. Let $\gamma$ be a basis for $W$. Since $T$ and $T_W$ is a linear operator, $T = L_A$ for some $A$ and $T_W = L_B$ for some $B$. We can see that $B$ is a submatrix of $A$. Thus the characteristic polynomial of $T_W$ is a factor of the characteristic polynomial of $T$.\\
        Thus if the characteristic polynomial of $T$ splits, then so does the characteristic polynomial of the restriction of $T$ to any $T$-invariant subspace of $V$.\\
        \textbf{Part (b)}\\
        Assume the characteristic polynomial of $T$ splits. Let $W$ be a non-trivial $T$-invariant subspace of $V$. By part (a) we know that the characteristic polynomial of $T_W$ is a factor of the characteristic polynomial of $T$. Since the roots of the characteristic polynomial are eigenvalues of a Linear operator, there must be some eigenvector of $T$ in $W$.
    \end{solution}
    \question Sec. 5.4 Problem 18\\
    Let $A$ be an $n \times n$ matrix with characteristic polynomial 
    $$f(t) = (-1)^n t^n + a_{n-1}t^{n-1} + \dots + a_1 t + a_0 $$
    \begin{parts}
        \part Prove that $A$ is invertable iff $a_0 \neq 0$.
        \part Prove that of $A$ is invertable, then $A^-1 = (-1/a_0)[(-1)^n A^{n-1} + a_{n-1}A^{n-2} + \dots + a_1I]$.
        \part Use (b) to compute $A^{-1}$ for 
        $$ A = \begin{bmatrix}
            1 & 2 & 1\\
            0 & 2 & 3\\
            0 & 0 & -1
        \end{bmatrix}$$
    \end{parts}
    \begin{solution}
        \textbf{Part (a)}\\
        We can see that $a_0 = det(A)$ as the characteristic polynomial is generated by $det(A-tI)$ and if we take $t = 0$ we get $det(A)$ and all elements of $f(t)$ (the characteristic polynomial ) go to zero except the constant $a_0$ term. Thus $a_0 = det(A)$.\\
        Thus $A$ is invertible iff $det(A) \neq 0$ iff $a_0 \neq 0$.\\
        \textbf{Part (b)}\\
        From the Cayley Hamilton theorem we know that a matrix $A$ satisfies its characteristic polynomial. Thus $f(A) = 0$. Thus we can see that
        $$ (-1)^n A^n + a_{n-1}A^{n-1} + \dots + a_1 A + a_0I = A_0$$
        were $A_0$ is the zero matrix. Thus we can see that
        $$A ( (-1)^n A^{n-1} + a_{n-1}A^{n-2} + \dots + a_1I) = -a_0I$$
        Thus $A^{-1} = (-1/a_0)[(-1)^n A^{n-1} + a_{n-1}A^{n-2} + \dots + a_1I]$.\\
        \textbf{Part (c)}\\
        First we can compute the characteristic polynomial of $A$ as
        $$det(A-tI) = \begin{vmatrix}
            1-t & 2 & 1\\
            0 & 2-t & 3\\
            0 & 0 & -1-t
        \end{vmatrix} = (1-t)(2-t)(-1-t) = -t^3 + 2t +t -2 $$
        Thus $a_0 = -2, a_1 = 1, a_2 = 2$
        We also can see that $A^2 = \begin{bmatrix}
            1 & 6 & 6\\
            0 & 4 & 3\\
            0 & 0 & 1
        \end{bmatrix}$
        Thus 
        $$ A^{-1} = (1/2)\left( A^2 + 2A +I\right)$$
        $$ = (1/2)\left(- \begin{bmatrix}
            1 & 6 & 6\\
            0 & 4 & 3\\
            0 & 0 & 1
        \end{bmatrix} + 2 \begin{bmatrix}
            1 & 2 & 1\\
            0 & 2 & 3\\
            0 & 0 & -1
        \end{bmatrix} + \begin{bmatrix}
            1 & 0 & 0\\
            0 & 1 & 0\\
            0 & 0 & 1
        \end{bmatrix} \right)$$
        $$ = \begin{bmatrix}
            1 & -1 & -2\\
            0 & 1/2 & -3/2\\
            0 & 0 & -1
        \end{bmatrix}$$
        Clearly we can see that this is the inverse of $A$.
    \end{solution}
    \question Sec. 5.4 Problem 21\\
    Let $T$ be a linear operator on a two-dimensional vector space $V$. Prove that either $V$ is a $T$-cyclic subspace of itself or $T = cI$ for some scalar $c$.
    \begin{solution}
        Let $T$ be a linear operator on a two-dimensional vector space $V$. Let $z$ be a vector in $V$. \\
        We need to prove that either $V$ is a $T$-cyclic subspace of itself or $T = cI$ for some scalar $c$.\\
        Let us consider two cases $T(z) = cIz$ and $T(z) \neq cIz$.\\
        In other words we will consider if $z$ is an eigenvector of $T$ or not.\\
        \textbf{Case 1: $T(z) = cIz$}\\
        Clearly we get the our second condition that we need to prove of $T = cI$ for some scalar $c$.\\
        \textbf{Case 2: $T(z) \neq cIz$}\\
        Let us consider the set $\setof{z, T(z)}$. Since $T(z) \neq cIz$, we can see that $z$ and $T(z)$ are linearly independent. Thus $\setof{z, T(z)}$ is a basis for $V$ since it is a linearly independent set of vectors in a two-dimensional vector space.
    \end{solution}
\end{questions}

\end{document}